\section{Das charakteristische Polynom}
\subsection{Definition}\index{Eigenwert,-vektor,-raum}
\begin{Definition}[Eigenwert,Eigenvektor,Eigenraum]
	Seien $ V $ ein $ K $-VR und $ f\in\End(V) $. Dann heißen
		\begin{enumerate}[(i)]
			\item $ x\in K $ ein Eigenwert von $ f $, falls
				\[ \exists v\in V^\times: f(v)=vx; \]
			\item $ v\in V^\times $ ein Eigenvektor von $ f $, falls
				\[ \exists x\in K:f(v)=vx; \]
			\item $ \ker(f-\id_Vx) \subset V $ ein Eigenraum, falls
				\[ \ker(f-\id_Vx) \neq \{0\}.\]
		\end{enumerate}
	\end{Definition}
\paragraph{Bemerkung}
	Der Skalar $ x\in K $ ist genau dann ein Eigenwert von $ f\in \End(V) $, wenn $ \ker(f-\id_Vx)\neq \{0\} $, d.h., wenn ein Eigenvektor $ v\in V^\times $ zu $ x $ existiert.
\paragraph{Beispiel}
	Für $ \frac{d}{ds} \in \End(C^\infty(\mathbb{R}))$ ist jedes $ x\in \mathbb{R} $ ein Eigenwert, da
		\[ \Big(\frac{d}{ds}-\id_Vx\Big)v = 0 \text{ für } v:\mathbb{R}\to\mathbb{R},s\mapsto v(s):= e^{xs}, \]
	wobei $ v\in C^\infty(\mathbb{R})\setminus \{0\} $, d.h. $ s\mapsto v(s)=e^{xs} $ ist ein Eigenvektor zum Eigenwert $ x\in\mathbb{R} $.
\paragraph{Beispiel}
	Ist $ \dim V < \infty $, so kann die Determinante zur Bestimmung von Eigenwerten von Endomorphismen $ f\in\End(V) $ benutzt werden, da
		\[ \ker(f-\id_Vx)\neq \{0\} \Leftrightarrow (f-\id_Vx) \text{ nicht injektiv}\Leftrightarrow \det(f-\id_Vx) = 0, \]
	d.h. das Auffinden von Eigenwerten $ x\in K $ von $ f $ ist reduziert auf die Bestimmung der Nullstellen der Funktion
		\[ K\ni x\mapsto \det(f-\id_Vx)\in K. \]
		
\paragraph{Beispiel}	
	Ist z.B. $ (b_1,b_2) $ Basis von $ V $ und $ f\in \End(V) $ durch $ f(B)=BX $ gegeben, so liefern die Nullstellen der Polynomfunktion
		\begin{gather*}
		\det(f-\id_Vx) = \det(X-E_2 x)= \det \begin{pmatrix}
		x_{11}-x & x_{12}\\
		x_{21} & x_{22} -x
		\end{pmatrix}\\
	= (x_{11}-x)(x_{22}-x)-x_{12}x_{21}
	= x^2 - x(x_{11}+x_{22}) + (x_{11}x_{22}-x_{12}x_{21})
		\end{gather*}
	die Eigenwerte von $ f $ -- beispielsweise erhalten wir für
		\[ X = \begin{pmatrix} 2 &3\\1 & 0 \end{pmatrix}:\ 
			\det(f-\id_Vx) = x^2-2x-4 = (x+1)(x-3), \]
	also Eigenwerte $ x_1 = -1 $ und $ x_2 = 3 $ mit zugehörigen Eigenvektoren als Lösungen von
		\[ v_i \in \ker(f-\id_Vx_i), \]
	also durch Lösungen der linearen Gleichungssysteme
		\[ \begin{pmatrix}
		2-(-1) & 3\\ 1 & -(-1)
		\end{pmatrix}
		\begin{pmatrix}
		v_1^1\\v_1^2
		\end{pmatrix} = \begin{pmatrix}
		3 & 3\\ 1 & 1
		\end{pmatrix}
		\begin{pmatrix}
		v_1^1\\v_1^2
		\end{pmatrix} \text{ und} \]
		\[ \begin{pmatrix}
		2-3 & 3\\ 1 & -3
		\end{pmatrix}
		\begin{pmatrix}
		v_2^1\\v_2^2
		\end{pmatrix}=
		\begin{pmatrix}
		-1 & 3\\ 1 & -3
		\end{pmatrix}
		\begin{pmatrix}
		v_2^1\\v_2^2
		\end{pmatrix}  \]
	sodass
		\[ v_1 = b_1-b_2 \text{ und } v_2 = b_13+b_2 \]
	Eigenvektoren zu den Eigenwerten $ x_1,x_2 $ liefert.

\paragraph{Rechenbeispiel 1}
	Für $ X = \begin{pmatrix}2&-1\\1&0\end{pmatrix} $ erhält man
		\[ \det(f-\id_Vx) = \det\begin{pmatrix}2-x&-1\\1&-x	\end{pmatrix} =x^2-2x+1 \]
	und Eigenvektoren zum Eigenwert $ x = 1 $ durch Lösung der LGS
		\[ \begin{pmatrix}
		2-1&-1\\1&-1
		\end{pmatrix}\begin{pmatrix}
		v_1^1\\v_1^2
		\end{pmatrix} =  \begin{pmatrix}
		1&-1\\1&-1
		\end{pmatrix}\begin{pmatrix}
		v_1^1\\v_1^2
		\end{pmatrix} \]
	d.h. der Eigenraum zum Eigenwert $ x $,
		\[ \ker(f-\id_V) = [\{b_1+b_2\}] \]
	hat
		\[ \dim \ker(f-\id_V)<\dim V. \]
\paragraph{Rechenbeispiel 2}
	Ist $ K=\mathbb{R} $ und
		\[ \det(f-\id_Vx)=x^2+1, \]
	so hat $ f $ keine Eigenwerte: z.B., wenn
		$ X=\begin{pmatrix} 0&1\\-1&0 \end{pmatrix} $.
		
\subsection{Definition} \index{Charakteristisches Polynom}
\begin{Definition}[Charakteristisches Polynom]
	Sei $ V $ ein $ K $-VR, für $ f\in\End(V) $ ist das \emph{charakteristische Polynom} von $ f $:
		\[ \chi_f(t) := \det (\id_Vt-f)\in K[t]. \]
	Analog definiert man für $ X\in K^{n\times n} $ das charakteristische Polynom
		\[ \chi_f(t) := \det (E_nt-X)\in K[t]. \]
\end{Definition}
\paragraph{Bemerkung}
	Oft wird auch das andere Vorzeichen in der Determinante verwendet, also $ \det(f-\id_Vt) $ bzw. $ \det(X-E_nt) $.
\paragraph{Bemerkung}
	\emph{Diese Definition ist erklärungsbedürftig!}
	
	Da $ t\notin K $ ist $ \id_Vt-f\notin \End(V) $, sondern $ \id_Vt-f\in\End(V)[t] $. Zwei Lösungsstrategien bieten sich an:
		\begin{enumerate}
			\item Erweiterung der Determinante auf $ \End(V)[t] $.
			\item Benutzung von Darstellungsmatrizen.
		\end{enumerate}
	Beide führen schließlich zur Leibniz-Formel:
	
	Ist $ B $ eine Basis von $ V $ und $ \xi_B^B(f) = X = (x_{ij})_{i,j\in\{1,\dots,n\}}$, so erhält man 
		\[ \chi_f(t)=\sum_{\sigma\in S_n}\sgn(\sigma)\prod_{j=1}^{n}\underset{\in K[t]}{\underbrace{\left(\delta_{\sigma(j)j}-x_{\sigma(j)j}\right)}} \in K[t]. \]
	Die Unabhängigkeit von der Basis $ B $ folgt aus der Transformationsformel für Darstellungsmatrizen und dem Determinanten-Multiplikationssatz (wie vorher für $ \det f = \det \xi_B^B(f) $).

% VO 2016-03-15 %

\subsection{Bemerkung \& Definition}\index{Spur}
\begin{Definition}[Spur]
	Ist $ \dim V=n $, so ist $ \chi_f(t) $ ein normiertes Polynom vom Grad $ \deg\left(\chi_f(t)\right)=n $,
		\[ \chi_f(t)=t^n-t^{n-1}\tr f + \dots + (-1)^n\det f,\] % = \det(-f) = \chi_f(0)
	wobei die \emph{Spur} $ \tr f $ (\glqq tr \grqq $\widehat{=}$ trace) von $ f $ durch diese Gleichung (wohl-)defininiert ist.
\end{Definition}	
	Ist $ (x_{ij})_{i,j\in\{1,\dots,n\}} = X = \xi_B^B(f) $ Darstellungsmatrix von $ f $, so gilt
		\[ \tr f = \sum_{j=1}^{n}x_{jj} = \sum_{j=1}^{n} b_j^*f(b_j). \]
	Oft wird $ \det(f-\id_vt)=(-1)^n\chi_f(t) $ als charakteristisches Polynom definiert -- dieses Polynom ist dann nur für gerade $ n $ normiert.
\subsection{Korollar}
\begin{Korollar}[Eigenwerte sind Nullstellen des char. Polynoms]
	Ein $ x\in K $ ist genau dann Eigenwert von $ f $, wenn $ \chi_f(x)=0 $.
	
	Also: Die Eigenwerte von $ f $ sind genau die Nullstellen des charakteristischen Polynoms $ \chi_f(t) $.
\end{Korollar}
\paragraph{Beweis}
	Klar -- das war die Idee hinter der Definition des charakteristischen Polynoms.
\subsection{Korollar \& Definition}\index{Algebraische/geometrische Vielfachheit}
\begin{Korollar}[Eigenwert ist Nullstelle des charakteristischen Polynoms]
	Ist $ x\in K $ Eigenwert von $ f\in\End(V) $, so ist $ (t-x) $ Teiler des charakteristischen Polynoms. Insbesondere gilt:
		\[ \exists!k\in \mathbb{N}^\times:
			\begin{cases}
				(t-x)^k\mid \chi_f(t)\\
				(t-x)^{k+1}\nmid \chi_f(t)
			\end{cases} \]
\end{Korollar}
\begin{Definition}[algebraische Vielfachheit, geometrische Vielfachheit]
	Diese Zahl $ k $ heißt die \emph{algebraische Vielfachheit} von $ x $;
		\[ g:= \dfkt(\id_Vx-f) \leq k \]
	ist die \emph{geometrische Vielfachheit} von $ x $.
\end{Definition}
\paragraph{Beweis}
	Da $ x $ Eigenwert von $ f $ ist, ist die Existenz und Eindeutigkeit von $ k $ klar. Außerdem gilt analog auch $ g\geq 1 $.
	
	Zu zeigen bleibt: $ g\leq k $, d.h. $ (t-x)^g \mid \chi_f(t) $:
	
	Für eine Basis $ B = (b_1,\dots,b_n) $ von $ V $ mit
	$ \ker (\id_v x - f) = [(b_1,\dots,b_g)]$
	hat
		\[ \xi_B^B(f) =
		\begin{pmatrix}
			E_gx & Y\\
			0 & X
		\end{pmatrix}
		\text{ mit } Y\in K^{g\times (n-g)}, X\in K^{(n-g)\times(n-g)} \]
	Blockgestalt, also ist
		\[ \chi_f(t)=(t-x)^g\cdot \chi_X(t), \]
	d.h. $ (t-x)^g \mid \chi_f(t)$, da $ (t-x)^{k+1}\nmid \chi_f(t) $, gilt also $ g\leq k $.
\paragraph{Beispiel}
	Ist $ f\in\End(V) $ wie oben durch $ f(B)=BX $ gegeben, so haben die Eigenwerte
		\[ x_1 = -1 \text{ und } x_2 = 3 \text{ für }
		X=\begin{pmatrix} 2 &3\\1 & 0 \end{pmatrix} \]
	algebraische und geometrische Vielfachheiten 
		\[ 1 = g_i = k_i, \text{ da } 1\leq g_i \leq k_i \text{ und } k_1+k_2 \leq 2; \]
	der Eigenwert
		\[ x=1 \text{ für } X = \begin{pmatrix} 2&-1\\1&0 \end{pmatrix} \]
	hat algebraische und geometrische Vielfachheiten
		\[ k = 2 \text{ und } g = 1 \]
	da
		\[ f\neq \id_V x = \id_V \]
	und $ \chi_f(t)=(t-x)^2 \in \mathbb{R}[t] $, da ein quadratisches Polynom zwei (relle oder komplex konjugierte) Nullstellen hat, oder aber eine doppelte reelle.

\subsection{Definition \& Lemma}
	Das Schlüsselargument im Beweis oben kann man verallgemeinern:

\begin{Definition}[$ f $-invarianter Unterraum]
	Sei $ f\in \End(V) $ und $ U\subset V $ ein \emph{$ f $-invarianter Unterraum}, d.h. $ f(U)\subset U $. 

\end{Definition}
\begin{Lemma}[]
	Ist dann $ V=U\oplus U' $ eine direkte Zerlegung und $ p,p'\in \End(V) $ die zugehörigen Projektionen, so gilt
		\[ \chi_f(t)=\chi_{f|_U}(t)\cdot \chi_{f'}(t), \]
	wobei
		\[ f':= p'\circ f|_{U'}\in \End(U'). \]

\end{Lemma}
\paragraph{Bemerkung}
	Man kann $ f|_U $ als Endomorphismus $ f|_U\in \End(U) $ auffassen, da $ f(U)\subset U $.
\paragraph{Beweis}
	Wie oben: Sei $ B=(b_1,\dots,b_n) $ Basis von $ V $, sodass
		\begin{itemize}
			\item $ C=(b_1,\dots,b_k) $ Basis von $ U $ und
			\item $ C'=(b_{k+1},\dots,b_n) $ Basis von $ U' $ ist.
		\end{itemize}
	Die Darstellungsmatrix von $ f $ bzgl. $ B $ hat dann Blockgestalt,
		\[ \xi_B^B(f) =
			\begin{pmatrix}
				X&Y\\0&X'
			\end{pmatrix}
		\text{ mit } X=\xi_C^C(f|_U), X' = \xi_{C'}^{C'}(f') \]
	Damit folgt die Behauptung (wie oben) mit der Leibniz-Formel.
\paragraph{Bemerkung}
	Alternativ kann man das Lemma mit der von $ f $ induzierten Quotientenabbildung $ f'\in \End(V/U) $ formulieren, wobei
		\[ f':V/U\to V/U, v+U\mapsto f'(v+U) := f(v)+U \]
\subsection{Definition}\index{Diagonalisierbarkeit}\index{Triagonalisierbarkeit}
\begin{Definition}[Diagonalisierbarkeit, Triagonalisierbarkeit von Endomorphismen]
	Ein Endomorphismus $ f\in\End(f) $ heißt \emph{diagonalisierbar} bzw. \emph{trigonalisierbar}, falls es eine Basis $ B $ von $ V $ gibt, sodass $ \xi_B^B(f)=(x_{ij})_{i,j\in\{1,\dots,n\}} $ eine Diagonalmatrix 
		\[ i\neq j\Rightarrow x_{ij} = 0 \]
	bzw. obere Dreiecksmatrix ist,
		\[ i>j \Rightarrow x_{ij} = 0 \]
\end{Definition}
\paragraph{Bemerkung}
	Falls $ \dim V<\infty $, so ist $ f\in\End(V) $ genau dann diagonalisierbar, wenn $ V $ eine Basis aus Eigenvektoren von $ f $ besitzt. Damit kann man "`Diagonalisierbarkeit"' auch im Falle $ \dim V=\infty $ definieren.
\paragraph{Bemerkung}
	Ist $ f $ trigonalisierbar (oder gar diagonalisierbar), so zerfällt $ \chi_f (t) $ in Linearfaktoren: für geeignete $ x_1,\dots,x_n\in K $ ist
		\[ \chi_f(t)=\prod_{j=1}^{n}(t-x_j). \]
\subsection{Bemerkung \& Definition}
\begin{Definition}[Diagonalisierbarkeit, Triagonalisierbarkeit von Matrizen]
	Man nennt eine Matrix $ K\in K^{n\times n} $ diagonalisierbar (bzw. trigonalisierbar), falls $ f_X\in \End(K^n) $ diagonalisierbar (bzw. trigonalisierbar) ist.
\end{Definition}	

	Dies ist genau dann der Fall, falls es $ P\in Gl(n) $ gibt, sodass $ PXP^{-1} $ Diagonalmatrix (bzw. obere Dreiecksmatrix) ist.