%VO19-2015-12-15
\section{Lineare Gleichungssysteme}
 Mission: Viele Probleme in Anwendungen oder Naturwissenschaften werden zu "`linearen Problemen"' reduziert, d.h. auf lineare Gleichungssysteme unterschiedlicher Komplexität.
 Diese Reduktion ist etwa eine wichtige Aufgabe der Analysis; Aufgabe der linearen Algebra ist dann die Lösung bzw. Strukturanalyse der linearen Gleichungssysteme.
 \subsection{Definition}
 	\begin{Definition}[Lineares Gleichungssystem]
 		Ein \emph{lineares Gleichungssystem} (LGS) ist ein System von $ m $ Gleichungen
 		\[
 			\begin{array}{cccc}\tag{$\star\star$}
 				a_{11}x_1+  & \dots & + a_{1n}x_n & =y_1   \\
 				\vdots      &       & \vdots      & \vdots \\
 				a_{m1}x_1 + & \dots & +a_{mn}x_n  & = y_m
 			\end{array}
 		\]
 		für $ n $ Unbekannte $ x_1,\dots,x_n\in K $, wobei die Parameter $ a_{ij},y_i\in K $ gegeben sind. Ist $ y_1 = \dots = y_m = 0 $, so heißt das System \emph{homogen}, anderenfalls \emph{inhomogen}.
 	\end{Definition}
 	\paragraph{Bemerkung}
 		Mit Matrizen $ A\in K^{m\times n},X\in K^{n\times 1} $ und $ Y\in K^{m\times 1} $ lässt sich ein lineares Gleichungssystem kompakter schreiben als
 		\[
 			AX = Y \tag{$\star$}
 		\]
 		Die Standardbasen $ E $ und $ E' $ von $ K^n $ bzw. $ K^m $ liefern den Isomorphismus
 		\[
 			K^{m\times n}\ni A\mapsto f_A\in \hom(K^n,K^m)\text{, wobei }f_A(E) = E'A,
 		\]
 		damit lässt sich ($\star$) umformulieren als Gleichung eines affinen Unterraumes von $ K^n: $
 		\[
 			f_A(x) = y \text{ mit } x=EX \text{ und } y=E'Y.
 		\]
 		Nämlich: Existiert eine Lösung $ x\in f_A^{-1}(\{y\})\neq \emptyset $, so ist der Lösungsraum
 		\[
 			f_A^{-1}(\{y\}) = x+\ker f_A\subset K^n
 		\]
 		ein affiner Unterraum.

 		Das nächste Lemma folgt dann mit dem Basisisomorphismus:
 		\[
 			K^{n\times 1} \ni X \mapsto EX =: x\in K^n
 		\]
 \subsection{Definition \& Lemma}
 	\begin{Lemma}[Lösungsraum]
 		Der \emph{Lösungsraum} $ L_{A,Y} $ eines linearen Gleichungssystems,
 		\[
 			L_{A,Y}:=\{X\in K^{n\times 1}\mid AX=Y\}\subset K^{n\times 1}
 		\]
 		ist leer oder ein affiner Unterraum der Dimension $ k = n-\rg A $.

 		Ist $ Y = 0 $, so gilt $ 0\in L_{A,Y} $ und $ L_{A,Y}\subset K^{n\times 1} $ ist ein linearer Unterraum (UVR).
 	\end{Lemma}

%VO20-2015-12-17
 	\paragraph{Bemerkung}
 		Jede Lösung $ X_l $ eines (inhomogenen) LGS $AX=Y$ lässt sich schreiben als Summe einer \emph{Partikulärlösung} $ X_0 \in K^{n\times 1},AX_0 = Y $, und einer Lösung $V$ des homogenen LGS $ AX = 0 $:
 		\[
 			\forall X_l\in L_{A,Y}\exists V\in L_{A,0}:X_l=X_0+V=\tau_V(X_0)
 		\]
 	\paragraph{Bemerkung}
 		Der Lösungsraum eines "`unendlichen linearen Gleichungssystems"' hat die gleiche Struktur eines affinen Unterraums wie im endlichen Fall, z.B.:
 		\[
 			\{x\in C^\infty(\mathbb{R})\mid \forall t\in \mathbb{R}:x''(t) = t^2 \}
 		\]
 		ist ein (2-dim) AUR, des unendlich-dim. R-VR $C^\infty$, wobei $ C^\infty(\mathbb{R}) $ den (Vektor-)Raum der beliebig oft differenzierbaren Funktionen auf $ \mathbb{R} $ notiert.
 \subsection{Bemerkung \& Definition}
 	\begin{Definition}[Erweiterte Koeffizientenmatrix]
 		Ist $ AX=Y \neq 0$ ein inhomogenes LGS, so gilt
 		\[
 			L_{A,Y} = \emptyset \Leftrightarrow y\notin f_A(K^n)
 		\]
 		mit der \emph{erweiterten Koeffizientenmatrix}
 		\[
 			(A\mid Y) \in K^{m\times (n+1)}
 		\]
 		lässt sich dies formulieren als
 		\[
 			f_{(A\mid Y)}(K^{n+1})\neq f_A(K^n) \Leftrightarrow \rg f_{(A\mid Y)}\neq \rg f_A \Leftrightarrow \rg (A\mid Y) \neq \rg A.
 		\]
 		Folglich ist
 		\[
 			L_{A,Y} \neq \emptyset \Leftrightarrow \rg (A\mid Y) = \rg A
 		\]
 	\end{Definition}
 \subsection{Bemerkung \& Definition, Gaußsches Eliminationsverfahren}
 	\begin{Definition}[Äquivalente LGS]
 		Eine Idee zur Lösung eines LGS ist, das Gleichungssystem zu "`vereinfachen"', ohne dabei den Lösungsraum zu verändern: Man nennt zwei LGS $ AX=Y$ und $A'X=Y' $ \emph{äquivalent}, wenn sie den gleichen Lösungsraum haben,
 		\[
 			(AX=Y)\sim (A'X=Y'):\Leftrightarrow L_{A,Y} = L_{A',Y'}
 		\]
 	\end{Definition}
 	\emph{Links}multiplikation der erweiterten Koeffizientenmatrix $ (A\mid Y) $ mit den folgenden Matrizen (mit $ i\neq j $) liefert z.B. äquivalente Systeme:

 	$ D_i = (d_{kl})\in \mathrm{Gl}(m), \qquad
 	d_{kl} := \delta_{kl}+(d-1)\delta_{ik}\delta_{il}\quad (d\in K^x) $
 	\[
 		\bordermatrix{
 			&   &   & i & &\cr
 			& 1 & 0 & \dots & \dots & 0\cr
 			& 0 & \ddots & \ddots &   & \vdots \cr
 			i & \vdots & \ddots & d & \ddots & \vdots \cr
 			& \vdots &   &  \ddots & \ddots & 0 \cr
 			& 0 &  \dots & \dots  & 0 & 1 \cr
 		}
 	\]
 	$ T_{ij} = (t_{kl})\in \mathrm{Gl}(m), \qquad
 	t_{kl} := \delta_{kl}-(\delta_{ik}-\delta_{jk})(\delta_{il}-\delta_{jl})$
 	\[
 		\bordermatrix{
 			&        & i      & \dots  & j     &         \cr
 			& 1      &        &        &       &         \cr
 			i     &        & 0      &        & 1     &         \cr
 			\vdots&        &        & 1      &       &         \cr
 			j     &        & 1      &        & 0     &         \cr
 			&        &        &        &       & 1       \cr
 		}
 	\]

 	$ S_{ij}=(s_{kl})\in \mathrm{Gl}(m), \qquad
 	s_{kl} := \delta_{kl}+s\delta_{ik}\delta_{jl} \quad (s\in K)$
 	\[
 		\bordermatrix{
 			&   &   &  & j &\cr
 			& 1 & 0 & \dots & \dots & 0\cr
 			i & 0 & \ddots & \ddots & s & \vdots \cr
 			& \vdots & \ddots & 1 & \ddots & \vdots \cr
 			& \vdots &   &  \ddots & \ddots & 0 \cr
 			& 0 &  \dots & \dots  & 0 & 1 \cr
 		}
 	\]

 	Die entsprechenden Operationen auf dem LGS werden als \emph{elementare Zeilenoperationen/-umformungen} bezeichnet (elZumf) bezeichnet:
 	\begin{itemize}
 		\item $ (A\mid Y) \to D_i (A\mid Y) $, Multiplikation der $ i $-ten Gleichung mit $ d\neq 0 $;
 		\item $ (A\mid Y) \to T_{ij} (A\mid Y) $, Vertauschung der $ i $-ten und $ j $-ten Gleichung;
 		\item $ (A\mid Y) \to S_{ij} (A\mid Y)$, Addition des $ s $-fachen der $ j $-ten Gleichung zur $ i $-ten Gleichung.
 	\end{itemize}
 	Da $ D_i,T_{ij},S_{ij}\in \mathrm{Gl}(m) $, sind die elementaren Zeilenumformungen reversibel, verändern daher den Lösungsraum nicht: für $ D_i $ und $ T_{ij} $ ist das klar; $ S_{ij} = S_{ij}(s) $ ist invertierbar mit
 	\[
 		(S_{ij})^{-1} = (S_{ij}(s))^{-1} = S_{ij}(-s).
 	\]
 	Geometrisch ist $ S_{ij} $ Darstellungsmatrix einer Scherung.

 	\begin{Definition}[Zeilenstufenform]
 		Mit Hilfe der elementaren Zeilenumformungen kann man das LGS auf Zeilenstufenform bringen:
 		\[
 			\begin{pmatrix}
 				a_{11} & \dots & a_{1n} & y_1    \\
 				a_{21} & \dots & a_{2n} & y_2    \\
 				a_{31} & \dots & a_{3n} & y_3    \\
 				\vdots &       & \vdots & \vdots \\
 				a_{m1} & \dots & a_{mn} & y_m
 			\end{pmatrix}
 			\xrightarrow{\text{elZUmf}}
 			\begin{pmatrix*}[l]
 				1 & \dots & \dots &\dots & \dots & y'_1\\
 				0 & 1 & \dots &\dots & \dots & y'_2\\
 				\vdots &\ddots & \ddots & &  &\vdots \\
 				0  & \dots  &  0 & 1 & \dots &y'_r\\
 				0  & \dots  & \dots    & 0 & 0 & y'_{r+1}\\
 				\vdots & & & & \vdots & \vdots \\
 				0 & \dots & \dots   & \dots &  0 & y'_{m}\\
 			\end{pmatrix*}
 		\]
 		Ein System in Zeilenstufenform kann dann einfach gelöst werden -- oder auch nicht, falls eine Gleichung $ 0 = y' \neq 0 $ auftaucht.
 	\end{Definition}
 	\paragraph{Beispiel}
 		Wir betrachten das LGS $ AX=Y $ mit
 		\[
 			A =
 			\begin{pmatrix}
 				0 & 3 & 6 \\
 				1 & 4 & 7 \\
 				2 & 5 & 8
 			\end{pmatrix}
 			\text{ und }
 			Y =
 			\begin{pmatrix}
 				y_1 \\ y_2 \\ y_3
 			\end{pmatrix}
 			.
 		\]
 		Elementare Zeilenumformungen liefern dann:
 		\begin{align*}
 			\begin{pmatrix}
 			0 & 3  & 6  & y_1             \\
 			1 & 4  & 7  & y_2             \\
 			2 & 5  & 8  & y_3
 			\end{pmatrix}
 			\quad
 			\overset{T_{12}}{\to}\quad
 			&
 			\begin{pmatrix}
 			1 & 4  & 7  & y_2             \\
 			0 & 3  & 6  & y_1             \\
 			2 & 5  & 8  & y_3
 			\end{pmatrix}
 			\\
 			\overset{S_{31}(-2)}{\to}\quad
 			&
 			\begin{pmatrix}
 			1 & 4  & 7  & y_2             \\
 			0 & 3  & 6  & y_1             \\
 			0 & -3 & -6 & y_3-2y_2
 			\end{pmatrix}
 			\\
 			\overset{S_{32}(1)}{\to}\quad
 			&
 			\begin{pmatrix}
 			1 & 4  & 7  & y_2             \\
 			0 & 3  & 6  & y_1             \\
 			0 & 0  & 0  & y_3-2y_2+y_1
 			\end{pmatrix}
 			\\
 			\overset{D_2(\frac{1}{3})}{\to}\quad
 			&
 			\begin{pmatrix}
 			1 & 4  & 7  & y_2             \\
 			0 & 1  & 2  & y_1 \frac{1}{3} \\
 			0 & 0  & 0  & y_3-2y_2
 			\end{pmatrix}
 		\end{align*}
 		d.h. ein äquivalentes LGS $ A'X=Y' $ ist gefunden mit
 		\[
 			A' =
 			\begin{pmatrix}
 				1 & 4 & 7 \\
 				0 & 1 & 2 \\
 				0 & 0 & 0
 			\end{pmatrix}
 			\text{ und }
 			Y =
 			\begin{pmatrix}
 				y_2 \\ y_1\frac{1}{3} \\ y_1-2y_2+y_3
 			\end{pmatrix}
 			.
 		\]
 		Das LGS $ AX=Y $ ist also genau dann lösbar, wenn $ y_1-2y_2+y_3 = 0 $; in diesem Falle ist dann
 		\[
 			L_{A,Y} = \{X =
 			\begin{pmatrix} y_2-7t-4(-2t+\frac{1}{3}y_1)\\-2t+\frac{1}{3}y_1\\t\end{pmatrix}
 			, t\in \mathbb{R}\}
 			= \{X =
 			\begin{pmatrix} t-\frac{4}{3}y_1+y_2\\-2t+\frac{1}{3}y_1\\t\end{pmatrix}
 			, t\in \mathbb{R}\}
 		\]
 	\paragraph{Historische Bemerkung}
 		Das Gaußsche Eliminationsverfahren ist schon seit ca. 2000 Jahren bekannt, also schon lange vor Gauß (1777-1855) entwickelt worden.

 	\paragraph{Nutzen der Methode}
 		\begin{itemize}
 			\item lässt sich einfach programmieren (leider ggf. numerisch instabil)
 			\item nützlich für mittelgroße Systeme (tausende Gleichungen)
 			\item nicht effizient für große Systeme (Millionen von Gleichungen)
 		\end{itemize}
 	\paragraph{Bemerkung}
 		Mehrere LGS $ AX = Y_1, AX = Y_2, \dots AX = Y_k $ mit derselben Koeffizientenmatrix $ A $ können simultan gelöst werden, indem man elementare Zeilenumformungen auf die um alle $ Y $ erweiterte Koeffizientenmatrix $ (A\mid Y_1\mid Y_2\mid \dots \mid Y_k) $ anwendet.
 	\paragraph{Bemerkung}
 		Das Gaußsche Eliminationsverfahren kann zur Bestimmung der Inversen einer Matrix $ A\in \mathrm{Gl}(n) $ verwendet werden. Insbesondere ist eine \emph{untere Dreiecksmatrix} $ A\in K^{n\times n} $, d.h $ a_{ij} = 0 $ für $ i<j $ genau dann invertierbar, wenn $ a_{ii}\neq 0 $ für alle $ i=1,\dots,n $.
