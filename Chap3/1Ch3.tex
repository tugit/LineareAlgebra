%VO18-2015-12-10
\chapter{Buchhaltung}
Dieses Kapitel zeigt eine Art "`Tabellenkalkül"' -- eine effiziente Rechenmethode in der linearen Algebra.

Vorteil: Selbst durch einen Trottel (e.g. einen Computer) ausführbar.

Nachteil: Selbst durch einen Trottel ausführbar.

\paragraph{Generalvoraussetzung} Alle VR haben in diesem Kapitel endliche Dimension.
\section{Matrizen}
 \paragraph{Idee}
 	Ein Homomorphismus $ f\in \hom(V,W) $ wird (nach Fortsetzungssatz) durch die Bilder $ f(b_j) $ der Vektoren einer Basis $ (b_j)_{j\in J} $ eindeutig festgelegt; ist $ (c_i)_{i\in I} $ eine Basis von $ W $, so hat jedes dieser $ f(b_j) $ eine eindeutige Basisdarstellung.
 	\[
 		\forall {j\in J}\exists! (x_i)_{i\in I}:f(b_j) = \sum_{i\in I}c_ix_{ij}
 	\]
 	Sind $ n=\dim V $ und $ m=\dim W $ endlich, so kann man also $ f $ mithilfe der Basen $ (b_j)_{j\in J} $ von $ V $ und $ (c_i)_{i\in I} $ von $ W $ komplett durch die Tabelle der Koeffizienten beschreiben:

 	\begin{figure}[H]\centering
 		$
 		\begin{array}{c|cccccc}
 			f      & f(b_1) & f(b_2) & \dots & f(b_j) & \dots & f(b_n) \\\hline
 			c_1    & x_{11} & x_{12} & \dots & x_{1j} & \dots & x_{1n} \\
 			c_2    & x_{21} & x_{22} & \dots & x_{2j} & \dots & x_{2n} \\
 			\vdots & \vdots & \vdots &       & \vdots &       & \vdots \\
 			c_i    & x_{i1} & x_{i2} & \dots & x_{ij} & \dots & x_{in} \\
 			\vdots & \vdots & \vdots &       & \vdots &       & \vdots \\
 			c_m    & x_{m1} & x_{m2} & \dots & x_{mj} & \dots & x_{mn}
 		\end{array}
 		$
 	\end{figure}

 	Dabei spielt es prinzipiell keine Rolle, ob die Bilder $ f(b_j) $ der Basisvektoren in den Spalten stehen (wie oben) oder in den Zeilen der Tabelle -- es ist aber wichtig, dass dies konsistent gemacht wird.

 	In dieser LVA: Bilder $ f(b_j) $ der Basisvektoren werden durch Spalten beschrieben.

\subsection{Definition}
	\begin{Definition}[Matrix]
		Eine \emph{Matrix} $ X\in K^{m\times n} $ ist eine Tabelle von Elementen $ x_{ij}\in K $ mit $ m $ Zeilen und $ n $ Spalten:
		\[
			X =
			\begin{pmatrix}
				x_{11} & \dots & x_{1n} \\
				\vdots &       & \vdots \\
				x_{m1} & \dots & x_{mn}
			\end{pmatrix}
		\]
		Die \emph{(Darstellungs-)Matrix} eines $ f\in \hom(V,W) $ bzgl. Basen $ B= (b_1,\dots,b_n) $ und $ C=(c_1,\dots,c_m) $ von $ V $ bzw. $ W $, ist die Matrix
		\[
			X = \xi^C_B(f)\in K^{m\times n}\text{ mit }\forall j=1,\dots,n:f(b_j) = \sum_{i=1}^{m}c_ix_{ij}.
		\]
	\end{Definition}

	\paragraph{Bemerkung}
		Mit $ I:= \{1,\dots,m\} $ und $ J:= \{1,\dots,n\} $ kann eine Matrix auch als Abbildung aufgefasst werden
		\[
			X = (x_{ij})_{i\in I,j\in J} \quad\text{ bzw. }\quad X:I\times J\to K,\ (i,j)\mapsto x_{ij}.
		\]
		Ist $ f\in \hom(V,W) $ und sind $ B=(b_1,\dots,b_n) $ und $ C=(c_1,\dots,c_m) $ Basen von $ V $ bzw. $ W $, so sind
		\[
			x_{ij} = c_i^*(f(b_j))
		\]
		die Komponenten der Darstellungsmatrix $ \xi_B^C(f) $ von $ f $ bzgl. der Basen $ B $ und $ C $ mit der zu $ C $ dualen Basis $ C^*=(c_1^*,\dots c_m^*) $ von $ W^* $.\\
		Mit der zu $ B $ dualen Basis $ B^*= (b_1^*,\dots,b_n^*) $ von $ V^* $ ist dann auch
		\[
			f=\sum_{i=1}^{m}\sum_{j=1}^{n} c_ix_{ij}b_j^*.
		\]
\subsection{Lemma}
	\begin{Lemma}[Matrizen als VR]
		Mit der komponentenweisen Addition und Skalarmultiplikation auf $ K^{m\times n} $,
		\[
			(x_{ij})+(y_{ij}) := (x_{ij}+y_{ij}) \text{ und } (x_{ij})\cdot z := (x_{ij}\cdot z),
		\]
		wird $ K^{m\times n} $ ein Vektorraum und man erhält einen Isomorphismus zu Basen $B$ und $C$ von $V$ bzw. $W$.
		\[
			\xi_B^C:\hom(V,W)\to K^{m\times n},\ f\mapsto \xi_B^C(f).
		\]
	\end{Lemma}
	\paragraph{Bemerkung}
		Die komponentenweise Addition und Skalarmultiplikation sind gerade die Addition und Skalarmultiplikation von Matrizen als Abbildungen.

	\paragraph{Beweis}
		Dass $ \hom(V,W) $ und $ K^{m\times n} \ K $-VR sind, ist bekannt (vgl. Kap. 1.4 bzw. Kap. 1.1). Die Linearität von $ \xi_B^C $ folgt direkt, da mit der zu $ C $ dualen Basis $ C^* $ von $ W^* $
		\[
			\forall_{i=1,\dots,m} \forall_{j=1,\dots,n}: x_{ij}= c_i^*(f(b_j)).
		\]
		Nämlich: für $ f,g\in \hom(V,W) $ und $ x,y\in K $ ist dann
		\begin{align*}
			\forall_{i= 1,\dots,m} \forall_{j=1,\dots, n} : c_i^*((fx+gy)(b_j)) & = c_i^*(f(b_j)x+g(b_j)y)         \\
			                                                                    & = c_i^*(f(b_j))x+c_i^*(g(b_j))y.
		\end{align*}
		Die Abbildung
		\[
			K^{m\times n}\ni X=(x_{ij})\mapsto \sum_{i=1}^{m}\sum_{j=1}^{n}c_ix_{ij}b_j^* = f\in \hom(V,W)
		\]
		liefert die Inverse von $ f\mapsto\xi_B^C(f) $, also ist $ \xi_B^C $ ein Isomorphismus.
	\paragraph{Bemerkung}
		Damit folgt (vgl. Kap. 1.4): $ \dim \hom(V,W) = \dim K^{m\times n} = m\cdot n $.
\subsection{Lemma \& Definition}
	\begin{Lemma}[Darstellungsmatrix einer Komposition]
		Sind $ U,V,W \ K$-VR mit Basen $ A=(a_1,\dots,a_p),\ B=(b_1,\dots,b_n),\ C=(c_1,\dots,c_m) $, so gilt für $ g\in \hom(U,V) $ und $ f\in \hom(V,W) $
		\[
			\xi_A^C(f\circ g) = \xi_B^C(f)\cdot \xi_A^B(g),
		\]
	\end{Lemma}
	\begin{Definition}
		wobei die \emph{Matrixmultiplikation}
		\[
			\cdot:K^{m\times n}\times K^{n\times p} \to K^{m\times p},\ (X,Y)\mapsto X\cdot Y = Z
		\]
		definiert ist durch
		\[
			z_{ik} := \sum_{j=1}^{n}x_{ij}y_{jk}.
		\]
	\end{Definition}
	\paragraph{Bemerkung}
		Das Element $ z_{ik} $ in der $ i $-ten Zeile und $ k $-ten Spalte von $ Z = XY $ wird also aus der $ i $-ten Zeile von $ X $ und der $k$-ten Spalte von $ Y $ berechnet.
		\begin{align*}
			\left(
			\begin{array}{ccc}
			       &        &        \\
			x_{i1} & \dots  & x_{in} \\
			       &        &        
			\end{array}
			\right)
			\cdot
			\left(
			\begin{array}{ccc}
			       & y_{1k} &        \\
			       & \vdots &        \\
			       & y_{nk} &        
			\end{array}
			\right)
			=
			\left(
			\begin{array}{ccc}
			       &        &	     \\
			       & z_{ik} &        \\
			       &        &        
			\end{array}
			\right)
		\end{align*}
	\paragraph{Beweis}
		Wir verwenden die Darstellungsmatrizen
		\[
			\begin{cases}
				X = \xi^C_B(f)\in K^{m\times n} & \text{ von } f\in \hom(V,W) \\
				Y = \xi_A^B(g)\in K^{n\times p} & \text{ von } g\in \hom(U,V)
			\end{cases}
		\]
		bezüglich $ B $ und $ C $ bzw. $ A $ und $ B $, dann gilt für $ k=1,\dots,p $
		\[
			(f\circ g)(a_k)=f\Big(\sum_{j=1}^{n}b_jy_{jk}\Big) = \sum_{j=1}^{n}f(b_j)y_{jk} = \sum_{j=1}^{n}\sum_{i=1}^{m}c_ix_{ij}y_{jk} = \sum_{i=1}^{m}c_i\Big(\sum_{j=1}^{n}x_{ij}y_{jk}\Big),
		\]
		d.h. durch $ I=\{1,\dots,m\},\ J=\{1,\dots,n\},\ K=\{1,\dots,p\} $ und
		\[
			\xi_A^C(f\circ g) = Z = (z_{ik})_{i\in I,k\in K} \quad\text{mit}\quad \forall i\in I\ \forall k\in K: z_{ik}= \sum_{j=1}^{n}x_{ij}y_{jk}
		\]
		erhält man die Darstellungsmatrix
		\[
			\xi_A^C(f\circ g) = \xi_B^C(f)\xi_A^B(g)
		\]
		der Komposition als Produkt der Darstellungsmatrizen von $ f $ und $ g $.
\subsection{Notation \& Definition}
	\begin{Definition}[Kurzform der def. Gleichung einer Darst.-Matrix]
		Wir notieren nun die definierende Gleichung einer Darstellungsmatrix $ X=\xi_B^C(f) $ von $ f\in \hom(V,W) $ auch in der Kurzform
		\[
			CX=(c_1,\dots,c_m)X = (f(b_1),\dots,f(b_n)) = f(B).
		\]
		Für die \emph{Koordinatenspalte eines Vektors}
		\[
			Y\in K^{n\times 1} \text{ mit } v=\sum_{j=1}^{n}b_jy_{j1}
		\]
		ist dann
		\[
			f(v) = (f(b_1),\dots,f(b_n))Y = (c_1,\dots,c_m)XY.
		\]
	\end{Definition}
	Die Familien $ (c_1,\dots,c_m) $ und $ (f(b_1),\dots,f(b_n)) $ sind keine Matrizen, denn die Elemente sind Vektoren!

%VO19-2015-12-15
	\paragraph{Bemerkung}
		Wir schreiben die Skalarmultiplikation als Rechts-Multiplikation.
	\paragraph{Beispiel}
		Die neue Notation liefert einen alternativen "`Beweis"' für $ \xi_A^C(f\circ g) = \xi_B^C(f)\xi_A^B(g) $:

		Gilt für jeden Vektor $ a_k,\ k=1,\dots,p $
		\[
			g(a_k) = \sum_{j=1}^{n}b_jy_{jk}, \text{ wobei } Y = \xi_A^B(g)
		\]
		so erhalten wir
		\[
			(f(g(a_1)),\dots, f(g(a_p))) = (f(b_1),\dots , f(b_n))Y = (c_1,\dots,c_m)\xi_B^C(f)\cdot Y = C\cdot XY
		\]
		womit nun
		\[
			(f\circ g)(A) = C\cdot XY,
		\]
		also
		\[
			\xi_A^C(f\circ g) = X\cdot Y = \xi_B^C(f)\cdot \xi_A^B(g).
		\]

		Einfacher (aber weniger überzeugend) ist die folgende, die Linearität von $ f $ benutzende Version:
		\[
			( f\circ g )(A) = f(g(A)) = f(BY) = f(B)\cdot Y = C\cdot XY
		\]
	\paragraph{Bemerkung}
		Sei $ f\in \hom(V,W) $ mit $ r:= \rg f $, dann existieren Basen $ B $ und $ C $ von $ V $ bzw. $ W $, sodass
		\[
			\xi_B^C(f) = X \text{ mit } x_{ij} =
			\begin{cases}
				1, & \text{falls }i=j\leq r \\
				0, & \text{sonst}
			\end{cases}
		\]
		d.h.
		\[
			X = \left(
			\begin{array}{ccc|ccc}
				1      & \dots  & 0      & 0      & \dots  & 0      \\
				\vdots & \ddots & \vdots & \vdots &        & \vdots \\
				0      & \dots  & 1_{rr} & 0      & \dots  & 0      \\\hline
				0      & \dots  & 0      & 0      & \dots  & 0      \\
				\vdots &        & \vdots & \vdots &        & \vdots \\
				0      & \dots  & 0      & 0      & \dots  & 0      \\
			\end{array}
			\right)
		\]

		Nämlich -- wie im Beweis des Rangsatzes: Die Basen $ B $ und $ C $ werden so gewählt, dass
		\begin{enumerate}[(i)]
			\item $ (b_{r+1},\dots,b_n) $ Basis von $ \ker f $ ist, und dann
			\item $ c_i := f(b_i) $ für $ i= 1,\dots,r $ eine Basis von $ f(V) $ liefert.
		\end{enumerate}
		Offenbar hat $ \xi_B^C(f)$ dann die gewünschte Form:
		\[
			f(b_1)=c_1,\dots,f(b_r)=c_r,f(b_{r+1})=0,\dots,f(b_n)=0,
		\]
		d.h.
		\[
			f(B) = (f(b_1),\dots,f(b_n))=(c_1,\dots,c_m)\left(
			\begin{array}{ccc|ccc}
				1      & \dots  & 0      & 0      & \dots  & 0      \\
				\vdots & \ddots & \vdots & \vdots &        & \vdots \\
				0      & \dots  & 1_{rr} & 0      & \dots  & 0      \\\hline
				0      & \dots  & 0      & 0      & \dots  & 0      \\
				\vdots &        & \vdots & \vdots &        & \vdots \\
				0      & \dots  & 0      & 0      & \dots  & 0      \\
			\end{array}
			\right) = CX.
		\]
		Umgekehrt: gibt es eine Darstellungmatrix von $ f $ dieser Form, so ist $ \rg f = r $.
\subsection{Beispiel \& Definition}
	\begin{Definition}[Einheitsmatrix]
		Ist $ B=(b_1,\dots,b_n) $ Basis von $ V $, so hat der Isomorphismus
		\[
			\phi:V\to K^n \text{ mit } \forall j=1,\dots,n:\phi(b_j)=e_j
		\]
		bezüglich $ B $ und der Standardbasis $ E = (e_1,\dots,e_n) $ von $ K^n $ die \emph{$ n $-reihige Einheitsmatrix} als Darstellungsmatrix:
		\[
			\xi_B^E(\phi) = E_n := (\delta_{ij})_{i,j = 1,\dots,n}
		\]
	\end{Definition}
	\paragraph{Beispiel}
		Sind $ B=(b_1,\dots,b_n) $ und $ B'=(b'_1,\dots,b'_n) $ Basen von $ V $, wobei
		\[
			\forall j=1,\dots,n:b_j = \sum_{i=1}^{n}b'_ix_{ij},
		\]
		so hat die Identität $ \id_V $ die Darstellungsmatrix
		\[
			\xi_B^{B'}(\id_V)= X = (x_{ij})_{i,j=1\dots,n}.
		\]
		Sind dann $ B $ und $ B' $ Basen von $ V $ und $ C $ und $ C' $ Basen von $ W $, so erhält man für $ f\in\hom(V,W) $ die Transformationsformel
		\[
			\xi_{B'}^{C'}(f) = \xi_{B'}^{C'}(\id_W\circ f\circ \id_V) = \xi_C^{C'}(\id_W)\cdot \xi_B^C(f)\cdot \xi_{B'}^{B}(\id_V)
		\]
\subsection{Beispiel \& Definition}
	Ist $ f\in \Iso(V,W) $ mit Basen $ B $ und $ C $ von $ V $ bzw. $ W $, so gilt (mit $ n=\dim V = \dim W $)
	\[
		\xi_C^B(f^{-1})\cdot \xi_B^C(f) = \xi_B^B(f^{-1}\circ f) = \xi_B^B(\id_V) = E_n
	\]
	und
	\[
		\xi_B^C(f)\cdot \xi_C^B(f^{-1})=\xi_C^C(f\circ f^{-1}) = \xi_C^C(\id_W)=E_n.
	\]

	\begin{Definition}[Invertierbare Matrix]
		Eine Matrix $ X\in K^{n\times n} $ nennt man invertierbar mit Inverser $ X^{-1} $, falls
		\[
			\exists X^{-1}\in K^{n\times n}:X^{-1}X = E_n
		\]
		Damit ist die Darstellungsmatrix der Inversen die Inverse der Darstellungsmatrix:
		\[
			\xi_C^B(f^{-1}) = (\xi_B^C(f))^{-1}
		\]
	\end{Definition}
\subsection{Bemerkung \& Definition}
	Jedes $ X\in K^{m\times n} $ liefert (eindeutig) $ f_X\in \hom(K^n,K^m) $ nach Fortsetzungssatz via
	\[
		f_X:K^n\to K^m,\ f_X(e_j) = \sum_{i=1}^{m} e'_ix_{ij} \text{ für } j=1,\dots,n.
	\]
	Bezüglich der Standardbasen $ E = (e_1,\dots,e_n) $ von $ K^n $ und $ E'=(e'_1,\dots,e'_m) $ von $ K^m $ ist dann
	\[
		\xi_E^{E'}(f_X) = X.
	\]

	\begin{Definition}[Rang einer Matrix]
		Damit definiert man den Rang einer Matrix $ X\in K^{m\times n} $ als
		\[
			\rg X:=\rg f_X
		\]
		Eine Matrix $ X\in K^{n\times n} $ ist genau dann invertierbar, wenn $ \rg X = n $. Man setzt
		\[
			\mathrm{Gl}(n):= \{X\in K^{n\times n}\mid \rg X =n\}.
		\]
	\end{Definition}
\subsection{Bemerkung \& Definition}
	Nach der Transformationsformel für Darstellungsmatrizen gilt bei Basiswechseln in $ V $ und $ W $ für $ f\in \hom(V,W) $
	\[
		\xi_{B'}^{C'}(f) = \xi_C^{C'}(\id_W)\cdot \xi_B^C(f)\cdot \xi_{B'}^B(\id_V).
	\]
	Dabei sind $ \xi_{B'}^B(\id_V)\in \mathrm{Gl}(n) $ und $ \xi_C^{C'}(\id_W)\in \mathrm{Gl}(m) $ invertierbar, da etwa
	\[
		\xi^B_{B'}(\id_V)\cdot\xi_B^{B'}(\id_V) = \xi_B^B(\id_V)=E_n;
	\]
	Sind andererseits die Basis $ B $ und $ P\in \mathrm{Gl}(n) $ gegeben, so ist
	\[
		\xi_B^{B'}(\id_V)=P^{-1} \text{ für } B':= BP,
	\]
	d.h. jedes $ P\in \mathrm{Gl}(n) $ realisiert einen Basiswechsel in $ V $, kommt also in der Transformationsformel vor.

	\begin{Definition}[Äquivalente Matrizen]
		Daher definiert man auch Matrizen $ X,X'\in K^{m\times n} $ als \emph{äquivalent},
		\[
			X\sim X',\quad \text{falls }\exists P\in \mathrm{Gl}(n)\exists Q\in \mathrm{Gl}(m):X' = QXP^{-1}.
		\]
	\end{Definition}
