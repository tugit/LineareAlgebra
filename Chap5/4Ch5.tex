\section{Euklidische Geometrie}
\subsection{Definition}\index{Euklidischer Raum}\index{Länge}\index{Abstand}\index{Winkel}
\begin{Definition}[Euklidischer Raum, Abstand, Länge, Winkel ]
	Ein \emph{Euklidischer Raum} ist eine affiner Raum $ (A,V,\tau) $ über einem Euklidischen Vektorraum $ (V,\Skl{.}{.}) $ mit induzierter Norm $ \|.\| $.
		\begin{itemize}
			\item Die \emph{Länge} eines Vektors $ v\in V $ ist seine Norm, der \emph{Abstand} zweier Punkte $ a,b\in A $ ist die Länge ihres Verbindungsvektors,
				\[ d(a,b) := \|b-a\| = \sqrt{\Skl{b-a}{b-a}}. \]
			\item Der \emph{Winkel} $ \alpha\in [0,\pi] $ zweier Vektoren $ v,w\in V^\times $ ist durch die Gleichung
				\[ \Skl{v}{w} = \|v\|\cdot \|w\|\cdot \cos \alpha \]
			definiert; der \emph{Winkel} (am Punkt $ a $) in einem nicht-degenerierten Dreieck $ \{a,b,c\} \subset A$ ist der Winkel der beiden Seitenvektoren $ v=b-a $ und $ w = c-a $.
		\end{itemize}
\end{Definition}

\begin{figure}[ht]
\centering	
\definecolor{qqwuqq}{rgb}{0.,0.39215686274509803,0.}
\definecolor{qqqqff}{rgb}{0.,0.,1.}
\begin{tikzpicture}[line cap=round,line join=round,>=triangle 45,x=1.0cm,y=1.0cm]
\draw[->,color=black] (-1.9743290273232554,0.) -- (5.524000640987365,0.);
\foreach \x in {-1.,1.,2.,3.,4.,5.}
\draw[shift={(\x,0)},color=black] (0pt,2pt) -- (0pt,-2pt) node[below] {\footnotesize $\x$};
\draw[->,color=black] (0.,-1.133095836146613) -- (0.,4.526265243390068);
\foreach \y in {-1.,1.,2.,3.,4.}
\draw[shift={(0,\y)},color=black] (2pt,0pt) -- (-2pt,0pt) node[left] {\footnotesize $\y$};
\draw[color=black] (0pt,-10pt) node[right] {\footnotesize $0$};
\clip(-1.9743290273232554,-1.133095836146613) rectangle (5.524000640987365,4.526265243390068);
\draw [shift={(1.,1.)},color=qqwuqq,fill=qqwuqq,fill opacity=0.1] (0,0) -- (13.366930696316846:0.38851449058604254) arc (13.366930696316846:70.38212059188172:0.38851449058604254) -- cycle;
\draw [->] (1.,1.) -- (1.72,3.02);
\draw [->] (1.,1.) -- (3.02,1.48);
\begin{scriptsize}
\draw [fill=qqqqff] (1.,1.) circle (2.5pt);
\draw[color=qqqqff] (0.9654306181111331,0.848328065842202) node {$A$};
\draw [fill=qqqqff] (1.72,3.02) circle (2.5pt);
\draw[color=qqqqff] (1.8072120143808919,3.257117907475663) node {$B$};
\draw [fill=qqqqff] (3.02,1.48) circle (2.5pt);
\draw[color=qqqqff] (3.115210799353902,1.7160104281510296) node {$C$};
\draw[color=black] (1.1726383464236891,2.130425884776141) node {$v$};
\draw[color=black] (2.014419742693448,1.1202882092524316) node {$w$};
\draw[color=qqwuqq] (1.5093509049315927,1.495852216818939) node {$\alpha$};
\end{scriptsize}
\end{tikzpicture}
\end{figure}

\paragraph{Bemerkung}
	Nach der Cauchy-Schwarzschen Ungleichung ist für $ v,w\in V^\times $
		\[ \frac{\Skl{v}{w}}{\|v\|\cdot \|w\|}\in [-1,1]; \]
	andererseits ist 
		\[ \cos:[0,\pi]\to [-1,1]\text{ bijektiv} \]
	Damit ist der Winkel von Vektoren bzw. im Dreieck wohldefiniert.

\subsection{Definition}\index{Kongruenzabbildung}\index{Isometrie}\index{Ähnlichkeits!transformation}
\begin{Definition}[Kongruenzabbildung, Ähnlichkeitstransformation]
    Eine affine Transformation eines Euklidischen Raumes heißt
		\begin{itemize}
			\item \emph{Kongruenzabbildung} oder \emph{Isometrie}, falls sie Abstandstreu ist,
			\item \emph{Ähnlichkeitstransformation}, falls sie winkeltreu ist.
		\end{itemize}
\end{Definition}
\paragraph{Bemerkung}
	Jede Kongruenzabbildung ist Ähnlichkeitstransformation (Polarisation).
\paragraph{Bemerkung}
	Offenbar bilden die Kongruenz- bzw. Ähnlichkeitsabbildungen eines Euklidischen Raumes $ A $ auf $ A $ operierende (Transformations-)Gruppen.
	
\subsection{Definition (Geometrie)}\index{Euklidische Geometrie}\index{Ähnlichkeits!geometrie}
\begin{Definition}[Euklidische Geometrie, Ähnlichkeitsgeometrie]
	Die auf einem Euklidischen Raum operierende Gruppe der Kongruenzabbildungen bestimmt eine Euklidische Geometrie.
	
	Die Gruppe der Ähnlichkeitstransformationen eines Euklidischen Raumes $ A $ bestimmt eine Ähnlichkeitsgeometrie.
\end{Definition}

% VO 19-05-2016 %

\paragraph{Beispiel}
Jede Translation $ \tau_v:A\to A $ ist eine Isometrie:
	Für $ a,b\in A $ gilt
		\[ \exists!w\in V: b=\tau_w(a) \]
	d.h. $ w=b-a $; also
		\[ \tau_v(b) = \tau_v(\tau_w(a)) = \tau_{v+w}(a) = \tau_w(\tau_v (a)) \]
	d.h. $ w = \tau_v(b)-\tau_v(a) $.
	Damit folgt:
	\[ \|\tau_v(b)-\tau_v(a)\| = \|w\| = \|b-a\| \]
	d.h. $ \tau_v $ ist abstandstreu, da $ a,b\in A $ beliebig waren.
	
	%------------------ IsometrieTranslation ----------------
 	%\begin{figure}[h]\centering
 	%	\include{Chap5/IsometrieTranslation.tikz}
	%\end{figure}
	%------------------ IsometrieTranslation ----------------

	

\paragraph{Beispiel}
	Die Streckung mit Zentrum $ o\in A $ um den Faktor $ s\in \mathbb{R}^\times $,
		\[ o+v=a\overset{\delta_s}{\mapsto}\delta_s(a) = \delta_s(o+v):= o+vs \]
	ist winkeltreu, denn für $ a=o+v, b=o+w $ gilt
		\[ \delta_s(b)-\delta_s(a) = (o+ws)-(o+vs) = \dots = (w-v)s \]
	und damit für drei paarweise verschiedene Punkte $ a,b,c\in A $
		\[ \cos \alpha = \frac{\Skl{\delta_s(b)-\delta_s(a)}{\delta_s(c)-\delta_s(a)}}{\|\delta_s(b)-\delta_s(a)\|\|\delta_s(c)-\delta_s(a)\|} = \frac{\Skl{(b-a)s}{(c-a)s}}{\|(b-a)s\|\|(c-a)s\|} =\frac{s^2}{|s^2|} \cdot \frac{\Skl{b-a}{c-a}}{\|b-a\|\|c-a\|} \]
	d.h. $ \delta_s $ ist winkeltreu; andererseits ist $ \delta_s $ für $ s\neq \pm 1 $ nicht abstandstreu.
	Ist $ a \neq b $, so gilt dann
	\[ \|\delta_s(b)-\delta_s(a)\| = \|b-a\|\cdot |s| \neq \|b-a\|. \] 
\begin{minipage}[t]{0.45\linewidth}
	%------------------ IsometrieTranslation ----------------
		\centering
 		\tdplotsetmaincoords{0}{0} %-27
 	\begin{tikzpicture}[yscale=1,xscale=0.9,tdplot_main_coords]

 		\def\xstart{0} %x Koordinate der Startposition der Grafik
 		\def\ystart{0} %y Koordinate der Startposition der Grafik
 		\def\myscale{1.0} %ändert die Größe der Grafik (Skalierung der Grafik)
        \def\myscalex{(\myscale)}
        \def\myscaley{(\myscale)}
                
 		\def\xstartdraw{(\xstart + 2.0)} %xKoordinate des Referenzstartpunktes (in dieser Zeichnung: a)
 		\def\ystartdraw{(\ystart + 1.5)}%yKoordinate des Referenzstartpunktes (in dieser Zeichnung: a)

 		\def\balkenhoehe{(3.5)}% Länge des vertikalen blauen Balkens
 		\def\balkenlaenge{(5.5)}% Länge des horizontalen blauen Balkens
 		\def\balkenbreite{0.4} %Balkenbreite

 		%---------Begin Balken----------
 		\def\drehwinkel{0}
 		\node (VekV) at ({\xstart+0.2*cos(\drehwinkel)-\balkenbreite*sin(\drehwinkel)},{\ystart+0.5*sin(\drehwinkel)+\balkenbreite*cos(\drehwinkel)})[right, xshift=1,color=blue] {$V$};
 		\node (AffA) at ({\xstart+(\balkenlaenge-0.5)*cos(\drehwinkel)},{\ystart+(\balkenlaenge-0.5)*sin(\drehwinkel)+\balkenbreite*cos(\drehwinkel)})[color=red] {$A$};

 		\path[ shade, top color=white, bottom color=blue, opacity=.6]
 		({\xstart},{\ystart},0)  -- ({\xstart - \balkenbreite * cos(\drehwinkel)- (-\balkenbreite+0)*sin(\drehwinkel)},{\ystart - \balkenbreite * sin(\drehwinkel)+ (-\balkenbreite+0)*cos(\drehwinkel)},0)  -- ({\xstart - \balkenbreite * cos(\drehwinkel)- (\balkenhoehe+0.5)*sin(\drehwinkel)},{\ystart - \balkenbreite * sin(\drehwinkel)+ (\balkenhoehe+0.5)*cos(\drehwinkel)},0) -- ({\xstart - 0 * cos(\drehwinkel)- (\balkenhoehe+0)*sin(\drehwinkel)},{\ystart - 0 * sin(\drehwinkel)+ (\balkenhoehe+0)*cos(\drehwinkel)},0) -- cycle;

 		\path[ shade, right color=white, left color=blue, opacity=.6]
 		({\xstart},{\ystart},0)  -- ({\xstart - \balkenbreite * cos(\drehwinkel)- (-\balkenbreite+0)*sin(\drehwinkel)},{\ystart - \balkenbreite * sin(\drehwinkel)+ (-\balkenbreite+0)*cos(\drehwinkel)},0) --
 		({\xstart + (\balkenlaenge+0.5) * cos(\drehwinkel)- (-\balkenbreite+0)*sin(\drehwinkel)},{\ystart + (\balkenlaenge+0.5) * sin(\drehwinkel)+ (-\balkenbreite+0)*cos(\drehwinkel)},0) --
 		({\xstart + \balkenlaenge * cos(\drehwinkel)},{\ystart + \balkenlaenge * sin(\drehwinkel)},0)--
 		cycle;
 		%---------End Balken----------
 		\def\lightoffset{0.2*\myscale} %offeset der Vektoren

 		% rote Punkte Definition
 		
 		\node (offsetx) at ({(2.5*\myscalex},{0.0}) {}; %just an offset
 		\node (offsety) at ({0.0},{1.5*\myscaley}) {}; %just an offset
 		
 		\node (pointintersection) at ({\xstartdraw},{\ystartdraw}) {};
 		
 		
 	%	\draw[red] (fov) -- ++(295:2cm);
    %\draw[red] (fov) -- ++(335:2cm);
       %\coordinate (B) at (45:2cm) ;
        
        \node (pointa2) at ($(pointintersection) + (70:2.5)$) {};
 		\node (pointb2) at ($(pointintersection) + (25:4)$) {};
 		
 		\node (pointa1) at ($(pointintersection) + (250:1.25)$) {};
 		\node (pointb1) at ($(pointintersection) + (205:2)$) {};
 		
 	%	\node (pointa2) at ($(pointa1) - 0.15*(offsetx) + 1.0*(offsety)$) {};
 		
 	
 		\node[ xshift=3mm, yshift=0mm,color=red] (labela1) at (pointa1) {$a$};
 		\node[ xshift=6mm, yshift=-1mm,color=red] (labela2) at (pointa2) {$\delta_{\text{\tiny  -2}} (a)$};
 		\node[ xshift=1mm, yshift=4mm,color=red] (labelataub) at (pointb2) {$\delta_{\text{\tiny  -2}} (b)$};
 		\node[ xshift=0mm, yshift=4mm,color=red] (labelataua) at (pointb1) {$b$};
 	
 	%    \draw[name path=line 1] (0,0) -- (2,2);
     %   \draw[name path=line 2] (2,0) -- (0,2);
%\fill[red,name intersections={of=line 1 and line 2,total=\t}]
 %   \foreach \s in {1,...,\t}{(intersection-\s) circle (2pt) node {\footnotesize\s}};
    
    
 		%Vektoren blau
 	    %waagrecht
 		\draw[name path=a--da,{<[scale=1,length=6,width=6]}-{>[scale=1,length=6,width=6]},shorten >=2pt, shorten <=2pt,line width=0.2pt,color=blue] (pointa1) -- (pointa2);
 		\draw[name path=b--db,{<[scale=1,length=6,width=6]}-{>[scale=1,length=6,width=6]},shorten >=2pt, shorten <=2pt,line width=0.2pt,color=blue] (pointb1) -- (pointb2);
 		
 		\draw[line width=0.2pt,color=red] ($(pointintersection) + (28:0.7)$) arc[radius=0.7, start angle=28, end angle=67] ($(pointintersection) + (67:0.7)$);
 		\draw[line width=0.2pt,color=red] ($(pointintersection) + (28:0.62)$) arc[radius=0.62, start angle=28, end angle=67] ($(pointintersection) + (67:0.62)$);
 		
 		\draw[line width=0.2pt,color=red] ($(pointintersection) + (208:0.7)$) arc[radius=0.7, start angle=208, end angle=247] ($(pointintersection) + (247:0.7)$);
 		\draw[line width=0.2pt,color=red] ($(pointintersection) + (208:0.62)$) arc[radius=0.62, start angle=208, end angle=247] ($(pointintersection) + (247:0.62)$);
 	
 		%\path [name intersections={of=a--da and b--db,by=E}];
 		
 	
 		%Beschriftung der Vektoren
 		\node [color=blue] (pointlabelvu) at ($(pointintersection)!0.5!(pointa1)$) [ xshift=2mm, yshift=0mm] {\small $v$} ;
 		\node [color=blue] (pointlabelvo) at ($(pointintersection)!0.5!(pointb1)$) [above, xshift=0, yshift=0mm] {\small $w$} ;
 		
 		\node [color=blue] (pointlabelwl) at ($(pointintersection)!0.5!(pointa2)$) [ xshift=-8mm, yshift=0mm] {\small $v \cdot (-2)$} ;
 		\node [color=blue] (pointlabelwr) at ($(pointintersection)!0.5!(pointb2)$) [ xshift=7mm, yshift=-2mm] {\small $w \cdot (-2)$} ;
 		
 	

 		%Punkte malen
 		\draw[fill,color=red] (pointa1) circle [x=1cm,y=1cm,radius=0.08]node[above, xshift=0, yshift=0]{};
 		\draw[fill,color=red] (pointb1) circle [x=1cm,y=1cm,radius=0.08]node[above, xshift=0, yshift=0]{};
 		\draw[fill,color=red] (pointa2) circle [x=1cm,y=1cm,radius=0.08]node[below, xshift=5, yshift=0]{};
 		\draw[fill,color=red] (pointb2) circle [x=1cm,y=1cm,radius=0.08]node[below, xshift=5, yshift=0]{};
 		
 		\draw[fill,color=white] (pointintersection) circle [x=1cm,y=1cm,radius=0.18];
 		
 		\draw[fill,color=red] (pointintersection) circle [x=1cm,y=1cm,radius=0.08]node[below, xshift=5, yshift=0]{};
 		
 		
 		
\end{tikzpicture}
	%------------------ IsometrieTranslation ----------------
\end{minipage}
\hfill
\begin{minipage}[t]{0.45\linewidth}
	%------------------ WinkeltreuAberNichtAbstandstreu ----------------
		\centering
		\tdplotsetmaincoords{0}{0} %-27
 	\begin{tikzpicture}[yscale=0.9,xscale=0.9,tdplot_main_coords]

 		\def\xstart{0} %x Koordinate der Startposition der Grafik
 		\def\ystart{0} %y Koordinate der Startposition der Grafik
 		\def\myscale{1.0} %ändert die Größe der Grafik (Skalierung der Grafik)
        \def\myscalex{(\myscale)}
        \def\myscaley{(\myscale)}
                
 		\def\xstartdraw{(\xstart + 2.5)} %xKoordinate des Referenzstartpunktes (in dieser Zeichnung: a)
 		\def\ystartdraw{(\ystart + 2.0)}%yKoordinate des Referenzstartpunktes (in dieser Zeichnung: a)

 		\def\balkenhoehe{(4.0)}% Länge des vertikalen blauen Balkens
 		\def\balkenlaenge{(6.5)}% Länge des horizontalen blauen Balkens
 		\def\balkenbreite{0.4} %Balkenbreite

 		%---------Begin Balken----------
 		\def\drehwinkel{0}
 		\node (VekV) at ({\xstart+0.2*cos(\drehwinkel)-\balkenbreite*sin(\drehwinkel)},{\ystart+0.5*sin(\drehwinkel)+\balkenbreite*cos(\drehwinkel)})[right, xshift=1,color=blue] {$V$};
 		\node (AffA) at ({\xstart+(\balkenlaenge-0.5)*cos(\drehwinkel)},{\ystart+(\balkenlaenge-0.5)*sin(\drehwinkel)+\balkenbreite*cos(\drehwinkel)})[color=red] {$A$};

 		\path[ shade, top color=white, bottom color=blue, opacity=.6]
 		({\xstart},{\ystart},0)  -- ({\xstart - \balkenbreite * cos(\drehwinkel)- (-\balkenbreite+0)*sin(\drehwinkel)},{\ystart - \balkenbreite * sin(\drehwinkel)+ (-\balkenbreite+0)*cos(\drehwinkel)},0)  -- ({\xstart - \balkenbreite * cos(\drehwinkel)- (\balkenhoehe+0.5)*sin(\drehwinkel)},{\ystart - \balkenbreite * sin(\drehwinkel)+ (\balkenhoehe+0.5)*cos(\drehwinkel)},0) -- ({\xstart - 0 * cos(\drehwinkel)- (\balkenhoehe+0)*sin(\drehwinkel)},{\ystart - 0 * sin(\drehwinkel)+ (\balkenhoehe+0)*cos(\drehwinkel)},0) -- cycle;

 		\path[ shade, right color=white, left color=blue, opacity=.6]
 		({\xstart},{\ystart},0)  -- ({\xstart - \balkenbreite * cos(\drehwinkel)- (-\balkenbreite+0)*sin(\drehwinkel)},{\ystart - \balkenbreite * sin(\drehwinkel)+ (-\balkenbreite+0)*cos(\drehwinkel)},0) --
 		({\xstart + (\balkenlaenge+0.5) * cos(\drehwinkel)- (-\balkenbreite+0)*sin(\drehwinkel)},{\ystart + (\balkenlaenge+0.5) * sin(\drehwinkel)+ (-\balkenbreite+0)*cos(\drehwinkel)},0) --
 		({\xstart + \balkenlaenge * cos(\drehwinkel)},{\ystart + \balkenlaenge * sin(\drehwinkel)},0)--
 		cycle;
 		%---------End Balken----------
 		\def\lightoffset{0.2*\myscale} %offeset der Vektoren

 		% rote Punkte Definition
 		
 		\node (offsetx) at ({(2.5*\myscalex},{0.0}) {}; %just an offset
 		\node (offsety) at ({0.0},{1.5*\myscaley}) {}; %just an offset
 		
 		\node (pointintersection) at ({\xstartdraw},{\ystartdraw}) {};
 		
 		
  		\node (pointa2) at ($(pointintersection) + (80:2)$) {};
 		\node (pointb2) at ($(pointintersection) + (35:3)$) {};
 		\node (pointc2) at ($(pointintersection) + (5:2.25)$) {};
 		
 		\node (pointa1) at ($(pointintersection) + (260:1.25)$) {};
 		\node (pointb1) at ($(pointintersection) + (215:2)$) {};
 		\node (pointc1) at ($(pointintersection) + (185:1.5)$) {};
 
 		
 	
 		\node[ xshift=3mm, yshift=0mm,color=red] (labela1) at (pointa1) {$a$};
 		\node[ xshift=-6mm, yshift=-1mm,color=red] (labela2) at (pointa2) {$\delta_{\text{\tiny  -2}} (a)$};
 		\node[ xshift=1mm, yshift=-4mm,color=red] (labelataub) at (pointb2) {$\delta_{\text{\tiny  -2}} (b)$};
 		\node[ xshift=1mm, yshift=-4mm,color=red] (labelatauc) at (pointc2) {$\delta_{\text{\tiny  -2}} (c)$};
 		\node[ xshift=0mm, yshift=4mm,color=red] (labelb) at (pointb1) {$b$};
 		\node[ xshift=-1mm, yshift=2mm,color=red] (labelc) at (pointc1) {$c$};
 		
 	
 		%Vektoren blau
 		\draw[name path=a--da,-{>[scale=1,length=6,width=6]},shorten >=2pt, shorten <=2pt,line width=0.2pt,color=blue] (pointa2) -- (pointb2);
 		\draw[name path=b--db,-{>[scale=1,length=6,width=6]},shorten >=2pt, shorten <=2pt,line width=0.2pt,color=blue] (pointa2) -- (pointc2);
 		
 		\draw[name path=a--da,-{>[scale=1,length=6,width=6]},shorten >=2pt, shorten <=2pt,line width=0.2pt,color=blue] (pointa1) -- (pointb1);
 		\draw[name path=b--db,-{>[scale=1,length=6,width=6]},shorten >=2pt, shorten <=2pt,line width=0.2pt,color=blue] (pointa1) -- (pointc1);
 	
 	    %punktierte Linien	
 		\draw[line width=0.2pt,color=blue,dotted] (pointa1) -- (pointa2);
 		\draw[line width=0.2pt,color=blue,dotted] (pointb1) -- (pointb2);
 		\draw[line width=0.2pt,color=blue,dotted] (pointc1) -- (pointc2);
 		
 		\draw[line width=0.2pt,color=red] ($(pointintersection) + (38:0.6)$) arc[radius=0.6, start angle=38, end angle=77] ($(pointintersection) + (77:0.6)$);
 		\draw[line width=0.2pt,color=red] ($(pointintersection) + (38:0.52)$) arc[radius=0.52, start angle=38, end angle=77] ($(pointintersection) + (77:0.52)$);
 		
 		\draw[line width=0.2pt,color=red] ($(pointintersection) + (218:0.6)$) arc[radius=0.6, start angle=218, end angle=257] ($(pointintersection) + (257:0.6)$);
 		\draw[line width=0.2pt,color=red] ($(pointintersection) + (218:0.52)$) arc[radius=0.52, start angle=218, end angle=257] ($(pointintersection) + (257:0.52)$);
 	
 	    \draw[line width=0.2pt,color=red] ($(pointintersection) + (8:0.9)$) arc[radius=0.9, start angle=8, end angle=33] ($(pointintersection) + (33:0.9)$);
 		\draw[line width=0.2pt,color=red] ($(pointintersection) + (8:0.82)$) arc[radius=0.82, start angle=8, end angle=33] ($(pointintersection) + (33:0.82)$);
 		
 		\draw[line width=0.2pt,color=red] ($(pointintersection) + (188:0.9)$) arc[radius=0.9, start angle=188, end angle=213] ($(pointintersection) + (213:0.9)$);
 		\draw[line width=0.2pt,color=red] ($(pointintersection) + (188:0.82)$) arc[radius=0.82, start angle=188, end angle=213] ($(pointintersection) + (213:0.82)$);
 		
 		%Beschriftung der Vektoren
 		
 		\node [color=blue] (pointlabelvr) at ($(pointa1)!0.5!(pointb1)$) [ xshift=0mm, yshift=-3mm] {\footnotesize $b- a$} ;
 		
 		\node [color=blue] (pointlabelwr) at ($(pointa2)!0.5!(pointb2)$) [ xshift=0mm, yshift=4mm] {\footnotesize $ \delta_{\text{\tiny  -2}} (b) - \delta_{\text{\tiny  -2}} (a) $} ;
 	
 	
 		%Punkte malen
 		\draw[fill,color=red] (pointa1) circle [x=1cm,y=1cm,radius=0.08]node[above, xshift=0, yshift=0]{};
 		\draw[fill,color=red] (pointb1) circle [x=1cm,y=1cm,radius=0.08]node[above, xshift=0, yshift=0]{};
 		\draw[fill,color=red] (pointa2) circle [x=1cm,y=1cm,radius=0.08]node[below, xshift=5, yshift=0]{};
 		\draw[fill,color=red] (pointb2) circle [x=1cm,y=1cm,radius=0.08]node[below, xshift=5, yshift=0]{};
 		\draw[fill,color=red] (pointc1) circle [x=1cm,y=1cm,radius=0.08]node[above, xshift=0, yshift=0]{};
 		\draw[fill,color=red] (pointc2) circle [x=1cm,y=1cm,radius=0.08]node[above, xshift=0, yshift=0]{};
 		
 		\draw[fill,color=white] (pointintersection) circle [x=1cm,y=1cm,radius=0.18];
 		
 		\draw[fill,color=red] (pointintersection) circle [x=1cm,y=1cm,radius=0.08]node[below, xshift=5, yshift=0]{};
 		\node[ xshift=-2mm, yshift=3mm,color=red] (label0) at (pointintersection) {\small $0$};
 		
 		
 		
\end{tikzpicture}
	%------------------ WinkeltreuAberNichtAbstandstreu ----------------
\end{minipage}		

\paragraph{Zur Erinnerung}
	Jede affine Abbildung $ \alpha:A\to A' $ besitzt einen (eindeutigen) \emph{linearen Anteil} $ \lambda:V\to V' $, sodass
		\[ \forall a\in A\forall v\in V: \alpha(a+v) = \alpha(a)+\lambda(v); \]
	ist $ \alpha $ eine affine Transformation, so ist $ \lambda \in Gl(V) $.
\paragraph{Bemerkung}
	Jede Ähnlichkeitstransformation ist Komposition einer Streckung und einer Kongruenzabbildung.
	
	Nämlich: Ist $ \alpha $ Ähnlichkeitstransformation mit linearem Anteil $ \lambda\in Gl(V) $, so erhält $ \lambda $ Winkel von Vektoren, insbesondere also Orthogonalität.
	Nun wähle $ w\in V^\times $ und setze
		\[ s := \frac{\|w\|}{\|\lambda w\|}. \]
	Ist dann $ v\in V $ mit $ \|v\|=\|w\| $, so folgt
		\[ v+w \perp v-w \Rightarrow \lambda(v+w)\perp \lambda(v-w) \Rightarrow \|\lambda(v)\| = \|\lambda(w)\|, \]
	also 
		\[ \forall v\in V^\times: \frac{\|\lambda(v)\|}{\|v\|} = \|\lambda(v\frac{\|w\|}{\|v\|})\|\frac{1}{\|w\|} = \frac{\|\lambda (w)\|}{\|w\|} = \frac{1}{s}. \]
	Mit einem beliebigen Streckungszentrum $ o\in A $ erhält man also eine Isometrie durch
		\[ \delta_s\circ \alpha :A\to A. \]
\paragraph{Beispiel}
	Eine \emph{nicht-triviale} Scherung ist \emph{keine} Ähnlichkeitstransformation. Beweis in der Übung. % 3 Zeilen Rechnung, 5 Zeilen Begründung!

\subsection{Lemma \& Definition}
\begin{Lemma}
	Eine affine Transformation $ \alpha:A\to A $ eines Euklidischen Raumes $ A $ ist genau dann eine Kongruenzabbildung, wenn ihr linearer Anteil $ \lambda $ \emph{orthogonal} ist:
\end{Lemma}
\begin{Definition}[orthonogale Gruppe]
		\[ \lambda\in O(V):= \{f\in Gl(V)\mid \forall v,w\in V: \Skl{f(v)}{f(w)} = \Skl{v}{w}\}. \]
	$ O(V) $ heißt die \emph{orthonogale Gruppe} von $ (V,\Skl{.}{.}) $.
\end{Definition}

\paragraph{Bemerkung}
	$ O(V)\subset Gl(V) $ ist eine Gruppe. Beweis in der Übung.
\paragraph{Bemerkung}
	Ist $ f\in \End(V) $, so folgt die Injektivität von $ f $ aus
		\[ \forall v,w\in V: \Skl{f(v)}{f(w)} = \Skl{v}{w}. \]
	Aus $ f(v) = 0 $ folgt nämlich
		\[ 0 = \|f(v)\| = 0 = \|v\| \Rightarrow v = 0, \text{ da } \Skl{.}{.}\text{ pos. definit.} \]
	Ist $ \dim V <\infty $, so folgt mit dem Rangsatz, $ \dim V = \rg f + \dfkt f = \rg $, dass $ f\in Gl(V) $.\\
	Im Fall $ \dim V = \infty $ ist $ f $ nicht notwendigerweise surjektiv, wie der \emph{Shiftoperator}
		\[ f\in \End(\R^\N), \forall n\in \N: f(e_n) = e_{n+1} \]
	zeigt.
\paragraph{Beweis (Lemma)}
	Sei $ (A,V,\tau) $ Euklidischer Raum über einem Euklidischen VR $ (V,\Skl{.}{.}) $ und $\alpha:A\to A $ Affinität mit linearem Anteil $ \lambda\in Gl(V) $. Dann ist $ \alpha $ genau dann Isometrie, wenn
		\[ \forall a,b\in A: \|\lambda(b-a)\| = \|\alpha(b)-\alpha(a)\| = \|b-a\|,  \]
	also (Polarisation), wenn $ \lambda\in O(V) $.
\subsection{Definition}
\begin{Definition}[unitäre Gruppe]
	Ist $ (V,\Skl{.}{.}) $ unitärer VR, so heißt $ f\in Gl(V) $ mit
		\[ \forall v,w\in V: \Skl{f(v)}{f(w)} = \Skl{v}{w} \]
	\emph{unitär}; die \emph{unitäre Gruppe} von $ (V,\Skl{.}{.}) $ ist die Gruppe
		\[ U(V) := \{f\in Gl(V)\mid \forall v,w\in V: \Skl{f(v)}{f(w)} = \Skl{v}{w} \}. \]
\end{Definition}

% VO 24-05-2016 %

\subsection{Schulgeometrie}
	Betrachte eine Euklidische Ebene $ A^2 $ über Euklidischem VR $ (\R^2,\Skl{.}{.}) $ mit kanonischem Skalarprodukt $ \forall i,j\in \{1,2\}:\Skl{e_i}{e_j} = \delta_{ij}. $\\
	Weiter (vgl. Abschnitt \ref{JDrehung}) bezeichne $ J\in \End(\R^2) $ den durch
		\[ J(e_1) = e_2 \text{ und } J(e_2) = -e_1 \]
	definierten Endomorphismus, also eine "`$ 90^{\circ}$-Drehung"',
	bzw. die $ \R^2 $ mit $ \C $ identifizierende komplexe Multiplikation mit $ i $,
		\[ v(x+iy) = vx+J(v)y \text{ für }
		\begin{cases}
			v\in \R^2\\ (x+iy)\in \C.
		\end{cases} \]
	Man bemerke: Für $ v\in \R^2\setminus \{0\} $ ist $ \{Jv\}^\perp = [v] $ und damit
		\[ \forall w\in \R^2: w\perp Jv \Leftrightarrow w\parallel v. \]
    So ermöglicht $ J $ einen einfachen Wechsel zwischen \emph{parametrischer} und \emph{impliziter Darstellung} (e.g. \emph{Hessesche Normalform}) einer Geraden
   		\begin{align*}
		    g &= \{p = o + vx\mid x\in \R\} \Leftrightarrow \\
		    g &= \{p\in A^2\mid \Skl{p-o}{Jv} = 0\} 
		\end{align*} 

    	%------------------ ImpliziteDarstellungGerade ----------------
     	\begin{figure}[h]\centering
     		\include{Chap5/ImpliziteDarstellungGerade.tikz}
     		\caption{Implizite Darstellung einer Geraden}
    	\end{figure}
    	%------------------ ImpliziteDarstellungGerade ----------------

\subsection{Definition}\index{Kreis}
\begin{Definition}[Kreis, Mittelpunkt]
    Ein \emph{Kreis} mit \emph{Mittelpunkt} $ z\in A^2 $ und \emph{Radius} $ r\geq 0 $ ist die Menge
		\[ k = \{p\in A^2\mid \|p-z\| = r\}. \]
\end{Definition}
\paragraph{Bemerkung}
	Es ist mitunter sinnvoll, Punkte als Kreise mit Radius $ r=0 $ zu betrachten.
	
\subsection{Umkreissatz}\index{Umkreis}\index{Streckensymmetrale}
\begin{Satz}[Umkreissatz]
	Sei $ \{a,b,c\} \subset A^2 $ ein nicht-degeneriertes Dreieck. Dann gibt es genau einen Kreis $ k\subset A^2 $, den \emph{Umkreis} des Dreiecks, der die Eckpunkte $ a,b $ und $ c $ des Dreiecks enthält.
\end{Satz}	
	Sein Mittelpunkt ist der Schnittpunkt der drei \emph{Streckensymmetralen}/\emph{Mittelsenkrechten} $ m_{ab}, m_{bc}$ und $ m_{ca} $ des Dreiecks, wobei
    		\[ m_{ab} = \{p\in A^2\mid \Skl{p-s_{ab}}{b-a} = 0\} \quad\text{mit}\quad s_{ab} = a\frac{1}{2}+b\frac{1}{2} \text{ etc}. \]
 
    	%------------------ Umkreis ----------------
     	\begin{figure}[h]\centering
     		\include{Chap5/Umkreis.tikz}
    	\end{figure}
    	%------------------ Umkreis ----------------

	
\paragraph{Beweis}
	Definiere
		\[ g_{ab} : A^2\to \R, p\mapsto g_{ab}(p):= 2\Skl{p-s_{ab}}{b-a} \]
	und analog $ g_{bc} $ und $ g_{ca} $ (zyklische Vertauschung).
	Für $ p\in A^2 $ gilt dann mit
		\[ g_{ab}(p) \overset{!}{=} \Skl{(p-a)+(p-b)}{(p-a)-(p-b)}  = \|p-a\|^2-\|p-b\|^2 \tag{$ * $} \]
	damit folgt
		\[ \forall p\in A^2: (g_{ab}+g_{bc}+g_{ca})(p) = 0, \]
	also
		\[ p\in m_{ab}\cap m_{bc} \Rightarrow p\in m_{ca}. \]
	Nun ist
		\[ m_{ab} = \{p(x) = s_{ab}+J(b-a)x\mid x\in \R\} \]
	mit $ J(b-a)\not\perp b-c $, da das Dreieck $ \{a,b,c\} $ nicht-degeneriert ist. Dies liefert einen eindeutigen Schnittpunkt $ z\in p(x)\in m_{ab}\cap m_{bc} $ als Lösung der linearen Gleichung
		\[ 0 = g_{bc}(p(x)) = 2\Skl{s_{ab}+J(b-a)x-s_{bc}}{c-b} \]
		\[ = 2\Skl{J(b-a)}{c-b}x+\Skl{a-c}{c-b}. \]
	Wegen $ (*) $ gilt nun für diesen Schnittpunkt $ z $
		\[ \|z-a\| = \|z-b\| = \|z-c\| \tag{$ ** $} \]
	d.h. $ a,b $ und $ c $ liegen auf einem Kreis mit Mittelpunkt $ z $.
	Andererseits: Wegen $ (*) $ impliziert $ (**) $, dass $ z\in m_{ab}\cap m_{bc} $, womit die Eindeutigkeit von $ z $ und damit des Umkreises folgt.

\subsection{Höhensatz}\index{Höhen!-schnittpunkt}\index{Höhen}
\begin{Satz}[Höhensatz]
	Die \emph{Höhen} $ h_a,h_b $ und $ h_c $ eines nicht-degenerierten Dreiecks $ \{a,b,c\} \subset A^2 $ schneiden sich in einem Punkt, dem \emph{Höhenschnittpunkt}, wobei
		\[ h_a = \{p\in A^2\mid \Skl{p-a}{b-c} = 0 \},\text{ etc.} \]
\end{Satz}

	%------------------ Hoehensatz ----------------
     	\begin{figure}[ht]\centering
     		\include{Chap5/Hoehensatz.tikz}
    	\end{figure}
   %------------------ Hoehensatz ----------------


	Beweis in der Übung, analog zum Umkreissatz.

\subsection{Euler-Gerade}\index{Euler-Gerade}
\begin{Satz}[Euler-Gerade]
	Seien $ s$, $h$ und $ z $ Schwerpunkt, Höhenschnittpunkt und Umkreismittelpunkt eines nicht-degenerierten Dreiecks $ a,b,c\subset A^2 $.
	Dann gilt
		\[ s=z\frac{2}{3}+h\frac{1}{3}. \]
	Ist $ s\neq z $, so liegen die drei Punkte also auf einer eindeutig bestimmten Geraden, \emph{Euler-Geraden}, mit einem Teilverhältnis $ (zs:hs)=-\frac{1}{2} $.
\end{Satz}
	Beweis in der Übung.

\subsection{Satz von Pythagoras}
\begin{Satz}[Satz von Pythagoras]
	 In einem Dreieck $ \{a,b,c\}\subset A^2 $ mit einem rechten Winkel $ \alpha = \frac{\pi}{2} $ bei $ a $ gilt stets
		 \[ \|c-a\|^2+\|a-b\|^2 = \|c-b\|^2. \]
\end{Satz}
\paragraph{Beweis}
	Offenbar gilt $ c-b = (c-a)+(a-b) $, daher
		\[ \|c-b\|^2 = \|c-a\|^2 + 2\Skl{c-a}{a-b}+\|a-b\|^2 = \|c-a\|^2+\|a-b\|^2. \]
\paragraph{Bemerkung}
	Für allgemeine Dreiecke liefert die gleiche Rechnung den Cosinussatz:
		\[ \|b-c\|^2 = \|c-a\|^2+\|a-b\|^2- 2\|c-a\|\|a-b\|\cos \alpha. \]
\paragraph{Bemerkung}
	Ist $ (o;e_1,e_2) $ ein affines Bezugssystem in $ A^2 $ mit
		\[ e_1 \perp e_2 \text{ und } \|e_1\| = \|e_2\| = 1, \]
	so ist jeder Punkt $ a\in A^2 $ Eckpunkt eines \emph{rechtwinkligen} Dreiecks
		\[ \{o,o+e_1x_1,o+e_1x_1+e_2x_2\} \text{ für } a = o+e_1x_1+e_2x_2; \]
	der Abstand vom Ursprung ist also (Pythagoras)
		\[ \|a-o\| = \sqrt{x_1^2+x_2^2}. \]
	Wegen seiner Translationsinvarianz kann der Abstand zwischen beliebigen Punkten genau so berechnet werden.
	
\subsection{Definition}\index{Kartesisches Bezugssystem}
\begin{Definition}[Kartesisches Bezugssystem]
	Ein \emph{kartesisches Bezugssystem} $ (o,E) $ eines Euklidischen Raumes $ (A,V,\tau) $ über einem Euklidischen VR $ (V,\Skl{.}{.}) $ besteht aus einem Ursprung $ o\in A $ und einer ONB $ E $ von $ (V,\Skl{.}{.}) $.
\end{Definition}
\paragraph{Bemerkung}
	In jedem endlichdimensionalen Euklidischen Raum gibt es ein kartesisches Bezugssystem, im Allgemeinen ist dies nicht so (vgl. Übung).
	
\subsection{Lemma}
\begin{Lemma}[]
	Ist $ (o;E) $ mit $ E=(e_i)_{i\in I} $ kartesisches Bezugssystem eines Euklidischen Raumes $ (A,V,\tau) $ über $ (V,\Skl{.}{.}) $, so ist 
		\[ \forall a\in A: a = o + \sum_{i\in I} e_i \Skl{e_i}{a-o} \]
\end{Lemma}
\paragraph{Beweis}
	Da $ E $ Basis ist, existiert zu $ a\in A $ eine Familie $ (x_i)_{i\in I} $ in $ \R $ mit
		\[ a = o + \sum_{i\in I}e_ix_i, \]
	wobei
		\[ \forall i\in I: \Skl{e_i}{a-o} = \Skl{e_i}{\sum_{j\in I}e_jx_j} = \sum_{j\in I}\delta_{ij}x_j = x_i. \]