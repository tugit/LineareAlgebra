% VO 14-04-2016 %
\chapter{Längen- und Winkelmessung}
Plan: 
	Längen und Winkel (in "Punkträumen" $ \cong $ affinen Räumen) verstehen.
	
Algebraisch:
	via Produkte (bilineare -- oder fast bilineare -- Abbildungen).
	
	\definecolor{qqwuqq}{rgb}{0.,0.39215686274509803,0.}
	\definecolor{qqqqff}{rgb}{0.,0.,1.}
	
	\begin{tikzpicture}[line cap=round,line join=round,>=triangle 45,scale=1.8]
	\clip(0,0) rectangle (10,3);
	\draw [shift={(6.635,1.07)},color=qqwuqq,fill=qqwuqq,fill opacity=0.1] (0,0) -- (34.85:0.14) arc (34.85:143.6:0.14) -- cycle;
	\draw [->] (3.58,1) -- (1.56,2.3);
	\draw [->] (7.6,0.35) -- (5.25,2.1);
	\draw [->] (5.5,0.26) -- (7.8,1.9);

	\draw [fill=qqqqff] (3.58,1) circle (1pt);
	\draw[color=qqqqff] (3.6,1) node[below] {$A$};
	\draw [fill=qqqqff] (1.56,2.3) circle (1pt);
	\draw[color=blue] (1.5,2.3) node[above] {$B$};
	\draw (2.5,0.1) node {Abstand a bis b $ \cong $ Länge b-a};
	\draw (6.4,0.1) node {Winkel $ \cong $ Winkel zwischen Richtungsvektoren};
	\end{tikzpicture}

\section{Bilinearformen \& Sesquilinearformen}
\paragraph{Zur Erinnerung}
	Sind $ V $ und  $W$ $ K $-VR, so nennt man eine Abbildung
		\[ \beta: V\times V\to W \]
	\emph{bilinear} oder ein \emph{Produkt}, wenn sie in jedem Argument linear ist:
		\begin{enumerate}[(i)]
			\item $ \forall w\in V :V\ni v \mapsto \beta(v,w)\in W $ ist linear;
			\item $ \forall v\in V: V\ni w\mapsto \beta(v,w)\in W $ ist linear.
		\end{enumerate}
	Zu vorgegebenen Werten $ \beta_{ij} \in W$ auf einer Basis $ (b_i)_{i\in I} $ von $ V $ existiert dann eine eindeutige Bilinearform $ \beta $ (Fortsetzungssatz Abschnitt 4.3):
		\[ \exists! \beta:V\times V\to W \text{ bilinear}: \forall i,j\in I: \beta(b_i,b_j) = \beta_{ij}. \]
\paragraph{Bemerkung}
	Man kann auch bilineare Abbildungen $ V\times V'\to W $ betrachten und, zum Beispiel, auch einen Fortsetzungssatz beweisen.
	
	Wir benötigen eine Verallgemeinerung in eine andere Richtung:
\subsection{Definition} \index{Sesquilinearform}\index{Semilinearität}
\begin{Definition}[Sesquilinearform]
Seien $ V $ ein $ K $-VR und $ K\ni x\mapsto \overline{x}\in K $ ein (Körper-) Automorphismus, d.h. eine bijektive Abbildung mit
		\[ \overline{x+y} = \overline{x}+\overline{y} \text{ und } \overline{xy} = \overline{x}\cdot \overline{y} \]
	für alle $ x,y\in K $. Eine Abbildung $ \sigma: V\times V \to K $ heißt dann \emph{Sesquilinearform} (bzgl. $ \overline{\phantom{a}} $), falls
		\begin{enumerate}[(i)]
			\item $ \forall v\in V: V\ni w \mapsto \sigma(v,w)\in K $ ist linear, d.h. $ \sigma(v,.)\in V^* $;
			\item $ \forall w\in V: V\ni v \mapsto \sigma(v,w)\in K $ ist \emph{semilinear}, d.h.
				\begin{enumerate}[(a)]
					\item $ \forall v,v' \in V: \sigma(v+v',w) = \sigma(v,w)+\sigma(v',w) $ und
					\item $ \forall v\in V\forall x\in K: \sigma(vx,w) = \overline{x}\sigma(v,w) $.
				\end{enumerate}
		\end{enumerate}
\end{Definition}

\paragraph{Beispiel}
	Die Identität $ K\ni x\mapsto \overline{x}:= x\in K $ ist offensichtlich ein Körperautomorphismus für jeden Körper $ K $. \emph{Bilinearformen} sind genau die Sesquilinearformen bezüglich $ \id_K $.
\paragraph{Beispiel}
	Für $ K = \mathbb{C} $ liefert \emph{komplexe Konjugation} einen Körperautomorphismus (keinen VR-Automorphismus, vgl. Abschnitt 1.4):
		\[ \mathbb{C}\ni x+iy \mapsto \overline{x+iy}:= x-iy \in \mathbb{C}. \]
	Dieses Beispiel ist unser Grund für die Einführung des Begriffs der Sesquilinearform.
\paragraph{Bemerkung}
	Ist $ \sigma $ Bilinearform und Sesquilinearform bezüglich $ \overline{\phantom{a}} $, so ist $ \sigma $ oder $ \overline{\phantom{a}} $ trivial:
		\[ \forall x\in K\forall v,w\in V: 0 = \sigma(vx,w) - \sigma(vx,w) = (x-\overline{x})\sigma(v,w)  \]
		\[ \Rightarrow \begin{cases}
		\forall v,w\in V: \sigma(v,w) = 0 \text{ oder}\\
		\exists v,w\in V: \sigma(v,w)\neq 0 \land \forall x\in K: \overline{x} = x.
		\end{cases} \]
\paragraph{Bemerkung}
	In $ \mathbb{Z}_p, \mathbb{Q} $ und $ \mathbb{R} $ gibt es nur \emph{einen} Körperautomorphismus: $ \id_K $. Ein Automorphismus $ \overline{\phantom{a}} $ von $ \mathbb{C} $ mit $ \overline{\mathbb{R}} = \mathbb{R} $ ist trivial, $ \overline{\phantom{a}} = \id_\mathbb{C} $ oder die komplexe Konjugation.
	
\subsection{Fortsetzungssatz für Sesquilinearformen}
\begin{Satz}[Fortsetzungssatz für Sesquilinearformen]
	Sind $ V $ ein $ K $-VR und $ K\ni x\mapsto \overline{x}\in K $ ein Körperautomorphismus, $ (b_i)_{i\in I} $ Basis von $ V $ und $ (s_{ij})_{i,j\in I} $ eine Familie in $ K $, so existiert eine eindeutige Sesquilinearform $ \sigma $ mit
		\[ \forall i,j\in I:\sigma(b_i,b_j) = s_{ij}. \]
\end{Satz}

% VO 19-04-2016 % 
\paragraph{Beweis}
	Wir imitieren den Beweis unseres ersten Fortsetzungssatzes für lineare Abbildungen:
	
	{Eindeutigkeit:}
	Sei $ \sigma $ eine Sesquilinearform mit der gewünschten Eigenschaft oben; gilt
		\[ v = \sum_{i\in I}b_ix_i \text{ und }w = \sum_{i\in I}b_i y_i \]
	so folgt
		\[ \sigma(v,w) = \sum_{i,j\in I}\overline{x_i}\sigma(b_i,b_j)y_j = \sum_{i,j\in I}\overline{x_i}s_{ij}y_j \]
	d.h. $ \sigma $ ist durch die Familie $ (s_{ij})_{i,j\in I} $ eindeutig bestimmt.
	
	{Existenz:}
	Da jeder Vektor $ v\in V $ eine eindeutige Basisdarstellung $ v=\sum_{i\in I}b_ix_i $ hat, wird durch
	\[ \sigma:V\times V \to K, (v,w)= \left(\sum_{i\in I}b_ix_i, \sum_{j\in I}b_jy_j\right) \]
	\[ \mapsto \sigma(v,w) := \sum_{i,j\in I}\overline{x_i}s_{ij}y_j \]
	eine Abbildung wohldefiniert. Offenbar (nachrechnen) ist $ \sigma $ dann sesquilinear. 

\paragraph{Bemerkung}
	Jede Sesquilinearform $ \sigma: V\times V\to K $ liefert eine semi-lineare Abbildung
		\[ V\ni v\mapsto \sigma(v,.)\in V^*. \]
	Mit einem "`Fortsetzungssatz für semi-lineare Abbildungen"' (Aufgabe 34) hätte man auch den früher skizzierten Beweis für bilineare Abbildungen imitieren können.

\subsection{Buchhaltung}\index{Gramsche Matrix}
\paragraph{Gramsche Matrix}
	Ist $ n=\dim V < \infty $ und $ B=(b_1,\dots,b_n) $ Basis von $ V $, so kann man eine Sesquilinearform $ \sigma: V\times V\to K $ durch eine Matrix $ S $
	\[ \begin{array}{c|ccc}
	\sigma & b_1 & \dots & b_n \\ \hline
	b_1 & s_{11} &  & s_{1n} \\ 
	\vdots &  & \ddots &  \\ 
	b_n & s_{n1} &  & s_{nn}
	\end{array}  \]
	Diese Matrix
		\[ \Gamma_B(\sigma) = S = \left(\sigma(b_i,b_j)\right)_{i,j\in \{1,\dots,n\}} \]
	heißt die Darstellungsmatrix oder \emph{Gramsche Matrix} von $ \sigma $ bezüglich $ B $. Für Vektoren
		\[ v = \sum_{i=1}^{n}b_ix_i = BX \text{ und } w = \sum_{j=1}^{n}b_jy_j = BY \]
	ist dann
		\[ \sigma(v,w) = \sum_{i,j=1}^{n}\overline{x_i}s_{ij}y_j = \overline{X}^tSY \]
		\[ = (\overline{x_1},\dots,\overline{x_n})\begin{pmatrix}
		\sum_{i=1}^{n}s_{1j}y_j\\ \vdots\\ \sum_{j=1}^{n}s_{nj}y_j
		\end{pmatrix} = \sum_{i=1}^{n}\overline{x_i}\sum_{j=1}^{n}s_{ij}y_j. \]
\paragraph{Transformationsformel}
	Ein Basiswechsel $ B' = BP $ mit $ P = \xi_{B'}^B \in Gl(n)$ liefert dann
		\[ v = BX = (B'P^{-1})X = B'\underset{X'}{\underbrace{(P^{-1}X)}} \text{ und } w = B'\underset{Y'}{\underbrace{(P^{-1}Y)}} \]
	und damit für $ X,Y \in K^{n\times 1} $
		\[ \overline{X}^tSY = \overline{X'}^t\underset{S'}{\underbrace{(\overline{P}^tSP)}}Y' \]
	woraus die \emph{Transformationsformel für Gramsche Matrizen} folgt
		\[ S' = \overline{P}^tSP, \]
	wobei $ \overline{P}^t $ die Transponierte der Matrix mit Einträgen $ \overline{p_{ij}} $ ist.
\paragraph{Äquivalenz von Matrizen}
Dies liefert einen weiteren Äquivalenzbegriff für quadratische Matrizen $ S\in K^{n\times n} $:
		\[ S' \sim S :\Leftrightarrow \exists  P\in Gl(n): S' = \overline{P}^tSP. \]
	Die verschiedenen Begriffe der Äquivalenz von Matrizen (vgl. 3.1 \& 4.2) spiegeln die verschiedenen Funktionen/Bedeutungen von Matrizen wider.
	
\paragraph{Bemerkung}
	Die Menge der Sesquilinearformen auf einem $ K $-VR ist selbst ein $ K $-VR. Ist $ n=\dim V< \infty $ und $ B $ Basis von $ V $, so erhält man (Fortsetzungssatz) einen Isomorphismus
		\[ K^{V\times V}\supset \{\sigma:V\times V\to K \text{ Sesquilinearform}\}\ni \sigma \mapsto \Gamma_B(\sigma)\in K^{n\times n}. \]
\subsection{Beispiel \& Definition} \index{Sesquilinearform!kanonische}\index{Sesquilinearform!assoziierte}
	Sei $ \overline{\phantom{a}}:K\to K $ Körperautomorphismus; jedes $ S\in K^{n\times n} $ liefert dann eine eindeutige Sesquilinearform
		\[ \sigma_S:K^n\times K^n \to K \text{ mit } (e_i,e_j)\mapsto \sigma_S(e_i,e_j):= s_{ij}, \] 
	die zu \emph{$ S $ assoziierte Sesquilinearform}.

	Für $ S = E_n $ bezeichnet man $ \sigma_S $ auch als \emph{kanonische Sesquilinearform}.
\subsection{Definition}
	Eine Sesquilinearform $ \sigma:V\times V\to K $ auf einem $ K $-VR bzgl. eines Automorphismus $ \overline{\phantom{a}}:K\to K $ nennen wir
		\begin{enumerate}[(i)]
			\item \emph{symmetrisch}, falls
				\[ \forall v,w\in V: \sigma(w,v) = \overline{\sigma(v,w)} \]
			\item \emph{schiefsymmetrisch}, falls
				\[ \forall v,w\in V: \sigma(w,v) = - \overline{\sigma(v,w)} \]
			\item \emph{alternierend}, falls
				\[ \forall v \in V: \sigma(v,v) = 0. \]
		\end{enumerate}
	Falls $ K =\mathbb{C} $ und $ \overline{\phantom{a}} $ komplexe Konjugation sind, so nennt man eine symmetrische Sesquilinearform auch \emph{Hermitesche Sesquilinearform}.
\paragraph{Bemerkung}
	Ist $ \sigma $ nicht-trivial und (schief-)symmetrisch, so muss $ \overline{\phantom{a}} $ eine Involution sein.
	
	Nämlich: Wähle $ v,w\in V $ mit $ \sigma(v,w) = 1$; dann gilt
		\[ \forall x\in K: \overline{\overline{x}} = \overline{\sigma(vx,w)} = \pm \sigma(w,vx) = \overline{\overline{x}\sigma(v,w)} = \pm \sigma(w,v)x = \overline{\sigma(v,w)}x = x. \]