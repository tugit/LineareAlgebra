\section{Orthogonalprojektion}
\subsection{Definition}
\begin{Definition}[Orthogonalprojektion]\index{Orthogonalprojektion}
	Sei $ (A,V,\tau) $ ein Euklidischer Raum über einem Euklidischen VR $ (V, \Skl{.}{.}) $. Dann heißt
		\begin{itemize}
			\item $ p\in \End(V) $ \emph{Orthogonalprojektion}, falls $ p $ Projektion ist, $ p^2 = p $, mit
				\[ \ker p \perp p(V) \]
			\item $ \pi: A\to A $ \emph{Orthogonalprojektion}, falls $ \pi $ Parallelprojektion ist, mit einer Orthogonalprojektion $ p\in End(V) $ als linearem Anteil.
		\end{itemize}
\end{Definition}
\paragraph{Bemerkung}
	Ist $ p\in \End(V) $ Orthogonalprojektion, so ist auch die komplementäre Projektion $ p' = \id_V-p $ Orthogonalprojektion, denn
		\[ \ker p' = p(V)\perp \ker p = p'(V.) \]
\paragraph{Bemerkung}
	Ist $ (o;E) $ mit $ E=(e_i)_{i\in I} $ kartesisches Bezugssystem eines Euklidischen Raumes $ A $ und $ J\subset I $, so liefert
		\[ \pi: A\to A,\ a = o+v \mapsto o+p(v) := o+\sum_{i\in J}e_i\Skl{e_i}{v} \]
	eine Orthogonalprojektion von $ A $ auf
		\[ \pi(A) = o + p(V) = o + [(e_i)_{i\in J}]. \]

% VO 31-05-2016 %
\subsection{Gram-Schmidtsches Orthogonalisierungsverfahren}\index{Gram-Schmidtsches Orthogonalisierungsverfahren}
	Sei $ (V,\Skl{.}{.}) $ ein Euklidischer VR und $ (v_1,\dots,v_n) $ linear unabhängig in $ V $; dann existiert ein ONS $ (e_1,\dots,e_n) $ mit
		\[ [(v_1,\dots,v_k)] = [(e_1,\dots,e_k)] \text{ und } \Skl{e_k}{v_k} > 0 \text{ für } k = 1,\dots,n \tag{$ * $} \]
\paragraph{Beweis}
	Induktion über $ n $. Ist $ n=1 $, so liefert $ e_1 := v_1\cdot\frac{1}{\|v_1\|} $ das gewünschte Orthonormalsystem. 
	
	Ist $ (v_1,\dots,v_{n+1}) $ linear unabhängig und (nach Induktions-Annahme) $ (e_1,\dots,e_n) $ ONS mit
		\[ V_k := [(v_1,\dots,v_k)] = [(e_1,\dots,e_k)] \text{ und } \Skl{e_k}{v_k} > 0 \text{ für } k = 1,\dots,n, \]
	so setzen wir
		\[ p:V\to V,\ v\mapsto p(v):= v-\sum_{i=1}^{n}e_i\Skl{e_i}{v}\in \{e_1,\dots,e_n\}^\perp; \]
	da $ v_{n+1}\notin V_n $ ist\footnote{da nur $e_1,\dots,e_n$ skaliert abgezogen werden!}  $p(v_{n+1}) \neq 0$ und 
		\[e_{n+1} := p(v_{n+1})\frac{1}{\|p(v_{n+1})\|} \]
	ergänzt dann $ (e_1,\dots,e_n) $ zum gesuchten Orthonormalsystem\footnote{da $p$ Orthogonalprojektion ist, $v_{n+1}\notin \ker p$ und $e_1,\dots,e_n\in \ker p$}.
\paragraph{Bemerkung}
	Das ONS $ (e_1,\dots,e_n) $ im Gram-Schmidtschen Verfahren ist durch die Bedingungen $ (*) $ eindeutig festgelegt.
\paragraph{Bemerkung}
	Der Beweis lässt sich wörtlich auf unitäre VR übertragen.

\subsection{Korollar \& Definition}\index{orthogonales Komplement}
\begin{Korollar}
	Ist $ U\subset V $ UVR eines Euklidischen VR (oder unitären VR) $ (V,\Skl{.}{.}) $ mit $ \dim V<\infty $, so gilt
		\[ V = U\oplus U^\perp. \]
\end{Korollar}
\begin{Definition}[orthogonale Komplement]
	Der UVR $ U^\perp $ heißt dann das \emph{orthogonale Komplement} von $ U $ (in $ (V,\Skl{.}{.}) $).
\end{Definition}
\paragraph{Beweis}
	Für $ v\in U\cap U^\perp $ ist $ \Skl{v}{v} = 0 $, also $ v = 0 $, da das Skalarprodukt positiv definit ist. Sei $ (e_1,\dots,e_k) $ ONB von $ U $ (Gram-Schmidt) und
		\[ p:V\to V,\ v\mapsto p(v) := \sum_{i=1}^{k}e_i\Skl{e_i}{v}\in U. \]
	Wegen
		\[ \Skl{e_j}{v-p(v)} = \Skl{e_j}{v}-\sum_{i=1}^{k}\delta_{ij}\Skl{e_i}{v} = 0 \]
	für $ j = 1,\dots, k $ ist dann
		\[ \forall v\in V:v=p(v) + (v-p(v))\in U+U^\perp, \]
	also $ V = U+U^\perp $.
\paragraph{Bemerkung}
	Die Einschränkung $ \dim V < \infty $ wurde nur benutzt, um die Orthogonalprojektion $ p\in \End(V) $ zu definieren/konstruieren. Insbesondere reicht es, $ \dim U < \infty $ anzunehmen.
\paragraph{Bemerkung}
	Ist $ \dim V < \infty $, so ist $ U^{\perp\perp} = U$.

\subsection{Beispiel \& Definition}\index{Spiegelung}
	Ist $ p\in \End(V) $ eine Projektion und $ p' = \id_V -p $, so erhält man eine Involution
		\[ s := p-p' \in \End(V) \]
% TODO Grafik %
	Im Falle einer Orthogonalprojektion $ p $ nennt man die zugehörige Transformation	
		\[ \sigma: A\to A,\ o+v\mapsto \sigma(o+v):= o+s(v) \]
	eines Euklidischen Raumes eine \emph{Spiegelung}: $ \sigma $ ist eine Isometrie, da
		\[ \forall v\in V: \|p(v)\pm p'(v)\|^2 =
			\begin{cases}
				\|v\|^2 &\text{für }+\\ \|s(v)\|^2 &\text{für } -
			\end{cases}
			= \|p(v)\|^2+\|p'(v)\|^2 \pm \underset{ = 0\text{, da }p(V)\perp p'(V)}{2\underbrace{\Skl{p(v)}{p'(v)}}} \]
\paragraph{Bemerkung}
	Jede Kongruenzabbildung eines endlichdimensionalen Euklidischen Raumes ist Komposition von Spiegelungen.

\subsection{Beispiel \& Definition}\index{Drehung}
	Ist $ A^2 $ Euklidische Ebene mit kartesischem Bezugssystem $ (o;e_1,e_2) $ und $ J\in \End(\R^2) $ wie oben,
	$ J(e_1) = e_2 \text{ und } J(e_2) = -e_1 $
	so liefert 
		\[ \rho_\theta : A^2\to A^2,\ o+v\mapsto \rho_\vartheta(o+v):= o+v\cos \vartheta + J(v)\sin\vartheta \]
	eine \emph{Drehung} mit \emph{Zentrum} $ o\in A^2 $ und \emph{Drehwinkel} $ \vartheta \in \R $. Die affine Abbildung $ \rho_\vartheta $ ist dann Komposition zweier Spiegelungen, 
		\[ \rho_\vartheta= \sigma'\circ \sigma, \]
	die durch ihre Fixpunktgeraden festgelegt sind:
		\[ g = o + [e_1] \text{ und }g' = o+[e_1'] \text{ mit } e_1' = e_1 \cos\frac{\vartheta}{2}+e_s\sin \frac{\vartheta}{2}. \]
	
% TODO Grafik Drehung % 

\subsection{Lemma}
\begin{Lemma}\label{sadj}
	Eine Projektion $ p\in \End(V) $ ist genau dann Orthogonalprojektion, wenn
		\[ \forall v,w\in V: \Skl{p(v)}{w} = \Skl{v}{p(w)} \]
\end{Lemma}
\paragraph{Beweis}
	Sei $ p\in \End(V) $ Projektion und $ p' = \id_V - p $ die komplementäre Projektion mit
		\[ \ker p = p'(V) \text{ und } p(V) = \ker p'. \]
	Ist $ p $ Orthogonalprojektion, $ p(V)\perp \ker p = p'(V) $, so ist für $ v,w\in V $
		\[ \Skl{p(v)}{w}-\Skl{v}{p(w)} = \Skl{p(v)}{p(w)}+\underset{0}{\underbrace{\Skl{p(v)}{p'(w)}}} - \Skl{p(v)}{p(w)}-\underset{0}{\underbrace{\Skl{p'(v)}{p(w)}}} = 0. \]
	Gilt andererseits für $ v,w\in V $, also insbesondere für $ v\in p(V), w\in \ker p $, stets
		\[ 0 = \Skl{p(v)}{w}-\Skl{v}{p(w)} = \Skl{v}{w}, \]
	so ist $ p(V)\perp \ker p $, also $ p $ Orthogonalprojektion.