\documentclass[a4paper, fontsize=11pt, DIV=14, parskip=half]{scrreprt}
\usepackage{microtype}
\usepackage[utf8]{inputenc}
\usepackage[T1]{fontenc}
\usepackage{lmodern}%schoeneres Schriftbild
\usepackage[ngerman]{babel}%deutsche Silbentrennung
\usepackage[onehalfspacing]{setspace}

\usepackage{graphicx}
\usepackage{float}%fuer H Positionierung
\usepackage{multicol}
\usepackage{tikz}%Zeichnungen
\usepackage{tikz-3dplot}
\usetikzlibrary{matrix,calc,arrows.meta,intersections,positioning}
\newcommand{\equal}{=}

\usepackage{amsmath,amsfonts,amssymb}%Mathematik-Pakete
\usepackage{mathtools} %für \mathrlap command
\allowdisplaybreaks%[3]%Zeilenumbrüche in Formeln erlaubt

\usepackage{enumerate}%enumerate mit roemischen Zahlen
\usepackage[colorlinks]{hyperref} %Verlinktes Inhaltsverzeichnis
\usepackage{makeidx} %Stichwortverzeichnis

\makeindex

\author{Studierendenmitschrift des Studienjahres 2015/16}
\title{Lineare Algebra \& Geometrie}
\subtitle{Prof. Udo Hertrich-Jeromin}

\setcounter{tocdepth}{1}

\newenvironment*{Satz}[1][]{\ignorespaces}{\ignorespacesafterend}
\newenvironment*{Lemma}[1][]{\ignorespaces}{\ignorespacesafterend}
\newenvironment*{Definition}[1][]{\ignorespaces}{\ignorespacesafterend}
\newenvironment*{Korollar}[1][]{\ignorespaces}{\ignorespacesafterend}

% Eigene Operatoren:
\let\hom\relax
\DeclareMathOperator{\Char}{Char}
\DeclareMathOperator{\End}{End}
\DeclareMathOperator{\Aut}{Aut}
\DeclareMathOperator{\Iso}{Iso}
\DeclareMathOperator{\hom}{Hom}
\DeclareMathOperator{\Hom}{Hom} % NOTE duplicate!
\DeclareMathOperator{\rg}{rg}
\DeclareMathOperator{\dfkt}{def}
\DeclareMathOperator{\id}{id}
\DeclareMathOperator{\sgn}{sgn} % Signum
\DeclareMathOperator{\vol}{vol} % Volumen
\DeclareMathOperator{\ggT}{ggT} 
\DeclareMathOperator{\tr}{tr} 

\newcommand{\R}{\mathbb{R}}
\newcommand{\C}{\mathbb{C}}
\newcommand{\K}{\mathbb{K}}
\newcommand{\N}{\mathbb{N}}
\newcommand{\Z}{\mathbb{Z}}
\renewcommand{\Re}{\operatorname{Re}}
\newcommand{\Skl}[2]{\langle #1,#2 \rangle}
\newcommand{\SSkl}[2]{\langle\! \Skl{#1}{#2}\! \rangle}


\begin{document}
\maketitle
\tableofcontents

% % % %Kapitel 0 - Grundlagen % % % %
\chapter{Grundlagen}
\section*{Einleitung}
	Es existieren zwei Methoden zur präzisen Formulierung:
	\begin{itemize}
	\item Funktion einer Formulierung wird präzisiert durch:
		\begin{itemize}
			\item Definition: Begriffsklärung
			\item Satz (Lemma, Proposition, Korollar): Aussage über einen (mathematischen) Sachverhalt
			\item Beweis: eine (logische) Argumentationskette, die erklärt, warum ein Satz/Lemma wahr ist
			\item Bemerkung, Beispiel: zusätzliche Information/Illustration, die oft Eigenarbeit (Beweis) erfordert
		\end{itemize}
	\item Formeln und (logische) Symbole werden verwendet:
		\begin{itemize}
			\item $\forall$ -- All-Quantor: \glqq für alle\grqq
			\item $\exists(!)$ -- Existenz-Quantor: \glqq es existiert (genau) ein\grqq
			\item $\lnot$ -- logische Verneinung: $\lnot A$ ist wahr, wenn $A$ falsch ist
			\item $\land ,\lor$ -- logisches \glqq und\grqq{} und \glqq oder\grqq
			\item $\Rightarrow ,\Leftrightarrow$ -- Implikation und Äquivalenz
		\end{itemize}
	\end{itemize}

	\begin{figure}[H]\centering
		\begin{tabular}{c|c|c|c|c|c|c}
			$A$ & $B$ & $\lnot A$ & $A\land B$ &$A\lor B$&$A \Rightarrow B$ & $A\Leftrightarrow B$\\\hline
			w & w & f & w & w & w & w\\
			w & f & f & f & w & f & f\\
			f & w & w & f & w & w & f\\
			f & f & w & f & f & w & w\\
		\end{tabular}
	\caption{Wahrheitstafel}
	\end{figure}

	Beispiele:
	\begin{itemize}
		\item Implikation: Für $x,y\in\mathbb{R}: xy = 0 \Rightarrow (x = 0\lor y = 0)$
		\item Für Aussagen $ A $ und $ B $ gilt: $(A\Rightarrow B)\Leftrightarrow (\lnot A \lor B)$, Beweis durch Wahrheitstafel
	\end{itemize}
	
	\begin{figure}[H]\centering
		\begin{tabular}{c|c|c|c|c|c}
			$A$ & $B$ & $\lnot A$ & $\lnot A\lor B$ & $A \Rightarrow B$ & $(A\Rightarrow B)\Leftrightarrow (\lnot A \lor B)$\\\hline
			w & w & f & w & w & w \\
			w & f & f & f & f & w \\
			f & w & w & w & w & w \\
			f & f & w & w & w & w \\
		\end{tabular}
	\caption{Beweis durch Wahrheitstafel}
	\end{figure}

\paragraph{Bemerkung}
	$\land$, $\lor$, und $\Leftrightarrow$ sind kommutativ (symmetisch), $\Rightarrow$ jedoch nicht, d.h.:
	\begin{gather*}
		(A\land B)\Leftrightarrow (B\land A)\\
		(A\lor B)\Leftrightarrow (B\lor A)\\
		(A\Leftrightarrow B)\Leftrightarrow (B\Leftrightarrow A)\\
		(A\Rightarrow B)\nLeftrightarrow (B\Rightarrow A)\\
	\end{gather*}
	
	weil beispielsweise formal gilt: $x,y\in\mathbb{R}: x = 0 \Rightarrow xy = 0$, aber nicht $xy = 0 \Rightarrow x = 0$.

\paragraph{Bemerkung (Beweisformen der Implikation)}
	Um eine Implikation $A\Rightarrow B$ zu zeigen, bedient man sich häufig auch folgender Äquivalenzen:
	\begin{equation*}
		(A\Rightarrow B)\Leftrightarrow
		\begin{cases}
			\lnot B\Rightarrow \lnot A&\text{(Indirekter Schluss)}\\
			\lnot (A\land \lnot B)&\text{(Widerspruchsbeweis)}
		\end{cases}
	\end{equation*}

\paragraph{Beispiel}
	Für reelle Zahlen $x,y\in\mathbb{R}$ gilt:
	\begin{equation*}
		\left((xy = 0)\Rightarrow (x=0 \lor y=0)\right) \Leftrightarrow \left((xy=0 \land x \neq 0)\Rightarrow (y =0)\right)
	\end{equation*}
	
	bzw. allgemein:
	\begin{equation*}
		(A\Rightarrow (B\lor C))\Leftrightarrow ((A\land\lnot B)\Rightarrow C)
	\end{equation*}

\paragraph{Bemerkung (Mengenlehre)}
	Die Ähnlichkeit mit der Mengensymbolik ist nicht zufällig, z.B. Mengen $X, Y$:
	\begin{gather*}
		(x\in X\cap Y)\Leftrightarrow (x\in X\land x\in Y)\\
		(x\in X\cup Y)\Leftrightarrow (x\in X\lor x\in Y)\\
		(X\subset Y) \Leftrightarrow \{\forall x : (x\in X \Rightarrow x\in Y)\}
	\end{gather*}

\section*{Definition (Abbildung)}
	\begin{Definition}[Abbildung]
		Eine Zuordnung $f: X\to Y$ zwischen zwei Mengen $X$ und $Y$ heißt eine Abbildung, falls $\forall x\in X: \exists ! y\in Y: y=f(x)$.

	X heißt der Definitionsbereich der Abbildung und $f(X):=\{f(x)\mid x\in X \}\subseteq Y$ das Bild.

	Eine Abbildung $f: X\to Y$ heißt
	\begin{itemize}
		\item injektiv, falls $\forall x,x'\in X:f(x) = f(x') \Rightarrow x=x'$
		\item surjektiv, falls $\forall y\in Y:\exists x\in X: y = f(x)$
		\item bijektiv, falls $\forall y\in Y:\exists !x\in X: y = f(x)$
	\end{itemize}
	\end{Definition}

\paragraph{Beispiel}
	Mit $X=Y=\mathbb{R}$ definiert
	\begin{itemize}
		\item die Relation $x^2 = y$ eine Abbildung $f:X\to Y, x\mapsto f(x)=x^2$
		\item die Relation $x=y^2$ keine Abbildung $f:X\to Y$, denn
		\begin{itemize}
			\item für ein $x$ gibt es zwei $y$-Werte
			\item $x < 0$ ist nicht definiert
		\end{itemize}
	\end{itemize}

\paragraph{Beispiel}
	Die Identität $id_X :X\to X, x\mapsto id_X(x):= x$ ist eine bijektive Abbildung.
	
\paragraph{Bemerkung}
	Eine Abbildung ist genau dann bijektiv, wenn sie injektiv und surjektiv ist.
	
\section*{Definition (Komposition)}
	\begin{Definition}[Komposition]
		Sind $ f:X\to Y $ und $ g:Y\to Z$ Abbildungen, so ist ihre Komposition/Verkettung die Abbildung $ g\circ f:X\to Z, x\mapsto (g\circ f)(x):= g(f(x)) $.
	\end{Definition}
	
\paragraph{Beispiel:}
	Seien $ X = Y = Z = \mathbb{R} $ und $ f:X\to Y, x\mapsto f(x) :=x^2 $, $ g:Y\to Z, y\mapsto g(y):=y^3 + y $, so ist die Verkettung $ g\circ f: X\to Z, x\mapsto (g\circ f)(x) = (x^2)^3+x^2 = x^6 + x^2 $.

\section*{Lemma}
	\begin{Lemma}[Inverse]
		Seien $ f:X\to Y $ und $ g:Y\to X $ Abbildungen. Dann gilt:
	\begin{enumerate}[i)]
		\item ist $ g $ Linksinverse von $ f $, d.h. $ g\circ f = id_X $, so ist f injektiv
		\item ist $ g $ Rechtsinverse von $ f $, d.h. $ f\circ g = id_Y$, so ist f surjektiv
		\item ist $ g $ Links- und Rechtsinverse von $ f $, so heißt $ g =f^{-1}$ Inverse von $ f $
	\end{enumerate}
	\end{Lemma}

\paragraph{Beispiel}
	$ f:\mathbb{N}\to \mathbb{N}, n\mapsto f(n):= n+1 $ hat Linksinverse
	\begin{equation*}
		g:\mathbb{N} \to \mathbb{N}, n\mapsto g(n):=
		\begin{cases}
			15700, & \text{falls } n=0\\
			n-1, & \text{falls } n\neq 0
		\end{cases}
	\end{equation*}

	Tatsächlich ist $ f $ injektiv, da
	\begin{equation*}
		\forall n,n'\in \mathbb{N} : n+1 = f(n) = f(n') = n'+1 \Rightarrow n=n'
	\end{equation*}
	
	jedoch $ f(\mathbb{N}) = \mathbb{N}\setminus \{0\} $, daher kann keine Rechtsinverse existieren.

\paragraph{Beweis}
	Zwei Aussagen sind zu beweisen:
	\begin{enumerate}[i)]
		\item Sei $ g $ Linksinverse von $ f $. Dann gilt für $ x,x'\in X $ mit \\$ f(x) = f(x'): x = g(f(x)) = g(f(x')) = x' $, also ist $ f $ injektiv.
		\item Sei $g $ Rechtsinverse von $ f $ und $ y\in Y $. Setze $ x:= g(y)\in X $, dann gilt $f(x) = f(g(y)) = y$. Damit existiert zu jedem $ y\in Y $ (mindestens) ein $ x = g(y) $, sodass  $ y=f(x) $.
	\end{enumerate}


%VO02-2015-10-08
\chapter{Lineare Räume und Abbildungen}
\section{Von Geometrie zu Algebra}
	Euklid führte in den \glqq Elementen\grqq{} (ca. 300 v. Chr.) das bis heute gültige Schema ein:
	\begin{itemize}
		\item Definition
		\item Axiom/Postulat
		\item Lehrsatz
		\item Beweis
	\end{itemize}

\subsection{Parallelenaxiom/-problem (Euklid, Formulierung nach Playfair)}
	Es existiert genau eine Parallele $ g' $ zum Punkt $ P \notin g $ zur Geraden $ g $.

	Kann das Axiom aus den anderen Axiomen hergeleitet/bewiesen werden? Nein, denn es existieren nichteuklidische, hyperbolische Geometrien (18. Jh.) in denen es mehrere derartige Parallelen gibt. Als Beispiel lässt sich eine Geometrie anführen, die nicht auf einer Ebene sondern auf einem Kreis operiert. Dort lassen sich zu einer Sekante mehrere parallele Sekanten betrachten (also Sekanten, die die ursprüngliche nicht schneiden).

	\begin{figure}[H]
		\begin{minipage}{.45\textwidth}
			\begin{tikzpicture}[line cap=round,line join=round,>=triangle 45,x=1.0cm,y=1.0cm]
				\clip(-1.69,-0.64) rectangle (4.14,2.83);
				\draw [domain=-1.69:4.14] plot(\x,{(-1--1*\x)/1});
				\draw [domain=-1.69:4.14] plot(\x,{(-0--1*\x)/1});
				\draw (0.6,1) node[] {P};
				\draw (1.58,0.16) node[] {g};
				\draw (1.52,1.78) node[] {g'};
				\begin{scriptsize}
				\fill [color=blue] (1,1) circle (2pt);
				\end{scriptsize}
			\end{tikzpicture}
		\end{minipage}
		\begin{minipage}{.45\textwidth}
			\begin{tikzpicture}[line cap=round,line join=round,>=triangle 45,x=1.0cm,y=1.0cm]
				\clip(-2.24,-3.38) rectangle (3.15,1.76);
				\draw(0,0) circle (1cm);
				\draw (-0.94,0.35)-- (0.66,0.75);
				\draw (-0.13,0.74) node[] {g};
				\draw (0.32,-0.56) node[] {P};
				\draw (-1,-0.02)-- (0.88,-0.48);
				\draw (-0.35,-0.94)-- (0.91,0.42);
				\begin{scriptsize}
				\fill [color=blue] (0.22,-0.32) circle (1.5pt);
				\end{scriptsize}
			\end{tikzpicture}
		\end{minipage}
	\end{figure}

\paragraph{Was ist eine Geometrie?}
	Eine Geometrie ist durch eine Menge X und eine auf X operierende Transformationsgruppe gegeben.

%VO3-2015-10-13
\subsection{Definition (Gruppe)}
	\begin{Definition}[Gruppe]
		Ein Paar $(G,\circ)$ bestehend aus einer Menge $G$ und einer Verknüpfung 
		\[\circ : G\times G \to G : (g,h) \mapsto g \circ h\]
		heißt Gruppe, falls:
                \begin{enumerate}[(i)]
                        \item $\forall f,g,h \in G : f\circ (g\circ h) = (f\circ g)\circ h$ \hfill (Assoziativität)
                        \item $\exists e\in G\ \forall g\in G : e\circ g = g$ \hfill (Existenz eines neutralen Elements)
                        \item $\forall g \in G\ \exists g^{-1} \in G : g^{-1}\circ g = e$ \hfill (Existenz eines inversen Elements)
                \end{enumerate}
                Die Gruppe heißt \emph{kommutativ} oder \emph{abelsch}, falls zusätzlich gilt:
                        \[\forall g,h\in G: g\circ h = h\circ g\] %\text{ (Kommutativität)}
	\end{Definition}

\paragraph{Bemerkung}
	Das ist eine axiomatische Definition, d.h. der Begriff \glqq Gruppe\grqq{} wird durch (aus vielen (!) Beispielen abstrahierten) \glqq Rechenregeln\grqq{} definiert.
\paragraph{Beispiel}
	Die rationalen Zahlen $\mathbb{Q}$ bilden mit der Addition eine Gruppe $(\mathbb{Q} ,+)$.
	Die rationalen Zahlen ohne $0$, $\mathbb{Q}^{\times} := \mathbb{Q}\setminus \{0\}$, bilden mit der Multiplikation eine Gruppe $(\mathbb{Q}^\times ,\cdot)$.

\subsection{Definition (Gruppenoperation)}
	\begin{Definition}[Gruppenoperation]
		Sind $(G,\circ )$ eine Gruppe und $X$ eine Menge, so heißt eine Abbildung
		\[ \cdot : G\times X\to X, (g,x)\mapsto g\cdot x \]
	eine \emph{Gruppenoperation} (von $(G,\circ )$ auf $X$), falls
	\begin{enumerate}[(i)]
		\item $\forall g,h\in G :\forall x\in X: g\cdot (h\cdot x) = (g\circ h)\cdot x$ (entspricht nicht der Assoziativität!)
		\item $\forall x\in X: e\cdot x = x$ für das neutrale Element $e$ der Gruppe $(G,\circ )$
	\end{enumerate}
	$(G,\circ )$ heißt dann \emph{Transformationsgruppe} von X.
	\end{Definition}

\paragraph{Bemerkung}
	Operiert $G$ (kurz für $(G,\circ )$, aus dem Zusammenhang ersichtlich) auf $X$, so ist für jedes $g\in G$ die Abbildung
		\[ g:X\to X, x\mapsto g\cdot x \]
	eine bijektive Abbildung von $X$ auf sich. Wegen der Axiome (i) und (ii) aus der Definition erhält man $g^{-1}: X\to X$ als Inverse der Abbildung.
	
\subsection{Beispiel und Definition (Permutationsgruppe)}
	\begin{Definition}[Permutationsgruppe]
		Die bijektiven Abbildungen einer Menge $X$ auf sich, 
		\[ G:= \{g:X\to X\mid g \text{ bij.}\}, \]
	bilden (mit der Komposition $\circ$) eine (Transformations-)Gruppe $(G,\circ )$ (die auf $X$ operiert): die \emph{Permutationsgruppe} oder \emph{symmetrische Gruppe} $S_X$ von $X$.
	
	Für $X=\{1,2,...,n\}$ schreibt man auch $S_n$ statt $S_{\{1,...,n\}}$.
	\end{Definition}
\paragraph{Bemerkung}
	Im Gegensatz zu allgemeinen Abbildungen stimmen in (Permutations-)Gruppen Links- und Rechtsinverse stets überein.
\subsection{Lemma (Eindeutigkeit des neutralen Elements)}
	\begin{Lemma}[Eindeutigkeit des neutralen Elements]
		Das neutrale Element einer Gruppe $(G,\circ )$ ist eindeutig und $\forall g\in G: g\circ e = g$. Weiters: 
		\[\forall g\in G\ \exists ! g^{-1} \in G: g^{-1}\circ g = g \circ g^{-1} = e\]
	\end{Lemma}

\paragraph{Beweis}
	Sei $g\in G$ gegeben und (gemäß Gruppenaxiom (iii)):
	\begin{itemize}
		\item $h:= g^{-1}$ (Linksinverse von $g$)
		\item $k:= h^{-1}$ (Linksinverse von $h$)
	\end{itemize}
	Damit berechnen wir (multiplikative Schreibweise: $ab$ statt $a\circ b$):
	\begin{align*}
		hg = e = kh &= k((hg)h) \\
                            &= k(h(gh)) \tag{$\star$}\\
                            &= (kh)(gh) = gh
	\intertext{und }
                ge = g(hg) &\stackrel{(\star)}{=} (gh)g = eg
	\end{align*}
	
	Jedes (links-)neutrale Element $e$ ist also auch rechtsneutral
	\[\forall g\in G: eg = ge = g\tag{$\star\star$}\]
	und ist $e'\in G$ auch neutrales Element, dann:
	\[e' = ee' \stackrel{(\star\star)}{=} e'e = e \]
	Weiters ist jedes (Links-)Inverse auch rechtsinvers
	\[\forall g \in G: gg^{-1}=g^{-1}g = e \]
	und sind $h,h'\in G$ Inverse von $g\in G$, so gilt:
	\[h' = h'(gh) = (h'g)h = h \]
	d.h. Eindeutigkeit des Inversen.

\subsection{Definition (Körper)}
	\begin{Definition}[Körper]
		Ein Tripel $(K,+,\cdot)$, bestehend aus einer Menge $K$ und zwei Verknüpfungen
                \begin{align*}
                        +:&K\times K\to K,(x,y)\mapsto x+y\\
                        \cdot : &K\times K\to K, (x,y)\mapsto xy
                \end{align*}
                heißt \emph{Körper}, falls:
                \begin{enumerate}[(i)]
                        \item $(K,+)$ ist abelsche Gruppe (mit neutralem Element $0$ und inversem Element $-x$ von $x$)
                        \item $(K^\times,\cdot)$ ist abelsche Gruppe (mit neutralem Element $1$ und inversem Element $\frac{1}{x} = x^{-1}$ von $x\in K^\times$)
                        \item die Distributivgesetze gelten:
                                \[ \forall x,y,z\in K :\begin{cases}x\cdot (y+z) = x\cdot y+x\cdot z\\ (x+y)\cdot z = x\cdot z+y\cdot z \end{cases} \]
                \end{enumerate}
	\end{Definition}

\paragraph{Bemerkung}
	In einem Körper gilt stets:
		\[0\cdot x = x\cdot 0 = 0\]
        denn
		\begin{gather*}
		0\cdot x = (0+0)\cdot x = 0\cdot x + 0\cdot x \\
		\Rightarrow 0 = 0\cdot x + (-(0\cdot x)) = 0\cdot x + 0\cdot x + (-(0\cdot y)) = 0\cdot x
		\end{gather*}
	und analog $x\cdot 0 = 0$\\
	Insbesondere folgt damit 
	\[\forall x,y\in K: x\cdot y = y\cdot x\]
	(im zweiten Axiom für Körper wird die abelsche Gruppe für $K^\times$ festgelegt.)
	
\paragraph{Beispiel}
	Die rationalen Zahlen $\mathbb{Q}$, die reellen Zahlen $\mathbb{R}$ oder die komplexen Zahlen $\mathbb{C}$ bilden mit den üblichen Verknüpfungen Körper.

%VO04-2015-10-15
\paragraph{Bemerkung und Beispiel}
	Aufgrund der Axiome (i) und (ii) enthält $ K $ mindestens 2 Elemente, also $ \# K \geq 2 $, nämlich:
	\begin{itemize}
		\item $ 0 $, das neutrale Elemente bezüglich $+$ und
		\item $1$ $(\neq 0)$, das neutrale Elemente (in $K^\times$ = $K\setminus\{0\}$) bezüglich $\cdot$
	\end{itemize}
	Es gibt auch einen Körper mit genau 2 Elementen $(\{0,1\},+,\cdot)$, wobei
	\begin{minipage}{0.45\textwidth}
		\begin{equation*}
			\begin{tabular}{c|cc}
				$+$ & 0 & 1\\\hline
				0 & 0 & 1\\
				1 & 1 & 0\\
			\end{tabular}
		\end{equation*}
	\end{minipage}
	\begin{minipage}{0.45\textwidth}
		\begin{equation*}
			\begin{tabular}{c|cc}
				$\cdot$ & 0 & 1\\\hline
				0 & 0 & 1\\
				1 & 1 & 1\\
			\end{tabular}
		\end{equation*}
	\end{minipage}
	Dieser Körper wird auch $\mathbb{Z}_2$ bezeichnet.

\subsection{Bemerkung und Definition (Charakteristik)}
	\begin{Definition}[Charakteristik]
		In $\mathbb{Z}_2$ gilt $1 + 1 = 0$. Allgemeiner definiert man die \emph{Charakteristik} eines Körpers $(K,+,\cdot)$ (mit neutralen Elementen 0 und 1 von + bzw. $\cdot$) durch
                \begin{equation*}
                        \Char(K,+,\cdot):=
                        \begin{cases}
                                0,\text{falls } \forall n \in \mathbb{N}^\times: \sum_{j = 1}^{n} 1 \neq 0\\
                                \min\{n \in \mathbb{N}^\times\mid \sum_{j = 1}^{n} 1 = 0\}
                        \end{cases}
                \end{equation*}
	\end{Definition}
	z.B. $\Char(\mathbb{Z}_2) = 2$, da
	\begin{align*}
		\{n\in\mathbb{N}^\times\mid \underbrace{1+...+1=0}_{\text{n-mal}}=0\}
		&=\{n\in\mathbb{N}^\times\mid n=0 \text{ mod } 2\}\\
		&=\{n\in\mathbb{N}^\times\mid n \text{ gerade}\}
        \end{align*}
        und damit $\min\{n\in\mathbb{N}^\times\mid \underbrace{1+...+1=0}_{\text{n-mal}}\}=2$
	
	Wir werden mitunter $\Char(K,+,\cdot)\neq 0$ oder (öfter) $\Char(K,+,\cdot)=2$ ausschließen (müssen).

%VO05-2015-10-20
\section{Unterräume und Lineare Hülle}
 \subsection{Definition (Untervektorraum)}
 	\begin{Definition}[Untervektorraum]
 		Eine Teilmenge $U\subset V$ eines $K$-VR $V$ heißt \emph{Unter(vektor)raum} (UVR), falls $U$ mit der eingeschränkten Addition und Skalarmultiplikation
 		\begin{align*}
 			^+    & \mid_{U\times U}: U\times U \to V,\ (v,w) \mapsto v+w \\
 			\cdot & \mid_{K\times U}: K\times U \to V,\ (x,v) \mapsto vx
 		\end{align*}
 		selbst ein Vektorraum ist, d.h. wenn insbesondere
 		\begin{align*}
 			  & \forall v,w \in U: v+w\in U \text{ und}  \\
 			  & \forall x\in K\ \forall v\in U: vx\in U.
 		\end{align*}
 	\end{Definition}

 	\paragraph{Bemerkung}
 		Eine nicht-leere Teilmenge $U\subset V, U\neq\emptyset$, ist genau dann ein UVR, wenn die auf $U$ eingeschränkten Operationen wohldefiniert sind, d.h. wenn $ U $ bzgl. $ + $ und $ \cdot $ abgeschlossen ist.

 		Dies kann zum \emph{Unterraumkriterium} zusammengefasst werden:
 		\begin{equation*}
 			U\subset V \text{ ist UVR }\Leftrightarrow
 			\begin{cases}
 				U\neq\emptyset                              \\
 				\forall v,w\in U\ \forall x\in K: vx+w\in U
 			\end{cases}
 		\end{equation*}

 	\paragraph{Beispiel}
 		Sei $I=\{1,...,n\}$. Für jedes (feste) $i\in I$ ist
 		\[
 			U_i := \{v:I\to K\mid v_i =0\}
 		\]
 		ein UVR von $K^n$, denn
 		\begin{enumerate}
 			\item $v = 0 \in U_i\text{, also } U_i \neq \emptyset$
 			\item Seien $v,w\in U_i$, d.h. $v,w\in K^n$ mit $v_i =w_i =0$, und $x\in K$; dann gilt $(vx+w)_i = v_ix+ w_i = 0\cdot x + 0 = 0$, also $vx+w\in U_i$ und damit ist $U_i$ UVR nach Unterraumkriterium.
 		\end{enumerate}
 		Kein UVR von $K^n, n\geq 2$, ist jedoch die Menge
 		\[
 			N:=\{v:I\to K\mid v_1\cdot v_2 = 0\},
 		\]
 		denn
 		\begin{enumerate}
 			\item $N$ ist zwar nicht-leer, $N\neq \emptyset$, aber
 			\item $^+\mid_{N\times N}: N\times N\to N$ nicht wohldefiniert: seien $v,w\in N$, so dass
 			      \begin{gather*}
 			      	v_1=0, v_2=1\text{ }(v_3 ... v_n \text{ irrelevant})\\
 			      	w_1=1, w_2 = 0\text{ }(w_3 ... w_n \text{ irrelevant})
 			      \end{gather*}
 		\end{enumerate}
 		dann gilt:
 		\begin{gather*}
 			(v+w)_1 = v_1 + w_1 = 0+1=1\\
 			(v+w)_2 = v_2 + w_2 = 1+0 = 1
 		\end{gather*}
 		und damit
 		\[
 			(v+w)_1(v+w)_2 = 1 \Rightarrow v+w\notin N.
 		\]

 	\paragraph{Bemerkung und Beispiel}
 		In analoger Weise definiert man die Begriffe
 		\begin{itemize}
 			\item einer \emph{Untergruppe} $H\subset G$ einer Gruppe $(G,\cdot)$, bzw.
 			\item eines \emph{Unter-} oder \emph{Teilkörpers} $T\subset K$ eines Körpers $(K,+,\cdot )$
 		\end{itemize}

 		z.B. bildet jeder UVR $U\subset V$ eines $K$-VR $V$ (mit der Addition) eine Untergruppe der Gruppe $(V,+)$.\\
 		In gleicher Weise bildet eine nicht-leere Teilmenge eine \emph{Untergruppe} bzw. einen \emph{Unterkörper}, falls die eingeschränkten Operationen wohldefiniert sind.

 		z.B. ist $H\subset G$ eine Untergruppe, falls (Untergruppenkriterium):
 		\begin{enumerate}
 			\item $H\neq \emptyset$
 			\item $\forall g,h\in H: g\circ h^{-1} \in H$
 		\end{enumerate}

 		Achtung: Inversenbildung muss im Kriterium explizit formuliert werden, sonst würde z.B.: $\mathbb{N}\subset\mathbb{Z}$ als Teilmenge von der Gruppe $(\mathbb{Z}, +)$ ein Gegenbeispiel liefern.

 		Weitere Beispiele:
 		\begin{itemize}
 			\item die Translationen bilden eine Untergruppe der Bewegungsgruppe
 			\item $\mathbb{Q}\subset\mathbb{R}$ und $\mathbb{R}\cong \{x+iy\mid y=0\}\subset\mathbb{C}$ bilden Teilkörper von $\mathbb{R}$ bzw. $\mathbb{C}$.
 		\end{itemize}

 \subsection{Lemma (Schnitt von UVR)}
 	\begin{Lemma}[Schnitt von UVR]
 		Ist $(U_i)_{i\in I}$ eine Familie von UVR $U_i\subset V$ eines $K$-VR $V$, so ist ihr Schnitt
 		\[
 			U:= \bigcap_{i\in I}U_i =\{ u\in V\mid \forall i\in I: u\in U_i\}
 		\]
 		ein UVR von $V$. (Beweis durch UR-Krit. in Aufgabe 17)
 	\end{Lemma}

 \subsection{Definition (Lineare Hülle)}
 	\begin{Definition}
 		Die \emph{lineare Hülle} $[S]$ einer Teilmenge $S\subset V$ eines $ K $-VR $ V $ ist der Schnitt aller $S$ enthaltenden UVR $U\subset V$:
 		\[
 			[S] := \bigcap_{S\subset U \text{ UVR}} U
 		\]
 		Die lineare Hülle einer Familie $(v_i)_{i\in I}$ von Vektoren $v_i\in V$ in einem $ K $-VR $ V $ ist
 		\[
 			[(v_i)_{i\in I}] := [\{v_i\mid i\in I\}]
 		\]
 	\end{Definition}

 	\paragraph{Bemerkung}
 		$[S]$ ist ein UVR (nach Lemma), der \glqq kleinste\grqq{} UVR, der $S$ enthält, d.h. ist $U\subset V$ UVR mit $S\subset U$, so gilt $[S]\subset U$; da aber $[S] = \bigcap_{S\subset \tilde{U}  \text{ UVR}}\tilde{U}\subset U$,
 		da $S\subset U$, also $U$ am Schnitt beteiligt ist.

 	\paragraph{Bemerkung}
 		$[\emptyset ] = \{0\}$ und $[V] = V$.

 	\paragraph{Beispiel}
 		Ist $U\subset V$ UVR, so gilt $[U] = U$.

 	\paragraph{Beispiel}
 		$N=\{v:I\to K\mid v_1v_2=0\} \subset K^n,\ I=\{1,...,n\},\ n\geq 2$, hat die lineare Hülle $[N]=K^n$.

 	\paragraph{Beispiel}
 		Für $I=\{1,...,n\}$ und $i\in I$ definiere
 		$e_i:I\to K ,\ j\mapsto e_i(j):= \delta_{ij}$, wobei
 		\begin{equation*}
 			\delta_{ij} :=
 			\begin{cases}
 				1, & \text{falls }i=j \\
 				0, & \text{sonst}
 			\end{cases}
 		\end{equation*}
 		das \emph{Kroneckersymbol} bezeichnet.

 		Damit ist die lineare Hülle der Familie $(e_i)_{i\in I}$
 		\[
 			[(e_i)_{i\in I}] = K^n.
 		\]
 		Nämlich: Da $[(e_i)_{i\in I}]\subset K^n$ ist, gilt für beliebige $x_1,...,x_n\in K$
 		\begin{gather*}
 			\underbrace{e_1x_1\underbrace{+...+\underbrace{e_nx_n + 0}_{\in [(e_i)_{i\in I}]}}_{\in [(e_i)_{i\in I}]}}_{\in [(e_i)_{i\in I}]}\in [(e_i)_{i\in I}]
 		\end{gather*}
 		da $[(e_i)_{i\in I}] \subset K^n$ UVR ist.
 		Andererseits gilt für beliebiges $v\in K^n$:
 		\[
 			v=\sum^n_{i=1}e_iv(i): I\to K,
 		\]
 		denn
 		\[
 			\forall j\in I: \bigg(\sum^n_{i=1} e_iv(i)\bigg)(j) = \sum^n_{i=1}e_i(j)v(i) = v(j)
 		\]
 		Damit ist gezeigt, dass die beiden Abbildungen übereinstimmen; da $v\in K^n$ beliebig war, folgt $K^n \subset [(e_i)_{i\in I}]$

 \subsection{Definition (Linearkombination)}
 	\begin{Definition}
 		Seien $(v_i)_{i\in I}$ und $(x_i)_{i\in I}$ Familien in einem $ K $-VR bzw. dem Körper $ K $, wobei
 		\begin{align*}
 			\# & \{i\in I\mid x_i \neq 0\} < \infty\text{ , also} \\
 			   & \{i\in I \mid x_i \neq 0\} = \{i_1,...,i_n\}
 		\end{align*}
 		für ein geeignetes  $n\in \mathbb{N}$;
 		Dann heißt die endliche Summe
 		\[
 			\sum_{i\in I} v_ix_i:= \sum^n_{j=1}v_{i_j}x_{i_j}
 		\]
 		eine \emph{Linearkombination}.
 	\end{Definition}

 	\paragraph{Bemerkung}
 		Die Bedingung $\#\{i\in I \mid x_i\neq 0\} <\infty$
 		garantiert, dass die Summe wohldefiniert ist $\rightarrow$ vgl. Reihen in der Analysis.

%VO06-2015-10-22
 \subsection{Lemma (Lineare Hülle und Linearkombinationen)}
 	\begin{Lemma}[Lineare Hülle und Linearkombinationen]
 		Ist $(v_i)_{i\in I}$, $I \neq \emptyset$, Familie in einem $K$-VR, so gilt:
 		\[
 			[(v_i)_{i\in I}] = \bigg\{\sum_{i\in I} v_ix_i\mid x: I\to K,\ \# \{i\in I \mid x_i \neq 0\}< \infty\bigg\},
 		\]
 		d.h. die lineare Hülle der Familie ist die Menge aller Linearkombinationen der Familie.
 	\end{Lemma}

 	\paragraph{Beweis}
 		Wir zeigen (wie üblich) zwei Inklusionen:

 		"$\supseteq$":

 		Sei also $(x_i)_{i\in I}$ eine geeignete Familie in $ K $, dann gilt:
 		\[
 			\sum_{i\in I} v_i x_i = \underbrace{v_{i_1} x_{i_1} + ... + \underbrace{(v_{i_n}x_{i_n}+0)}_{\in [(v_i)_{i\in I}]}}_{\mathrlap{ \in [(v_i)_{i\in I}] \text{ nach UR-Krit. (nach $n$ Schritten)}}}
 		\]

 		"$\subseteq$":

 		Setze $U := \{{\sum_{i\in I} v_ix_i\mid x: I\to K \text{ mit } \#\{{i\in I\mid x_i \neq 0\}} < \infty\}}$, offenbar gilt:
 		\[
 			\forall i\in I: v_i\in U
 		\]
 		Wir zeigen, dass $U$ ein Untervektorraum ist. Das heißt:
 		\begin{align*}
 			^+    & \mid_{U\times U}: U\times U \to U \subset V  \\
 			\cdot & \mid_{K\times U}: K\times U \to U \subset V,
 		\end{align*}
 		also die Addition und Skalarmultiplikation vererben sich auf $ U $.

 		Zur Skalarmultiplikation:
 		\begin{addmargin}[25pt]{0pt}
 			Sind $(x_i)_{i\in I}$ mit $\#\{i\in I \mid x_i \neq 0\}<\infty$ eine Familie in $ K $ und $x\in K$, so gilt für ein geeignetes $n\in \mathbb{N}$
 			\[
 				\{i\in I\mid x_i \neq 0\} = \{i_1, ... , i_n\}
 			\]
 			und damit
 			\begin{equation*}
 				\{i\in I\mid x_ix\neq 0\} =
 				\begin{cases}
 					\{{i_1,...,i_n\}}, & \text{falls }x \neq 0 \\
 					\emptyset,         & \text{falls }x = 0.
 				\end{cases}
 			\end{equation*}
 			Also folgt
 			\begin{align*}
 				\bigg(\sum_{i\in I}v_i x_i\bigg) x & = \bigg(\sum_{j=1}^{n} v_{i_j}x_{i_j}\bigg)x                          \\
 				                                   & = \sum_{j=1}^{n} v_{i_j}(x_{i_j}x) = \sum_{i\in I} v_i(x_ix) \in U_i,
 			\end{align*}
 			da $\sum_{i\in I} v_i(x_ix)$ Linearkombination (mit der Familie $(x_ix)_{i\in I}$ in K) ist.
 		\end{addmargin}

 		Zur Addition:
 		\begin{addmargin}[25pt]{0pt}
 			Ähnlich, siehe Aufgabe.
 		\end{addmargin}

 	\paragraph{Bemerkung}
 		Um triviale Diskussionen zu vermeiden, setzt man $\sum_{i\in \emptyset} ...:=0$.

%VO06-2015-10-22
\section{Basis und Dimensionen}

\subsection{Definition (Basis)}
	\begin{Definition}[Basis]
		Eine Teilmenge $S\subset V$ oder eine Familie $(v_i)_{i\in I}$ in $ V $ heißt:
	\begin{itemize}
		\item \emph{Erzeugendensystem} von $ V $, falls $[S] = V$ bzw. $[(v_i)_{i\in I}] = V$
		\item \emph{linear unabhängig}, falls $\forall v\in S: v \notin [S\setminus\{{v\}}]$ bzw. $\forall i\in I: v_i \notin [(v_j)_{j\in I\setminus\{{i\}}}]$ und sonst \emph{linear abhängig}.
	\end{itemize}
        Eine \emph{Basis} ist ein linear unabhängiges Erzeugendensystem.
	\end{Definition}

\paragraph{Bemerkung}
	Man kann jede (Teil-)Menge $S\subset V$ als Familie in $V$ auffassen mit
		\[v_i: S \to V: v\mapsto id_S(v) = v.\]
	Andererseits gilt für eine Familie $(v_i)_{i\in I} $:
		\[(v_i)_{i\in I} \text{ linear unabhängig } \Rightarrow \{v_i \mid i\in I\} \text{ linear unabhängig.}\]
	Die Umkehrung gilt im Allgemeinen nicht: Eine Familie (in $ V $) enthält mehr Information als eine Teilmenge von $ V $.
	
\subsection{Beispiel und Definition (Standardbasis)}
	\begin{Definition}[Standardbasis]
		Für $V = K^n$ ist $(e_1, ... , e_n)$,
	\begin{equation*}
		e_i:\{{1, ... ,n\}} =: I\to K: j\mapsto e_i(j)= \delta_{ij}=
		\begin{cases}
			1,& \text{falls } i=j\\
			0,& \text{sonst}
		\end{cases}
	\end{equation*}
	für $i=1,\dots,n$, eine Basis -- die Standardbasis des (Standard-)Vektorraumes $K^n$.
	\end{Definition}

\paragraph{Beweis}
	z.z.: $ (e_i)_{i\in I} $ ist ein linear unabhängiges Erzeugendensystem. Wir wissen bereits $ [(e_i)_{i\in I}] = K^n $. Andererseits gilt für jedes $i\in I$ und jede Familie $(x_j\mid j\in I)$ in $ K $
	\begin{gather*}
		\left(\sum_{j\in I\setminus\{i\}}e_jx_j\right)(i) = \sum_{j\in I\setminus\{i\}}e_j(i)x_j = 0 \neq 1 = e_i(i)\\
		\Rightarrow \sum_{j\in I\setminus\{i\}} e_jx_j \neq e_i
	\end{gather*}
	also gilt
	\begin{equation*}
		\forall i\in I: e_i \notin [(e_j)_{j\in I\setminus\{i\}}] = \left\{\sum_{j=I\setminus\{i\}} e_jx_j\mid (x_j)_{ j\in I}\right\} \text{ mit } \#\{j\in I\mid x_j \neq 0\}<\infty
	\end{equation*}
	
\subsection{Lemma}
	\begin{Lemma}
		Eine Familie $(v_i)_{i\in I}$ ist linear unabhängig gdw. für jede Linearkombination
		\[0 = \sum_{i\in I} v_ix_i \Rightarrow \forall i\in I: x_i = 0.\]
	\end{Lemma}

\paragraph{Beweis}
	Wir zeigen zwei Richtungen der Äquivalenz der Negationen: 
		\[(v_i)_{i\in I} \text{ linear abhängig } \Leftrightarrow \exists(x_i)_{i\in I} \neq (0)_{i\in I}: \sum_{i\in I} v_ix_i = 0.\]
	"$\Leftarrow$":
	Wir nehmen an, es gäbe eine \emph{nicht-triviale}\footnote{d.h. $(x_i)_{i\in I}\neq 0$} Linearkombination der Null,
		\[0 = \sum_{i\in I} v_ix_i, \text{ wobei } \exists j\in I: x_j \neq 0.\]
	Für $(y_i)_{i\in I}, y_i := - \frac{x_i}{x_j}$ ist dann
	\begin{gather*}
		0 = v_jx_j + \sum_{i\in I\setminus\{j\}} v_ix_i \\
		\Rightarrow v_j = -\left(\sum_{i\in I\setminus\{j\}}v_ix_i\right)x_j^{-1} = \sum_{i\in I\setminus\{j\}} v_iy_i \in [(v_i)_{i\in I\setminus\{j\}}]
	\end{gather*}
	insbesondere ist also $(v_i)_{i\in I}$ linear abhängig.
	"$\Rightarrow$": siehe Aufgabe.
	
\subsection{Korollar}
	\begin{Korollar}
		Ist $(v_i)_{i\in I}$ Basis von $ V $, so ist jeder Vektor $v\in V$ eindeutig in den $v_i$ darstellbar:
		\[\forall v\in V \exists! (x_i)_{i\in I}: v = \sum_{i\in I} v_ix_i\]
	\end{Korollar}

\paragraph{Beweis}
	Sei $v\in V$ beliebig, dann gilt:
		\[V = [(v_i)_{i\in I}] \Rightarrow \exists (x_i)_{i\in I}: v = \sum_{i\in I} v_ix_i\]
	liefern $(x_i)_{i\in I}$ und $(y_i)_{i\in I}$
	\begin{align*}
		v = \sum_{i\in I} v_ix_i = \sum_{i\in I}v_iy_i \Rightarrow\footnotemark 0 &= \sum_{i\in I} v_i(x_i-y_i)\\
                &\Rightarrow \forall i\in I: x_i = y_i \Rightarrow (x_i)_{i\in I} = (y_i)_{i\in I}
	\end{align*}
	\footnotetext{Bemerke: Die Summe von Linearkombinationen ist wieder eine Linearkombination, siehe Aufgabe 19}
	Damit ist die Basisdarstellung $v = \sum_{i\in I} v_ix_i$ von $v$ auch eindeutig.

%VO07-2015-10-27
\subsection{Basislemma}
    \begin{Lemma}[Basislemma]
    	Sei $S\subset V$ lin. unabh. und $E\subset V$ ein Erzeugendensystem mit $S\subset E$. Dann existiert eine Basis $B$ von $V$ mit $S\subset B\subset E$.
    \end{Lemma}

\paragraph{Beweis}
    Wir gehen für den Beweis davon aus, dass $\#E<\infty$. Betrachte alle Teilmengen $X\subset V$ mit $S\subset X\subset E$ und $X$ lin. unabh. Sei $B$ eine solche Menge, die maximal ist, d.h.
        \[\forall X\subset E: ((B\subset X\land X\text{ lin. unabh.}) \Rightarrow X= B)\]
    Nach Konstruktion ist $B=\{b_1,...,b_n\}$ lin. unabh. Zu zeigen: $V=[B]$.\\
    Ist $B=E$, so folgt $[B]=[E]=V$.\\
    Ist $B\neq E$, so ist $B\cup \{v\} $ für (jedes) $v\in E\setminus B$ lin. abh., da $B$ maximal lin. unabh. ist; also existiert eine nicht-triviale Linearkombination des Nullvektors.
        \[\exists x,x_1,...,x_n \in K: 0=vx+\sum^n_{i=1}b_ix_i\]
    Wäre $x=0$, so würde folgen $x_1=...=x_n=0$, da $B$ lin. unabh. ist. 
    Also ist $x\neq 0$ und 
    	\[v=-\sum^n_{i=1} b_i\frac{x_i}{x} \in [B].\]
    Da dies für beliebiges $v\in E\setminus B$ gilt, folgt
    	\[E\subset [B] \Rightarrow V=[E]\subset [[B]] = [B],\]
    d.h., $ B $ ist Erzeugendensystem und damit eine Basis mit $S\subset B\subset E$.

\paragraph{Bemerkung}
    Ist $\#E = \infty$, so kann man einen analogen Beweis führen, falls man die Existenz einer maximalen Menge voraussetzt: Dies garantiert das \emph{Zornsche Lemma} bzw. \emph{Auswahlaxiom}.
    Wir werden das Lemma auch im Falle $\#E = \infty$ benutzen!

\paragraph{Beispiel}
    Für $V=K^3=K^I$ mit $I=\{1,2,3\}$ betrachte die Standardbasisvektoren 
    \begin{align*}
        e_i &:I\to K,\ j\mapsto e_i(j) = \delta_{ij}\text{, und}\\
        f_i &: I\to K,\ j\mapsto f_i(j):= 1-\delta_{ij}
    \end{align*}
    dann sind $S:= \{e_1,f_1\}$ und $E:= \{e_i,f_i\mid i\in I\}$ lin. unabh. bzw. Erzeugendensystem von $K^3$. Ergänzung von $S$ durch einen Vektor $e_i$ oder $f_i, i = 2,3$ liefert eine Basis $B$ mit $S\subset B\subset E$.
    
    Zum Beispiel: $B=\{e_1,f_1,f_2\}$ eine Basis, da sich jede Funktion $v\in K^3$ aus den Funktionen $e_1,f_1$ und $f_2$ linear kombinieren lässt.
    \begin{gather*}
        v=e_1x_1+f_1y_1 + f_2y_2\Leftrightarrow \left\{
            \begin{array}{l}
                v(2)=y_1\\
                v(3) - v(2) = y_1 + y_2 - y_1 = y_2\\
                v(1) + v(2) - v(3) = x_1 + y_2 - y_2 = x_1
            \end{array}
    	\right.
    \end{gather*}
    Dass $B$ lin. unabh. folgt dann: Wäre $B$ lin. abh., so würde folgen $f_2\in [\{e_1,f_1\}]\Rightarrow [B] \subset [\{e_1,f_1\}] \neq K^3$, was nicht der Fall ist.

\subsection{Basisergänzungssatz}
    \begin{Satz}[Basisergänzungssatz]
    	Jede lin. unabh. Menge $S\subset V$ kann zu einer Basis $B$ von $V$ ergänzt werden: Es existiert eine Basis $B$ von $V$ mit $S\subset B$.
    \end{Satz}

\paragraph{Beweis}
    Sei $E\subset V$ ein Erzeugendensystem von $V$ (z.B. $E=V$). Dann ist $S\cup E$ ein Erzeugendensystem von $V$ mit $S\subset S\cup E$, das Basislemma liefert dann die\footnote{nicht eindeutig!} gesuchte Basis.

\subsection{Bemerkung}
    Strikt genommen haben wir den Basisergänzungssatz (BES) nur unter der Annahme bewiesen, dass $V$ \emph{endlich erzeugt} sei, d.h. $V$ ein endliches Erz. Syst. $E$ besitzt, $V=[E]$ und $\#E<\infty$.

\paragraph{Bemerkung}
    Wir haben für den BES die (in diesem Falle einfachere) Mengenschreibweise (anstelle der Familienschreibweise) verwendet.

\paragraph{Bemerkung}
    Ähnlich kann man einen Verkürzungssatz beweisen: Jedes Erzeugendensystem eines Vektorraums $V$ kann zu einer Basis verkürzt werden.

\subsection{Austauschlemma}
    \begin{Lemma}[Austauschlemma]
    	Seien $B,B' \subset V$ Basen von $V$. Dann gilt:
        \[\forall b\in B \exists b' \in B': (B\setminus\{b\})\cup\{b'\} \text{ ist Basis}\]
    \end{Lemma}
    
\paragraph{Beweis}
    Sei $b\in B$ beliebig gewählt und $S:= B\setminus \{b\}$. Da $B$ lin. unabh. ist, gilt 
        \[b\notin [S] \Rightarrow \emptyset \neq V\setminus [S] = [B']\setminus [S] \Rightarrow B' \not\subset [S]\]
    d.h. es existiert $b' \in B'$ mit $b' \notin [S]$. Wir zeigen, dass $B'' := S\cup \{b'\} = (B\setminus\{b\})\cup \{b'\}$ Basis ist.
    
    $B''$ ist Erzeugendensystem: Da $b'\in [B]$ existiert $(x_j)_{j\in B}$ mit $$b' = \sum_{j\in B} jx_j $$ mit $x_b \neq 0$, da $b' \notin [S]$.
    Damit ist $b=(b'-\sum_{j\in S} jx_j)\frac{1}{x_b} \in [B''] \Rightarrow V = [B] \subset [B'' \cup \{b\}] \subset [B'']$.
    
    $B''$ ist linear unabhängig: $B''$ ist Erz. Syst. und $S\subset B' = S \cup \{b'\}$ lin unabh., kann also (nach Basislemma) erg"anzt werden zu einer Basis $\tilde{B}$ mit $S\subset \tilde{B}\subset B''$.
    Da $[S] \neq V$ gilt $\tilde{B} \neq S$ und damit $\tilde{B} = B''$ Basis, insbesondere linear unabhängig.
    
\paragraph{Bemerkung}
    Hier haben wir die Familienschreibweise (mit $B$ bzw. $S$ als Indexmenge) verwendet, um Linearkombinationen darzustellen.
    
\subsection{Basissatz}
	\begin{Satz}[Basissatz]
	Sei $V$ ein endlich erzeugter $K$-VR, $V=[E]$ mit $\#E < \infty$. Dann gilt:
	\begin{enumerate}[(i)]
		\item $V$ besitzt eine endliche Basis $B$ mit $n:= \#B \leq \#E$.
		\item Ist $B'\subset V$ eine Basis von $V$, so ist $\#B' = \#B = n$.
	\end{enumerate}
	\end{Satz}
    
\paragraph{Beweis}
    \begin{enumerate}[(i)]
        \item  Dies folgt direkt aus dem Basislemma (mit $S=\emptyset$).
        \item Seien $B,B'$ Basen von V, $B = (b_1,...,b_n)$.\\
        Annahme: $\#B' < n, B' = (b'_1,...,b'_k)$ mit $k < n$. Wiederholte Anwendung des Austauschlemmas auf die Basen $B$ und $B'$ liefert nach (spätestens) $k+1\leq n$ Schritten einen Widerspruch zur linearen Unabhängigkeit der neuen Basis $B''$, da Vektoren $b'_i$ doppelt vorkommen müssen.\\
        Annahme: $\#B' > n, B' = (b'_1,...,b'_n,b'_{n+1})$: Das gleiche Argument mit vertauschten Rollen der Basen führt wieder zum Widerspruch.
     \end{enumerate}

\subsection{Definition (Dimension)}
    \begin{Definition}[Dimension]
    	Sei $V$ ein $K$-VR, die \emph{Dimension} von $ V $ ist dann:
        \begin{itemize}
            \item $\dim V:= \#B$, falls $ V $ endlich erzeugt und $B$ eine Basis von $V$ ist;
            \item $\dim V:= \infty$, falls $V$ nicht endlich erzeugt ist.
        \end{itemize}
    \end{Definition}
    
\paragraph{Bemerkung}
    Nach dem Basissatz hängt $\dim V = \#B$ (falls $V$ endlich erz.) nicht von der Basis $B$ ab, d.h. $\dim V$ ist wohldefiniert.
    
\paragraph{Beispiel}
    $\dim K^n = \#\{e_1,...,e_n\} = n$ (Standardbasis).

%VO08-2015-10-29
\subsection{Korollar (Dimension und Teilmengen)}
	\begin{Korollar}[Dimension und Teilmengen]
		Sei $ V $ ein $ K $-VR mit $\dim V =: n\in \mathbb{N}$. Dann gilt:
    \begin{enumerate}[(i)]
    	\item Ist $S \subset V$ linear unabhängig, so ist $\# S \leq n$ und $\# S = n$ gdw. $ S $ Basis ist.
    	\item Ist $E \subset V$ Erzeugendensystem, so ist $\#E \geq n$, bzw. $\#E = n$ gdw. $ E $ eine Basis ist.
    \end{enumerate}
	\end{Korollar}
    
\paragraph{Bemerkung}
	Insbesondere: Ist $U\subset V$ UVR mit $\dim U=\dim V < \infty$, so gilt $ U=V $.
   
\paragraph{Beweis}
    \begin{enumerate}[(i)]
        \item Ist $ S $ linear unabhängig, so existiert (nach BES) eine Basis $ B $ von $ V $ mit 
            \begin{gather*}
                S\subset B\Leftrightarrow \left\{
                \begin{array}{l}
                    \#S \leq \#B\\
                    \#S = \#B \Leftrightarrow S = B
                \end{array}\right.
            \end{gather*}
        \item Analog (mit Basislemma), siehe Aufgabe 23.
        \end{enumerate} 

%VO08-2015-10-29
\section{Homomorphismen}
\subsection{Definition}
	\begin{Definition}[Homomorphismus]
		Sind $ V $ und $ W $ $ K $-VR, so heißt eine Abbildung $f: V \rightarrow W$ \emph{($K$-)linear} oder ein \emph{(Vektorraum-)Homomorphismus} $f\in \hom(V,W)$, falls gilt:
	

\begin{enumerate}[(i)]
	\item $\forall v,w \in V: f(v+w) = f(v) + f(w)$;
	\item $\forall v\in V\ \forall x\in K: f(vx) = f(v)x$
\end{enumerate}

    das heißt, $ f $ ist verträglich mit den Vektorraumoperationen in $ V $ und $ W $.
	\end{Definition}
\paragraph{Bemerkung}
	Damit die Verträglichkeit mit der Skalarmultiplikation sinnvoll ist, müssen $ V $ und $ W $ Vektorräume über demselben Körper $ K $ sein.

\paragraph{Bemerkung}
	Für $f\in \hom(V,W)$ gilt stets $f(0_V) = f(0_V\cdot0_K) = f(0_V)\cdot0_K = 0_W$.

\subsection{Bemerkung \& Definition}
        Ebenso erklärt man zum Beispiel \emph{Gruppenhomomorphismen} oder \emph{Körperhomomorphismen}. Sind etwa $(G,\circ)$ und $(H,\cdot)$ Gruppen, so ist eine Abbildung $f: G \to H$ ein Gruppenhomomorphismus, falls
        \begin{equation*}
            \forall g,h \in G: f(g\circ h) = f(g) \cdot f(h)
        \end{equation*}
  
\paragraph{Beispiel}
	Ist $f\in \hom(V,W)$ ein Vektorraumhomomorphismus so ist $ f $ nach (i) Gruppenhomomorphismus von $ (V,+) $ in $ (W,+) $.
  
\paragraph{Beispiel}
	Sei $ V $ ein $ K $-VR und $y\in K$ fest, dann ist die Streckung um $y: \eta_y:V\to V: v\mapsto \eta_y(v) := vy$ ein Homomorphismus von $ V $ in sich, $\eta_y\in \hom(V,V)$. Eine Streckung nennt man auch Homothetie.
  	
\paragraph{Beispiel}
	Sei $V = \mathbb{C} = \{z = x+iy\mid x,y\in \mathbb{R}\}$, dann ist die komplexe Konjugation $\mathbb{C}\ni z = x+iy \mapsto x-iy =: \bar{z} \in \mathbb{C}$ kein Homomorphismus von $\mathbb{C}$ in sich, wenn man $\mathbb{C}$ als $\mathbb{C}$-VR auffasst. Hingegen ist sie ein Homomorphismus von $\mathbb{C}$ in sich, wenn man $\mathbb{C}$ als $ \mathbb{R} $-VR auffasst.
	
\subsection{Lemma (Linearkombinationen und Homomorphismen)}
	\begin{Lemma}[Linearkombinationen und Homomorphismen]
		$f:V\to W$ ist genau dann ein Homomorphismus, wenn für jede beliebige Linearkombination gilt:
		\begin{equation*}
                    f\left(\sum_{i\in I}v_ix_i\right) = \sum_{i\in I}f(v_i)x_i
		\end{equation*}
	\end{Lemma}

\paragraph{Beweis}
	Eine Richtung ist trivial, die andere mit vollständiger Induktion zu zeigen.

\subsection{Fortsetzungssatz} 
	\begin{Satz}[Fortsetzungssatz]
		Seien $ V $ und $ W $ $K$-VR, $(b_i)_{i\in I}$ eine Basis von $ V $ und $(c_i)_{i\in I}$ eine Familie in $ W $.
	Dann gilt:
	\begin{equation*}
            \exists!f\in \hom(V,W), \forall i\in I: f(b_i) = c_i
        \end{equation*}
	\end{Satz}
    
\paragraph{Bemerkung}
        Anders ausgedrückt: ist $B\subset V$ eine Basis von $ V $, so kann jede Abbildung $f: B\to C\subset W$ eindeutig zu einem Homomorphismus $f: V\to W$ fortgesetzt werden.
    
\paragraph{Beweis}
	Wir beweisen die Existenz und die Eindeutigkeit getrennt. 
	\begin{enumerate}
		\item Eindeutigkeit: Sei $f\in \hom(V,W)$ so, dass $\forall i\in I: f(b_i)=c_i$. Sei $v\in V$ beliebig. Da $ B $ Erzeugendensystem ist, lässt sich $ v $ als Linearkombination in $(b_i)_{i\in I}$ mit geeigneten Koeffizienten $(x_i)_{i\in I}$ in $ K $ darstellen.
			\begin{gather*}
    				v=\sum_{i\in I}b_ix_i \Rightarrow f(v) = \sum_{i\in I} f(b_i)x_i = \sum_{i\in I}c_ix_i
    			\end{gather*}
    
                        Damit ist $ f(v) $ eindeutig durch $ v $ und die $c_i = f(b_i)x_i$ bestimmt.
    
    		\item Existenz: Da $(b_i)_{i\in I}$ auch linear unabhängig ist, ist jedes $v\in V$ eindeutig als Linearkombination in $(b_i)_{i\in I}$ dargestellt, damit ist durch
    		\begin{equation*}
                    f:V\to W,\ v=\sum_{i\in I}b_ix_i \mapsto f(v):=\sum_{i\in I}c_iv_i
                \end{equation*}
    		eine Abbildung wohldefiniert.
    
                        Weiters ist $f\in\hom(V,W)$ wegen
                        \begin{align*}
                                &f(v+w) =\sum_{i\in I}c_i(x_i+y_i)= f(v) + f(w) \text{ für alle } v=\sum_{i\in I}b_ix_i \in V, w=\sum_{i\in I}b_iy_i \in V    
                        \intertext{und}
                                &f(vx) =\sum_{i\in I}c_i(x_ix)=\sum_{i\in I}(c_ix_i)x = (\sum_{i\in I}c_ix_i)x= f(v)x \text{ für }  x\in K\text{ und }v= \sum_{i\in I}b_ix_i \in V.
                        \end{align*}
                        Damit ist die Linearität von $ f $ gezeigt.
        \end{enumerate}
    
\subsection{Beispiel und Definition (Dualraum)}
	\begin{Definition}[Dualraum]
		Der Dualraum $V^\ast := \hom(V,K)$ eines $K$-VRs $V$ ist ein $ K $-VR $(\subset K^V)$. Ist $\dim V=:n<\infty$ so ist $\dim V^\ast=n$.
	Ist $B=(b_i, ... ,b_n)$ eine Basis von $ V (\dim V < \infty)$, so definieren wir für $ i = \{1, ... ,n\} $ die Linearform (nach Fortsetzungssatz):
	\begin{equation*}
		b_i^\ast\in V^*:V\to K, \forall j\in \{1,...,n\}:b_i^*(b_j)=\delta_{ij}
	\end{equation*} die zu $ B $ duale Basis $ B^* $ von $V^\ast$.
	\end{Definition}

%VO09-2015-11-03
\paragraph{Beweis} $ V^* $ ist $ K $-VR. Wir zeigen $ V^*\subset K^V $ ist UVR.
        \begin{itemize}
                \item $ 0: V\to K $ ist linear, d.h. $ 0 \in V^* \Rightarrow V^* \neq \emptyset $
                \item Seien $ f,g \in V^* $ und $ x\in K $; dann gilt
			\begin{align*}
				\forall v,w\in V: (fx+g)(v+w) &= f(v+w)x+g(v+w)\\
                                                              &= (f(v)+ f(w))x+(g(v)+g(w))\\
                                                              &= (f(v)x+g(v))+(f(w)x+g(w))\\
                                                              &= (fx+g)(v)+(fx+g)(w)
			\intertext{genauso:}
                                \forall v\in V, y\in K: (fx+g)(vy) &= f(vy)x+g(vy)\\
                                                                   &= f(v)yx + g(v)y\\ 
                                                                   &= (f(v)x +g(v))y = (fx+g)(v)y
                        \end{align*}
                        Damit gilt: $ fx+g\in \hom (V,K) = V^* $
        \end{itemize}
	
	Da $ f,g\in V^* $ und $ x\in K $ beliebig waren, zeigt das UR-Kriterium, dass $ V^*\subset K^V $ ein UVR ist und damit selbst $ K $-VR ist.
	
\paragraph{Beweis} $B^*$ ist Basis. Wir zeigen $B^*$ ist linear unabhängig und Erzeugendensystem.
	\begin{itemize}
            \item $ B^* $ ist linear unabhängig: Seien $ x_1,...,x_n $ so, dass
                    \begin{equation*}
                    0 = \sum_{i=1}^{n}b_i^*x_i \Rightarrow \forall j=1,...,n: 0=(\sum_{i=1}^{n}b_i^*x_i)(b_j) = \sum_{i=1}^{n}b_i^*(b_j)x_i = \sum_{i=1}^{n}\delta_{ij}x_i = x_j.
                    \end{equation*}
            Also $ x_1 = ... = x_n = 0 $ und damit ist $ B^* $ linear unabhängig.
            \item $ B^* $ ist Erzeugendensystem: Sei $ f\in V^* $ beliebig, dann gilt:
            \begin{equation*}
                    \forall j = 1,...,n:f(b_i) = \sum_{i=1}^{n}b_i^*(b_j)f(b_i) = (\sum_{i=1}^{n}b_i^*f(b_i))b_j \Rightarrow f = \sum_{i=1}^{n}b_i^*f(b_i)\in [B^*].
            \end{equation*}
            
            Da $ f\in V^* $ beliebig war, ist also $ V^* = [B^*]$.
	\end{itemize}
	
	Damit ist $ B^* = \{b_1^*,...,b_n^*\}$ eine Basis von $ V^* $ -- insbesondere also $ \dim V^* = n = \dim V = \dim K\cdot \dim V $.

\paragraph{Bemerkung}
	Ist $\dim V = \infty$ und $B=(b_i)_{i\in I}$ eine Basis von $V$, so liefert $B^\ast=(b_i^\ast)_{i\in I}$ mit $\forall j\in I:b_i^\ast(b_j)=\delta_{ij}$ eine lineare unabhängige Familie. Diese ist jedoch kein Erzeugendensystem von $V^\ast: f\in\hom(V,K)=V^\ast$ mit $\forall j\in I:f(b_j)=1$ lässt sich nicht in $B^\ast$ linear kombinieren. Wäre $f=\sum_{i\in I}b_i^\ast x_i$, so gälte $\forall j\in I: x_j =\sum_{i\in I}b_i^\ast(b_j)x_j= \sum_{i\in I} \delta_{ij}x_j = f(b_j) = 1$.

	Das heißt, $(x_i)_{i\in I}$ wäre eine Familie in $ K $ mit $\#\{i\in I\mid x_i\neq 0\}=\infty$.

\subsection{Satz (Homomorphismen als VR)}
	\begin{Satz}[Homomorphismen als VR]
		$ \hom (V,W) $ ist ein VR. Die Dimension der Homomorphismen $\dim\hom (V,W) = m\cdot n$, falls $m:=\dim W<\infty, n:=\dim V< \infty$.
	\end{Satz}
	
\paragraph{Beweis}
	Addition und Skalarmultiplikation in $\hom (V,W)$ werde (wie für $K$-wertige Abbildungen oder in $V^*$) punktweise definiert:
	\begin{itemize}
		\item für $f,g \in \hom (V,W)$ setzt man $(f+g)(v) := f(v) + g(v)$ für alle $v\in V$,
		\item für $f\in \hom (V,W)$ und $x\in K$ setzt man $(fx)(v) := f(v)x$ für alle $v\in V$.
	\end{itemize}
	Die so definierten Abbildungen $f+g,fx: V\to W$ sind linear, $f+g, fx\in \hom (V,W)$, aufgrund der VR-Eigenschaften von $V$.
	
	Damit zeigt man: $\hom (V,W)$ ist $K$-VR (siehe Aufgabe 27).
	
	Seien nun $\dim V = n < \infty$ und $\dim W = m < \infty$.
	
	Wir wählen (nach BES) Basen $B = (b_1,...,b_n)$ von $V$ und $C=(c_1,...,c_m)$ von $W$ und definieren
		\begin{equation*}
			\hom (V,W) \ni f_{ij}:= c_i\cdot b_j^* \text{ für } 
				\begin{cases}
					i\in I := \{1,...,m\}\\
					j\in J := \{1,...,n\}
				\end{cases}
		\end{equation*}
	Behauptung: $F=(f_{ij})_{\substack{i\in I\\j \in J}}$ ist Basis von $\hom (V,W)$.
	
	Da $(c_i)_{i\in I}$ linear unabhängig in $W$ ist, gilt für jede Famlilie $(x_{ij})_{I,J}$ in $K$:
		\begin{align*}
			0 &= \sum_{\substack{i\in I\\j \in J}} f_{ij}x_{ij} \\
			\Rightarrow \forall k \in J: 0 &= \sum_{\substack{i\in I\\j \in J}} (f_{ij}x_{ij})(b_k) = \sum_{\substack{i\in I\\j \in J}} c_i b_j^* (b_k) x_{ij} = \sum_{i\in I} c_ix_{ik} \\
			&\Rightarrow \forall k\in J\forall i\in I:x_{ik} = 0
		\end{align*}
	Also ist $F$ linear unabhängig.
	
	Da $(c_i)_{i\in I}$ Erzeugendensystem von $W$ ist, existiert zu jedem (fest gegebenen) $f\in\hom (V,W)$ eine Familie $(x_{ij})_{I,J}$ in $K$, sodass
		\begin{align*}
                    \forall k\in J: f(b_k) &= \sum_{i\in I} c_i x_{ik} \quad\text{(da $(c_i)_{i\in I}$ Erzeugendensystem)}\\
                    &= \sum_{\substack{i\in I\\j \in J}}c_ib_j^*(b_k)x_{ij}
                    = (\sum_{\substack{i\in I\\j \in J}} f_{ij}x_{ij})(b_k)
                    \intertext{also (nach Fortsetzungssatz):}
                    f&=\sum_{\substack{i\in I\\j \in J}}f_{ij}x_{ij} \in [F]
                \end{align*}
	Da $f\in\hom (V,W)$ beliebig war, gilt also $\hom (V,W) = [F]$. Damit ist $F$ Basis von $\hom (V,W)$ und $\dim\hom (V,W) = \# F = m\cdot n$
	
	
\subsection{Lemma und Definition (Bild, Kern, Rang \& Defekt)}
	\begin{Definition}[Bild, Kern, Rang \& Defekt]
		Sei $f\in \hom (V,W)$. Dann sind Bild und Kern von f:
		\begin{equation*}
			f(V) = \{f(v)\in W\mid v\in V \}\subset W \text{ bzw. } \ker (f) := \{v\in V\mid f(v) = 0 \} \subset V
		\end{equation*}
	
	UVR von $W$ bzw. $V$. Ihre Dimensionen heißen Rang und Defekt von $f$:
		\begin{equation*}
			\rg f := \dim f(V) \text{ bzw. } \dfkt f := \dim \ker f
		\end{equation*}
	\end{Definition}

\paragraph{Bemerkung}
	Da $f(0)=0$ für  $f\in \hom (V,W)$ gilt $\{0_V \}\in \ker f$ und $\{0_W \}\in f(V)$.

\paragraph{Beweis}
	Zu zeigen: Das Bild $f(V)\subset W$ und $\ker f\subset V$ sind UVR. Nach Bemerkung gilt $f(V)\neq \emptyset$ und $\ker f \neq \emptyset$ -- wir verwenden dann das UR-Kriterium.
	
	Das Bild $f(V)$ ist UVR: $f(V) \neq \emptyset$. Es bleibt zu zeigen:
		\begin{equation*}
			\forall w_1,w_2\in f(V), \forall x\in K: w_1x+w_2 \in f(V).
		\end{equation*}
	
	Seien also $w_1 = f(v_1), w_2 = f(v_2) \in f(V)$ und $x\in K$; dann gilt:
		\begin{equation*}
			w_1x+w_2 = f(v_1)x+f(v_2) = f(v_1x+v_2)\in f(V)
		\end{equation*}
		
	Der Kern $\ker f$ ist UVR: $\ker f\neq \emptyset$; seien $v_1,v_2\in \ker f$ und $x\in K$, dann gilt:
		\begin{equation*}
			f(v_1x+v_2) = f(v_1)x+f(v_2) = 0\cdot x + 0 = 0 \Rightarrow v_1x+v_2\in \ker f
		\end{equation*}

\paragraph{Bemerkung}
	Allgemeiner kann man für $f\in \hom (V,W)$ zeigen:
		\begin{enumerate}
			\item Ist $U\subset V$ UVR, so ist $f(U)\subset W$ UVR.
			\item Ist $U\subset W$ UVR, so ist $f^{-1}(U) = \{v\in V\mid f(v) \in U \}\subset V$ ein UVR.
		\end{enumerate}

\paragraph{Bemerkung}
	Die Funktion $f\in \hom (V,W)$ ist genau dann injektiv, wenn $\ker f = \{0\}$. Nämlich:
		\begin{itemize}
			\item ist $f$ injektiv und $v\in \ker f$, so gilt $f(v) = 0 = f(0) \Rightarrow v=0$
			\item ist $\ker f = \{ 0 \}$ und sind $v,w \in V$ mit $f(v) = f(w)$, so folgt\\
				$0=f(v)-f(w) = f(v-w) \Rightarrow v-w\in \ker f = \{0\} \Rightarrow v = w$
		\end{itemize}

\paragraph{Bemerkung}
	Eine lineare Abbildung $ f\in \hom (V,W) $ ist genau dann
		\begin{enumerate}[(i)]
			\item injektiv, wenn $ \forall S\subset V: S$ lin. unabh. $ \Rightarrow f(S) $ lin. unabh.
			\item surjektiv, wenn $ \forall E \subset V:E $ Erz. Syst. $ \Rightarrow f(E)$ Erz. Syst.
			\item bijektiv, wenn $ \forall B\subset V: B$ Basis $ \Rightarrow f(B)$ Basis
		\end{enumerate}

	Ist $ f\in \hom (V,W) $ bijektiv, so ist $ f^{-1}\in \hom (W,V) $.

\subsection{Rangsatz}
	\begin{Satz}[Rangsatz]
		Sei $ f\in \hom (V,W) $. Ist $ \dim V = n < \infty $,  so gilt $\rg f + \dfkt f = \dim V$.  Ist $ \dim V = \infty $, so gilt $ \rg f = \infty $ oder $ \dfkt f = \infty $.
	\end{Satz}

%VO09-2015-11-05
\paragraph{Beweis}
	Wir nehmen an, dass $ \dfkt f = k \neq \infty $.
	Sei $ (b_1,...,b_k) $ eine Basis von $ \ker f $;
	nach BES ergänzen wir zu einer Basis $ (b_j)_{j\in J} $ von $ V $ (bemerke: $ \{1,...,k\}\subset J $).
	Wir sehen $ I:= J\setminus \{1,...,k\} $ und $ \forall i\in I: c_i := f(b_i) $.
	
	Behauptung: $(c_i)_{c\in I}$ ist eine Basis von $f(V)$.
	
\subparagraph{Lineare Unabhängigkeit}
	gilt für eine Linearkombination in $(c_i)_{i\in I}$:
		\[ 0=\sum_{i\in I}c_ix_i = \sum_{i\in I}f(b_i)x_i = f(\sum_{i\in I}b_ix_i)
		 \]
	so folgt
	\begin{gather*}
		\sum_{i\in I}b_ix_i \in \ker f\\
		\Rightarrow \exists y_1,...,y_n\in K:\sum_{i\in I}b_ix_i=\sum_{j=1}^{k}b_jy_j\\
		\Rightarrow 0 = \sum_{i\in I}b_ix_i - \sum_{j=1}^{k}b_jy_j\\
		\Rightarrow
		\begin{cases}
			\forall j = 1, ... ,k:y_j=0\\
			\forall i\in I: x_i = 0,
		\end{cases}
		\text{da $(b_j)_{j\in J}$ linear unabhängig ist.}
	\end{gather*}
			
	Insbesondere gilt also $\forall i\in I: x_i = 0$ damit folgt die lineare Unabhängigkeit nach Lemma.
	
\subparagraph{Erzeugendensystem}
	
	Sei $w\in f(V)$, also existiert $v\in V$ mit $w = f(v)$. Da $(b_j)_{j\in J}$ Basis von $V$ ist, existiert eine Familie $(x_j)_{j\in J}$ in $K$ so, dass 
	\begin{equation*}
		v = \sum_{j\in J} b_jx_j
	\end{equation*}
	
	Dann gilt
	\begin{gather*}
		w = f(v) = f(\sum_{j\in J} b_jx_j) = \sum_{j\in J}f(b_jx_j)\\
		J=I \cup\{{1,...,k\}} \Rightarrow \sum_{j=1}^{k}f(b_j)^{=0}x_j + \sum_{i\in I}f(b_i)^{=c_i}x_i = \sum_{i\in I}c_ix_i\in[(c_i)_{i\in I}].
	\end{gather*}
			
	Da $(c_i)_{i\in I}$ also Basis von $f(V)$ ist folgt:
			
	\begin{enumerate}[1.{ Fall}]
		\item $(\dim V = n<\infty)$ dann ist $\# J = n$ und $\# I = \# J-k$, also $\rg f = n-k = \dim V - \dfkt f.$
		\item $(\dim V = \infty)$, dann ist $\# J = \infty $ und damit auch $\#I =\#(J\setminus \{{1,...,k\}})=\infty $, also $\rg f= \infty$.
	\end{enumerate}
	
\paragraph{Bemerkung}
	Die Annahme $\dfkt f = k<\infty$, im Beweis ist keine Einschränkung:
	\begin{enumerate}
		\item ist $\dim V < \infty$, so folgt $\dfkt f<\infty$, da $\ker f\subset V$ Untervektorraum ist;			
		\item $\dim V = \infty$, so ist man mit dem Beweis fertig, falls $\dfkt f = \infty$.
	\end{enumerate}
			
\paragraph{Korollar (Homomorphismen zwischen gleichdimensionalen VR)} 
	\begin{Korollar}[Homomorphismen zwischen gleichdimensionalen VR]
		Sei $f\in \hom(V,W)$ und $\dim W = \dim V = n<\infty$.
	Dann gilt: Ist $f$ injektiv oder surjektiv, so ist $f$ bijektiv.
	\end{Korollar}
	
\paragraph{Beweis} 
	Der Rangsatz liefert:
	\begin{enumerate}
		\item Wenn $ f $ injektiv ist, dann ist $\ker f = \{{0\}}$, also ist $ \dfkt f = 0 \Rightarrow \rg f = \dim V-0 = \dim V \Rightarrow f(V) = W \Leftrightarrow f$ surjektiv
		\item $f(V) = W \Rightarrow \rg f = \dim W = \dim V \Rightarrow \dfkt f= \dim V - \rg f=0 \Rightarrow \ker f = \{{0}\}$
	\end{enumerate}
	
\paragraph{Beispiel}
	Der Shiftoperator für Folgen $(x_i)_{i\in \mathbb{N}}$ in $K$, $s: K^{\mathbb{N}} \to K^{\mathbb{N}}: (x_i)_{i\in \mathbb{N}} \mapsto (y_i)_{i\in \mathbb{N}}$ wobei
	
	\begin{equation*}
		y_i :=
		\begin{cases}
			0 &\text{ für } i = 0\\
			x_{i-1} &\text{ für } i \neq 0
		\end{cases}
	\end{equation*}
			
	ist ein injektiver Homomorphismus, $s\in \hom(K^\mathbb{N},K^\mathbb{N})$ von $K^\mathbb{N}$ in sich (damit gilt $\dim $ Definitionsbereich $= \dim K^\mathbb{N}= \dim$ Wertebereich). Aber $s$ ist nicht surjektiv, also auch nicht bijektiv.
	
\paragraph{Übrigens}
	Damit folgt $\dim K^\mathbb{N} =\infty$ (sonst hätte man einen Widerspruch zum Korollar).
		
\subsection{Definition (Spezielle Homomorphismen)}
	\begin{Definition}[Spezielle Homomorphismen]
		Sei $f\in \hom(V,W)$ ein Homomorphismus, dann heißt $f$:
	\begin{itemize}
		\item Endomorphismus, $f\in \End (V)$, falls $W = V$;
		\item Isomorphismus, $f\in \Iso(V,W)$, falls $f$ bijektiv ist;
		\item Automorphismus, $f\in \Aut(V)$, falls $W=V$ und $f$ bijektiv ist.
	\end{itemize}
	
	Zwei $K$-VR $V$ und $W$ heißen isomorph, $W \cong V$, falls $\Iso(V,W) \neq \emptyset$.
	\end{Definition}

\paragraph{Bemerkung}
	Ein Isomorphismus $f\in \Iso(V,W)$ bildet jede Basis $B$ von $V$ auf eine Basis $C = f(B)$ von $W$ ab.
	
	Andererseits: Bildet eine lineare Abbildung $f\in \hom(V,W)$, eine Basis $B$ von $V$ auf eine Basis $C = f(B)$ von $W$ ab, so ist $f$ ein Isomorphismus.
	
	Nämlich: Ist $B$ Basis von $V$ und $ C = f(B)$ Erzeugendensystem, so ist $f$ surjektiv, da $f(V) = f ([B]) = [f(B)] = [C] = W$;
	ist $C =f(B)$ linear unabhängig, so ist $f$ injektiv, denn für
			
	\begin{gather*}
		v = \sum_{b\in B} bx_b \in \ker f \Rightarrow 0 = f(v) = f(\sum_{b\in B}bx_b) = \sum_{b\in B}f(b)x_b\\
		\Rightarrow \forall b \in B: x_b = 0 \Rightarrow v = 0, \text{ d.h., } \ker f=\{{0}\}.
	\end{gather*}
			
\subsection{Isomorphielemma}
	\begin{Lemma}[Isomorphielemma]
		Seien $V$ und $W$ $ K $-VR mit $\dim V, \dim W < \infty$.
	Dann gilt: $V \cong W \Leftrightarrow \dim V = \dim W$.
	\end{Lemma}
	
\paragraph{Beweis}
	Folgt aus obiger Bemerkung. Ausführlich:
		
	$\Rightarrow$:
	
	Annahme: $V \cong W$; sei $f\in \Iso(V,W)(\neq 0)$.
	Wähle eine Basis $B = (b_1, ... b_n)$ von $V$ (BES); da $f$ bijektiv, ist dann:
	\begin{equation*}
		C = f(B) = (f(b_i), ... , f(b_n))
	\end{equation*}
	
	eine Basis von W, damit ist $\dim W = n = \dim V$.
	
	$\Leftarrow$:
	
	Sei $\dim W = \dim V = n$;
	wähle Basen $B = (b_1, ... ,b_n)$ von $V$ und $C = (c_1, ... ,c_n)$ von $W$ (BES und Basissatz) und definiere $f\in \hom(V,W)$ durch (Fortsetzungssatz):
	\begin{equation*}
		\forall i = 1, ... ,n : f(b_i) = c_i
	\end{equation*}

	Da $f$ eine Basis auf eine Basis abbildet ist $f\in \Iso(V,W)$.
	Damit folgt also $\Iso(V,W) \neq \emptyset \Rightarrow V \cong W$.
			
\paragraph{Beispiel: }
	Ist $V$ $K$-VR mit $\dim V < \infty$, so ist $V^\ast \cong V$. (Achtung: Es gibt aber viele Isomorphismen, keiner ist besonders d.h., \glqq kanonisch\grqq .)
	
\paragraph{Bemerkung: }
	Ist $f\in \Iso(V,W)$, so ist $f^{-1}\in \Iso(W,V)$, denn
	\begin{gather*}
		(f\circ f^{-1})(\sum_{i\in I}v_ix_i) = f(\sum_{i\in I}f^{-1}(v_i)x_i) = \sum_{i\in I}(f\circ f^{-1})^{(= id)}(v_i)x_i\\
		\Rightarrow f^{-1}(\sum_{i\in I}v_ix_i) = \sum_{i\in I}f^{-1}(v_i)x_i.
	\end{gather*}

%VO10-2015-11-10
\section{Summen, Produkte und Quotienten}
 \subsection{Definition (Summe von UVR)}
 	\begin{Definition}[Summe von UVR]
 		Die \emph{Summe einer Familie} $ (U_i)_{i\in I} $ von UVR $ U_i\subset V $ eines $ K $-VR ist die Menge
 		\[
 			\sum_{i\in I} U_i := \Big\{\sum_{i \in I}u_i\mid \forall i\in I: u_i\in U_i \land \# \{i\in I\mid u_i \neq 0\}<\infty\Big\}.
 		\]
 	\end{Definition}

 	\paragraph{Bemerkung}
 		Offenbar ist $ \sum_{i\in I} U_i\subset V $ UVR mit
 		\[
 			\bigcup_{i\in I}U_i \subset \sum_{i\in I} U_i \Rightarrow \Big[\bigcup_{i\in I}U_i\Big]\subset \sum_{i\in I} U_i;
 		\]
 		andererseits gilt:
 		\[
 			\sum_{i\in I}U_i \subset \Big\{\sum_{j\in J}v_jx_j\mid \forall j\in J: v_j\in \bigcup_{i\in I}U_i \land \#\{j\in J\mid x_j\neq 0\}<\infty\Big\}\subset \Big[\bigcup_{i\in I}U_i\Big].
 		\]
 		Damit ist die Summe einer Familie $ (U_i)_{i\in I} $ gerade die lineare Hülle ihrer Vereinigung $ \bigcup_{i\in I}U_i $,
 		\[
 			\sum_{i\in I}U_i= \Big[\bigcup_{i\in I}U_i\Big].
 		\]

 	\paragraph{Beispiel}
 		Sei $ V=\mathbb{R}^\mathbb{N} $ der Raum der reellen Folgen. Für $ n\in \mathbb{N} $ setze
 		\[
 			U_n := \{v\in \mathbb{R}^\mathbb{N}\mid \forall j\in \mathbb{N}: j>n\Rightarrow v_j = 0 \} \subset \mathbb{R}^\mathbb{N};
 		\]
 		dann gilt $ \forall n\in \mathbb{N}: U_n\subset U_{n+1} $, und damit auch
 		\begin{gather*}
 			\sum_{i\leq n} U_i = U_n = \bigcup_{i\leq n}U_i, \quad\text{aber}\quad \sum_{i\in \mathbb{N}}U_i = \bigcup_{i\in \mathbb{N}}U_i \neq V.
 		\end{gather*}

 		Nun setze für $ i\in \{0,1\} $
 		\[
 			\tilde{U}_i := \{v\in \mathbb{R}^\mathbb{N}\mid \forall j\in \mathbb{N}: j\equiv
 			i\operatorname{mod} 2\Rightarrow v_j = 0\}
 		\]
 		dann ist
 		\[
 			\bigcup_{i\in \{0,1\}}\tilde{U}_i \neq \sum_{i\in \{0,1\}}\tilde{U}_i = V.
 		\]

 \subsection{Dimensionssatz}
 	\begin{Satz}[Dimensionssatz]
 		Sind $ U_i \subset V $ UVR mit $ \dim U_i < \infty $ für $ i\in \{1,2\} $, so ist
 		\[
 			\dim (U_1+U_2) + \dim (U_1\cap U_2) = \dim U_1 + \dim U_2.
 		\]
 		Ist $ \dim U_1 = \infty$ oder $ \dim U_2=\infty $, so ist auch $ \dim (U_1+U_2)=\infty $.
 	\end{Satz}

 	\paragraph{Beweis}
 		Seien
 		\begin{itemize}
 			\item $ B_0 \subset U_1\cap U_2 $ eine Basis von $ U_0 := U_1\cap U_2 $;
 			\item $ S_i \subset U_i $ lin. unabh., so dass $ B_i = B_0 \cup S_i $ Basen von $ U_i $ sind ($ i = 1,2 $) (nach BES).
 		\end{itemize}
 		Offenbar gilt dann, da $ B_i = B_0\cup S_i $ lin. unabh. sind,
 		\[
 			B_0\cap S_1 = \emptyset \text{ und } B_0\cap S_2 = \emptyset
 		\]
 		und
 		\[
 			S_1\cap S_2 \subset U_1\cap U_2 = [B_0] \Rightarrow S_1\cap S_2 = \emptyset.
 		\]
 		Wir zeigen, dass $ B:= B_0\cup S_1\cup S_2 $ Basis von $ U_1 + U_2 =: U $ ist.
 		\begin{itemize}
 			\item	$ B\subset U $ ist Erz. Syst. nach Konstruktion:
 			      \begin{gather*}
 			      	\forall i\in \{1,2\} : U_i=[B_i]\subset [B]\\
 			      	\Rightarrow U_1+U_2 = [U_1\cup U_2]\subset [B]
 			      \end{gather*}

 			\item $ B $ ist linear unabhängig:
 			      Gegeben sei eine Linearkombination von $ 0\in U $,
 			      \[
 			      	0 = \sum_{b\in B}bx_b = \underbrace{\sum_{b\in B_0}bx_b}_{:=b_0} + \underbrace{\sum_{b\in S_1}bx_b}_{:=s_1} + \underbrace{\sum_{b\in S_2}bx_b}_{:=s_2}
 			      \]
 			      mit $b_0\in [B_0] = U_0$ und $s_i\in [S_i]$ für $i= 1,2$;

 			      dann gilt etwa, da $ B_1 = B_0 \cup S_1 $ lin. unabh. ist,
 			      \[
 			      	\underbrace{b_0+ s_1}_{\in U_1} = \underbrace{-s_2}_{\in [S_2]} \in U_1\cap [S_2]\subset U_0 \Rightarrow s_1 = 0
 			      \]

 			      und damit, da $ B_2 = B_0 \cup S_2 $ lin. unabh. ist,
 			      \[
 			      	0 = b_0 + \underbrace{s_1}_{=0} + s_2 \Rightarrow b_0=s_2 = 0.
 			      \]
 		\end{itemize}
 		Mit der linearen Unabhängigkeit von $ B_0$, $S_1 $ und $ S_2 $ folgt dann
 		\[
 			0 = \sum_{b\in B_0}bx_b = \sum_{b\in S_1}bx_b = \sum_{b\in S_2}bx_b \Rightarrow \forall b\in B: x_b = 0.
 		\]
 		Mit
 		\begin{align*}
 			\overbrace{\#B}^{\dim(U_1+U_2)} + \overbrace{\#B_0}^{\dim(U_1\cap U_2)} & = (\#B_0 + \#S_1 + \#S_2) + \#B_0                               \\
 			                                                                        & = (\#B_0+\#S_1)+(\#B_0 + \#S_2)                                 \\
 			                                                                        & = \underbrace{\#B_1}_{\dim(U_1)}+\underbrace{\#B_2}_{\dim(U_2)}
 		\end{align*}
 		folgt dann die Behauptung.

 	\paragraph{Bemerkung}
 		Im Beweis haben wir bemerkt:

 		Ist z.B. $ B_1 = B_0\cup S_1 $ lin. unabh., und $ b_0\in [B_0] $ und $ s_1\in [S_1] $ mit $b_0 + s_1 = 0$ so folgt
 		\[
 			b_0 = s_1 = 0
 		\]
 		Sind nämlich $ b_0 = \sum_{b\in B_0} bx_b $ und $ s_1 = \sum_{b\in S_1}bx_b $, so gilt
 		\begin{gather*}
 			0 = b_0+s_1 = \sum_{b\in B_0} bx_b+\sum_{b\in S_1} bx_b = \sum_{b\in B_1} bx_b \\ \Rightarrow \forall b\in B_1: x_b = 0\Rightarrow b_0 = s_1 = 0.
 		\end{gather*}

 	\paragraph{Bemerkung}
 		Ist $ U_1\cap U_2 = \{0\} $ bzw. $ \dim (U_1\cap U_2)  = 0 $, so zeigt der Beweis auch:
 		\[
 			\forall v\in U_1+U_2\ \exists ! u_1 \in U_1\ \exists ! u_2\in U_2: v= u_1+u_2
 		\]

 \subsection{Definition (Komplementäre UVR)}
 	\begin{Definition}[Komplementäre UVR]
 		Zwei UVR $ U_1,U_2 \subset V $ heißen \emph{komplementär} in $ V $, falls
 		\[
 			U_1+U_2=V\text{ und } U_1\cap U_2 = \{0\}.
 		\]
 	\end{Definition}

 \subsection{Lemma (Komplementäre UVR)}
 	\begin{Lemma}[Komplementäre UVR]
 		Zu jedem UVR $ U\subset V $ existiert ein (in $ V $) komplementärer UVR.
 	\end{Lemma}

 	\paragraph{Beweis}
 		Sei $ U\subset V $ UVR eines $ K $-VR $ V $.
 		Seien
 		\begin{itemize}
 			\item $ B\subset U $ eine Basis von $ U $;
 			\item $ S\subset V $ lin. unahb., so dass $ C=B\cup S $ Basis von $ V $ ist (BES).
 		\end{itemize}
 		Definiere $ U':= [S] $. Dann ist $ U'\subset V $ UVR mit
 		\begin{enumerate}[(i)]
 			% NOTE: wir zeigen, dass U+U' Übermenge ist, wollen aber letztendlich Gleichheit zeigen ... die andere Inklusion ist aber trivial, da U und U' Teilmengen von V sind
 			\item $ U+U' \supset [C] = V $, da $ C\subset U\cup U' $ Erz. Syst. von $ V $ ist;
 			\item $ U\cap U' = [B]\cap [S] = \{0\}$, da $ C=B\cup S $ linear unabhängig ist.
 		\end{enumerate}

 	\paragraph{Bemerkung}
 		Zu einem UVR $ U\subset V $ gibt es normalerweise viele komplementäre UVR $ U'\subset V $.
 		Zu
 		\begin{equation*}
 			U:= \{v\in K^2\mid v_2 = 0\}
 		\end{equation*}

 		ist beispielsweise jeder UVR $ U' = [u']$ mit $u'_2\neq 0 $ komplementär in $ K^2 $.

 \subsection{Lemma \& Definition (direkte Summe)}
 	\begin{Definition}[Direkte Summe]
 		Sei $ U= \sum_{i\in I}U_i\subset V $ Summe einer Familie von UVR $ U_i\in V $; dann besitzt jeder Vektor $ u\in U $ eine eindeutige Zerlegung als Summe von $ u_i $, genau dann, wenn
 		\begin{equation*}
 			\forall i\in I: U_i\cap \sum_{j\in I\setminus \{i\}}U_j = \{0 \}.
 		\end{equation*}

 		In diesem Falle heißt die Summe \emph{direkt} und man schreibt
 		\[
 			U = \bigoplus_{i\in I} U_i.
 		\]
 	\end{Definition}

 	\paragraph{Bemerkung}
 		Eine Summe $ V = \sum_{i\in I} U_i $ ist genau dann direkt, wenn
 		\[
 			\forall i\in I: U_i, \sum_{j\in I\setminus\{i\}}U_j \subset V
 		\]
 		komplementäre UVR in $ V $ sind.

 	\paragraph{Beweis}
 		Zu zeigen ist die Eindeutigkeitsaussage. Sei also $ u \in \bigoplus_{i\in I}U_i $,
 		\begin{gather*}
 			u = \sum_{i\in I} u_i = \sum_{i\in I} u_i' \text{ mit } \forall i\in I: u_i,u'_i\in U_i;
 		\end{gather*}
 		dann gilt für jedes $ i\in I$:
 		\[
 			u'_i-u_i = \sum _{j\neq i}u_j-\sum_{j\neq i} u'_j = \underbrace{\sum_{j\neq i}u_j-u'_j}_{\in \sum_{j\neq i}U_j}\ \in U_i\cap \sum_{j\neq i} U_j = \{0\},
 		\]
 		da die Summe als direkt angenommen wurde; damit folgt $ \forall i \in I: u_i = u'_i $, d.h. die Zerlegung ist eindeutig.

 		Die Umkehrung ist trivial:
 		\begin{gather*}
 			\exists i\in I:U_i\cap \sum_{j\neq i} U_j \neq \{0\}\\
 			\Rightarrow \exists i\in I\ \exists u_i\in U_i\setminus\{0\}\ \exists (u_j)_{j\in I\setminus\{i\}}:
 			(\forall j\in I\setminus\{i\}:u_j \in U_j)\land u_i = \sum_{j\neq i} u_j,
 		\end{gather*}
 		d.h., die Zerlegung von $ u_i\in \sum_{i\in I}U_i $ ist nicht eindeutig.

 	\paragraph{Bemerkung}
 		Sind $ \dim V <\infty $ und $ \# I < \infty $ so gilt
 		\begin{equation*}
 			\forall i\in I: \dim U_i < \infty
 		\end{equation*}

 		und es gilt die \emph{Dimensionsformel für direkte Summen} (Beweis: Aufgabe 35):
 		\begin{equation*}
 			\dim \bigoplus_{i\in I}U_i = \sum_{i\in I} \dim U_i.
 		\end{equation*}

 		Ist insbesondere $ B=(b_1,...,b_n) $ eine Basis von $ V $, so gilt
 		\begin{equation*}
 			\dim V = \dim \bigoplus_{i=1}^n [b_i]=\sum_{i=1}^{n}1 = n.
 		\end{equation*}

 	\paragraph{Bemerkung}
 		Seien $ U,U'\subset V $ komplementäre UVR, also $ V = U \oplus U' $, dann werden durch
 		\begin{equation*}
 			v = u+u' \mapsto
 			\begin{cases}
 				p(v):=u    \\
 				p'(v):= u'
 			\end{cases}
 		\end{equation*}
 		Endomorphismen $ p,p'\in \operatorname{End}(V) $ (wohl-)definiert, da $ u,u' $ durch $ v $ eindeutig bestimmt sind (Linearität von $ p,p' $ ist klar).
 		Offenbar ist
 		\[
 			p(V) = U \text{ und } \ker p = U'
 		\]
 		und es gilt
 		\[
 			p^2 := p\circ p = p
 		\]
 		und analog für $ p' $; außerdem gilt ($ \circ $ für die Komposition weggelassen)
 		\[
 			p+p' = id_V \text{ und } p'p = 0 = pp'.
 		\]

 \subsection{Definition (Projektion)}
 	\begin{Definition}[Projektion]
 		$ p\in \End(V) $ heißt \emph{Projektion}, falls $ p^2 = p $ (d.h. falls $ p $ idempotent ist).
 	\end{Definition}

 \subsection{Satz (Projektionen)}
 	\begin{Satz}[Projektionen]
 		Sei $ p\in \End(V) $ Projektion, dann ist
 		\[
 			p'= \id_V-p \quad\text{Projektion mit}\quad  pp' = p'p = 0.
 		\]
 		Gilt andererseits $ p+p' = id_V $ und $ pp' = 0 $ für $ p,p' \in \End(V) $, so sind $ p,p' $ Projektionen mit
 		\[
 			V = p(V)\oplus p'(V) = \ker p \oplus \ker p'.
 		\]
 	\end{Satz}

%VO11-2015-11-12
 	\paragraph{Beweis}
 		Seien $p\in \operatorname{End}(V)$ Projektion und $p' := id_V -p$; dann gilt:
 		\begin{align*}
 			pp' & = p(\id_V-p)=p-p^{2} = 0     \\
 			p'p & = (\id_V-p)p = p - p^{2} = 0
 		\end{align*}
 		und
 		\[
 			p'^2 = p'(\id_V-p)=p'-p'p = p'
 		\]
 		d.h., $p'\in\End(V)$ ist Projektion.

 		Anderseits: Seien $p$, $p'\in \End(V)$, so dass $p+p' = \id_V \text{ und } pp' = 0$.

 		Dann gilt:
 		\[
 			p-p^2 = p(\id_V-p) = pp' = 0 \Leftrightarrow p^2=p
 		\]
 		d.h., $p\in\operatorname{End}(V)$ ist Projektion -- damit ist auch $p'$ Projektion (erster Teil) und $p' p = 0 $.

 		Weiters liefert
 		\begin{align*}
 			\forall v\in V: v & =\id_V(v) = p(v) + p'(v) \\
 			\Rightarrow V     & = p(V)+p'(V)
 		\end{align*}
 		und ist $w = p(v)= p'(v')$ für geeignete $v,v'\in V$ (d.h., $w \in p(V)\cap p'(V)$), so gilt
 		\[
 			w = p(v) = p^2(v) = p(p(v)) = p(w) = p(p'(v')) = pp'(v') = 0,
 		\]
 		also $p(V)\cap p'(V) = {0}$ und damit $V = p(V)\oplus p'(V)$.

 		Ferner gilt
 		\[
 			0 = p \circ p' \Rightarrow p'(V)\subset \ker p
 		\]
 		und ist $v\in \ker p$, so folgt
 		\[
 			v = p(v) + p'(v) = 0 + p'(v)\in p'(V) \Rightarrow \ker p \subset p'(V).
 		\]
 		Für $p'$ gilt das Gleiche und wir haben
 		\[
 			\ker p = p'(V) \text{ und } \ker p' = p(V)
 		\]
 		Damit folgt die letzte Behauptung $V = \ker p \oplus\ker p'$.

 	\paragraph{Bemerkung}
 		Im Beweis haben wir etwas mehr bewiesen als behauptet -- nämlich:
 		\[
 			\ker p = p'(V)\text{ und }\ker p' = p(V)
 		\]

 \subsection{Beispiel und Definition (Involution)}
 	\begin{Definition}[Definition]
 		Sei $s\in \operatorname{End}(V)$ eine \emph{Involution} d.h. $s^2 = \id_V$  und
 		\[
 			p_\pm := \frac{1}{2}(\id_V\pm s).
 		\]
 		Offenbar gilt dann
 		\[
 			p_{+} + p_{-} = \id_V \text{ und } p_+ p_{-} = \frac{1}{4}(\id_V +s)(\id_V -s) =  \frac{1}{4}(\id_V^2-s^2)=0
 		\]
 		also (Satz) sind $p_\pm\in\End(V)$ Projektionen mit komplementären Bildern bzw. Kernen.
 	\end{Definition}

 \subsection{Lemma und Definition (Produkt von VR)}
 	\begin{Definition}[Produkt von VR]
 		Ist $(V_i)_{i\in I}$ eine Familie von $ K $-VR $V_i$, so wird das (mengenthoretische) \emph{Produkt}:
 		\[
 			V:= \prod_{i\in I}V_i=\{(v_i)_{i\in I}\mid\forall i\in I:v_i\in V_i\}
 		\]
 		mit den komponentenweise definierten VR-Operationen zu einem $ K $-VR. Dies ist der \emph{Produktraum} der Familie	$(V_i)_{i\in I}$.
 	\end{Definition}

 	\paragraph{Beweis} ($V$ is $K$-VR) Aufgabe!

 	\paragraph{Bemerkung}
 		Ist $V = \prod_{i\in I} V_i$ ein Produktraum, so erhält man kanonische UVR
 		\[
 			U_i:=\{v=(v_i)_{i\in I}\in V\mid\forall j \neq i:v_j = 0\}\subset V,
 		\]
 		die isomorph zu den $V_i$ sind mittels der \emph{Faktor-Projektionen}\footnote{Achtung: Keine Projektion in unserem obrigen Sinn!}
 		\[
 			\pi_i:V\to V_i,\ (v_j)_{j\in I} \mapsto v_i,
 		\]
 		bzw. mittels der \emph{Faktor-Injektionen}
 		\[
 			\iota_i:V_i\to V,\ v_i \mapsto(v_j)_{j\in I}\text{, wobei }v_j :=
 			\begin{cases}
 				v_i & \text{ falls } j=i \\
 				0   & \text{ sonst.}
 			\end{cases}
 		\]
 		Ist dann $\# I < \infty$, so erhält man
 		\[
 			\prod_{i\in I} V_i\cong \bigoplus_{i\in I}U_i\quad (=: \bigoplus_{i\in I}V_i);
 		\]
 		Ist $\#I=\infty$, so ist diese Identifikation im Allgemeinen falsch!

 	\paragraph{Beispiel}
 		% NOTE: Im Skript von Prof. Jeromin steht  K^n = \bigoplus ..., in der VO wurde aber K^n \cong \bigoplus ... geschrieben, wir schreiben = da diese Aussage stärker ist, also \cong daraus folgt
 		Für einen Körper $ K $ liefert das $ n $-fache Produkt den Standardraum
 		\[
 			\prod_{i=1}^{n}K = \{(x_i)_{i = 1,...,n}\mid\forall i \in \{1,...,n\}: x_i \in K\} = K^n = \bigoplus_{i=1}^n\{(x_i)_{i = 1,...,n}\mid\forall j\neq i: x_j = 0\};
 		\]
 		für den Raum der $K$-wertigen Folgen ist jedoch
 		% NOTE: Im Skript von Prof. Jeromin steht \ncong statt \neq, in der VO wurde aber \neq geschrieben, wir schreiben \ncong, da diese Aussage stärker ist, also \neq daraus folgt
 		\[
 			\prod_{i\in \mathbb{N}}K=K^{\mathbb{N}}\ncong\bigoplus_{i\in \mathbb{N}}\{(x_i)_{i\in \mathbb{N}}\mid\forall j\neq i: x_j=0\}.
 		\]

 \subsection{Lemma und Definition (Nebenklassen)}
 	\begin{Definition}[Nebenklassen]
 		Sei $U\subset V$ UVR. Die Menge der Nebenklassen
 		\[
 			V/U := \{v+U\mid v\in V\},
 		\]
 		wobei
 		\[
 			v+U:=\{v+u\mid u\in U\}
 		\]
 		die \emph{Nebenklasse}\footnote{Bemerke: Nebenklassen sind im Allgemeinen keine UVR, z.B. müssen sie $0$ nicht enthalten!} zu $v\in V$ bezeichnet, wird mit den durch
 		\begin{align*}
 			+     & : V/U \times V/U \to V/U,\ ((v+U),(w+U))\mapsto (v+w)+U, \\
 			\cdot & : K\times V/U \to V/U,\ (x,(v+U))\mapsto (vx)+U,
 		\end{align*}
 		definierten Operationen ein Vektorraum: der \emph{Quotientenraum} $V/U$.
 	\end{Definition}

 	\paragraph{Beweis}
 		Zu zeigen: Wohldefiniertheit der Operationen und VR-Axiome.

 		Wohldefiniertheit der Skalarmultiplikation:

 		Ist $x\in K$ und sind $(v+U),(v'+U)\in V/U$ gleich, also $v+U = v'+U$, so gilt
 		\begin{align*}
 			v+U = v'+U & \Leftrightarrow v - v'\in U \\
 			           & \Rightarrow (v-v')x \in U   \\
 			           & \Rightarrow vx+U=v' x+U
 		\end{align*}
 		Das Resultat der Skalarmultiplikation hängt also nicht von dem Repräsentanten $v$ einer Nebenklasse $v+U$ ab, sondern nur von der Nebenklasse.

 		Die Wohldefiniertheit der Addition ist analog zu beweisen.

 		Die Vektorraumaxiome sind zu überprüfen.

%VO12-2015-11-17
 \subsection{Bemerkung \& Definition (Äquivalenzrelation)}
 	\begin{Definition}[Äquivalenzrelation]
 		Der Definition von $ V/U $ liegt ein allgemeineres Prinzip zugrunde:
 		\[
 			v\sim w :\Leftrightarrow (v+U)= (w+U) \Leftrightarrow w-v \in U
 		\]
 		definiert für jeden UVR $ U\subset V $ eines $ K $-VR $ V $ eine \emph{Äquivalenzrelation} auf $ V $, d.h.:
 		\begin{enumerate}[(i)]
 			\item $ \forall v\in V: v\sim v $ (\emph{Reflexivität})
 			\item $ \forall v,w\in V: v\sim w\Leftrightarrow w\sim v $ (\emph{Symmetrie});
 			\item $ \forall u,v,w\in V: u\sim v\land v\sim w\Rightarrow u\sim w $ (\emph{Transitivität}).
 		\end{enumerate}
 	\end{Definition}

 	z.z.: $ v\sim w:\Leftrightarrow w-v\in U $ definiert eine Äquivalenzrelation

 	\begin{itemize}
 		\item Reflexivität: Sei $ v\in V $ beliebig, dann gilt $ v-v=0\in U $, da $ U $ UVR.
 		\item Symmetrie: Seien $ v,w\in V $ beliebig, dann gilt
 		      \begin{align*}
 		      	v\sim w & \Leftrightarrow w-v\in U                          \\
 		      	        & \Leftrightarrow v-w\in U \Leftrightarrow w\sim v.
 		      \end{align*}
 		\item Transivitität: Seien $ u,v,w\in V $ beliebig; gilt nun
 		      $u\sim v$, d.h. $v-u\in U$ und $v\sim w$, d.h. $w-v\in U$, so gilt auch:
 		      \[
 		      	\underbrace{(w-v)}_{\in U}+\underbrace{(v-u)}_{\in U}= w-u\in U \Leftrightarrow u\sim w
 		      \]
 	\end{itemize}

 	Ist dann $ \sim $ eine Äquivalenzrelation auf einer Menge $ X $, so zerfällt $ X $ in Äquivalenzklassen
 	\[
 		X_x = \{y\in X\mid y\sim x\}
 	\]
 	d.h., $ X $ ist disjunkte Vereinigung der Äquivalenzklassen:
 	\[
 		X = \dot{\bigcup}_{x\in X}X_x \qquad\text{ und } \underbrace{X_x \cap X_y \neq \emptyset \Rightarrow X_x = X_y}_{\text{zwei Äquivalenzklassen sind entweder disjunkt oder gleich}}
 	\]

 	Insbesondere zerfällt also ein VR $ V $ in Äquivalenzklassen (Nebenklassen)
 	\[
 		v+U (= V_v)\subset V,
 	\]
 	wenn $ U $ ein UVR von $ V $ ist -- nach Lemma wird die Menge der Neben- bzw. Äquivalenzklassen dann wieder ein VR. Ähnlich wie den Quotientenvektorraum definiert man (z.B.) die \emph{Faktorgruppe} [Havlicek §.1.11.11].

 	\paragraph{Bemerkung}
 		Allgemeiner definiert man eine (binäre) \emph{Relation} $ (X,Y,\Gamma) $ zwischen Mengen $ X, Y $ durch den \emph{Graph der Relation}, eine Teilmenge
 		\[
 			\Gamma \subset X\times Y = \{(x,y)\mid x\in X \land y\in Y\}.
 		\]
 		Zum Beispiel ist eine Abbildung $ f:X\to Y $ eine Relation $ (X,Y,\Gamma_f) $, sodass
 		\[
 			\forall x\in X\ \exists ! y\in Y:(x,y)\in \Gamma_f.
 		\]
 		Ein anderes Beispiel ist die \emph{Ordnungsrelation} auf $ \mathbb{Z} $, definiert durch
 		\[
 			x\leq y :\Leftrightarrow \{(x,y)\in \mathbb{Z}^2\mid y-x\in \mathbb{N}\},
 		\]
 		Eine Ordnungsrelation ist reflexiv, transitiv und \emph{antisymmetrisch}, d.h.
 		\[
 			\forall x,y\in \mathbb{Z}: x\leq y\land y\leq x\Rightarrow x=y
 		\]
 		Die \emph{Teilmengenrelation}
 		\[
 			Y\subset\tilde{Y} \text{ für } Y,\tilde{Y}\in \mathcal{P}(X):= \{Y\mid Y\subset X\}
 		\]
 		liefert auch eine Ordnungsrelation auf der Potenzmenge $ \mathcal{P}(X) $ von $ X $, jedoch nur eine \emph{Halbordnung} -- im Gegensatz zur \emph{Totalordnung} auf $ \mathbb{Z} $, wo je zwei Elemente vergleichbar sind, d.h.,
 		\[
 			\forall x,y\in \mathbb{Z}: x\leq y\lor y\leq x.
 		\]
 		Auf $ \mathcal{P}(X) $ gilt dies im Allgemeinen nicht, denn z.B. in $ \mathcal{P}(\{0,1\}) $ sind $ \{0\},\{1\} $ nicht vergleichbar, denn
 		\[
 			\{0\}\not\subset\{1\}\land \{1\}\not\subset \{0\}.
 		\]

 \subsection{Lemma (Dimensionen von komplementären UVR)}
 	\begin{Lemma}[Dimensionen von komplementären UVR]
 		Ist $ U\subset V $ UVR und $ U' $ komplementärer UVR zu $ U $ in $ V $, so gilt
 		\[
 			U'\cong V/U
 		\]
 		vermöge
 		\[
 			U'\ni u' \overset{\phi}{\mapsto} u'+U\in V/U.
 		\]
 		Ist $ \dim V<\infty $, so gilt $ \dim U+\dim V/U = \dim V $.
 	\end{Lemma}

 	\paragraph{Beweis} z.z.: $ \phi $ ist Isomorphismus.

 		Dass $ \phi $ Homomorphismus ist, folgt aus der Definition der VR-Operationen auf $ V/U $.

 		Injektivität: Sei $ u'\in \ker \phi $, d.h.\footnote{Bemerke: "`$=$"' steht für die Gleichheit von Nebenklassen!}
 		\begin{align*}
 			\phi(u') & =u'+U=0+U\in V/U                   \\
 			         & \Rightarrow u'\in U'\cap U = \{0\} \\
 			         & \Rightarrow u'=0
 		\end{align*}

 		Surjektivität: Sei $ v+U\in V/U $ mit $ v\in V = U \oplus U' $; mit der Projektion
 		\begin{gather*}
 			p':V\to V,\ v=u+u' \mapsto p'(v) := u'
 			\intertext{ist }
 			v + U = u' + \underbrace{u}_{\in U} + U = u' + U = \phi(p'(v))
 		\end{gather*}
 		also $V/U = \phi(U')$.

 		Die Dimensionsformel folgt dann aus der für direkte Summen:
 		\[
 			\dim V/U = \dim U' \Rightarrow \dim V/U = \dim V-\dim U
 		\]

 	\paragraph{Beispiel}
 		Ist $ p\in \End(V) $ eine Projektion, so auch $ p':= \id_V-p $ und es gilt
 		\[
 			V= \ker p \oplus \ker p' = \ker p \oplus p(V),
 		\]
 		also gilt
 		\[
 			p(V)\cong V/\ker p.
 		\]

 \subsection{Homomorphiesatz für lineare Abbildungen: }
 	\begin{Satz}[Homomorphiesatz für lineare Abbildungen]
 		Für $ f\in \hom(V,W) $ ist $ f(V)\cong V/\ker f $ vermöge
 		\[
 			V/\ker f\ni v+\ker f \overset{\phi}{\mapsto} f(v)\in f(V).
 		\]
 	\end{Satz}

 	\paragraph{Beweis}
 		Wohldefiniertheit von $ \phi $:

 		Sind $ v+\ker f, v'+\ker f \in V/\ker f$ und $ v+\ker f = v'+\ker f $, so gilt
 		\begin{align*}
 			v'-v \in \ker f & \Rightarrow f(v')-f(v) = f(v'-v) = 0 \\
 			                & \Rightarrow f(v') = f(v)
 		\end{align*}
 		d.h. $ \phi $ ist wohldefiniert.

 		Linearität:
 		\begin{itemize}
 			\item Für $ x\in K $ und $ v+\ker f \in V/\ker f $ ist
 			      \begin{align*}
 			      	\phi( (v+\ker f)x) & = f(vx)                  \\
 			      	                   & = f(v)x                  \\
 			      	                   & = \phi (v+\ker f)\cdot x
 			      \end{align*}
 			\item Für $ v+\ker f, w+\ker f\in V/\ker f $ ist
 			      \begin{align*}
 			      	\phi ((v+\ker f)+(w+\ker f) ) & = \phi((v+w)+\ker f)           \\
 			      	                              & =f(v+w)                        \\
 			      	                              & = f(v)+f(w)                    \\
 			      	                              & =\phi(v+\ker f)+\phi(w+\ker f)
 			      \end{align*}
 		\end{itemize}
 		Injektivität:
 		Sei $ v+\ker f\in \ker \phi $, also
 		\[
 			\phi(v+\ker f)= f(v) = 0;
 		\]
 		dann folgt
 		\[
 			v\in \ker f \Rightarrow v+\ker f = \ker f = \underbrace{0}_{0+\ker f}\in V/\ker f.
 		\]

 		Surjektivität:
 		folgt direkt aus der Definition.


% % 2015-11-19 % %
\chapter{Affine Geometrie}
\section{Affine Räume}
\begin{tikzpicture}[scale=1.5,>=triangle 45]
	\draw[->,color=black] (-0.1,0) -- (10,0);
	\draw[->,color=black] (0,-0.1) -- (0.,4);
	
	\coordinate[label=left:$x$] (x) at (1,1);
	\coordinate[label=below:$\tau_v(x)$] (y) at (5,1.5);
	\coordinate[label=above:$\tau_w(x)$] (y') at (2,2.5);
	\coordinate (z) at (6,3);
	
	\draw [fill] (x) circle (.5pt);
	\draw [fill] (y') circle (.5pt);
	\draw [fill] (y) circle (.5pt);
	\draw [fill] (z) circle (.5pt);
	
	\draw [->] (x) to node[below left]{$ v $} (y);
	\draw [->] (x) --node[above left]{$ w $} (y');
	\draw [->] (y) --node[below right]{$ w $} (z);
	\draw [->] (y') --node[above right]{$ v $} (z);
	
	\draw (z) node[above right] {$\tau_w(\tau_v(x))=(\tau_w\circ\tau_v)(x)=\tau_{w+v}(x)$};
	\draw (z) node[below right] {$\tau_v(\tau_w(x))=(\tau_v\circ\tau_w)(x) = \tau_{v+w}(x)$};
\end{tikzpicture}

\subsection{Definition (Geometrie nach Klein)}
\begin{Definition}[Geometrie]
	Eine Geometrie besteht aus einer Menge $ A $ (z.B. Punktmenge) und einer darauf operierenden Gruppe $ (G,*) $, d.h.,
	es gibt eine Gruppenoperation
		\[ \rho: G\times A\to A,(g,a)\mapsto \rho_g(a),  \]
	wobei gilt
		\begin{enumerate}[(i)]
			\item $ \forall a\in A\forall g,h,\in G:(\rho_g\circ \rho_h)(a) = \rho_{g*h}(a) $
			\item $ \forall a\in A:\rho_e(a) = a $ für das neutrale Element $ e \in G $.
		\end{enumerate}
\end{Definition}

\subsection{Definition (Affiner Raum)}
\begin{Definition}[Affiner Raum]
	Sei $ K $ ein Körper. Ein affiner Raum (AR) $ (A,V,\tau) $ über $ K $ besteht aus einer Menge $ A $, einem $ K $-Vektorraum $ V $ und einer Gruppenoperation
		\[ \tau:V\times A\to A,(v,a)\mapsto \tau_v(a) \]
	von $ V $ (als additive Gruppe $ (V,+) $) auf $ A $, die einfach transitiv ist, d.h.
		\[ \forall a,b\in A\exists!v\in V:b=\tau_v(a) \]
\end{Definition}

\begin{figure}[H]\centering
\begin{tikzpicture}[scale=1.5,>=triangle 45]
	\draw[->,color=black] (-0.1,0) -- (5,0);
	\draw[->,color=black] (0,-0.1) -- (0.,2);
	
	\coordinate[label=left:$a$] (x) at (1,0.5);
	\coordinate[label=right:${b=\tau_v(a)}$] (y) at (3,1.5);

	\draw [fill] (x) circle (.5pt);
	\draw [fill] (y) circle (.5pt);
	
	\draw [->] (x) to node[below]{$ v $} (y);
	\draw (5,0.5) node[] {Der Verbindungsvektor ist eindeutig.};
	
\end{tikzpicture}
\end{figure}

	Weiters nennen wir
		\begin{itemize}
			\item Elemente von $ A $ Punkte,
			\item $ V $ den Richtungsvektorraum oder Tangentialraum von $ A $,
			\item $ v $ mit $ \tau_v(a)=b $ den Verbindungsvektor von $ a $ nach $ b $,
			\item $ \tau_v:A\to A, a\mapsto \tau_v(a) $ die Translation von $ v $
			\item und $ \dim V $ die Dimension des affinen Raums $ A $
		\end{itemize}
		
\paragraph{Bemerkung}
	Die Translationen eines AR $ A $ bilden eine abelsche Gruppe.
	
	Alternative Notation:
		\[ a+v:=\tau_v(a) \text{ und } b-a:= v\text{, falls } b=\tau_v(a) \]
	Mit dieser alternativen Schreibweise für die Operation von $ (V,+) $ auf $ A $, erscheinen die Bedingungen, dass $ V=(V,+) $ einfach transitiv auf $ A $ operiert, "`offensichtlich"'.
	
	Gruppenoperation:
		\begin{enumerate}[(i)]
			\item $ \forall a\in A\forall v,w,\in V: (a+v)+w = a+(v+w) $ ist kurz für $ \tau_w(\tau_v(a)) = \tau_{v+w}(a) $, entspricht also nicht der Assoziativität.
			\item $ \forall a\in A:a+0=a $ entspricht $ \tau_0(a) = a $
		\end{enumerate}
	Transitivität:
		\[ \forall a,b\in A\exists v\in V: b=a+v \]
	Nämlich: sind $ a,b\in A $ gegeben, so liefert $ v:=b-a $ (weil $ V $ einfach transitiv operiert) eindeutig den gesuchten Vektor.

\subsection{Beispiel \& Definition (affiner Standardraum)}
	\begin{Definition}[Affiner Standardraum]
		Jeder $ K $-VR liefert einen affinen Raum $ (V,V,\tau) $ mit der Operation
		\[ \tau: V\times V\to V, (v,a)\mapsto \tau_v(a):= a+v \]
	von $ V $ auf sich selbst -- die Unterscheidung zwischen Punkten und Vektoren wird dann etwas undurchsichtig.
	
	Der affine Standardraum $ (K^n,K^n,\tau) $ wird mit $ A^n $ bezeichnet.
	\end{Definition}
	
\subsection{Beispiel \& Definition (Ursprung)}
	\begin{Definition}[Ursprung]
		Sei $ (A,V,\tau) $ AR, für jede Wahl eines Ursprungs $ o\in A $ ist
		\[ \tau(o) :V\to A,v\mapsto \tau_v(o) \]
	eine Bijektion -- ein VR ist also ein "`AR mit Ursprung"'.
	\end{Definition}
	
\paragraph{Beispiel}
	Auf einem Zylinder
	\[ Z^2 := S^1\times \mathbb{R}:= (\mathbb{R}/2\pi\mathbb{Z})\times \mathbb{R} \]
	liefert die Operation
		\[ \tau:\mathbb{R}^2\times Z^2\to Z^2,(v,a)\mapsto a+v \]
	keinen affinen Raum, da diese Operation nicht einfach transitiv ist: zu je zwei Punkten gibt es unendlich viele "`Verbindungsvektoren"'.

%-----------------------------------------------------------------------
\tdplotsetmaincoords{340}{0} 
\begin{tikzpicture}[scale=2,tdplot_main_coords]
\def\cyradius{1.2}
\def\cyhight{2.5}
\def\xstart{-6}
\def\ystart{1.2}
%\def\xstart{-\cyradius}
%\def\ystart{-2}

\def\xpostext{\xstart-0.1}
\def\ypostext{\ystart-1}

\coordinate[color=blue,label={[xshift=-10, yshift=5]:$\mathbb{R}^2$}] (Nullpunkt) at (\xstart,\ystart,0);
%x,Z,y
\foreach \t in {0,5,...,180}{%
\draw[line width=1pt,color=red] ({\cyradius*cos(\t)},{0},{\cyradius*sin(\t)})--({\cyradius*cos(\t+5)},{0},{\cyradius*sin(\t+5)});
}

\foreach \t in {0,5,...,360}{%
\draw[line width=1pt,color=red] ({\cyradius*cos(\t)},{\cyhight},{\cyradius*sin(\t)})--({\cyradius*cos(\t+5)},{\cyhight},{\cyradius*sin(\t+5)});
}

\draw[line width=1pt,color=red] ({\cyradius},{0},{0})--({\cyradius},{\cyhight},{0});
\draw[line width=1pt,color=red] ({-\cyradius},{0},{0})--({-\cyradius},{\cyhight},{0});

\foreach \t in {0,-5,...,-180}{%
\draw[line width=1pt,color=blue] ({\cyradius*cos(\t)},{1-\t/360},{sin(\t)})--({\cyradius*cos(\t+2)},{1-\t/360},{sin(\t +2)});
}%for end

\foreach \t in {-180,-181,...,-260}{%
\ifthenelse{\t=-260}{\draw[-{>[scale=1,length=10,width=6]},line width=1pt,color=blue] ({\cyradius*cos(\t)},{1-\t/360},{sin(\t)})--({\cyradius*cos(\t-5)},{1-\t/360},{sin(\t -5)});}{%else Zweig
\ifthenelse{\t=-225}{\draw[line width=1pt,color=blue] ({\cyradius*cos(\t)},{1-\t/360},{sin(\t)})--({\cyradius*cos(\t-5)},{1-\t/360},{sin(\t -5)})node[below]{$v$};}{\draw[line width=1pt,color=blue] ({\cyradius*cos(\t)},{1-\t/360},{sin(\t)})--({\cyradius*cos(\t-5)},{1-\t/360},{sin(\t -5)});}
}
}%for end

\foreach \t in {85,84,...,0}{%
\draw[line width=1pt,color=blue] ({\cyradius*cos(\t)},{1-\t/360},{sin(\t)})--({\cyradius*cos(\t+5)},{1-\t/360},{sin(\t +5)});
}
%zwei Punkte x,y und u,z
\draw[fill,color=red] (0,0.75,1) circle [x=1cm,y=1cm,radius=0.05]node[below,label={[xshift=0, yshift=-34]$\text{Äquivalenzklasse } \mathbb{R}^2_{(x,y)}$}]{$(x,y)\sim (\tilde{x},\tilde{y})$};
\draw[fill,color=red] (0,1.75,1) circle [x=1cm,y=1cm,radius=0.05]node[ yshift=20,label={[xshift=0, yshift=22]$Z^2 =S^{1} \times \mathbb{R}^{1}$}]{$(u,z)\sim (\tilde{u},\tilde{z})$};
% Richtungsvektor
\draw[-{>[scale=1,length=10,width=6]},shorten >=6pt, shorten <=6pt,line width=1pt,color=blue] (0,0.75,1) -- (0,1.75,1) node[midway, right]{$v$} ;
%%blause Koordinatensystem
\draw[-{>[scale=1,length=10,width=8]},line width=1pt,color=blue] ({\xstart},{\ystart-1.5},{0})--({\xstart},{\ystart+1.5},{0});
\draw[-{>[scale=1,length=10,width=8]},line width=0.75pt,color=blue] ({\xstart-1},{\ystart},{0})--({\xstart+3*\cyradius},{\ystart},{0});
%rote Linien
\draw[line width=1pt,color=red] ({\xstart+2*\cyradius},{\ystart-0.6},{0})--({\xstart+2*\cyradius},{\ystart+1.5},{0});
\draw[line width=1pt,color=red] ({\xstart},{\ystart-1},{0})--({\xstart},{\ystart+1},{0});

%rote Punkte 2d x,y und x1,y1
\draw[fill,color=red] ({\xstart+0.5},{\ystart-0.5},0) circle [x=1cm,y=1cm,radius=0.05]node[above]{$(x,y)$};
\draw[fill,color=red] ({\xstart+0.5+2*\cyradius},{\ystart-0.5},0) circle [x=1cm,y=1cm,radius=0.05]node[above]{$(\tilde{x},\tilde{y})$};

%rote Punkte 2d u,z und u1,z1
\draw[fill,color=red] ({\xstart+0.5},{\ystart+0.5},0) circle [x=1cm,y=1cm,radius=0.05]node[above]{$(u,z)$};
\draw[fill,color=red] ({\xstart+0.5+2*\cyradius},{\ystart+0.5},0) circle [x=1cm,y=1cm,radius=0.05]node[above]{$(\tilde{u},\tilde{z})$};

%text node unterhalb der 3d graphik    
\node[text width=6cm, anchor=north west, text centered] at (\xpostext,\ypostext,0)
    {$(x,y)\sim (\tilde{x},\tilde{y})$ \\ $:\Leftrightarrow \begin{cases} \exists k \in \mathbb{Z}: &  \tilde{x} =x + 2k \pi, \\  & \tilde{y} = y \end{cases}$};
    
\draw[->,line width=1pt,color=red, dashed, shorten >=7pt, shorten <=7pt] (\xpostext+0.89,\ypostext-0.15,0) -- ({\xstart+0.5},{\ystart-0.5},0);
   
\draw[->,line width=1pt,color=red, dashed, shorten >=7pt, shorten <=7pt] (\xpostext+2.21,\ypostext-0.15,0) -- ({\xstart+0.5+2*\cyradius},{\ystart-0.5},0);
\end{tikzpicture}
%-----------------------------------------------------------------------

\subsection{Beispiel \& Definition (affiner Unterraum)}
		Ist $ U\subset V $ UVR eines $ K $-VR $ V $, so liefert jedes $ v\in V $ die Nebenklasse
		\[ A = v+U \]
	einen affinen Raum $ (A,U,\tau) $ mit 
		\[ \tau:U\times A\to A,(u,a)\mapsto \tau_u(a):= a+u; \]
	offensichtlich ist die Operation wohldefiniert (operiert auf der Nebenklasse) und einfach transitiv.
	
	Eine Nebenklasse $ A= v+U\subset V $ nennt man daher auch einen affinen Unterraum des VR $ V $.
	
	\begin{Definition}[Affiner Unterraum]
	$ A'\subset A $ ist affiner Unterraum (AUR) des affinen Raumes $ (A,V,\tau) $, falls
		\[ \exists a\in A\exists U\subset V \text{UVR}:A' = a+U = \{\tau_u(a)\mid u\in U\}.\]
	Ist $ \dim A' =1 $ oder $ \dim A' = 2 $, so heißt $ A' $ (affine) Gerade bzw. Ebene; ist $ \dim A' < \infty $ und $ \dim A' = \dim A-1 $, so heißt $ A' $ (affine) Hyperebene.
	\end{Definition}

\paragraph{Bemerkung}
	Jeder AUR ist selbst AR mit der "`geerbten"' (eingeschränkten) Operation.
	
\paragraph{Beispiel}
	Ist $ f\in \hom(V,W) $ und $ w\in f(V) $, so erhält man einen affinen Raum
		\[ (f^{-1}(\{w\}),\ker f,\tau) \text{ mit }\tau_u(a):= a+u.\]
	Ist $ f\in V^*\setminus \{0\} $ (und $ \dim V<\infty $), so wird $ f^{-1}(\{x\})\subset V $ für jedes $ x\in K (=f(V)) $ eine affine Hyperebene in $ (V,V,\tau) $ -- nach Rangsatz.
	
% % % 2015-11-24

\paragraph{Bemerkung}
	Ist $ A' = a+U\subset A $ AUR des AR $ (A,V,\tau) $, so gilt
		\[ \forall b\in A'\exists u\in U:b=\tau_u(a)\]
	und damit
	\begin{align*}
		b+U&=\{\tau_{u'}(b)\mid u'\in U\}\\
		&=\{(\tau_{u'}\circ \tau_u)(a)=\tau_{u'+u}(a)\mid u'\in U\}\\
		&= \{\tau_{u''}(a)\mid u'' \in U\} = a+U = A'
	\end{align*}
	
	\begin{figure}[H]\centering
	\begin{tikzpicture}[scale=1.5,>=triangle 45]
		\draw[->,color=black] (-0.1,0) -- (5,0);
		\draw[->,color=black] (0,-0.1) -- (0.,2);
		
	
		\coordinate[label=below:$U$] (x) at (1,1);
		\coordinate (y) at (3,0.5);
		\coordinate[label=right:${A'}$] (A) at (3,1.25);
		\coordinate[label=below:$b$] (b) at (2,1.5);
		\coordinate[label=above right:$ {c=a+u'} $] (c) at (1.5,1.625);
		\coordinate[label=above:$a$] (a) at (1,1.75);
		\draw [-] (a) to (A);
		\draw [-] (x) to (y);
		\draw [->] (a) to node[below]{$ u'$} (c);
		\draw[fill] (b) circle (0.5pt);
		\draw[fill] (a) circle (0.5pt);
		\draw[fill] (c) circle (0.5pt);
		\draw (4,0.5) node[] {$ A $};
		\draw (0.25,0.25) node[] {$ V $};
	\end{tikzpicture}
	\end{figure}
	
	Damit zeigt man: Ist $ (A'_i)_{i\in I} $ eine Familie AUR $ A'_i\subset A $ eines AR $ A $, so ist der Schnitt leer oder ein affiner Unterraum. Ist nämlich der Schnitt nicht leer, d.h.,
	\[ \exists a\in A\forall i\in I:a\in A'_i, \]
	so erhält man
	\begin{gather*}
		\forall i\in I:A'_i = a+U_i \text{ mit einem geeigneten UVR } U_i\subset V\\
		\Rightarrow \bigcap_{i\in I}A'_i = a+\bigcap_{i\in I}U_i \text{ und } U:= \bigcap_{i\in I}U_i\subset V \text{ ist UVR.}
	\end{gather*}
\subsection{Definition (affine Hülle)}
	\begin{Definition}[Affine Hülle]
		Die affine Hülle $ [S] $ einer Teilmenge eines affinen Raumes $ A $ ist der Schnitt aller $ S $ enthaltenden AUR $ A'\subset A $,
		\[ [S] = \bigcap_{S\subset A' \text{AUR}}A'. \]
	\end{Definition}
	
\paragraph{Bemerkung}
	Die affine Hülle einer Teilmenge $ S\subset A $ ist also der kleinste $ S $ enthaltende affine Unterraum von $ A $.
	
	Achtung: In einem $ K $-VR $ V $ (den kann man auch als AR auffassen, siehe Beispiel vorher) sind die lineare Hülle und die affine Hülle (in $ V $ aufgefasst als AR) im Allgemeinen verschieden:
		\[ [S]_{\text{lin}} = \bigcap_{S\subset U\text{ UVR}}U \neq \bigcap_{S\subset A \text{ AUR}}A = [S]_{\text{aff}} \]
% % % % Grafik affine Hülle
	\begin{figure}[H]\centering
		\begin{tikzpicture}[scale=1.5,>=triangle 45]
			\draw[->,color=black] (-0.1,0) -- (5,0);
			\draw[->,color=black] (0,-0.1) -- (0.,2);
			
		
			\coordinate[label=below right:${V =[\{\vec{a},\vec{b}\}]_\text{lin}}$] (v) at (0.5,0.5);
			\coordinate[label=above:$ a $] (a) at (1,1.5);
			\coordinate[label=below:$b$] (b) at (3,1);

			\draw [->] (v) to node[left]{$ \vec{a}$} (a);
			\draw [->] (v) to node[above]{$ \vec{b}$} (b);
			\draw [-] (0.5,1.63) to node[above right]{${[\{a,b\}]_\text{aff}}$ Gerade} (4,0.76);
			\draw[fill] (v) circle (0.5pt);
			\draw[fill] (a) circle (0.5pt);
			\draw[fill] (b) circle (0.5pt);
			\draw (4,0.25) node[] {$ A $};
		\end{tikzpicture}
		\end{figure}
\paragraph{Beispiel}
	Für $ S=\{a\}\subset V $ mit $ a\neq 0 $ gilt
		\[ [S]_{\text{lin}} = \{ax\in A = V\mid x\in K\} \neq \{a\} = [S]_{\text{aff}} \]
	allgemein gilt:
		\[ [S]_{\text{aff}}\subset [S\cup \{0\}]_{\text{aff}}=[S]_{\text{lin}} \]
	Beweis in Aufgabe 45.
	
\subsection{Lemma \& Definition (baryzentrischer Kalkül)}
	\begin{Definition}[Affinkombination/Baryzentrum]
		Seien $(a_i)_{i\in I}$ und $(x_i)_{i\in I}$ Familien in einem AR $ A $ über $ K $ bzw. in $ K $, wobei
		\[ \#\{i\in I\mid x_i\neq 0\}<\infty \text{ und } \sum_{i\in I}x_i=1; \]
		dann ist die mit einem beliebigen Ursprung $ o\in A $ definierte Affinkombination	
		\[ \sum_{i\in I}a_ix_i := o+\sum_{i\in I} (a_i-o)x_i \]
		wohldefiniert, d.h. unabhängig von der Wahl des Ursprungs $ o\in A $.
		Dann heißt 
		\[ s:= \sum_{i\in I} a_ix_i \]
		Schwerpunkt oder Baryzentrum der Punkte $ a_i $ mit Gewichten $ x_i $.
	\end{Definition}
%-------------------------Begin Grafik Affinkombination---------------------------------    
\begin{figure}[H]\centering
\tdplotsetmaincoords{0}{-27} %-27
\begin{tikzpicture}[scale=1,tdplot_main_coords]
 
\def\xstart{0}
\def\ystart{0}

\def\xstartdraw{(\xstart + 2)}
\def\ystartdraw{(\ystart + 1)}

\def\xlength{3.5}
\def\ylength{1.7}

%---------Begin Balken----------
\def\drehwinkel{-27}
\def\balkenbreite{0.4}
\def\balkenhoehe{(\ylength*2+2)}
\def\balkenlaenge{(\xlength*2+3)}

\node (VekV) at ({\xstart+0.5*cos(\drehwinkel)-\balkenbreite*sin(\drehwinkel)},{\ystart+0.5*sin(\drehwinkel)+\balkenbreite*cos(\drehwinkel)})[color=blue] {$V$};
\node (AffA) at ({\xstart+(\balkenlaenge-1)*cos(\drehwinkel)},{\ystart+(\balkenlaenge-1)*sin(\drehwinkel)+\balkenbreite*cos(\drehwinkel)})[color=red] {$A$};

\path[ shade, top color=white, bottom color=blue, opacity=.6] 
    ({\xstart},{\ystart},0)  -- ({\xstart - \balkenbreite * cos(\drehwinkel)- (-\balkenbreite+0)*sin(\drehwinkel)},{\ystart - \balkenbreite * sin(\drehwinkel)+ (-\balkenbreite+0)*cos(\drehwinkel)},0)  -- ({\xstart - \balkenbreite * cos(\drehwinkel)- (\balkenhoehe+0.5)*sin(\drehwinkel)},{\ystart - \balkenbreite * sin(\drehwinkel)+ (\balkenhoehe+0.5)*cos(\drehwinkel)},0) -- ({\xstart - 0 * cos(\drehwinkel)- (\balkenhoehe+0)*sin(\drehwinkel)},{\ystart - 0 * sin(\drehwinkel)+ (\balkenhoehe+0)*cos(\drehwinkel)},0) -- cycle;
        
\path[ shade, right color=white, left color=blue, opacity=.6] 
	({\xstart},{\ystart},0)  -- ({\xstart - \balkenbreite * cos(\drehwinkel)- (-\balkenbreite+0)*sin(\drehwinkel)},{\ystart - \balkenbreite * sin(\drehwinkel)+ (-\balkenbreite+0)*cos(\drehwinkel)},0) --
    ({\xstart + (\balkenlaenge+0.5) * cos(\drehwinkel)- (-\balkenbreite+0)*sin(\drehwinkel)},{\ystart + (\balkenlaenge+0.5) * sin(\drehwinkel)+ (-\balkenbreite+0)*cos(\drehwinkel)},0) --   
    ({\xstart + \balkenlaenge * cos(\drehwinkel)},{\ystart + \balkenlaenge * sin(\drehwinkel)},0)--
    cycle;       
%---------End Balken----------
%Punkte Definition
\node (pointo) at ({\xstartdraw},{\ystartdraw}) {};
\node (pointostrich) at ({\xstartdraw+2*\xlength},{\ystartdraw}) {};
\node (pointmiddle) at ({\xstartdraw+\xlength},{\ystartdraw}) {};
\node (pointa1) at ({\xstartdraw+\xlength},{\ystartdraw+\ylength}) {};
\node (pointa2) at ({\xstartdraw+\xlength},{\ystartdraw-\ylength}) {};

%Vektoren
\draw[-{>[scale=1,length=10,width=6]},shorten >=4pt, shorten <=4pt,line width=1pt,color=blue] (pointo) -- (pointostrich) node[xshift=5, yshift=-30]{$(a_{1}-o)+(a_{2}-o)$} ;
\draw[-{>[scale=1,length=10,width=6]},shorten >=4pt, shorten <=4pt,line width=1pt,color=blue] (pointo) -- (pointa1) node[midway, left]{$a_{1}-o$} ;
\draw[-{>[scale=1,length=10,width=6]},shorten >=4pt, shorten <=4pt,line width=1pt,color=blue] (pointo) -- (pointa2) node[midway, below]{$a_{2}-o$} ;
\draw[-{>[scale=1,length=10,width=6]},shorten >=4pt, shorten <=4pt,line width=1pt,color=blue] (pointo) -- (pointmiddle) node[xshift=10, yshift=-28]{$(a_{1}-o)\frac{1}{2}+(a_{2}-o)\frac{1}{2}$};
%Hilfslinien
%\draw[-,shorten >=3pt, shorten <=3pt,line width=0.3pt,color=blue] (pointa1) -- (pointostrich) ;
%\draw[-,shorten >=3pt, shorten <=3pt,line width=0.3pt,color=blue] (pointa2) -- (pointostrich) ;

%Punkte malen
\draw[fill,color=red] (pointo) circle [x=1cm,y=1cm,radius=0.08]node[ xshift=-10]{$o$};
\draw[fill,color=red] (pointostrich) circle [x=1cm,y=1cm,radius=0.08]node[xshift=10]{$o'$};
\draw[fill,color=red] (pointa1) circle [x=1cm,y=1cm,radius=0.08]node[ xshift=-10]{$a_{1}$};
\draw[fill,color=red] (pointa2) circle [x=1cm,y=1cm,radius=0.08]node[ yshift=-10]{$a_{2}$};
\draw[fill,color=red] (pointmiddle) circle [x=1cm,y=1cm,radius=0.08]node[xshift=-8, yshift=15]{$a_{1}\frac{1}{2}+a_{2}\frac{1}{2}$};
\end{tikzpicture}
\end{figure}
%-------------------------End Grafik Affinkombination-------------------------------------- 

\paragraph{Beispiel}
	Sind etwa $ K=\mathbb{R} $ und $ I = \{1,...,n\} $, so erhält man mit $ x_i = \frac{1}{n} $ für $ {i\in I} $ den üblichen geometrischen Schwerpunkt der (endlichen) Punktmenge,
		\[ s =\sum_{i=1}^{n}a_i\frac{1}{n}. \]
	Achtung: Die Ausdrücke
		\[ \frac{\sum_{i =1}^{n}a_i}{n}\text{ oder } \frac{1}{n}\sum_{i=1}^{n}a_i \]
	sind sinnlos, da nicht definiert.
\paragraph{Beweis}
	Zu zeigen: Sind $ o,o'\in A $, so gilt
	\[ o'+\sum_{i\in I} v'_ix_i = o+\sum_{i\in I} v_ix_i \text{, wobei }
		\begin{cases}
		v_i := a_i-o\\
		v'_i := a_i-o'
		\end{cases}\]
	Zunächst bemerken wir, dass mit $ w:= o'-o $ für $ {i\in I} $ gilt: $ v'_i+w=v_i $, denn:
	\begin{gather*}
		\tau_{v'_i+w}(o) = \tau_{v'_i}(\tau_w(o)) = \tau_{v'_i}(o')\\
		= a_i = \tau_{v_i}(o),
	\end{gather*}
	also
	\begin{gather*}
		o+\sum_{i\in I}v_ix_i = o+\sum_{i\in I} (w+v'_i)x_i = o+ \sum_{i\in I}wx_i + \sum_{i\in I}v'_ix_i\\
		= o+ w\cdot \sum_{i\in I}x_i + \sum_{i\in I}v'_i x_i = o+w + \sum_{i\in I}v'_ix_i = o' + \sum_{i\in I}v'_i x_i
	\end{gather*}
	
\subsection{Lemma (Affine Hülle und Affinkombination)}
	\begin{Lemma}[Affine Hülle und Affinkombination]
		Ist $ S\subset A $ Teilmenge eines AR $ A $, so ist ihre affine Hülle
		\[ [S] = \{\sum_{a\in S}ax_a\mid \#\{a\in S\mid x_a\neq 0\}<\infty \land \sum_{a\in S}x_a = 1 \}. \]
	\end{Lemma}
	
\paragraph{Beweis}
	Wir setzen $ S\neq \emptyset $ voraus und wählen $ o\in S $, dann ist
	\[ [S] = o+[\{a-o\mid a\in S\}] \]
	und die Behauptung folgt aus der entsprechenden für die lineare Hülle.
	
\paragraph{Beispiel}
	Die affine Hülle zweier Punkte $ a,b\in A, a\neq b $ ist die (affine) Gerade
	\[ [ab] := [\{a,b\}] = \{a(1-t)+bt\mid t\in K\}. \]
	Die affine Hülle von drei verschiedenen Punkten $ a,b,c\in A $ ist eine Gerade oder Ebene, je nachdem, ob $ \dim[\{a,b,c\}] $ gleich 1 oder 2 ist. Im zweiten Fall sagen wir: das Dreieck $ \{a,b,c\} $ sei nicht-degeneriert.
	
\subsection{Definition (allgemeine Lage)}
	\begin{Definition}[Allgemeine Lage]
		Eine Familie $ (a_i)_{i\in I} $ von Punkten $ a_i\in A $ eines AR $ A $ ist affin unabhängig, bzw. in allgemeiner Lage, falls
		\[ \forall i\in I:a_i\notin [\{a_j\mid j\in I\setminus \{i\}\}], \]
		und sonst affin abhängig; Punkte heißen kollinear bzw. koplanar, falls sie in einer Geraden oder einer Ebene liegen.
	\end{Definition}
	
\subsection{Lemma (Affine und lineare (Un-)Abhängigkeit)}
	\begin{Lemma}[Affine und lineare (Un-)Abhängigkeit]
		Eine Familie $ (a_i)_{i\in I} $ ist genau dann affin unabhängig, wenn für jedes $ i\in I $ die Familie $ (a_j-a_i)_{j\in I\setminus \{i\}} $ linear unabhängig ist.
	\end{Lemma}
	
\paragraph{Beweis}
	Die Familie $ (a_i)_{i\in I} $ ist genau dann affin abhängig, wenn
	\begin{gather*}
		\exists i\in I:a_i\in [\{a_j\mid j\in I\setminus \{i\}\}] \Leftrightarrow \exists i\in I\exists(x_j)_{j\in I\setminus \{i\}}:a_i=\sum_{j\neq i}a_jx_j\land 1=\sum_{j\neq i}x_j\\
		\Leftrightarrow \exists i\in I\exists (x_j)_(j\in I\setminus \{i\}):0=\sum_{j\neq i}(a_j-a_i)x_j \land 1=\sum_{j\neq i}x_j,
	\end{gather*}
	d.h., wenn die Familie $ (a_j-a_i)_{j\in I\setminus \{i\}} $ eine nicht-triviale Linearkombination von 0 erlaubt, also linear abhängig ist.
% % % % Nicht degeneriertes 3-Eck
	\begin{figure}[H]\centering
		\begin{tikzpicture}[scale=1.5,>=triangle 45]
			\draw[->,color=black] (-0.1,0) -- (5,0);
			\draw[->,color=black] (0,-0.1) -- (0.,2);
			
		
			\coordinate[label=left:$a  $] (a) at (0.5,0.5);
			\coordinate[label=above:$ c $] (c) at (1,1.5);
			\coordinate[label=below:$b$] (b) at (3,1);

			\draw [->] (a) to node[right]{$ c-a $} (c);
			\draw [->] (a) to node[below]{$ b-a$} (b);

			\draw[fill] (v) circle (0.5pt);
			\draw[fill] (a) circle (0.5pt);
			\draw[fill] (b) circle (0.5pt);
			\draw (5,1.5) node[] {$ \{a,b,c\}$ nicht deg. $\Delta$ gdw. Vektoren lin. unab.};
		\end{tikzpicture}
		\end{figure}	
\subsection{Lemma (Eindeutigkeit der Punktdarstellung)}
	\begin{Lemma}[Eindeutigkeit der Punktdarstellung]
		Eine Familie $ (a_i)_{i\in I} $ ist genau dann affin unabhängig, wenn jeder Punkt ihrer affinen Hülle eine eindeutige Affinkombination hat:
		\[ \forall a\in [\{a_i\mid i\in I\}]\exists!(x_i)_{i\in I}:
			\begin{cases}
			1 = \sum_{i\in I}x_i\\
			a = \sum_{i\in I}a_ix_i
			\end{cases}\]
	\end{Lemma}
	
\paragraph{Beweis}
	Hat jeder Punkt $ a\in [\{a_i\mid i\in I\}] $ eine eindeutige Affinkombination, so gilt insbesondere
		\[ \forall i\in I: a_i = a_i\cdot 1 \notin [\{a_j\mid j\neq i\}]. \]
	Hat andererseits der Punkt $ a $ zwei Affindarstellungen,
		\[ a = \sum_{i\in I} a_ix_i = \sum_{i\in I}a_iy_i, \]
	so folgt mit einem Ursprung $ o\in A $ und $ v_i = a_i-o $
		\[ a=o+\sum_{i\in I}v_ix_i=o+\sum_{i\in I}v_iy_i \Rightarrow 0 = \sum_{i\in I}v_i(y_i-x_i). \]
	Ist $ (a_i)_{i\in I} $ affin unabhängig, so ist $ (v_j)_{j\in I\setminus \{i\}} $ linear unabhängig für ein beliebiges $ i\in I $ und $ o:= a_i $. Es folgt:
	\begin{gather*}
        \forall j\in I\setminus \{i\}:x_j=y_j \Rightarrow x_i = 1-\sum_{j\neq i}x_j = 1-\sum_{j\neq i}y_j = y_i 
        \\ \text{ also } (x_{i})_{i \in I} = (y_{i})_{i \in I}
	\end{gather*}

% % % 2015-11-26 % % %
\subsection{Definition (Affines/baryzentrisches Bezugssystem)}
	\begin{Definition}[Affines/baryzentrisches Bezugssystem]
	Ein affines Bezugssystem $ (o;B) $ eines affinen Raumes $ (A,V,\tau) $ besteht aus einem Ursprung $ o\in A $ und einer Basis $ B $ von $ V $;
	ein baryzentrisches Bezugssystem $ (a_i)_{i\in I} $ ist eine affin unabhängige Familie von Punkten, sodass 
		\[ [\{a_i\mid {i\in I}\}] = A. \]
	\end{Definition}
	
\paragraph{Bemerkung}
	Ist $ n=\dim A $, so enthält
		\begin{itemize}
		\item ein affines Bezugssystem $ (o;b_1,...,b_n) $ einen Punkt und $ n $ Vektoren;
		\item ein baryzentrisches Bezugssystem $ (a_0,...,a_n) $ $ n+1 $ Punkte (und keinen Vektor).
		\end{itemize}
		
\paragraph{Beispiel}
	% % % % Baryzentrisches Bezugssystem
	\begin{figure}[H]\centering
		\begin{tikzpicture}[scale=1.5,>=triangle 45]
			\draw[->,color=black] (-0.1,0) -- (5,0);
			\draw[->,color=black] (0,-0.1) -- (0.,2);
				
			\coordinate[label=above:$ a_0 $] (a0) at (2.5,1.5);
			\coordinate[label=left:$a_1  $] (a1) at (1.5,0.5);
			\coordinate[label=above:$a_2$] (a2) at (4,1);
			
			\draw [-] (a0) to node[above right]{${ a_1 \notin [\{a_2,a_0\}] }$} (a2);
			\draw [-] (a1) to node[above left]{${ a_2 \notin [\{a_1,a_0\}] }$} (a0);
			\draw [-] (a2) to node[below right]{${ a_0 \notin [\{a_1,a_2\}] }$} (a1);

			\draw[fill] (a0) circle (0.5pt);
			\draw[fill] (a1) circle (0.5pt);
			\draw[fill] (a2) circle (0.5pt);
		\end{tikzpicture}
	\end{figure}
			
	Drei Punkte $ a_0,a_1,a_2 \in A $ sind genau dann in allgemeiner Lage, wenn sie die Ecken eines nicht degenerierten Dreiecks sind. Sie bilden dann ein baryzentrisches Bezugssystem der Ebene des Dreiecks. Andernfalls sind sie kollinear.
	
\subsection{Definition (Teilverhältnis)}
\begin{Definition}[Teilverhältnis]
	Sind $ a,b,c\in A $ kollinear, $ c\neq b $, so ist ihr Teilverhältnis
		\[ (ac:bc) = t :\Leftrightarrow a=bt+c(1-t). \]
\end{Definition}		
\paragraph{Bemerkung}
	Sind $ a,b\in A,a\neq b $ gegeben, so bestimmt das Teilverhältnis $ t $ die Lage eines Punktes $ c $ auf der Verbindungsgeraden $ [\{a,b\}] $ eindeutig:
		\begin{gather*}
		(ac:bc) = t \Leftrightarrow a=bt+c(1-t) = c+ (b-c)t + (c-c)(1-t)\text{ (nach Affinkomb. mit $ o = c $)}\\
		\Leftrightarrow a=\tau_{(b-c)t}(c)\Leftrightarrow a-c = (b-c)t\\
		\Leftrightarrow a-b \overset{*}{=} (a-c)+(c-b) = (b-c)t + (c-b) \overset{*}{=} (c-b)(1-t)\\
		\Leftrightarrow (a-b)\frac{1}{1-t} = c-b \Leftrightarrow \tau_{(a-b)\frac{1}{1-t}}(b) = c\\
		\Leftrightarrow c = b+(a-b)\frac{1}{1-t} + (b-b)(1-\frac{1}{1-t}) = a\frac{1}{1-t}+b(1-\frac{1}{1-t})
		=a\frac{1}{1-t}+b\frac{-t}{1-t}
		\end{gather*}
	Dabei erhält man $ c = a $ mit $ t = 0 $, wegen $ a\neq b $ muss $ t=1 $ ausgeschlossen werden und $ c = b $ wird durch kein Teilverhältnis realisiert. (* vgl. Beweis baryzentrischer Kalkül)
	
	Ist $ K=\mathbb{R} $, so ist $ t<0 $ genau dann, wenn der Punkt $ c $ "`zwischen"' $ a $ und $ b $ liegt, d.h. wenn
		\[ c\in \{a(1-s)+bs\mid s\in (0,1)\}. \]
	Man sagt daher auch: "`$ c $ teilt die Strecke $ \overline{ab} $ im Verhältnis $ (ac:bc) $."'
	
	Bei nicht geordneten Körpern ist diese Aussage sinnlos!
	
\paragraph{Bemerkung}
	Das Teilungsverhältnis $ t = (ac:bc) = -\frac{s}{1-s} $ für $ c=a(1-s)+bs $.

% % % Ende Abschnitt 2.1 % % %
%VO15-2015-11-26
\section{Affine Abbildungen \& Transformationen}
\subsection{Definition}
	\begin{Definition}[Affine Abbildung/Affinität]
		Eine Abbildung $ \alpha:A\to A' $ zwischen affinen Räumen $ A $ und $ A' $ (über dem gleichen Körper $ K $) heißt affin, falls sie
			\begin{enumerate}[(i)]
				\item \emph{geradentreu} ist, d.h. die Bilder kollinearer Punkte sind kollinear;
				\item \emph{teilverhältnistreu} ist, d.h. das Teilverhältnis kollinearer Punkte wird erhalten (solange die Punkte nicht alle zusammenfallen).
			\end{enumerate}
		Eine bijektive affine Abbildung $ \alpha:A\to A $ heißt Affinität oder affine Transformation.
	\end{Definition}
	
\paragraph{Bemerkung}
	Sei $ \alpha:A\to A' $ und $ a,b\in A $ sodass $ \alpha(a)\neq \alpha(b) $; insbesondere ist dann auch $ a\neq b $. Ist $ \alpha $ geradentreu, so gilt für jeden Punkt
		\[ c_s = a(1-s)+bs;\ s=(ca:ba), \]
	dass $ c_s\in [\{a,b\}] $, d.h.
		\[ \forall s\in K\exists t\in K:\alpha(c_s) = \alpha(a(1-s)+bs) = \alpha(a)(1-t)+\alpha(b)t \in [\{\alpha(a),\alpha(b)\}] \]
	Ist $ \alpha $ dann auch teilverhältnistreu, so folgt
		\[ \frac{-t}{1-t} = (\alpha(a)\alpha(c_s):\alpha(b)\alpha(c_s)) = (ac_s:bc_s) = \frac{-s}{1-s} \Rightarrow t = s. \]
	Insbesondere bildet $ \alpha $ die Gerade $ [ab] $ dann bijektiv auf die Gerade $ [\alpha(a),\alpha(b)] $ durch die Bildpunkte von $ a $ und $ b $ ab, und 
		\[ \forall s\in K:\alpha(a(1-s)+bs)=\alpha(a)(1-s)+\alpha(b)s. \]
	Enthält die Gerade durch $ a $ und $ b $, $ a\neq b $ keine Punkte, deren Bilder verschieden sind, so wird die Gerade auf einen einzigen Punkt abgebildet -- und die vorherige Gleichung gilt ebenfalls.
	
\paragraph{Beispiel}
	Die Translationen eines affinen Raumes sind Affinitäten, denn für
		\[ c_s = a(1-s)+bs = a + ws, \text{ mit } w:=b-a \]
	gilt, mit Translationsvektor $ v\in V $,
		\[ \tau_v(c_s) = \tau_v(a+ws) = \tau_v(\tau_{ws}(a)) = \tau_{v+ws}(a) = \tau_{ws+v}(a) = \tau_{ws}(\tau_v(a)) =  \tau_v(a) + ws, \]
	insbesondere gilt also
		\[ \tau_v(b) = \tau_v(a)+w \]
	und damit
		\[ \tau_v(c_s) = \tau_v(a)+ws = \tau_v(a)+(\tau_v(b)-\tau_v(a))s = \tau_v(a)(1-s)+\tau_v(b)s.\]
	Also sind $ \tau_v(a),\tau_v(b) $ und $ \tau_v(c_s) $ kollinear und erhalten das Teilverhältnis
		\[ (\tau_v(a)\tau_v(c_s):\tau_v(b)\tau_v(c_s)) = (ac_s:bc_s). \]
		
\subsection{Lemma}
	\begin{Lemma}[]
		$ \alpha:A\to A' $ ist genau dann affin, wenn für jede Affinkombination in $ A $ gilt:
			\[ \alpha(\sum_{i\in I}a_ix_i) = \sum_{i\in I} \alpha(a_i)x_i. \]
	\end{Lemma}
	
\paragraph{Beweis}
	Wir haben schon gesehen: $ \alpha:A\to A' $ ist affin genau dann, wenn
		\[ \forall a,b,\in A\forall s\in K:\alpha(a(1-s)+bs) = \alpha(a)(1-s)+\alpha(b)s \]
	Offenbar ist die vorherige Bemerkung ein Spezialfall des Lemmas. Es bleibt die andere Richtung zu zeigen. Wir benutzen vollständige Induktion über $k = \#\{{i\in I}\mid x_i\neq 0\}<\infty $.
	
\subparagraph{Induktionsanfang}
	Für $ k=1 $ trivial.

\subparagraph{Induktionsannahme}
	Für $ a_1,...,a_k\in A $ und $ x_1,...,x_k \in K^\times$ mit $ \sum_{i=1}^{k}x_i=1 $ gelte
		\[ \alpha(\sum_{i=1}^{k}a_ix_i) = \sum_{i=1}^{k}\alpha(a_i)x_i. \]
	
\subparagraph{Induktionsschluss}
	Seien $ a_1,...,a_{k+1} \in A$ und $ x_1,...,x_{k+1} \in K^\times$ Gewichte, sodass $ \sum_{i=1}^{k+1}x_i = 1 $, o.B.d.A. $ x_{k+1}\neq 1 $; dann gilt
		\[ \alpha(\sum_{i=1}^{k+1}a_ix_i) = \alpha((\sum_{i=1}^{k}a_i\frac{x_i}{1-x_{k+1}})(1-x_{k+1})+a_{k+1}x_{k+1}) \]
		\[ = \alpha(\sum_{i=1}^{k}a_i\frac{x_i}{1-x_{k+1}})(1-x_{k+1})+\alpha(a_k+1)x_{k+1} \]
		\[ = \sum_{i=1}^{k}\alpha(a_i)\frac{x_i}{1-x_{k+1}}(1-x_{k+1})+\alpha(a_{k+1})x_{k+1} \]
		\[ = \sum_{i=1}^{k+1}\alpha(a_i)x_i. \]
	Damit ist die Behauptung für affine Abbildungen $ \alpha $ bewiesen.

%VO16-2015-12-01
\paragraph{Bemerkung}
	Im Beweis wurde benutzt: für Affinkombinationen ist (falls $ x_j \neq 1$)
		\[ \sum_{i\in I}a_ix_i = \left(\sum_{i\neq j}a_i\frac{x_i}{1-x_j}\right)(1-x_j)+a_jx_j \]
%-------------------Begin affin Kombinationen Aufteilungsbeispiel----------------  
\begin{figure}[H]\centering
\tdplotsetmaincoords{0}{0} %-27
\begin{tikzpicture}[yscale=1,tdplot_main_coords]

\def\xstart{0} %x Koordinate der Startposition der Grafik
\def\ystart{0} %y Koordinate der Startposition der Grafik
\def\myscale{0.015} %ändert die Größe der Grafik (Skalierung der Grafik) 

\def\xstartdraw{(\xstart + 2.0)} %xKoordinate des Referenzstartpunktes (in dieser Zeichnung: a)
\def\ystartdraw{(\ystart + 1.0)}%yKoordinate des Referenzstartpunktes (in dieser Zeichnung: a)

\def\balkenhoehe{(4.3)}% Länge des vertikalen blauen Balkens
\def\balkenlaenge{(10)}% Länge des horizontalen blauen Balkens
\def\balkenbreite{0.4} %Balkenbreite

%---------Begin Balken----------
\def\drehwinkel{0}
\node (VekV) at ({\xstart+0.7*cos(\drehwinkel)-\balkenbreite*sin(\drehwinkel)},{\ystart+0.5*sin(\drehwinkel)+\balkenbreite*cos(\drehwinkel)})[color=blue] {$V$};
\node (AffA) at ({\xstart+(\balkenlaenge-1)*cos(\drehwinkel)},{\ystart+(\balkenlaenge-1)*sin(\drehwinkel)+\balkenbreite*cos(\drehwinkel)})[color=red] {$A$};

\path[ shade, top color=white, bottom color=blue, opacity=.6] 
    ({\xstart},{\ystart},0)  -- ({\xstart - \balkenbreite * cos(\drehwinkel)- (-\balkenbreite+0)*sin(\drehwinkel)},{\ystart - \balkenbreite * sin(\drehwinkel)+ (-\balkenbreite+0)*cos(\drehwinkel)},0)  -- ({\xstart - \balkenbreite * cos(\drehwinkel)- (\balkenhoehe+0.5)*sin(\drehwinkel)},{\ystart - \balkenbreite * sin(\drehwinkel)+ (\balkenhoehe+0.5)*cos(\drehwinkel)},0) -- ({\xstart - 0 * cos(\drehwinkel)- (\balkenhoehe+0)*sin(\drehwinkel)},{\ystart - 0 * sin(\drehwinkel)+ (\balkenhoehe+0)*cos(\drehwinkel)},0) -- cycle;
        
\path[ shade, right color=white, left color=blue, opacity=.6] 
	({\xstart},{\ystart},0)  -- ({\xstart - \balkenbreite * cos(\drehwinkel)- (-\balkenbreite+0)*sin(\drehwinkel)},{\ystart - \balkenbreite * sin(\drehwinkel)+ (-\balkenbreite+0)*cos(\drehwinkel)},0) --
    ({\xstart + (\balkenlaenge+0.5) * cos(\drehwinkel)- (-\balkenbreite+0)*sin(\drehwinkel)},{\ystart + (\balkenlaenge+0.5) * sin(\drehwinkel)+ (-\balkenbreite+0)*cos(\drehwinkel)},0) --   
    ({\xstart + \balkenlaenge * cos(\drehwinkel)},{\ystart + \balkenlaenge * sin(\drehwinkel)},0)--
    cycle;       
%---------End Balken----------

%Punkte Definition
\node (pointa0) at ({\xstartdraw},{\ystartdraw}) {};
\node (pointa2) at ({\xstartdraw+(-20 *\myscale)},{\ystartdraw+(200*\myscale)}) {};
\node (pointa1) at ({\xstartdraw+(270*\myscale)},{\ystartdraw+(103*\myscale)}) {};
\node (pointax) at ($(pointa0)!0.6!(pointa1)$) {};
\node (pointa) at ($(pointa2)!0.5!(pointax)$) {};

%Geraden
\draw[-,shorten >=-20pt, shorten <=-20pt,line width=0.2pt,color=red] (pointa0) -- (pointa1) ;
\draw[-,shorten >=-20pt, shorten <=-20pt,line width=0.2pt,color=red] (pointa2) --  (pointax) ;

%Punkte malen
\draw[fill,color=white] (pointa0) circle [x=1cm,y=1cm,radius=0.18];
\draw[fill,color=white] (pointa1) circle [x=1cm,y=1cm,radius=0.18];
\draw[fill,color=white] (pointa2) circle [x=1cm,y=1cm,radius=0.18];
\draw[fill,color=white] (pointax) circle [x=1cm,y=1cm,radius=0.18];
\draw[fill,color=white] (pointa) circle [x=1cm,y=1cm,radius=0.18];

\draw[fill,color=red] (pointa0) circle [x=1cm,y=1cm,radius=0.08]node[ xshift=1, yshift=-10]{$a_0$};
\draw[fill,color=red] (pointa1) circle [x=1cm,y=1cm,radius=0.08]node[ xshift=5, yshift=-10]{$a_1$};
\draw[fill,color=red] (pointa2) circle [x=1cm,y=1cm,radius=0.08]node[ xshift=-10]{$a_2$};
\draw[fill,color=red] ([xshift=-2pt,yshift=-2pt]pointax) rectangle ++(4pt,4pt) node[xshift=40, yshift=-20]{$\displaystyle a_s = \sum_{i=0}^{1}a_i \frac{x_i}{1-x_2}$};
\draw[fill,color=red] (pointa) circle [x=1cm,y=1cm,radius=0.08]node[above right,xshift=0, yshift=-10]{$\displaystyle a = \sum_{i=0}^{2}a_i x_i = a_s (1-x_2) + a_2 x_2 $};

\end{tikzpicture}
\end{figure}
%-------------------End affin Kombinationen Aufteilungsbeispiel----------------

        
\paragraph{Bemerkung}
	Mit der Verträglichkeit affiner Abbildungen mit Affinkombinationen folgt, dass die Inverse $ \alpha^{-1}:A'\to A $ einer bijektiven affinen Abbildung $ \alpha:A\to A' $ ebenfalls affin ist:
		\begin{gather*}
		\alpha\left(\sum_{i\in I} \alpha^{-1}(a_i')x_i\right)=\sum_{i\in I}(\alpha\circ\alpha')(a_i')x_i = \sum_{i\in I}a_i'x_i = \alpha\left(\alpha^{-1}(\sum_{i\in I}a_i'x_i)\right) \\
		\Rightarrow \sum_{i\in I} \alpha^{-1}(a_i')x_i =\alpha^{-1}(\sum_{i\in I}a_i'x_i),
		\end{gather*}
	da die Affinkombination $ \sum_{i\in I}a_i'x_i\in A' $ beliebig war, folgt damit die Behauptung. Insbesondere sind damit auch die Inversen von Affinitäten Affinitäten.
\paragraph{Bemerkung}
	Sind $ \alpha:A \to A' $ und $ \beta:A'\to A'' $ geraden- und teilverhältnistreu, so ist auch
		\[ \beta\circ\alpha:A\to A'' \]
	geraden- und teilverhältnistreu, d.h. die Komposition affiner Abbildungen ist affin. Insbesondere ist damit die Menge $ G $ aller affinen Transformationen eines affinen Raumes $ A $ abgeschlossen unter der Komposition
		\[ \circ: G\times G\to G; \]
	außerdem ist $ G $ abgeschlossen unter Inversenbildung. Damit folgt: $ G $ ist Untergruppe der Permutationsgruppe (der symmetrischen Gruppe) des affinen Raumes $ A $: Diese Gruppe bezeichnet man als \emph{affine Gruppe}.
\subsection{Definition}
	\begin{Definition}[Affine Geometrie]
	Die auf einem affinen Raum $ A $ operierende Gruppe $ G $ der Affinitäten von $ A $ bestimmt eine \emph{affine Geometrie}.
	\end{Definition}
	
\paragraph{Bemerkung}
	Die Verträglichkeit einer affinen Abbildung $ \alpha:A\to A' $ mit Affinkombinationen lässt sich auch mithilfe von Vektoren formulieren (unabhängig von der Wahl des Ursprungs $ o \in A$):
		\begin{gather*}
		v_i = a_i-o \Rightarrow \alpha\left(\sum_{i\in I}a_ix_i\right)=\alpha\left(o+\sum_{i\in I}v_ix_i\right)\\
		\sum_{i\in I}\alpha(a_i)x_i = \sum_{i\in I}\alpha(o+v_i)x_i\\
		\Rightarrow \alpha\left(o+\sum_{i\in I}v_ix_i\right)-\alpha(o) = \sum_{i\in I}\alpha(o+v_i)x_i-\alpha(o)\sum_{i\in I}x_i = \sum_{i\in I}(\alpha(o-v_i)-\alpha(o))x_i,
		\end{gather*}
	setzt man also
		\[ \lambda:V\to V',v\mapsto \lambda(v):= \alpha(o+v)-\alpha(o),  \]
	wobei $ V $ und $ V' $ die zu $ A $ bzw. $ A' $ gehörenden Richtungsvektorräume sind, so erhält man einen Homomorphismus $ \lambda\in \hom(V,V') $, da sie mit Linearkombinationen verträglich ist:
		\[ \lambda\left(\sum_{i\in I}v_ix_i\right)=\alpha\left(o+\sum_{i\in I}v_ix_i\right)-\alpha(o)= \sum_{i\in I}\left(\alpha(o+v_i)-\alpha(o)\right)x_i = \sum_{i\in I}\lambda(v_i)x_i\]
		
\subsection{Lemma \& Definition}
	\begin{Lemma}
	Seien $ A $ und $ A' $ AR mit RVR $ V $ bzw. $ V' $; dann ist eine Abbildung $ \alpha:A\to A' $ genau dann affin, wenn es $ \lambda\in\hom(V,V') $ gibt, sodass 
		\[ \forall a\in A\forall v\in V:\alpha(a+v)=\alpha(a)+\lambda(v). \]
	\end{Lemma}
	\begin{Definition}
	Wir nennen $ \lambda $ den \emph{linearen Anteil} einer affinen Abbildung $ \alpha $.
	\end{Definition}
\paragraph{Beweis}
	Es sind zwei Richtungen zu zeigen:
	
	$ \Rightarrow: $ Sei $ \alpha:A\to A' $ affin. Zu zeigen ist nun die Existenz eines geeigneten $ \lambda \in\hom(V,V') $. Nämlich: Wähle $ o\in A $ und definiere
		\[ \lambda:V\to V',v\mapsto \lambda(v):=\alpha(o+v)-\alpha(o). \]
	Wegen der Verträglichkeit von $ \alpha $ mit Affinkombinationen ist $ \lambda $ linear. Für $ a\in A,v\in V $ gilt dann mit $ w:=a-o $:
		\begin{gather*}
		\alpha(a+v) = \alpha(o+w+v) = \alpha(o)+\lambda(w+v) =\\ \alpha(o)+\lambda(w)+\lambda(v) = \alpha(o+w)+\lambda(v) = \alpha(a)+\lambda(v)
		\end{gather*}
	Insbesondere ist der lineare Anteil $ \lambda \in\hom(V,V')$ von $ \alpha $ wohldefiniert, d.h. unabhängig von der Wahl des Ursprungs.
	
	$ \Leftarrow: $ Für $ \alpha:A\to A' $ gilt mit einem $ \lambda\in\hom(V,V') $
		\[ \forall a\in A\forall v\in V:\alpha(a+v)=\alpha(a)+\lambda(v) \]
	Wegen der Verträglichkeit von $ \lambda $ mit Linearkombinationen ist $ \alpha $ verträglich mit Affinkombinationen (siehe oben) und damit affin.
\paragraph{Bemerkung}
	Jede affine Transformation setzt sich also zusammen aus einer Translation und und einem Automorphismus $ \lambda \in \Aut(V) $. Insbesondere: Ist $ \tau_w:A\to A' $ Translation eines affinen Raumes $ A $ über $ V $, so ist für $ a\in A $ und $ v\in V $
		\[ \tau_w(a+v) = (a+v)+w = a+(v+w) = a+(w+v) = (a+w)+v = \tau_w(a)+v = \tau_w(a)+\id_V(v), \]
		d.h. der lineare Anteil einer Translation ist trivial -- also die Identität auf $ V $.
\subsection{Definition}
	\begin{Definition}[Allgemeine lineare Gruppe]
	Die Automorphismen eines VR $ V $ bilden seine \emph{allgemeine lineare Gruppe}
		\[ \mathrm{Gl}(V):= \{\lambda\in \End(V)\mid \lambda \text{ invertierbar}\}. \]
	\end{Definition}
\subsection{Bemerkung \& Definition}
	\begin{Definition}[Parallele Geraden]
	Sind $ g_i = [a_ib_i]=a_i + [v] $ mit $ b_i = a_i + v $ für $ i = 1,2 $ zwei Geraden mit dem gleichen RVR $ [v] $, d.h. \emph{parallel}, so sind auch ihre Bilder unter einer affinen Transformation $ \alpha $ parallele Geraden,
		\[ \alpha(g_i) = \alpha(a_i) + [\lambda(v)] \text{ mit } \lambda\in \mathrm{Gl}(V). \]
	\end{Definition}
	%-------------------Begin parallele Geraden ----------------  
\begin{figure}[H]\centering
\tdplotsetmaincoords{0}{0} %-27
\begin{tikzpicture}[yscale=1,tdplot_main_coords]

\def\xstart{0} %x Koordinate der Startposition der Grafik
\def\ystart{0} %y Koordinate der Startposition der Grafik
\def\myscale{0.20} %ändert die Größe der Grafik (Skalierung der Grafik) 

\def\xstartdraw{(\xstart + 2.0)} %xKoordinate des Referenzstartpunktes (in dieser Zeichnung: a)
\def\ystartdraw{(\ystart + 1.5)}%yKoordinate des Referenzstartpunktes (in dieser Zeichnung: a)

\def\balkenhoehe{(5.3)}% Länge des vertikalen blauen Balkens
\def\balkenlaenge{(10)}% Länge des horizontalen blauen Balkens
\def\balkenbreite{0.4} %Balkenbreite

%---------Begin Balken----------
\def\drehwinkel{0}
\node (VekV) at ({\xstart+0.7*cos(\drehwinkel)-\balkenbreite*sin(\drehwinkel)},{\ystart+0.5*sin(\drehwinkel)+\balkenbreite*cos(\drehwinkel)})[color=blue] {$V$};
\node (AffA) at ({\xstart+(\balkenlaenge-1)*cos(\drehwinkel)},{\ystart+(\balkenlaenge-1)*sin(\drehwinkel)+\balkenbreite*cos(\drehwinkel)})[color=red] {$A$};

\path[ shade, top color=white, bottom color=blue, opacity=.6] 
    ({\xstart},{\ystart},0)  -- ({\xstart - \balkenbreite * cos(\drehwinkel)- (-\balkenbreite+0)*sin(\drehwinkel)},{\ystart - \balkenbreite * sin(\drehwinkel)+ (-\balkenbreite+0)*cos(\drehwinkel)},0)  -- ({\xstart - \balkenbreite * cos(\drehwinkel)- (\balkenhoehe+0.5)*sin(\drehwinkel)},{\ystart - \balkenbreite * sin(\drehwinkel)+ (\balkenhoehe+0.5)*cos(\drehwinkel)},0) -- ({\xstart - 0 * cos(\drehwinkel)- (\balkenhoehe+0)*sin(\drehwinkel)},{\ystart - 0 * sin(\drehwinkel)+ (\balkenhoehe+0)*cos(\drehwinkel)},0) -- cycle;
        
\path[ shade, right color=white, left color=blue, opacity=.6] 
	({\xstart},{\ystart},0)  -- ({\xstart - \balkenbreite * cos(\drehwinkel)- (-\balkenbreite+0)*sin(\drehwinkel)},{\ystart - \balkenbreite * sin(\drehwinkel)+ (-\balkenbreite+0)*cos(\drehwinkel)},0) --
    ({\xstart + (\balkenlaenge+0.5) * cos(\drehwinkel)- (-\balkenbreite+0)*sin(\drehwinkel)},{\ystart + (\balkenlaenge+0.5) * sin(\drehwinkel)+ (-\balkenbreite+0)*cos(\drehwinkel)},0) --   
    ({\xstart + \balkenlaenge * cos(\drehwinkel)},{\ystart + \balkenlaenge * sin(\drehwinkel)},0)--
    cycle;       
%---------End Balken----------
\def\lightoffset{0.2*\myscale} %offeset der Vektoren

%Punkte Definition
\node (pointa1) at ({\xstartdraw},{\ystartdraw}) {};
\node (pointa2) at ({\xstartdraw+(-3 *\myscale)},{\ystartdraw+(6*\myscale)}) {};
\node (pointb1) at ($(pointa1) + (7*\myscale,2.0*\myscale) $) {};
\node (pointb2) at ($(pointa2) + (7*\myscale,2.0*\myscale) $) {};

\node (pointaa1) at ($(pointa1) + (18*\myscale,-2*\myscale) $) {};
\node (pointaa2) at ($(pointa1) + (9*\myscale,11*\myscale) $) {};
\node (pointab1) at ($(pointaa1) + (4*\myscale,8*\myscale) $) {};
\node (pointab2) at ($(pointaa2) + (4*\myscale,8*\myscale) $) {};

%Geraden
\draw[-,shorten >=-60pt, shorten <=-50pt,line width=0.2pt,color=red] (pointa1) -- (pointb1) ;
\draw[-,shorten >=-80pt, shorten <=-30pt,line width=0.2pt,color=red] (pointa2) -- (pointb2) ;
\draw[-,shorten >=-90pt, shorten <=-30pt,line width=0.2pt,color=red] (pointaa1) -- (pointab1) ;
\draw[-,shorten >=-20pt, shorten <=-110pt,line width=0.2pt,color=red] (pointaa2) -- (pointab2) ;

\node [color=red] (pointlabelg1) at ($(pointa1)+2.2*(pointb1)-2.2*(pointa1)$) [below, xshift=0, yshift=0] {$g_1$} ;
\node [color=red] (pointlabelg2) at ($(pointa2)+2.7*(pointb2)-2.7*(pointa2)$) [below, xshift=0, yshift=0] {$g_2$} ;

\node [color=red] (pointlabelag1) at ($(pointaa1)+2.2*(pointab1)-2.2*(pointaa1)$) [right, xshift=0, yshift=0] {$\alpha(g_1)$} ;
\node [color=red] (pointlabelag2) at ($(pointaa2)-2.1*(pointab2)+2.1*(pointaa2)$) [right, xshift=0, yshift=0] {$\alpha(g_2)$} ;

%Abbildung alpha
\draw [-{>[scale=1,length=10,width=6]},shorten >=8pt, shorten <=8pt,line width=0.4pt,color=blue!70!red!50] (pointa1) to [bend right=19] (pointaa1);
\draw [-{>[scale=1,length=10,width=6]},shorten >=8pt, shorten <=8pt,line width=0.4pt,color=blue!70!red!50] (pointa2) to [bend right=-25] (pointaa2);
\node [color=blue!70!red!50] (pointlabel1) at ($(pointa1)!0.5!(pointaa1)$) [below, xshift=0, yshift=2] {$\alpha$} ;
\node [color=blue!70!red!50] (pointlabel2) at ($(pointa2)!0.5!(pointaa2)$) [above, xshift=-10, yshift=10]{$\alpha$} ;

%Punkte malen
\draw[fill,color=white] (pointa1) circle [x=1cm,y=1cm,radius=0.18];
\draw[fill,color=white] (pointb1) circle [x=1cm,y=1cm,radius=0.18];
\draw[fill,color=white] (pointa2) circle [x=1cm,y=1cm,radius=0.18];
\draw[fill,color=white] (pointb2) circle [x=1cm,y=1cm,radius=0.18];
\draw[fill,color=white] (pointaa1) circle [x=1cm,y=1cm,radius=0.18];
\draw[fill,color=white] (pointaa2) circle [x=1cm,y=1cm,radius=0.18];
\draw[fill,color=white] (pointab1) circle [x=1cm,y=1cm,radius=0.18];
\draw[fill,color=white] (pointab2) circle [x=1cm,y=1cm,radius=0.18];


\draw[fill,color=red] (pointa1) circle [x=1cm,y=1cm,radius=0.08]node[above, xshift=0, yshift=0]{$a_1$};
\draw[fill,color=red] (pointb1) circle [x=1cm,y=1cm,radius=0.08]node[above, xshift=0, yshift=0]{$b_1$};
\draw[fill,color=red] (pointa2) circle [x=1cm,y=1cm,radius=0.08]node[below, xshift=5, yshift=0]{$a_2$};
\draw[fill,color=red] (pointb2) circle [x=1cm,y=1cm,radius=0.08]node[below, xshift=5, yshift=0]{$b_2$};
\draw[fill,color=red] (pointaa1) circle [x=1cm,y=1cm,radius=0.08]node[right, xshift=0, yshift=0]{$\alpha(a_1)$};
\draw[fill,color=red] (pointab1) circle [x=1cm,y=1cm,radius=0.08]node[right, xshift=0, yshift=0]{$\alpha(b_1)$};
\draw[fill,color=red] (pointaa2) circle [x=1cm,y=1cm,radius=0.08]node[right, xshift=2, yshift=5]{$\alpha(a_2)$};
\draw[fill,color=red] (pointab2) circle [x=1cm,y=1cm,radius=0.08]node[right, xshift=2, yshift=5]{$\alpha(b_2)$};

%Richtungsvektoren
\draw [-{>[scale=1,length=10,width=6]},shorten >=5pt, shorten <=5pt,line width=0.4pt,color=blue] ($(pointa1)+(\lightoffset,\lightoffset)$) to  ($(pointb1)+(\lightoffset,\lightoffset)$);
\node (pointa1b1v) at ($(pointa1)!0.5!(pointb1)$) [above,color=blue]{$v$};

\draw [-{>[scale=1,length=10,width=6]},shorten >=5pt, shorten <=5pt,line width=0.4pt,color=blue] ($(pointa2)+(\lightoffset,\lightoffset)$) to  ($(pointb2)+(\lightoffset,\lightoffset)$);
\node (pointa1b1v) at ($(pointa2)!0.5!(pointb2)$) [below,color=blue]{$v$};

\draw [-{>[scale=1,length=10,width=6]},shorten >=5pt, shorten <=5pt,line width=0.4pt,color=blue] ($(pointaa2)+(\lightoffset,\lightoffset)$) to  ($(pointab2)+(\lightoffset,\lightoffset)$);
\node (pointa1b1v) at ($(pointaa2)!0.5!(pointab2)$) [left,color=blue]{$\lambda(v)$};

\draw [-{>[scale=1,length=10,width=6]},shorten >=5pt, shorten <=5pt,line width=0.4pt,color=blue] ($(pointaa1)+(\lightoffset,\lightoffset)$) to  ($(pointab1)+(\lightoffset,\lightoffset)$);
\node (pointa1b1v) at ($(pointaa1)!0.5!(pointab1)$) [right,color=blue]{$\lambda(v)$};

\end{tikzpicture}
\end{figure}
%-------------------End parallele Geraden ----------------
	
\subsection{Beispiel \& Definition}
	\begin{Definition}[Streckung]
	Sei $ (A,V,\tau) $ ein AR über $ K $ und $ \lambda\in\End(V) $ eine \emph{Homothetie},\hfill
		$ \lambda= \id_V\cdot c \text{ für ein }c\in K. $
		
	Ist die zugehörige affine Abbildung\hfill
		$ \alpha:A\to A,o+v\mapsto \alpha(o+v):= o+v\cdot c $
		
	eine affine Transformation, d.h.,\hfill
		$ \lambda \in \mathrm{Gl}(V)\Leftrightarrow c\in K^\times, $
	
	so nennt man $ \alpha $ eine \emph{Streckung} mit \emph{Zentrum} $ o\in A $. Ist $ c\neq 1 $, d.h. $ \alpha \neq \id_A $, so gilt 
		\[ \alpha(a) = a\Leftrightarrow a = o. \]
	also hat die Abbildung $ \alpha $ genau einen \emph{Fixpunkt} $ a = o $.
	\end{Definition}
\subsection{Beispiel \& Definition}
	\begin{Definition}[Parallelprojektion]
	Sind $ p\in \End(V) $ eine Projektion ($ p^2 = p $) und $ o\in A $, so liefert
		\[ \pi:A\to A,o+v\mapsto \pi(o+v):= o+p(v) \]
	eine \emph{Parallelprojektion} von $ A $ auf dem affinen Unterraum $ o+p(V) $. Ist $ p\neq \id_V $, so ist $ p\notin \mathrm{Gl}(V) $ und also $ \pi $ keine affine Transformation (sondern eine nicht bijektive affine Abbildung), so hat $ \pi $ nicht-triviale \emph{Fasern} 
		\[ \pi^{-1}(\{a'\})\subset A \text{ für } a'\in \pi(A), \]
	wobei $ \dim\pi^{-1}(\{a'\}) = \dfkt p \geq 1 $.
	\end{Definition}
\subsection{Beispiel \& Definition}
	\begin{Definition}[Scherung]
	Seien $ \omega\in V^* $ und $ w\in \ker \omega $, sei $ o\in A $; die \emph{Scherung}
		\[ \sigma:A\to A, o+v\mapsto \sigma(o+v):= o+v+w\omega(v) \]
	ist dann eine affine Transformation, denn\hfill
		$ \lambda = \id_V + w\cdot \omega \in \mathrm{Gl}(V) $
		
	mit\hfill
		$ \lambda^{-1} = \id_v - w\cdot \omega. $
	
	Ist $ w\cdot\omega\in\End(V)\setminus\{o\} $, so hat $ \sigma $ Fixpunktmenge $ \text{Fix}_\sigma = o+\ker\omega $ und jeder Punkt und sein Bild liegen auf einer zu $ o+[w] $ parallelen Geraden:
		\[ \forall a\in A\setminus \text{Fix}_\sigma : [a,\sigma(a)] \parallel o+[w] \]
	\end{Definition}

%VO17-2015-12-03
\subsection{Korollar (Fortsetzungssatz für affine Abbildungen)}
	\begin{Korollar}[Fortsetzungssatz für affine Abbildungen]
		Eine affine Abbildung $ \alpha:A\to A' $ ist durch die (beliebige) Angabe der Bilder $ a_i' = \alpha(a_i) $ der Punkte eines baryzentrischen Bezugssystems $ (a_i)_{i\in I} $ von $ A $ eindeutig bestimmt.
	\end{Korollar}
	Beweis ist analog dem des Fortsetzungssatzes für lineare Abbildungen: Mit der Verträglichkeit der gesuchten affinen Abbildung mit Affinkombinationen muss gelten:
		\[ \alpha\left(\sum_{i\in I}a_ix_i\right)=\sum_{i\in I}\alpha(a_i)x_i \text{ für } a =\sum_{i\in I}a_ix_i \text{ mit } \sum_{i\in I}x_i = 1 \]
	Eindeutigkeit folgt, da jeder Punkt $ a\in A $ eine Affindarstellung $ a = \sum_{i\in I}a_ix_i $ besitzt. Existenz von $ \alpha $ folgt aus der Eindeutigkeit der Affindarstellung jedes Punktes $ a\in A $ im baryzentrischen Bezugssystem $ (a_i)_{i\in I} $.
	
	\paragraph{Beispiel}
    
%-------------------Begin Fortsetzungssatz für affine Abbildungen----------------  
\begin{figure}[H]\centering
\tdplotsetmaincoords{0}{0} %-27
\begin{tikzpicture}[yscale=1,tdplot_main_coords]

\def\xstart{0} %x Koordinate der Startposition der Grafik
\def\ystart{0} %y Koordinate der Startposition der Grafik
\def\myscale{0.022} %ändert die Größe der Grafik (Skalierung der Grafik) 

\def\xstartdraw{(\xstart + 1.5)} %xKoordinate des Referenzstartpunktes (in dieser Zeichnung: a)
\def\ystartdraw{(\ystart + 2.0)}%yKoordinate des Referenzstartpunktes (in dieser Zeichnung: a)

\def\balkenhoehe{(4.3)}% Länge des vertikalen blauen Balkens
\def\balkenlaenge{(10)}% Länge des horizontalen blauen Balkens
\def\balkenbreite{0.4} %Balkenbreite

%---------Begin Balken----------
\def\drehwinkel{0}
\node (VekV) at ({\xstart+1*cos(\drehwinkel)-\balkenbreite*sin(\drehwinkel)},{\ystart+0.5*sin(\drehwinkel)+\balkenbreite*cos(\drehwinkel)})[color=blue] {$V=K^2$};
\node (AffA) at ({\xstart+(\balkenlaenge-1)*cos(\drehwinkel)},{\ystart+(\balkenlaenge-1)*sin(\drehwinkel)+\balkenbreite*cos(\drehwinkel)})[color=red] {$A$};

\path[ shade, top color=white, bottom color=blue, opacity=.6] 
    ({\xstart},{\ystart},0)  -- ({\xstart - \balkenbreite * cos(\drehwinkel)- (-\balkenbreite+0)*sin(\drehwinkel)},{\ystart - \balkenbreite * sin(\drehwinkel)+ (-\balkenbreite+0)*cos(\drehwinkel)},0)  -- ({\xstart - \balkenbreite * cos(\drehwinkel)- (\balkenhoehe+0.5)*sin(\drehwinkel)},{\ystart - \balkenbreite * sin(\drehwinkel)+ (\balkenhoehe+0.5)*cos(\drehwinkel)},0) -- ({\xstart - 0 * cos(\drehwinkel)- (\balkenhoehe+0)*sin(\drehwinkel)},{\ystart - 0 * sin(\drehwinkel)+ (\balkenhoehe+0)*cos(\drehwinkel)},0) -- cycle;
        
\path[ shade, right color=white, left color=blue, opacity=.6] 
	({\xstart},{\ystart},0)  -- ({\xstart - \balkenbreite * cos(\drehwinkel)- (-\balkenbreite+0)*sin(\drehwinkel)},{\ystart - \balkenbreite * sin(\drehwinkel)+ (-\balkenbreite+0)*cos(\drehwinkel)},0) --
    ({\xstart + (\balkenlaenge+0.5) * cos(\drehwinkel)- (-\balkenbreite+0)*sin(\drehwinkel)},{\ystart + (\balkenlaenge+0.5) * sin(\drehwinkel)+ (-\balkenbreite+0)*cos(\drehwinkel)},0) --   
    ({\xstart + \balkenlaenge * cos(\drehwinkel)},{\ystart + \balkenlaenge * sin(\drehwinkel)},0)--
    cycle;       
%---------End Balken----------

\def\xdistanz{20} %Abstand zwischen den beiden Dreiecken

%Punkte Definition
\node (pointa) at ({\xstartdraw},{\ystartdraw}) {};
\node (pointc) at ({\xstartdraw+(10 *\myscale)},{\ystartdraw+(95*\myscale)}) {};
\node (pointb) at ({\xstartdraw+(90*\myscale)},{\ystartdraw-(30*\myscale)}) {};

\node (pointas) at ({\xstartdraw+((245+\xdistanz) *\myscale)},{\ystartdraw+(75*\myscale)}) {};
\node (pointbs) at ({\xstartdraw+((275+\xdistanz)*\myscale)},{\ystartdraw-(5*\myscale)}) {};
\node (pointcs) at ({\xstartdraw+((205+\xdistanz) *\myscale)},{\ystartdraw-(65*\myscale)}) {};

\node [color=blue!70!red!50] (pointlabel) at ($(pointc)!0.5!(pointas)$) {$\exists ! \alpha$} ;

%Geraden
\draw[-,shorten >=-20pt, shorten <=-20pt,line width=0.2pt,color=red] (pointa) -- (pointc) ;
\draw[-,shorten >=-20pt, shorten <=-20pt,line width=0.2pt,color=red] (pointa) --  (pointb) ;
\draw[-,shorten >=-20pt, shorten <=-20pt,line width=0.2pt,color=red] (pointc) -- (pointb) ;

\draw[-,shorten >=-20pt, shorten <=-20pt,line width=0.2pt,color=red] (pointas) -- (pointcs) ;
\draw[-,shorten >=-20pt, shorten <=-20pt,line width=0.2pt,color=red] (pointas) --  (pointbs) ;
\draw[-,shorten >=-20pt, shorten <=-20pt,line width=0.2pt,color=red] (pointcs) -- (pointbs) ;

%Abbildung alpha
\draw [-{>[scale=1,length=10,width=6]},shorten >=7pt, shorten <=7pt,line width=0.4pt,color=blue!70!red!50] (pointa) to [bend right=15] (pointas);
\draw [-{>[scale=1,length=10,width=6]},shorten >=7pt, shorten <=7pt,line width=0.4pt,color=blue!70!red!50] (pointb) to [bend right=-25] (pointbs);
\draw [-{>[scale=1,length=10,width=6]},shorten >=7pt, shorten <=7pt,line width=0.4pt,color=blue!70!red!50] (pointc) to [bend right=-25] (pointcs);

\draw [-{>[scale=1,length=10,width=6]},shorten >=7pt, shorten <=7pt,line width=0.4pt,color=blue!70!red!50] ($(pointc)!0.3!(pointas)$)  to [bend right=-25] ($(pointc)!0.7!(pointas)$) ;

%Punkte malen
\draw[fill,color=white] (pointa) circle [x=1cm,y=1cm,radius=0.18];
\draw[fill,color=white] (pointb) circle [x=1cm,y=1cm,radius=0.18];
\draw[fill,color=white] (pointc) circle [x=1cm,y=1cm,radius=0.18];

\draw[fill,color=red] (pointa) circle [x=1cm,y=1cm,radius=0.08]node[ xshift=-10, yshift=-10]{$a$};
\draw[fill,color=red] (pointb) circle [x=1cm,y=1cm,radius=0.08]node[ yshift=-10]{$b$};
\draw[fill,color=red] (pointc) circle [x=1cm,y=1cm,radius=0.08]node[ xshift=-10]{$c$};

\draw[fill,color=white] (pointas) circle [x=1cm,y=1cm,radius=0.18];
\draw[fill,color=white] (pointbs) circle [x=1cm,y=1cm,radius=0.18];
\draw[fill,color=white] (pointcs) circle [x=1cm,y=1cm,radius=0.18];

\draw[fill,color=red] (pointas) circle [x=1cm,y=1cm,radius=0.08]node[ xshift=10,yshift=5 ]{$a'$};
\draw[fill,color=red] (pointbs) circle [x=1cm,y=1cm,radius=0.08]node[ xshift=10,yshift=-2]{$b'$};
\draw[fill,color=red] (pointcs) circle [x=1cm,y=1cm,radius=0.08]node[ xshift=-10]{$c'$};

\end{tikzpicture}
\end{figure}
%-------------------End Fortsetzungssatz für affine Abbildungen----------------  

	
		Gegeben sind die Ecken eines nicht-degenerierten Dreiecks $ a,b,c\in A^2 := (A,K^2,\tau) $ und drei Punkte $ a',b',c'\in A^2 $; es existiert genau eine affine Abbildung $ \alpha:A^2\to A^2 $ mit $ (a,b,c)\mapsto (a',b',c') $. Dieses $ \alpha $ ist genau dann eine affine Transformation von $ A^2 $, wenn das Bilddreieck $ \{a',b',c'\} $ nicht-degeneriert ist, d.h. $ (a',b',c') $ ein baryzentrisches Bezugssystem ist (dann bekommt man die Inverse mittels Fortsetzungssatz durch $ (a',b',c') \overset{\alpha^{-1}}{\mapsto} (a,b,c) $).

%VO17-2015-12-03
\section{Dreiecke in der Affinen Geometrie}
\subsection{Beispiel \& Definition}
	\begin{Definition}[Mittelpunkt]
	Der (geometrische) Schwerpunkt zweier Punkte $ a,b\in A $ eines affinen Raumes über dem Körper $ K $ ist ihr Mittelpunkt
		\[ s_{a,b} = a\cdot \frac{1}{2}+b\cdot \frac{1}{2}. \]
	\end{Definition}
	
	Dies ist sinnlos, falls $ \Char K = 2 $ ist, was wir also ausschließen müssen.
	Ist etwa $ A $ AR über $ K=\mathbb{Z}_2 $, so enthält jede Gerade genau zwei Punkte,
		\[ \forall a,b\in A: [ab] = 
			\begin{cases}
				\{a,b\},& \text{falls }a\neq b\\
				\{a\},& \text{falls } a=b
			\end{cases} \]
	Für den Rest des Kapitels wird $ \Char K \neq 0 $ ausgeschlossen.
\paragraph{Bemerkung}
	Ist $ K $ ein geordneter Körper, e.g. $ K=\mathbb{R} $, so kann man die \emph{Strecke}
		\[ \overline{ab}:= \{a(1-s)+bs\mid 0\leq s\leq 1\} \]
	zwischen zwei Punkten $ a,b\in A $ definieren. Jeder Punkt $ c\in \overline{ab} $ auf der Strecke liegt \emph{zwischen} ihren \emph{Endpunkten} $ a $ und $ b $; $ s_{ab} $ ist dann auch Mittelpunkt der Strecke $ \overline{ab} $.
	
	Offenbar ist das sinnlos, wenn der Körper $ K $ nicht angeordnet ist.
	
	
\subsection{Schwerpunktsatz}
	\begin{Satz}[Schwerpunktsatz]
	Der Schwerpunkt eines nicht-degenerierten Dreieck $ \{a,b,c\}\subset A $ ist der Schnittpunkt der Seitenhalbierenden, die er im Verhältnis $ -\frac{1}{2} $ teilt.
	\end{Satz}
% % % Grafik Schwerpunktsatz
	\begin{figure}[H]\centering
	\definecolor{uququq}{rgb}{0.25,0.25,0.25}
	\definecolor{zzttqq}{rgb}{0.6,0.2,0}
	\definecolor{qqqqff}{rgb}{0,0,1}
		\begin{tikzpicture}[line cap=round,line join=round,>=triangle 45,scale=1.5] %,x=1.0cm,y=1.0cm]
		\clip(0.71,0.52) rectangle (5.68,4.35);
		\coordinate (a) at (1.36,2.12);
		\coordinate (b) at (3,4);
		\coordinate (c) at (4.56,0.82);
		%\fill[color=zzttqq,fill=zzttqq,fill opacity=0.1] (a) -- (b) -- (c) -- cycle;
		\draw [color=zzttqq] (a)-- (b);
		\draw [color=zzttqq] (b)-- (c);
		\draw [color=zzttqq] (c)-- (a);
		
		% Halbierungspunkte:
		\coordinate (Sab) at (2.18,3.06);
		\coordinate (Sbc) at (3.78,2.41);
		\coordinate (Sac) at (2.96,1.47);
		\coordinate (S) at (2.97,2.31);
		
		\fill [color=qqqqff] (a) circle (1.5pt);
		\draw[color=qqqqff] (a) node[left] {$a$};
		\fill [color=qqqqff] (b) circle (1.5pt);
		\draw[color=qqqqff] (b) node[above] {$b$};
		\fill [color=qqqqff] (c) circle (1.5pt);
		\draw[color=qqqqff] (c) node[right] {$c$};
		\fill [color=uququq] (Sab) circle (1.5pt);
		\draw[color=uququq] (Sab) node[left] {$S_{ab}$};
		\fill [color=uququq] (Sbc) circle (1.5pt);
		\draw[color=uququq] (Sbc) node[right] {$S_{bc}$};
		\fill [color=uququq] (Sac) circle (1.5pt);
		\draw[color=uququq] (Sac) node[below] {$S_{ac}$};
		\fill [color=uququq] (S) circle (1.5pt);
		\draw[color=uququq] (S) node[below left] {$S$};
		\draw [dash pattern=on 2pt off 2pt] (Sbc)-- (a);
		\draw [dash pattern=on 2pt off 2pt] (Sac)-- (b);
		\draw [dash pattern=on 2pt off 2pt] (Sab)-- (c);
		\end{tikzpicture}
	\end{figure}
\paragraph{Beweis}
	Der (geometrische) Schwerpunkt des Dreiecks $ \{a,b,c\}\subset A $ ist
		\[ s = a\cdot \frac{1}{3}+ b\cdot \frac{1}{3}+ c\cdot \frac{1}{3} = (a\cdot \frac{1}{2}+b\frac{1}{2})\frac{2}{3}+c\cdot \frac{1}{3} = s_{ab}\cdot\frac{2}{3}+c\cdot \frac{1}{3} \in [s_{ab}c];\]
	weiters gilt
			\[ (s_{ab}s:cs)= -\frac{\frac{1}{3}}{1-\frac{1}{3}} = -\frac{1}{2}, \]
	$ s $ teilt die Strecke $ \overline{s_{ab}c} $ im Verhältnis $ -\frac{1}{2} $. Aus Symmetriegründen gelten diese Resultate genauso für die anderen Seitenhalbierenden.
\paragraph{Bemerkung}
	Andere bekannte Schnittsätze im Dreieck machen in der affinen Geometrie keinen Sinn. Sätze wie der Höhensatz oder über den Umkreismittelpunkt können gar nicht erst formuliert werden: in der affinen Geometrie kennt man weder Längen- noch Winkelmessung.
	
	Dem gegenüber sind die Sätze von Menelaos und Ceva "`affine Sätze"', d.h. sie können rein affin formuliert werden und beschreiben unter affinen Transformationen \emph{invariante} Sachverhalte.
\subsection{Bemerkung \& Definition}
	Sind $ \alpha:A\to A' $ und $ \beta:A'\to A'' $ affine Abbildungen und bezeichnen $ \lambda:V\to V' $ bzw. $ \mu:V'\to V'' $ ihre linearen Anteile,
		\[ \forall a\in A\forall v\in V:\alpha(a+v) = \alpha(a)+\lambda(v)\text{ und }\forall a'\in A'\forall v'\in V':\beta(a'+v') = \beta(a')+\mu(v'), \]
	so gilt für ihre Komposition
		\[ (\beta\circ\alpha)(a+v) = \beta(\alpha(a)+\lambda(v)) = \beta(\alpha(a))+\mu(\lambda(v)) = (\beta\circ\alpha)(a)+(\mu\circ\lambda)(v), \]
	d.h. der lineare Anteil einer Komposition von affinen Abbildungen ist die Komposition der linearen Anteile.
	
	\begin{Definition}[Dilatationsgruppe]
	Da eine affine Transformation, deren linearer Anteil Vielfaches der Identität ist, eine Translation oder eine Streckung ist, bilden die Translationen und Streckungen eines affinen Raumes eine Gruppe, die \emph{Dilatationsgruppe}.
	\end{Definition}
	
\subsection{Satz von Menelaos}
	\begin{Satz}[Satz von Menelaos]
		Seien $ \{a,b,c\}\subset A $ ein nicht-degeneriertes Dreieck und $ g\subset A $ eine Gerade, die die drei Seiten des Dreiecks außerhalb der Ecken des Dreiecks schneidet;
			\[ a'\in g\cap [bc],b'\in g\cap [ca] \text{ und }c'\in g\cap [ab] \]
		bezeichne die Schnittpunkte. Dann gilt:
			\[ (ac':bc')(ba':ca')(cb':ab') = 1 \]
		Umgekehrt garantiert die TV-Bedingung, dass drei Punkte $ a'\in [bc],b'\in [ca] $ und $ c'\in [ab] $ auf den Seiten des Dreiecks kollinear sind.
	\end{Satz}
	
	\begin{figure}[H]\centering
	\definecolor{zzttqq}{rgb}{0.6,0.2,0}
	\definecolor{qqqqff}{rgb}{0,0,1}
	\begin{tikzpicture}[line cap=round,line join=round,>=triangle 45,x=1.0cm,y=1.0cm]
	\clip(1.05,-0.85) rectangle (9.76,4.01);
	%\fill[color=zzttqq,fill=zzttqq,fill opacity=0.1] (2.1,0.26) -- (6.42,0.81) -- (4.5,3) -- cycle;
	\draw [color=zzttqq] (2.1,0.26)-- (6.42,0.81);
	\draw [color=zzttqq] (6.42,0.81)-- (4.5,3);
	\draw [color=zzttqq] (4.5,3)-- (2.1,0.26);
	\draw [domain=1.05:9.76] plot(\x,{(-0.02--0.55*\x)/4.32});
	\draw [domain=1.05:9.76] plot(\x,{(-13.58--1.45*\x)/-3.33});

	\fill [color=qqqqff] (2.1,0.26) circle (1.5pt);
	\draw[color=qqqqff] (2.14,0.33) node[below] {$A$};
	\fill [color=qqqqff] (6.42,0.81) circle (1.5pt);
	\draw[color=qqqqff] (6.46,0.8) node[below] {$B$};
	\fill [color=qqqqff] (4.5,3) circle (1.5pt);
	\draw[color=qqqqff] (4.54,3.07) node[above] {$C$};
	\fill (3.94,2.36) circle (1.5pt);
	\draw (3.99,2.43) node[above] {$b'$};
	\fill (7.28,0.92) circle (1.5pt);
	\draw (7.33,0.99) node[above] {$c'$};
	\draw[color=black] (1.33,3.19) node {$g$};
	\fill (5.74,1.59) circle (1.5pt);
	\draw (5.78,1.66) node[above right] {$a'$};
	\end{tikzpicture}
	\end{figure}
	
\paragraph{Beweis}
	Betrachte Streckung $ \gamma $ mit Zentrum $ c' $ und Faktor $ s_{ab}\in K^\times $,
		\[ \gamma:A\to A, c'+ v\mapsto \gamma(c'+v) := c'+vs_{ab}; \]
	insbesondere ist für $ s_{ab}=\frac{1}{(ac':bc')} $
		\[ \gamma(a)=c'+(a-c')\frac{1}{(ac':bc')}=c'+(b-c') = b. \]
	Definiert man Streckungen $ \alpha $ und $ \beta $ entsprechend, mit Zentren $ a' $ bzw. $ b' $ und Faktoren $ s_{bc} = \frac{1}{(ba':ca')} $ bzw. $ s_{ca}=\frac{1}{cb':ab'} $, so liefert die Komposition eine affine Transformation
		\[ \delta:= \beta\circ\alpha\circ\gamma:A\to A:,a+v \mapsto \delta(a+v) := a+vs_{ab}s_{bc}s_{ca}, \]
	da
		\[ a\overset{\gamma}{\mapsto}b\overset{\alpha}{\mapsto}c\overset{\beta}{\mapsto}a. \]
	Damit gilt
		\[ (ac':bc')(ba':ca')(cb':ab') = 1 \Leftrightarrow \delta = \id_A \]
	Wegen $ a\notin [c'a'] $ ist andererseits
		\[ \delta = \id_A \Leftrightarrow [c'a'] = \delta([c'a']) = \beta([c'a']), \]
	da $ \gamma([c'a']) = [c'a'] $ und $ \alpha([c'a']) = [c'a'], $ was die letzte Gleichung liefert, damit ist
		\[ \delta = \id_A \Leftrightarrow [c'a']=\beta([c'a'])\Leftrightarrow b'\in [c'a'], \]
	da $ \beta $ Streckung mit Zentrum $ b' $ ist. Damit ist die Behauptung bewiesen.
\subsection{Satz von Ceva}
	\begin{Satz}[Satz von Ceva]
		Seien $ \{a,b,c\}\subset A $ ein nicht-degeneriertes Dreieck und
			\[ a'\in [bc]\setminus \{b,c\},b'\in [ac]\setminus \{a,c\},c'\in [ab]\setminus \{a,b\}. \]
		Schneiden sich die drei Transversalen $ [aa'],[bb'] $ und $ [cc'] $ in einem Punkt, so gilt
			\[ (ac':bc')(ba':ca')(cb':ab')=-1. \]
	\end{Satz}
	\begin{figure}[H]\centering
	\definecolor{zzttqq}{rgb}{0.6,0.2,0}
	\definecolor{qqqqff}{rgb}{0,0,1}
	\begin{tikzpicture}[line cap=round,line join=round,>=triangle 45,x=1.0cm,y=1.0cm]
	\clip(0.25,0.33) rectangle (9.57,6.25);
	%\fill[color=zzttqq,fill=zzttqq,fill opacity=0.1] (1.5,1.52) -- (6.8,1.78) -- (4.08,5.58) -- cycle;
	\draw [color=zzttqq] (1.5,1.52)-- (6.8,1.78);
	\draw [color=zzttqq] (6.8,1.78)-- (4.08,5.58);
	\draw [color=zzttqq] (4.08,5.58)-- (1.5,1.52);
	\draw [domain=0.25:9.57] plot(\x,{(--15.39-1.1*\x)/4.43});
	\draw [domain=0.25:9.57] plot(\x,{(-13.46--3.96*\x)/0.48});
	\draw [domain=0.25:9.57] plot(\x,{(--1.82--1.03*\x)/2.21});
	\fill [color=qqqqff] (1.5,1.52) circle (1.5pt);
	\draw[color=qqqqff] (1.5,1.64) node[left] {$a$};
	\fill [color=qqqqff] (6.8,1.78) circle (1.5pt);
	\draw[color=qqqqff] (6.88,1.9) node[right] {$b$};
	\fill [color=qqqqff] (4.08,5.58) circle (1.5pt);
	\draw[color=qqqqff] (4.14,5.7) node[right] {$c$};
	\fill (2.37,2.88) circle (1.5pt);
	\draw (2.45,3.01) node[above left] {$b'$};
	\fill (3.6,1.62) circle (1.5pt);
	\draw (3.68,1.75) node[below right] {$c'$};
	\fill (5.62,3.43) circle (1.5pt);
	\draw (5.7,3.56) node[above] {$a'$};
	\end{tikzpicture}
	\end{figure}
	Beweis in Aufgabe 59.
\paragraph{Bemerkung}
	Für die Seitenmitten gilt der Satz (Schwerpunktsatz).


%VO18-2015-12-10
\chapter{Buchhaltung}
Dieses Kapitel zeigt eine Art "`Tabellenkalkül"' -- eine effiziente Rechenmethode in der linearen Algebra.

Vorteil: Selbst durch einen Trottel (e.g. einen Computer) ausführbar.

Nachteil: Selbst durch einen Trottel ausführbar.

\paragraph{Generalvoraussetzung} Alle VR haben in diesem Kapitel endliche Dimension.
\section{Matrizen}
 \paragraph{Idee}
 	Ein Homomorphismus $ f\in \hom(V,W) $ wird (nach Fortsetzungssatz) durch die Bilder $ f(b_j) $ der Vektoren einer Basis $ (b_j)_{j\in J} $ eindeutig festgelegt; ist $ (c_i)_{i\in I} $ eine Basis von $ W $, so hat jedes dieser $ f(b_j) $ eine eindeutige Basisdarstellung.
 	\[
 		\forall {j\in J}\exists! (x_i)_{i\in I}:f(b_j) = \sum_{i\in I}c_ix_{ij}
 	\]
 	Sind $ n=\dim V $ und $ m=\dim W $ endlich, so kann man also $ f $ mithilfe der Basen $ (b_j)_{j\in J} $ von $ V $ und $ (c_i)_{i\in I} $ von $ W $ komplett durch die Tabelle der Koeffizienten beschreiben:

 	\begin{figure}[H]\centering
 		$
 		\begin{array}{c|cccccc}
 			f      & f(b_1) & f(b_2) & \dots & f(b_j) & \dots & f(b_n) \\\hline
 			c_1    & x_{11} & x_{12} & \dots & x_{1j} & \dots & x_{1n} \\
 			c_2    & x_{21} & x_{22} & \dots & x_{2j} & \dots & x_{2n} \\
 			\vdots & \vdots & \vdots &       & \vdots &       & \vdots \\
 			c_i    & x_{i1} & x_{i2} & \dots & x_{ij} & \dots & x_{in} \\
 			\vdots & \vdots & \vdots &       & \vdots &       & \vdots \\
 			c_m    & x_{m1} & x_{m2} & \dots & x_{mj} & \dots & x_{mn}
 		\end{array}
 		$
 	\end{figure}

 	Dabei spielt es prinzipiell keine Rolle, ob die Bilder $ f(b_j) $ der Basisvektoren in den Spalten stehen (wie oben) oder in den Zeilen der Tabelle -- es ist aber wichtig, dass dies konsistent gemacht wird.

 	In dieser LVA: Bilder $ f(b_j) $ der Basisvektoren werden durch Spalten beschrieben.

\subsection{Definition}
	\begin{Definition}[Matrix]
		Eine \emph{Matrix} $ X\in K^{m\times n} $ ist eine Tabelle von Elementen $ x_{ij}\in K $ mit $ m $ Zeilen und $ n $ Spalten:
		\[
			X =
			\begin{pmatrix}
				x_{11} & \dots & x_{1n} \\
				\vdots &       & \vdots \\
				x_{m1} & \dots & x_{mn}
			\end{pmatrix}
		\]
		Die \emph{(Darstellungs-)Matrix} eines $ f\in \hom(V,W) $ bzgl. Basen $ B= (b_1,\dots,b_n) $ und $ C=(c_1,\dots,c_m) $ von $ V $ bzw. $ W $, ist die Matrix
		\[
			X = \xi^C_B(f)\in K^{m\times n}\text{ mit }\forall j=1,\dots,n:f(b_j) = \sum_{i=1}^{m}c_ix_{ij}.
		\]
	\end{Definition}

	\paragraph{Bemerkung}
		Mit $ I:= \{1,\dots,m\} $ und $ J:= \{1,\dots,n\} $ kann eine Matrix auch als Abbildung aufgefasst werden
		\[
			X = (x_{ij})_{i\in I,j\in J} \quad\text{ bzw. }\quad X:I\times J\to K,\ (i,j)\mapsto x_{ij}.
		\]
		Ist $ f\in \hom(V,W) $ und sind $ B=(b_1,\dots,b_n) $ und $ C=(c_1,\dots,c_m) $ Basen von $ V $ bzw. $ W $, so sind
		\[
			x_{ij} = c_i^*(f(b_j))
		\]
		die Komponenten der Darstellungsmatrix $ \xi_B^C(f) $ von $ f $ bzgl. der Basen $ B $ und $ C $ mit der zu $ C $ dualen Basis $ C^*=(c_1^*,\dots c_m^*) $ von $ W^* $.\\
		Mit der zu $ B $ dualen Basis $ B^*= (b_1^*,\dots,b_n^*) $ von $ V^* $ ist dann auch
		\[
			f=\sum_{i=1}^{m}\sum_{j=1}^{n} c_ix_{ij}b_j^*.
		\]
\subsection{Lemma}
	\begin{Lemma}[Matrizen als VR]
		Mit der komponentenweisen Addition und Skalarmultiplikation auf $ K^{m\times n} $,
		\[
			(x_{ij})+(y_{ij}) := (x_{ij}+y_{ij}) \text{ und } (x_{ij})\cdot z := (x_{ij}\cdot z),
		\]
		wird $ K^{m\times n} $ ein Vektorraum und man erhält einen Isomorphismus zu Basen $B$ und $C$ von $V$ bzw. $W$.
		\[
			\xi_B^C:\hom(V,W)\to K^{m\times n},\ f\mapsto \xi_B^C(f).
		\]
	\end{Lemma}
	\paragraph{Bemerkung}
		Die komponentenweise Addition und Skalarmultiplikation sind gerade die Addition und Skalarmultiplikation von Matrizen als Abbildungen.

	\paragraph{Beweis}
		Dass $ \hom(V,W) $ und $ K^{m\times n} \ K $-VR sind, ist bekannt (vgl. Kap. 1.4 bzw. Kap. 1.1). Die Linearität von $ \xi_B^C $ folgt direkt, da mit der zu $ C $ dualen Basis $ C^* $ von $ W^* $
		\[
			\forall_{i=1,\dots,m} \forall_{j=1,\dots,n}: x_{ij}= c_i^*(f(b_j)).
		\]
		Nämlich: für $ f,g\in \hom(V,W) $ und $ x,y\in K $ ist dann
		\begin{align*}
			\forall_{i= 1,\dots,m} \forall_{j=1,\dots, n} : c_i^*((fx+gy)(b_j)) & = c_i^*(f(b_j)x+g(b_j)y)         \\
			                                                                    & = c_i^*(f(b_j))x+c_i^*(g(b_j))y.
		\end{align*}
		Die Abbildung
		\[
			K^{m\times n}\ni X=(x_{ij})\mapsto \sum_{i=1}^{m}\sum_{j=1}^{n}c_ix_{ij}b_j^* = f\in \hom(V,W)
		\]
		liefert die Inverse von $ f\mapsto\xi_B^C(f) $, also ist $ \xi_B^C $ ein Isomorphismus.
	\paragraph{Bemerkung}
		Damit folgt (vgl. Kap. 1.4): $ \dim \hom(V,W) = \dim K^{m\times n} = m\cdot n $.
\subsection{Lemma \& Definition}
	\begin{Lemma}[Darstellungsmatrix einer Komposition]
		Sind $ U,V,W \ K$-VR mit Basen $ A=(a_1,\dots,a_p),\ B=(b_1,\dots,b_n),\ C=(c_1,\dots,c_m) $, so gilt für $ g\in \hom(U,V) $ und $ f\in \hom(V,W) $
		\[
			\xi_A^C(f\circ g) = \xi_B^C(f)\cdot \xi_A^B(g),
		\]
	\end{Lemma}
	\begin{Definition}
		wobei die \emph{Matrixmultiplikation}
		\[
			\cdot:K^{m\times n}\times K^{n\times p} \to K^{m\times p},\ (X,Y)\mapsto X\cdot Y = Z
		\]
		definiert ist durch
		\[
			z_{ik} := \sum_{j=1}^{n}x_{ij}y_{jk}.
		\]
	\end{Definition}
	\paragraph{Bemerkung}
		Das Element $ z_{ik} $ in der $ i $-ten Zeile und $ k $-ten Spalte von $ Z = XY $ wird also aus der $ i $-ten Zeile von $ X $ und der $k$-ten Spalte von $ Y $ berechnet.
		\begin{align*}
			\left(
			\begin{array}{ccc}
			       &        &        \\
			x_{i1} & \dots  & x_{in} \\
			       &        &        \\
			\end{array}
			\right)
			\cdot
			\left(
			\begin{array}{ccc}
			       & y_{1k} &        \\
			       & \vdots &        \\
			       & y_{nk} &        \\
			\end{array}
			\right)
			=
			\left(
			\begin{array}{ccc}
			       & z_{ik} &        \\
			       &        &        \\
			\end{array}
			\right)
		\end{align*}
	\paragraph{Beweis}
		Wir verwenden die Darstellungsmatrizen
		\[
			\begin{cases}
				X = \xi^C_B(f)\in K^{m\times n} & \text{ von } f\in \hom(V,W) \\
				Y = \xi_A^B(g)\in K^{n\times p} & \text{ von } g\in \hom(U,V)
			\end{cases}
		\]
		bezüglich $ B $ und $ C $ bzw. $ A $ und $ B $, dann gilt für $ k=1,\dots,p $
		\[
			(f\circ g)(a_k)=f\Big(\sum_{j=1}^{n}b_jy_{jk}\Big) = \sum_{j=1}^{n}f(b_j)y_{jk} = \sum_{j=1}^{n}\sum_{i=1}^{m}c_ix_{ij}y_{jk} = \sum_{i=1}^{m}c_i\Big(\sum_{j=1}^{n}x_{ij}y_{jk}\Big),
		\]
		d.h. durch $ I=\{1,\dots,m\},\ J=\{1,\dots,n\},\ K=\{1,\dots,p\} $ und
		\[
			\xi_A^C(f\circ g) = Z = (z_{ik})_{i\in I,k\in K} \quad\text{mit}\quad \forall i\in I\ \forall k\in K: z_{ik}= \sum_{j=1}^{n}x_{ij}y_{jk}
		\]
		erhält man die Darstellungsmatrix
		\[
			\xi_A^C(f\circ g) = \xi_B^C(f)\xi_A^B(g)
		\]
		der Komposition als Produkt der Darstellungsmatrizen von $ f $ und $ g $.
\subsection{Notation \& Definition}
	\begin{Definition}[Kurzform der def. Gleichung einer Darst.-Matrix]
		Wir notieren die definierende Gleichung einer Darstellungsmatrix $ X=\xi_B^C(f) $ von $ f\in \hom(V,W) $ auch in Kurzform
		\[
			CX=(c_1,\dots,c_m)X = (f(b_1),\dots,f(b_n)) = f(B).
		\]
		Für die \emph{Koordinatenspalte eines Vektors}
		\[
			Y\in K^{n\times 1} \text{ mit } v=\sum_{j=1}^{n}b_jy_{j1}
		\]
		ist dann
		\[
			f(v) = (f(b_1),\dots,f(b_n))Y = (c_1,\dots,c_m)XY.
		\]
	\end{Definition}
	Die Familien $ (c_1,\dots,c_m) $ und $ (f(b_1),\dots,f(b_n)) $ sind keine Matrizen, denn die Elemente sind Vektoren!

%VO19-2015-12-15
	\paragraph{Bemerkung}
		Wir schreiben die Skalarmultiplikation als Rechts-Multiplikation.
	\paragraph{Beispiel}
		Die neue Notation liefert einen alternativen "`Beweis"' für $ \xi_A^C(f\circ g) = \xi_B^C(f)\xi_A^B(g) $:

		Gilt für jeden Vektor $ a_k,\ k=1,\dots,p $
		\[
			g(a_k) = \sum_{j=1}^{n}b_jy_{jk}, \text{ wobei } Y = \xi_A^B(g)
		\]
		so erhalten wir
		\[
			(f(g(a_1)),\dots, f(g(a_p))) = (f(b_1),\dots , f(b_n))Y = (c_1,\dots,c_m)\xi_B^C(f)\cdot Y = C\cdot XY
		\]
		womit nun
		\[
			(f\circ g)(A) = C\cdot XY,
		\]
		also
		\[
			\xi_A^C(f\circ g) = X\cdot Y = \xi_B^C(f)\cdot \xi_A^B(g).
		\]

		Einfacher (aber weniger überzeugend) ist die folgende, die Linearität von $ f $ benutzende Version:
		\[
			( f\circ g )(A) = f(g(A)) = f(BY) = f(B)\cdot Y = C\cdot XY
		\]
	\paragraph{Bemerkung}
		Sei $ f\in \hom(V,W) $ mit $ r:= \rg f $, dann existieren Basen $ B $ und $ C $ von $ V $ bzw. $ W $, sodass
		\[
			\xi_B^C(f) = X \text{ mit } x_{ij} =
			\begin{cases}
				1, & \text{falls }i=j\leq r \\
				0, & \text{sonst}
			\end{cases}
		\]
		d.h.
		\[
			X = \left(
			\begin{array}{ccc|ccc}
				1      & \dots  & 0      & 0      & \dots  & 0      \\
				\vdots & \ddots & \vdots & \vdots &        & \vdots \\
				0      & \dots  & 1_{rr} & 0      & \dots  & 0      \\\hline
				0      & \dots  & 0      & 0      & \dots  & 0      \\
				\vdots &        & \vdots & \vdots &        & \vdots \\
				0      & \dots  & 0      & 0      & \dots  & 0      \\
			\end{array}
			\right)
		\]

		Nämlich -- wie im Beweis des Rangsatzes: Die Basen $ B $ und $ C $ werden so gewählt, dass
		\begin{enumerate}[(i)]
			\item $ (b_{r+1},\dots,b_n) $ Basis von $ \ker f $ ist, und dann
			\item $ c_i := f(b_i) $ für $ i= 1,\dots,r $ eine Basis von $ f(V) $ liefert.
		\end{enumerate}
		Offenbar hat $ \xi_B^C(f)$ dann die gewünschte Form:
		\[
			f(b_1)=c_1,\dots,f(b_r)=c_r,f(b_{r+1})=0,\dots,f(b_n)=0,
		\]
		d.h.
		\[
			f(B) = (f(b_1),\dots,f(b_n))=(c_1,\dots,c_m)\left(
			\begin{array}{ccc|ccc}
				1      & \dots  & 0      & 0      & \dots  & 0      \\
				\vdots & \ddots & \vdots & \vdots &        & \vdots \\
				0      & \dots  & 1_{rr} & 0      & \dots  & 0      \\\hline
				0      & \dots  & 0      & 0      & \dots  & 0      \\
				\vdots &        & \vdots & \vdots &        & \vdots \\
				0      & \dots  & 0      & 0      & \dots  & 0      \\
			\end{array}
			\right) = CX.
		\]
		Umgekehrt: gibt es eine Darstellungmatrix von $ f $ dieser Form, so ist $ \rg f = r $.
\subsection{Beispiel \& Definition}
	\begin{Definition}[Einheitsmatrix]
		Ist $ B=(b_1,\dots,b_n) $ Basis von $ V $, so hat der Isomorphismus
		\[
			\phi:V\to K^n \text{ mit } \forall j=1,\dots,n:\phi(b_j)=e_j
		\]
		bezüglich $ B $ und der Standardbasis $ E = (e_1,\dots,e_n) $ von $ K^n $ die \emph{$ n $-reihige Einheitsmatrix} als Darstellungsmatrix:
		\[
			\xi_B^E(\phi) = E_n := (\delta_{ij})_{i,j = 1,\dots,n}
		\]
	\end{Definition}
	\paragraph{Beispiel}
		Sind $ B=(b_1,\dots,b_n) $ und $ B'=(b'_1,\dots,b'_n) $ Basen von $ V $, wobei
		\[
			\forall j=1,\dots,n:b_j = \sum_{i=1}^{n}b'_ix_{ij},
		\]
		so hat die Identität $ \id_V $ die Darstellungsmatrix
		\[
			\xi_B^{B'}(\id_V)= X = (x_{ij})_{i,j=1\dots,n}.
		\]
		Sind dann $ B $ und $ B' $ Basen von $ V $ und $ C $ und $ C' $ Basen von $ W $, so erhält man für $ f\in\hom(V,W) $ die Transformationsformel
		\[
			\xi_{B'}^{C'}(f) = \xi_{B'}^{C'}(\id_W\circ f\circ \id_V) = \xi_C^{C'}(\id_W)\cdot \xi_B^C(f)\cdot \xi_{B'}^{B}(\id_V)
		\]
\subsection{Beispiel \& Definition}
	Ist $ f\in \Iso(V,W) $ mit Basen $ B $ und $ C $ von $ V $ bzw. $ W $, so gilt (mit $ n=\dim V = \dim W $)
	\[
		\xi_C^B(f^{-1})\cdot \xi_B^C(f) = \xi_B^B(f^{-1}\circ f) = \xi_B^B(\id_V) = E_n
	\]
	und
	\[
		\xi_B^C(f)\cdot \xi_C^B(f^{-1})=\xi_C^C(f\circ f^{-1}) = \xi_C^C(\id_W)=E_n.
	\]

	\begin{Definition}[Invertierbare Matrix]
		Eine Matrix $ X\in K^{n\times n} $ nennt man invertierbar mit Inverser $ X^{-1} $, falls
		\[
			\exists X^{-1}\in K^{n\times n}:X^{-1}X = E_n
		\]
		Damit ist die Darstellungsmatrix der Inversen die Inverse der Darstellungsmatrix:
		\[
			\xi_C^B(f^{-1}) = (\xi_B^C(f))^{-1}
		\]
	\end{Definition}
\subsection{Bemerkung \& Definition}
	Jedes $ X\in K^{m\times n} $ liefert (eindeutig) $ f_X\in \hom(K^n,K^m) $ nach Fortsetzungssatz via
	\[
		f_X:K^n\to K^m,\ f_X(e_j) = \sum_{i=1}^{m} e'_ix_{ij} \text{ für } j=1,\dots,n.
	\]
	Bezüglich der Standardbasen $ E = (e_1,\dots,e_n) $ von $ K^n $ und $ E'=(e'_1,\dots,e'_m) $ von $ K^m $ ist dann
	\[
		\xi_E^{E'}(f_X) = X.
	\]

	\begin{Definition}[Rang einer Matrix]
		Damit definiert man den Rang einer Matrix $ X\in K^{m\times n} $ als
		\[
			\rg X:=\rg f_X
		\]
		Eine Matrix $ X\in K^{n\times n} $ ist genau dann invertierbar, wenn $ \rg X = n $. Man setzt
		\[
			\mathrm{Gl}(n):= \{X\in K^{n\times n}\mid \rg X =n\}.
		\]
	\end{Definition}
\subsection{Bemerkung \& Definition}
	Nach der Transformationsformel für Darstellungsmatrizen gilt bei Basiswechseln in $ V $ und $ W $ für $ f\in \hom(V,W) $
	\[
		\xi_{B'}^{C'}(f) = \xi_C^{C'}(\id_W)\cdot \xi_B^C(f)\cdot \xi_{B'}^B(\id_V).
	\]
	Dabei sind $ \xi_{B'}^B(\id_V)\in \mathrm{Gl}(n) $ und $ \xi_C^{C'}(\id_W)\in \mathrm{Gl}(m) $ invertierbar, da etwa
	\[
		\xi^B_{B'}(\id_V)\cdot\xi_B^{B'}(\id_V) = \xi_B^B(\id_V)=E_n;
	\]
	Sind andererseits die Basis $ B $ und $ P\in \mathrm{Gl}(n) $ gegeben, so ist
	\[
		\xi_B^{B'}(\id_V)=P^{-1} \text{ für } B':= BP,
	\]
	d.h. jedes $ P\in \mathrm{Gl}(n) $ realisiert einen Basiswechsel in $ V $, kommt also in der Transformationsformel vor.

	\begin{Definition}[Äquivalente Matrizen]
		Daher definiert man auch Matrizen $ X,X'\in K^{m\times n} $ als \emph{äquivalent},
		\[
			X\sim X',\quad \text{falls }\exists P\in \mathrm{Gl}(n)\exists Q\in \mathrm{Gl}(m):X' = QXP^{-1}.
		\]
	\end{Definition}

%VO19-2015-12-15
\section{Lineare Gleichungssysteme}
	Mission: Viele Probleme in Anwendungen oder Naturwissenschaften werden zu "`linearen Problemen"' reduziert, d.h. auf lineare Gleichungssysteme unterschiedlicher Komplexität.
	Diese Reduktion ist etwa eine wichtige Aufgabe der Analysis; Aufgabe der linearen Algebra ist dann die Lösung bzw. Strukturanalyse der linearen Gleichungssysteme.
\subsection{Definition}
	\begin{Definition}[Lineares Gleichungssystem]
	Ein \emph{lineares Gleichungssystem} (LGS) ist ein System von $ m $ Gleichungen
		\[ \begin{array}{cccc}\tag{$\star\star$}
		a_{11}x_1+&\dots &+ a_{1n}x_n &=y_1\\
		\vdots & &\vdots & \vdots\\
		a_{m1}x_1 +& \dots &+a_{mn}x_n &= y_m
		\end{array} \]
	für $ n $ Unbekannte $ x_1,\dots,x_n\in K $, wobei die Parameter $ a_{ij},y_i\in K $ gegeben sind. Ist $ y_1 = \dots = y_m = 0 $, so heißt das System \emph{homogen}, anderenfalls \emph{inhomogen}.
	\end{Definition}
\paragraph{Bemerkung}
	Mit Matrizen $ A\in K^{m\times n},X\in K^{n\times 1} $ und $ Y\in K^{m\times 1} $ lässt sich ein lineares Gleichungssystem kompakter schreiben als
		\[ AX = Y \tag{$\star$}\]
	Die Standardbasen $ E $ und $ E' $ von $ K^n $ bzw. $ K^m $ liefern den Isomorphismus
		\[ K^{m\times n}\ni A\mapsto f_A\in \hom(K^n,K^m)\text{, wobei }f_A(E) = E'A, \]
	damit lässt sich ($\star$) umformulieren als Gleichung eines affinen Unterraumes von $ K^n: $
		\[ f_A(x) = y \text{ mit } x=EX \text{ und } y=E'Y. \]
	Nämlich: Existiert eine Lösung $ x\in f_A^{-1}(\{y\})\neq \emptyset $, so ist der Lösungsraum
		\[ f_A^{-1}(\{y\}) = x+\ker f_A\subset K^n \]
	ein affiner Unterraum.
	
	Das nächste Lemma folgt dann mit dem Basisisomorphismus:
		\[ K^{n\times 1} \ni X \mapsto EX =: x\in K^n \]
\subsection{Definition \& Lemma}
	\begin{Lemma}[Lösungsraum]
	Der \emph{Lösungsraum} $ L_{A,Y} $ eines linearen Gleichungssystems,
		\[ L_{A,Y}:=\{X\in K^{n\times 1}\mid AX=Y\}\subset K^{n\times 1} \]
	ist leer oder ein affiner Unterraum der Dimension $ k = n-\rg A $.
	
	Ist $ Y = 0 $, so gilt $ 0\in L_{A,Y} $ und $ L_{A,Y}\subset K^{n\times 1} $ ist ein linearer Unterraum (UVR).
	\end{Lemma}

%VO20-2015-12-17
\paragraph{Bemerkung}
	Jede Lösung $ X_l $ eines (inhomogenen) LGS $AX=Y$ lässt sich schreiben als Summe einer \emph{Partikulärlösung} $ X_0 \in K^{n\times 1},AX_0 = Y $, und einer Lösung $V$ des homogenen LGS $ AX = 0 $:
		\[ \forall X_l\in L_{A,Y}\exists V\in L_{A,0}:X_l=X_0+V=\tau_V(X_0) \]
\paragraph{Bemerkung}
	Der Lösungsraum eines "`unendlichen linearen Gleichungssystems"' hat die gleiche Struktur eines affinen Unterraums wie im endlichen Fall, z.B.:
		\[ \{x\in C^\infty(\mathbb{R})\mid \forall t\in \mathbb{R}:x''(t) = t^2 \} \]
	ist ein (2-dim) AUR, des unendlich-dim. R-VR $C^\infty$, wobei $ C^\infty(\mathbb{R}) $ den (Vektor-)Raum der beliebig oft differenzierbaren Funktionen auf $ \mathbb{R} $ notiert. 
\subsection{Bemerkung \& Definition}
	\begin{Definition}[Erweiterte Koeffizientenmatrix]
	Ist $ AX=Y \neq 0$ ein inhomogenes LGS, so gilt
		\[ L_{A,Y} = \emptyset \Leftrightarrow y\notin f_A(K^n) \]
	mit der \emph{erweiterten Koeffizientenmatrix}
		\[ (A\mid Y) \in K^{m\times (n+1)} \]
	lässt sich dies formulieren als
		\[ f_{(A\mid Y)}(K^{n+1})\neq f_A(K^n) \Leftrightarrow \rg f_{(A\mid Y)}\neq \rg f_A \Leftrightarrow \rg (A\mid Y) \neq \rg A. \]
	Folglich ist
		\[ L_{A,Y} \neq \emptyset \Leftrightarrow \rg (A\mid Y) = \rg A \]
	\end{Definition}
\subsection{Bemerkung \& Definition, Gaußsches Eliminationsverfahren}
	\begin{Definition}[Äquivalente LGS]
	Eine Idee zur Lösung eines LGS ist, das Gleichungssystem zu "`vereinfachen"', ohne dabei den Lösungsraum zu verändern: Man nennt zwei LGS $ AX=Y$ und $A'X=Y' $ \emph{äquivalent}, wenn sie den gleichen Lösungsraum haben,
		\[ (AX=Y)\sim (A'X=Y'):\Leftrightarrow L_{A,Y} = L_{A',Y'} \]
	\end{Definition}
	\emph{Links}multiplikation der erweiterten Koeffizientenmatrix $ (A\mid Y) $ mit den folgenden Matrizen (mit $ i\neq j $) liefert z.B. äquivalente Systeme:

		$ D_i = (d_{kl})\in Gl(m), \qquad
			d_{kl} := \delta_{kl}+(d-1)\delta_{ik}\delta_{il}\quad (d\in K^x); $
                \[
                \bordermatrix{
                    &   &   & i & &\cr
                    & 1 & 0 & \dots & \dots & 0\cr 
                    & 0 & \ddots & \ddots & 0 & \vdots \cr
                i & \vdots & \ddots & d & \ddots & \vdots \cr
                    & \vdots & 0  &  \ddots & \ddots & 0 \cr
                    & 0 &  \dots & \dots  & 0 & 1 \cr
                }
                \]
		$ T_{ij} = (t_{kl})\in Gl(m), \qquad
			 t_{kl} := \delta_{kl}-(\delta_{ik}-\delta_{jk})(\delta_{il}-\delta_{jl})$
                \[
                \bordermatrix{
                      &        & i      & \dots  & j     &         \cr
                      & \ddots &        &        &       &         \cr 
                i     &        & 0      & 1      &       &         \cr
                \vdots&        &        & \ddots &       &         \cr
                j     &        &        & 1      & 0     &         \cr
                      &        &        &        &       & \ddots  \cr
                }
                \]
                
		$ S_{ij}=(s_{kl})\in Gl(m), \qquad
			 s_{kl} := \delta_{kl}+s\delta_{ik}\delta_{jl} \quad s\in K$
                \[
                \bordermatrix{
                    &   &   &  & j &\cr
                    & 1 & 0 & \dots & \dots & 0\cr 
                i & 0 & \ddots & \ddots & s & \vdots \cr
                    & \vdots & \ddots & 1 & \ddots & \vdots \cr
                    & \vdots & 0  &  \ddots & \ddots & 0 \cr
                    & 0 &  \dots & \dots  & 0 & 1 \cr
                }
                \]
	
	Die entsprechenden Operationen auf dem LGS werden als \emph{elementare Zeilenoperationen/-umformungen} bezeichnet (elZumf) bezeichnet:
	\begin{itemize}
		\item $ (A\mid Y) \to D_i (A\mid Y) $, Multiplikation der $ i $-ten Gleichung mit $ d\neq 0 $;
		\item $ (A\mid Y) \to T_{ij} (A\mid Y) $, Vertauschung der $ i $-ten und $ j $-ten Gleichung;
		\item $ (A\mid Y) \to S_{ij} (A\mid Y)$, Addition des $ s $-fachen der $ j $-ten Gleichung zur $ i $-ten Gleichung.
	\end{itemize}
	Da $ D_i,T_{ij},S_{ij}\in Gl(m) $, sind die elementaren Zeilenumformungen reversibel, verändern daher den Lösungsraum nicht: für $ D_i $ und $ T_{ij} $ ist das klar; $ S_{ij} = S_{ij}(s) $ ist invertierbar mit
		\[ (S_{ij})^{-1} = (S_{ij}(s))^{-1} = S_{ij}(-s). \]
	Geometrisch ist $ S_{ij} $ Darstellungsmatrix einer Scherung.
	
	\begin{Definition}[Zeilenstufenform]
	Mit Hilfe der elementaren Zeilenumformungen kann man das LGS auf Zeilenstufenform bringen:
		\[
		\begin{pmatrix}
		a_{11} & \dots & a_{1n} & y_1 \\
		a_{21} & \dots & a_{2n} & y_2 \\
		a_{31} & \dots & a_{3n} & y_3 \\
		\vdots &       & \vdots & \vdots \\
		a_{m1} & \dots & a_{mn} & y_m
		\end{pmatrix}
		\xrightarrow{\text{elZUmf}}
                \begin{pmatrix*}[l]
			1 & \dots & \dots &\dots & \dots & y'_1\\
			0 & 1 & \dots &\dots & \dots & y'_2\\
			\vdots &\ddots & \ddots & & \dots &\vdots \\
			0  & \dots  &  0 & 1 & \dots &y'_r\\
			 0  & \dots  & \dots    & 0 & 0 & y'_{r+1}\\
			\vdots & & & & \vdots & \vdots \\
			 0 & \dots & \dots   & \dots &  0 & y'_{m}\\
		\end{pmatrix*} \]
	Ein System in Zeilenstufenform kann dann einfach gelöst werden -- oder auch nicht, falls eine Gleichung $ 0 = y' \neq 0 $ auftaucht.
	\end{Definition}
\paragraph{Beispiel}
	Wir betrachten das LGS $ AX=Y $ mit 
	\[
            A = \begin{pmatrix}
		0 & 3 & 6\\
		1 & 4 & 7\\
		2 & 5 & 8
		\end{pmatrix}
		\text{ und }
		Y = \begin{pmatrix}
		y_1 \\ y_2 \\ y_3
		\end{pmatrix}. \]
	Elementare Zeilenumformungen liefern dann:
	\begin{align*}
	\begin{pmatrix}
		0 & 3 & 6 & y_1\\
		1 & 4 & 7 & y_2\\
		2 & 5 & 8 & y_3
	\end{pmatrix}\quad
	\overset{T_{12}}{\to}\quad
	&\begin{pmatrix}	
		1 & 4 & 7 & y_2\\
		0 & 3 & 6 & y_1\\
		2 & 5 & 8 & y_3
	\end{pmatrix}
	\\
	\overset{S_{31}(-2)}{\to}\quad
	&\begin{pmatrix}	
		1 & 4 & 7 & y_2\\
		0 & 3 & 6 & y_1\\
		0 & -3 & -6 & y_3-2y_2
	\end{pmatrix}\\
	\overset{S_{32}(1)}{\to}\quad
	&\begin{pmatrix}	
		1 & 4 & 7 & y_2\\
		0 & 3 & 6 & y_1\\
		0 & 0 & 0 & y_3-2y_2+y_1
	\end{pmatrix}\\
	\overset{D_2(\frac{1}{3})}{\to}\quad
	&\begin{pmatrix}	
		1 & 4 & 7 & y_2\\
		0 & 1 & 2 & y_1 \frac{1}{3}\\
		0 & 0 & 0 & y_3-2y_2
	\end{pmatrix}
	 \end{align*}
	 d.h. ein äquivalentes LGS $ A'X=Y' $ ist gefunden mit
	 	\[ A' = 
	 	\begin{pmatrix}	
	 		1 & 4 & 7 \\
	 		0 & 1 & 2 \\
	 		0 & 0 & 0 
	 	\end{pmatrix}
	 	\text{ und }
	 	Y = \begin{pmatrix}
	 	y_2 \\ y_1\frac{1}{3} \\ y_1-2y_2+y_3
	 	\end{pmatrix}.\]
	 Das LGS $ AX=Y $ ist also genau dann lösbar, wenn $ y_1-2y_2+y_3 = 0 $; in diesem Falle ist dann
	 	\[ L_{A,Y} = \{X =
	 		\begin{pmatrix} y_2-7t-4(-2t+\frac{1}{3}y_1)\\-2t+\frac{1}{3}y_1\\t\end{pmatrix}
	 	, t\in \mathbb{R}\}
	 	= \{X =
	 		\begin{pmatrix} t-\frac{4}{3}y_1+y_2\\-2t+\frac{1}{3}y_1\\t\end{pmatrix}
	 	, t\in \mathbb{R}\} \]
\paragraph{Historische Bemerkung}
	Das Gaußsche Eliminationsverfahren ist schon seit ca. 2000 Jahren bekannt, also schon lange vor Gauß (1777-1855) entwickelt worden.
	
\paragraph{Nutzen der Methode}
	\begin{itemize}
		\item lässt sich einfach programmieren (leider ggf. numerisch instabil)
		\item nützlich für mittelgroße Systeme (tausende Gleichungen)
		\item nicht effizient für große Systeme (Millionen von Gleichungen)
	\end{itemize}
\paragraph{Bemerkung}
	Mehrere LGS $ AX = Y_1, AX = Y_2, \dots AX = Y_k $ mit derselben Koeffizientenmatrix $ A $ können simultan gelöst werden, indem man elementare Zeileinumformungen auf die um alle $ Y $ erweiterte Koeffizientenmatrix $ (A\mid Y_1\mid Y_2\mid \dots \mid Y_k) $ anwendet.
\paragraph{Bemerkung}
	Das Gaußsche Eliminationsverfahren kann zur Bestimmung der Inversen einer Matrix $ A\in Gl(n) $ verwendet werden. Insbesondere ist eine \emph{untere Dreiecksmatrix} $ A\in K^{n\times n} $, d.h $ a_{ij} = 0 $ für $ i<j $ genau dann invertierbar, wenn $ a_{ii}\neq 0 $ für alle $ i=1,\dots,n $.


%VO21-2016-01-07
\chapter{Volumenmessung}
Grundlegende Idee: Wir definieren ein Spat- oder Parallelotop-Volumen.

Algebraisch: Dieses Volumen kann dann benutzt werden, um zu testen, wann ein Spat/Pa"-ral"-lelotop "`zusammenklappt"'.
\section{Determinantenformen}
 Idee: Für den Flächeninhalt $ F(v,w) $ eines von zwei Vektoren $ v,w\in V $ aufgespannten Parallelogramms gilt
 \begin{align*}
 	F{(vx,w)} & = F{(v,w)}\cdot x \\
 	F(v+v',w) & = F(v,w)+F(v',w)
 \end{align*}
 und entsprechend für das zweite Argument.

 \definecolor{ttttff}{rgb}{0.2,0.2,1}
 \definecolor{ttfftt}{rgb}{0.2,1,0.2}
 \definecolor{uququq}{rgb}{0.25,0.25,0.25}
 \definecolor{qqqqff}{rgb}{0,0,1}

 %-------------------Begin Addition mit einem Vektor ----------------
 \begin{figure}[H]\centering
 	\tdplotsetmaincoords{0}{0} %-27
 	\begin{tikzpicture}[yscale=1,tdplot_main_coords]

 		\def\xstart{0} %x Koordinate der Startposition der Grafik
 		\def\ystart{0} %y Koordinate der Startposition der Grafik
 		\def\myscale{0.9} %ändert die Größe der Grafik (Skalierung der Grafik)

 		\def\xstartdraw{(\xstart + 1.5)} %xKoordinate des Referenzstartpunktes (in dieser Zeichnung: a)
 		\def\ystartdraw{(\ystart + 3.5)}%yKoordinate des Referenzstartpunktes (in dieser Zeichnung: a)

 		\def\balkenhoehe{(5.3)}% Länge des vertikalen blauen Balkens
 		\def\balkenlaenge{(10)}% Länge des horizontalen blauen Balkens
 		\def\balkenbreite{0.4} %Balkenbreite

 		%---------Begin Balken----------
 		\def\drehwinkel{0}
 		\node (VekV) at ({\xstart+0.7*cos(\drehwinkel)-\balkenbreite*sin(\drehwinkel)},{\ystart+0.5*sin(\drehwinkel)+\balkenbreite*cos(\drehwinkel)})[right, xshift=1,color=blue] {$V$};
 		\node (AffA) at ({\xstart+(\balkenlaenge-1)*cos(\drehwinkel)},{\ystart+(\balkenlaenge-1)*sin(\drehwinkel)+\balkenbreite*cos(\drehwinkel)})[color=red] {$A^2$};

 		\path[ shade, top color=white, bottom color=blue, opacity=.6]
 		({\xstart},{\ystart},0)  -- ({\xstart - \balkenbreite * cos(\drehwinkel)- (-\balkenbreite+0)*sin(\drehwinkel)},{\ystart - \balkenbreite * sin(\drehwinkel)+ (-\balkenbreite+0)*cos(\drehwinkel)},0)  -- ({\xstart - \balkenbreite * cos(\drehwinkel)- (\balkenhoehe+0.5)*sin(\drehwinkel)},{\ystart - \balkenbreite * sin(\drehwinkel)+ (\balkenhoehe+0.5)*cos(\drehwinkel)},0) -- ({\xstart - 0 * cos(\drehwinkel)- (\balkenhoehe+0)*sin(\drehwinkel)},{\ystart - 0 * sin(\drehwinkel)+ (\balkenhoehe+0)*cos(\drehwinkel)},0) -- cycle;

 		\path[ shade, right color=white, left color=blue, opacity=.6]
 		({\xstart},{\ystart},0)  -- ({\xstart - \balkenbreite * cos(\drehwinkel)- (-\balkenbreite+0)*sin(\drehwinkel)},{\ystart - \balkenbreite * sin(\drehwinkel)+ (-\balkenbreite+0)*cos(\drehwinkel)},0) --
 		({\xstart + (\balkenlaenge+0.5) * cos(\drehwinkel)- (-\balkenbreite+0)*sin(\drehwinkel)},{\ystart + (\balkenlaenge+0.5) * sin(\drehwinkel)+ (-\balkenbreite+0)*cos(\drehwinkel)},0) --
 		({\xstart + \balkenlaenge * cos(\drehwinkel)},{\ystart + \balkenlaenge * sin(\drehwinkel)},0)--
 		cycle;
 		%---------End Balken----------
 		\def\lightoffset{0.2*\myscale} %offeset der Vektoren

 		%Punkte Definition
 		\node (pointa1) at ({\xstartdraw},{\ystartdraw}) {};
 		\node (pointa2) at ({\xstartdraw+(1 *\myscale)},{\ystartdraw-(2.0*\myscale)}) {};
 		\node (pointb1) at ($(pointa1) + (3.0*\myscale,-1.0*\myscale) $) {};
 		\node (pointb2) at ($(pointb1) + (1.0*\myscale,-2.0*\myscale) $) {};

 		\node (pointc1) at ($(pointa1) + (6.5*\myscale,1.3*\myscale) $) {};
 		\node (pointc2) at ($(pointa2) + (6.5*\myscale,1.3*\myscale) $) {};

 		\node (pointFvwi) at ($(pointb2) + (-0.9*\myscale,0.9*\myscale) $) {};
 		\node (pointFvwa) at ($(pointb2) + (-1.5*\myscale,-0.3*\myscale) $) {};

 		\node (pointFvswi) at ($(pointc2) + (-2.9*\myscale,-1.2*\myscale) $) {};
 		\node (pointFvswa) at ($(pointc2) + (-1.3*\myscale,-1.9*\myscale) $) {};

 		\node (pointfgi) at ($(pointa1) + (2.2*\myscale,-0.2*\myscale) $) {};
 		\node (pointfga) at ($(pointfgi) + (-1.4*\myscale,1.8*\myscale) $) {};



 		%Flächen füllen
 		%blaue Flaeche
 		\fill[color=ttttff,fill=ttttff,fill opacity=0.15] (pointa1.center) -- (pointb1.center) -- (pointc1.center) -- (pointc2.center) -- (pointb2.center)-- (pointa2.center)-- cycle;
 		%gruene Flaeche
 		\fill[color=ttfftt,fill=ttfftt,fill opacity=0.5] (pointa1.center) -- (pointc1.center) -- (pointc2.center) -- (pointa2.center)-- cycle;

 		%Vektoren blau
 		\draw[-{>[scale=1,length=10,width=6]},shorten >=2pt, shorten <=2pt,line width=0.2pt,color=blue] (pointa1) -- (pointb1);
 		\draw[-{>[scale=1,length=10,width=6]},shorten >=2pt, shorten <=2pt,line width=0.2pt,color=blue] (pointa2) -- (pointb2);
 		\node [color=blue] (pointlabelg1) at ($(pointa1)!0.5!(pointb1)$) [above, xshift=0, yshift=0] {$v$} ;
 		\node [color=blue] (pointlabelg2) at ($(pointa2)!0.5!(pointb2)$) [above, xshift=0, yshift=0] {$v$} ;

 		\draw[-{>[scale=1,length=10,width=6]},shorten >=2pt, shorten <=2pt,line width=0.2pt,color=blue] (pointb1) -- (pointc1);
 		\draw[-{>[scale=1,length=10,width=6]},shorten >=2pt, shorten <=2pt,line width=0.2pt,color=blue] (pointb2) -- (pointc2);
 		\node [color=blue] (pointlabelg3) at ($(pointb1)!0.5!(pointc1)$) [above, xshift=0, yshift=0] {$v'$} ;
 		\node [color=blue] (pointlabelg4) at ($(pointb2)!0.5!(pointc2)$) [above, xshift=0, yshift=0] {$v'$} ;

 		\draw[-{>[scale=1,length=10,width=6]},shorten >=2pt, shorten <=2pt,line width=0.2pt,color=blue] (pointa2) -- (pointa1);
 		\draw[-{>[scale=1,length=10,width=6]},shorten >=2pt, shorten <=2pt,line width=0.2pt,color=blue] (pointb2) -- (pointb1);
 		\draw[-{>[scale=1,length=10,width=6]},shorten >=2pt, shorten <=2pt,line width=0.2pt,color=blue] (pointc2) -- (pointc1);

 		\node [color=blue] (pointlabelga2a1) at ($(pointa2)!0.5!(pointa1)$) [left, xshift=0, yshift=0] {$w$} ;
 		\node [color=blue] (pointlabelgb2b1) at ($(pointb2)!0.5!(pointb1)$) [right, xshift=0, yshift=0] {$w$} ;
 		\node [color=blue] (pointlabelgc2c1) at ($(pointc2)!0.5!(pointc1)$) [right, xshift=0, yshift=0] {$w$} ;

 		%Vektoren gruen
 		\draw[-{>[scale=1,length=10,width=6]},shorten >=4pt, shorten <=4pt,line width=0.2pt,color=green] (pointa1) -- (pointc1);
 		\draw[-{>[scale=1,length=10,width=6]},shorten >=4pt, shorten <=4pt,line width=0.2pt,color=green] (pointa2) -- (pointc2);
 		\node [color=green] (pointlabelac1) at ($(pointa1)!0.5!(pointc1)$) [above, xshift=0, yshift=0] {$v+v'$} ;
 		\node [color=green] (pointlabelac2) at ($(pointa2)!0.5!(pointc2)$) [below, xshift=10, yshift=5] {$v+v'$} ;

 		%Punkte malen
 		\draw[fill,color=red] (pointa1) circle [x=1cm,y=1cm,radius=0.08]node[above, xshift=0, yshift=0]{};
 		\draw[fill,color=red] (pointb1) circle [x=1cm,y=1cm,radius=0.08]node[above, xshift=0, yshift=0]{};
 		\draw[fill,color=red] (pointa2) circle [x=1cm,y=1cm,radius=0.08]node[below, xshift=5, yshift=0]{};
 		\draw[fill,color=red] (pointb2) circle [x=1cm,y=1cm,radius=0.08]node[below, xshift=5, yshift=0]{};
 		\draw[fill,color=red] (pointc1) circle [x=1cm,y=1cm,radius=0.08]node[below, xshift=5, yshift=0]{};
 		\draw[fill,color=red] (pointc2) circle [x=1cm,y=1cm,radius=0.08]node[below, xshift=5, yshift=0]{};

 		\draw[->,shorten >=2pt, shorten <=2pt,line width=0.2pt,color=blue] (pointFvwa) -- (pointFvwi);
 		\draw[->,shorten >=2pt, shorten <=2pt,line width=0.2pt,color=blue] (pointFvswa) -- (pointFvswi);
 		\draw[->,shorten >=2pt, shorten <=2pt,line width=0.2pt,color=green] (pointfga) -- (pointfgi);

 		\node [color=blue] (pointlabelFvwl) at (pointFvwa) [xshift=0.5, yshift=-0.5] {$F(v,w)$} ;
 		\node [color=blue] (pointlabelFvswl) at (pointFvswa) [xshift=-0.5,  yshift=-5] {$F(v',w)$} ;
 		\node [color=green] (pointlabelFvswl) at (pointfga) [xshift=35 ] {$F(v+v',w)=\textcolor{blue}{F(v,w)+F(v',w)}$} ;

 	\end{tikzpicture}
 \end{figure}
 %-------------------End Addition mit einem Vektor ----------------


 Außerdem verschwindet der Flächeninhalt, wenn das Parallelogramm "`zusammenklappt"', also insbesondere gilt
 \[
 	w=v\Rightarrow F(v,w)=0
 \]
 Die folgende Definition verallgemeinert diese Eigenschaften:

 \subsection{Definition}
 	\begin{Definition}[Linearform/Determinantenform]
 		Sei $ V $ ein $ K $-VR. Eine Abbildung $ \omega:V^m\to K $ heißt
 		\begin{itemize}
 			\item \emph{$ m $-linear}, bzw. eine \emph{$ m $-(Linear-)Form}, falls $ \omega $ in jedem Argument linear ist, d.h.
 			      \[
 			      	\forall i=1,\dots, m: V\ni v_i\mapsto \omega(v_1,\dots,v_{i-1},v_i,v_{i+1},\dots,v_m)\in K
 			      \]
 			      ist linear;
 			\item \emph{alternierend}, falls $ \omega(v_1,\dots,v_m)=0 $ wann immer zwei Vektoren gleich sind, d.h.
 			      \[
 			      	v_i = v_j \text{ für } i\neq j \Rightarrow \omega(v_1,\dots, v_m) = 0.
 			      \]
 		\end{itemize}
 		Die Menge der alternierenden $ m $-Formen wird mit $ \Lambda^mV^* $ bezeichnet. Ist $ \dim V = n $, so heißt ein $ \omega\in \Lambda^nV^* $ auch \emph{Determinantenform}.
 	\end{Definition}

 	\paragraph{Beispiel}
 		Jede Linearform $ \omega\in V^* $ ist eine (alternierende) 1-Form, $ \Lambda^1V^*=V^* $.
 	\paragraph{Bemerkung}
 		$ \Lambda^mV^* $ ist für jedes $ m\in \mathbb{N} $ selbst ein $ K $-VR.
 \subsection{Lemma}
 	\begin{Lemma}
 		Für eine alternierende $ m $-Form $ \omega \in \Lambda^mV^* $ und $ i\neq j $ gilt:
 		\begin{enumerate}[(i)]
 			\item $ \omega(\dots,v_i,\dots,v_j,\dots) = -\omega (\dots,v_j,\dots,v_i,\dots)$;
 			\item $ \omega(\dots,v_i,\dots,v_is+v_j,\dots) = \omega(\dots,v_i,\dots,v_j,\dots) $ für $ s\in K $ \footnote{Geometrisch entspricht dies einer Scherung!};
 			\item $ \omega(v_1,\dots,v_m)=0 $, falls $ (v_i)_{i\in \{1,\dots,m\}} $ linear abhängig ist.
 		\end{enumerate}
 		\paragraph{Beweis}
 			Seien $ v_1,\dots,v_m\in V $ und $ i,j\in \{1,\dots,m\} $ mit $ i\neq j $. Dann gilt:
 			\begin{align*}
 				0 & = \omega(\dots,v_i+v_j,\dots,v_i+v_j,\dots)                                                                                               \\
 				  & = \omega(\dots,v_i,\dots,v_i,\dots)+\omega(\dots,v_j,\dots,v_j,\dots)\\& +\omega(\dots,v_i,\dots,v_j,\dots)+\omega(\dots,v_j,\dots,v_i,\dots) \\
 				  & = \omega(\dots,v_i,\dots,v_j,\dots)+\omega(\dots,v_j,\dots,v_i,\dots)
 				\intertext{und}
 				0 & =\omega(\dots,v_i,\dots,v_is,\dots)                                                                                                       \\
 				  & = \omega(\dots,v_i,\dots,v_is+v_j-v_j,\dots)                                                                                              \\
 				  & = \omega(\dots,v_i,\dots,v_is+v_j,\dots)-\omega(\dots,v_i,\dots,v_j,\dots)
 			\end{align*}
 			Dies beweist (i) und (ii).

 			Ist die Familie $ (v_i)_{i\in \{1,\dots,m\}} $ linear abhängig, o.B.d.A
 			\[
 				v_m = \sum_{i=1}^{m-1}v_ix_i \in [(v_i)_{i\in \{1,\dots,m-1\}}]
 			\]
 			so gilt
 			\begin{align*}
 				\omega(v_1,\dots,v_m) & =\omega(v_1,\dots,v_{m-1},\sum_{i=1}^{m-1}v_ix_i)                       \\
 				                      & = \sum_{i=1}^{m-1}\underbrace{\omega(v_1,\dots,v_{m-1},v_i)}_{0}x_i = 0
 			\end{align*}
 			womit (iii) bewiesen ist.
 		\end{Lemma}
 	\paragraph{Bemerkung}
 		(i) liefert eine äquivalente Formulierung von "`alternierend"' für $ m $-Linearformen, wenn $ \Char (K)\neq 2 $.
 		Nämlich: sind $ v_1,\dots,v_m\in V $ mit $ v_i=v_j $ für $ i\neq j $, so gilt
 		\begin{align*}
 			0             & = \omega(\dots,v_i\dots,v_j,\dots)+\omega(\dots,v_j,\dots,v_i,\dots) \\
 			              & = 2\omega(\dots,v_i,\dots,v_j,\dots)                                 \\
 			\Rightarrow 0 & = \omega(\dots,v_i,\dots,v_j,\dots)
 		\end{align*}
 	\paragraph{Buchhaltung}
 		Benutzt man (vgl. Gausssches Eliminationsverfahren) die Elementarmatrizen
 		\begin{gather*}
 			D_i = (d_{kl}) \in \mathrm{Gl}(m);\ d_{kl} = \delta_{kl}+(d-1)\delta_{ik}\delta_{il}\quad (d\in K^\times);\\
 			T_{ij} = (t_{kl}) \in \mathrm{Gl}(m);\ t_{kl} = \delta_{kl}-(\delta_{ik}-\delta_{jk})(\delta_{il}-\delta_{jl});\\
 			S_{ij} = (s_{kl})\in \mathrm{Gl}(m);\ s_{kl}=\delta_{kl}+s\delta_{ik}\delta_{jl}\quad (s\in K)
 		\end{gather*}
 		und beschreibt man eine Familie $ (v_i)_{i\in \{1,\dots,m\}} $ von Vektoren $ v_i\in V $ durch ein \emph{$ m $-Tupel} $ A=(v_1,\dots,v_m) $ von Werten der Familie, so lassen sich die \emph{Homogenität} und Eigenschaften (i) und (ii) des Lemmas einfach schreiben als
 		\[
 			\omega(AD_i(d)) = \omega(A)d,\ \omega(AT_{ij}) = -\omega(A),\ \omega(AS_{ij}(s)) = \omega(A)
 		\]
 \subsection{Wiederholung \& Definition}
 	Die bijektiven Abbildungen (Permutationen)
 	\[
 		\sigma: I\to I,\ i\mapsto \sigma(i),\text{ der Menge }I = \{1,\dots,m\}
 	\]
 	bilden mit der Komposition eine Gruppe: die Permutationsgruppe $ S_m $ der Menge $ I $.
 	\begin{Definition}[Transposition]
 		Eine \emph{Transposition} $ \tau_{ij} \in S_m, i\neq j $ ist eine Permutation, die zwei Indizes vertauscht,
 		\[
 			\tau_{ij}:I\to I,\ k\mapsto \tau_{ij}(k):=
 			\begin{cases}
 				j, & \text{falls } k=i, \\
 				i, & \text{falls } k=j, \\
 				k  & \text{sonst}.
 			\end{cases}
 		\]
 		Jede Permutation ist eine Komposition von Transpositionen, wie man leicht durch Induktion über $m$ zeigt:

 		Ist $ \sigma(m) = i<m$, so ist $ \tau_{im}\circ \sigma $ eine Permutation, die $ m $ fixiert, also
 		\[
 			\tau_{im}\circ\sigma\mid_{\{1,\dots,m-1\}}\in S_{m-1}
 		\]
 	\end{Definition}
 	\paragraph{Bemerkung}
 		Die Eigenschaft (i) des Lemmas, $ \omega(AT_{ij})=-\omega(A) $, lässt sich mit $ \tau_{ij} $ dann formulieren als
 		\[
 			\omega(v_{\tau_{ij}(1)}, \dots, v_{\tau_{ij}(m)}) = - \omega(v_1,\dots,v_m)
 		\]
 		Da jede Permutation $ \sigma\in S_m $ Komposition von Transpositionen ist, folgt
 		\[
 			\forall \sigma\in S_m: \omega(v_{\sigma(1)},\dots,v_{\sigma(m)})=\pm \omega(v_1,\dots,v_{m}).
 		\]
 		Frage: Was ist das Vorzeichen bzw. wie kann man es berechnen?
 \subsection{Lemma \& Definition}
 	\begin{Definition}[Signum einer Permutation]
 		Das Signum einer Permutation $ \sigma\in S_m $ ist die Zahl
 		\[
 			\operatorname{sgn}\sigma := \prod_{i<j} \frac{\sigma(i)-\sigma(j)}{i-j}\in \{\pm 1\};
 		\]
 		ist $ \sgn\sigma = 1 $, so heißt $ \sigma $ gerade, sonst ungerade. Signum liefert einen Gruppenhomomorphismus
 		\[
 			\sgn: S_m\to (\{\pm 1\},\cdot).
 		\]
 	\end{Definition}
 	\paragraph{Beispiel}
 		Eine Transposition $ \tau_{ij} $ ist eine ungerade Permutation, da
 		\[
 			\sgn\tau_{ij} = \prod_{k<l}\frac{\tau_{ij}(k)-\tau_{ij}(l)}{k-l} = \frac{j-i}{i-j}\prod_{k\neq i,j}\frac{i-k}{j-k}\frac{j-k}{i-k} = -1
 		\]

%VO22-2016-01-12
 	\paragraph{Beweis}
 		Seien $ \sigma,\tau\in S_m $ beliebig, dann gilt
 		\begin{align*} \sgn(\tau\circ\sigma) &= \prod_{i<j}\frac{\tau(\sigma(i))-\tau(\sigma(j))}{\sigma(i)-\sigma(j)}\cdot\frac{\sigma(i)-\sigma(j)}{i-j} \\
 			  & = \prod_{i<j}\frac{\tau(\sigma(i))-\tau(\sigma(j))}{\sigma(i)-\sigma(j)}\prod_{i<j}\frac{\sigma(i)-\sigma(j)}{i-j} \\
 			\intertext{Setze $i':=\sigma(i), j':=\sigma(j) $}
 			  & =\prod_{i'<j'}\frac{\tau(i')-\tau(j')}{i'-j'}\prod_{i<j}\frac{\sigma(i)-\sigma(j)}{i-j}                            \\
 			  & = \sgn(\tau)\cdot \sgn(\sigma).
 		\end{align*}
 		Da jede Permutation Komposition von Transpositionen ist, folgt daraus
 		\[
 			\forall \sigma \in S_m:\sgn(\sigma) = \pm 1
 		\]
 		und dass
 		\[
 			\sgn:S_m \to (\{\pm 1\},\cdot)
 		\]
 		Gruppenhomomorphismus ist.
 	\paragraph{Bemerkung}
 		Damit folgt für $ \omega\in \Lambda^mV^* $ und $ \sigma \in S_m $
 		\[
 			\omega(v_{\sigma(1)},\dots,v_{\sigma(m)}) =\omega(v_1,\dots,v_m)\sgn\sigma.
 		\]
 \subsection{Leibniz-Formel}
 	\begin{Satz}[Leibniz-Formel]
 		Seien $ \omega\in \Lambda^mV^* $, $ (b_i)_{i\in \{1,\dots,m\}} $ lin. unabhängig und $ (v_j)_{j\in \{1,\dots,m\}} $ eine Familie in $ [(b_i)_{i\in\{1,\dots,m\}}] \subset V$,
 		\[
 			\forall j=1,\dots,m: v_j = \sum_{i=1}^{m}b_ix_{ij}
 		\]
 		dann gilt
 		\[
 			\omega(v_1,\dots,v_m)=\omega(b_1,\dots,b_m)\sum_{\sigma\in S_m}\sgn(\sigma)x_{\sigma(1)1} \cdots x_{\sigma(m)m}
 		\]
 	\end{Satz}
 	\paragraph{Beweis}
 		Ausmultiplizieren ergibt:
 		\begin{align*} \omega(v_1,\dots,v_m)&=\sum_{i_1=1}^{m}\cdots \sum_{i_m=1}^{m}\omega(b_{i_1},\dots,b_{i_m})x_{i_11}\cdots x_{i_mm}
 			\intertext{Es gilt: $ \omega(\dots)=0 $, wenn zwei $ b $'s gleich sind, d.h. wann immer $ \{1,\dots,m \}\ni j\mapsto i_j \in \{1,\dots,m\} $ nicht injektiv ist, also keine Permutation ist.}
 			  & = \sum_{\sigma\in S_m}\omega(b_{\sigma(1)},\dots,b_{\sigma(m)})x_{\sigma(1)1}\cdots x_{\sigma(m)m} \\
 			  & = \sum_{\sigma\in S_m}\omega(b_1,\dots,b_m)\sgn(\sigma)x_{\sigma(1)1}\cdots x_{\sigma(m)m}
 		\end{align*}
 	\paragraph{Beispiel}
 		Ist die \emph{Koeffizientenmatrix} $ x=(x_{ij})_{i,j\in\{1,\dots,m\}} $ in der Leibniz-Formel eine obere Dreiecksmatrix, d.h.
 		\[
 			X=
 			\begin{pmatrix}
 				x_{11} & \cdots & \cdots & x_{1m} \\
 				0      & x_{22} &        & \vdots \\
 				\vdots & \ddots & \ddots & \vdots \\
 				0      & \cdots & 0      & x_{mm}
 			\end{pmatrix}
 			\quad\text{und}\quad \forall j=1,\dots,m :v_j=\sum_{i=1}^{j}b_ix_{ij}
 		\]
 		so gilt für jede Permutation $ \sigma\in S_m $ von $ I=\{1,\dots,m \} $
 		\begin{align*} x_{\sigma(1)1},\dots,x_{\sigma(m)m}\neq 0 &\Rightarrow \forall j\in I:\sigma(j)\leq j\\
 			  & \Rightarrow \sigma = \id_I
 		\end{align*}
 		und damit
 		\[
 			\omega(v_1,\dots,v_m) = \omega(b_1,\dots,b_m) x_{11}\cdots x_{mm}.
 		\]

 \subsection{Buchhaltung}
 	\begin{Definition}[Determinante]
 		Mit
 		\[
 			A:= (v_1,\dots,v_m)=\underbrace{(b_1,\dots,b_m)}_{:=B}X = BX
 		\]
 		und der \emph{Determinante}
 		\[
 			\det X := \sum_{\sigma\in S_m}\sgn(\sigma)x_{\sigma(1)1}\cdots x_{\sigma(m)m}
 		\]
 		der Koeffizientenmatrix $ X\in K^{m\times m} $ lässt sich die Leibniz-Formel auch kürzer schreiben als
 		\[
 			\omega(A) = \omega(B)\cdot\det X.
 		\]
 	\end{Definition}
 	Für $ m=2 $ und $ m=3 $ lässt sich $ \det X $ einfach berechnen:
 	\begin{itemize}
 		\item für $ m=2 $ ist
 		      \[
 		      	\det
 		      	\begin{pmatrix}
 		      		x_{11} & x_{12} \\ x_{21} & x_{22}
 		      	\end{pmatrix}
 		      	= x_{11}x_{22}-x_{21}x_{12}
 		      \]
 		\item für $ m=3 $ mit Hilfe der \emph{Regel von Sarrus} (zuerst zyklische (gerade) Permutationen, dann mit einem Fixpunkt, also Transpositionen $\tau_{1,3}, \tau_{1,2}, \tau_{2,3}$)
 		      \[
 		      	\det
 		      	\begin{pmatrix}
 		      		x_{11} & x_{12} & x_{13} \\
 		      		x_{21} & x_{22} & x_{23} \\
 		      		x_{31} & x_{32} & x_{33}
 		      	\end{pmatrix}
 		      	=
 		      	\begin{matrix*}[r]
 		      		x_{11}x_{22}x_{33}
 		      		+x_{21}x_{32}x_{13}
 		      		+x_{31}x_{12}x_{23}\\
 		      		-x_{31}x_{22}x_{13}
 		      		-x_{21}x_{12}x_{33}
 		      		-x_{11}x_{32}x_{23}
 		      	\end{matrix*}
 		      \]
 		      \begin{center}
 		      	% source: http://www.texample.net/tikz/examples/mnemonic-rule-for-matrix-determinant/
 		      	\begin{tikzpicture}[baseline=(A.center)]
 		      		\tikzset{node style ge/.style={circle}}
 		      		\tikzset{BarreStyle/.style = {opacity=.4,line width=4 mm,line cap=round,color=#1}}
 		      		\tikzset{SignePlus/.style = {above left,,opacity=1,circle,fill=#1!50}}
 		      		\tikzset{SigneMoins/.style = {below left,,opacity=1,circle,fill=#1!50}}

 		      		\matrix (A) [matrix of math nodes, nodes = {node style ge},,column sep=0 mm]
 		      		{
                                        x_{11} & x_{12} & x_{13}  \\
 		      			x_{21} & x_{22} & x_{23}  \\
 		      			x_{31} & x_{32} & x_{33}  \\\hline
 		      			x_{11} & x_{12} & x_{13}  \\
 		      			x_{21} & x_{22} & x_{13}  \\
 		      		};

 		      		\draw [BarreStyle=blue] (A-1-1.north west) node[SignePlus=blue] {$+$} to (A-3-3.south east);
 		      		\draw [BarreStyle=blue] (A-2-1.north west) node[SignePlus=blue] {$+$} to (A-4-3.south east);
 		      		\draw [BarreStyle=blue] (A-3-1.north west) node[SignePlus=blue] {$+$} to (A-5-3.south east);
 		      		\draw [BarreStyle=red]  (A-3-1.south west) node[SigneMoins=red] {$-$} to (A-1-3.north east);
 		      		\draw [BarreStyle=red]  (A-4-1.south west) node[SigneMoins=red] {$-$} to (A-2-3.north east);
 		      		\draw [BarreStyle=red]  (A-5-1.south west) node[SigneMoins=red] {$-$} to (A-3-3.north east);
 		      	\end{tikzpicture}
 		      \end{center}
 	\end{itemize}
 	Für $ m>3 $ liefert der Laplacesche Entwicklungssatz eine Methode, die Terme (Permutationen) zu sortieren: Für fest gewähltes $ j\in \{1,\dots, m \} $ gilt\footnote{Entwicklung nach $j$-ter Spalte}
 	\[
 		\det X = \sum_{i=1}^{m}(-1)^{i+j}x_{ij}\det X_{ij}
 	\]
 	mit\footnote{Die $i$-te Zeile und die $j$-te Spalte sind in dieser Matrix "`gestrichen"', d.h. die Matrix ist aus $K^{(m-1)\times(m-1)}$}
 	\[
 		X_{ij} := (x_{kl})_{\substack{k\neq i\\ l\neq j}} =
 		% TODO: die i-te Zeile und j-te Spalte gehören durchgestrichen! % Q&D-Fix durch markierung in rot
 		\begin{pmatrix}
 			x_{11}     & \cdots & x_{1(j-1)} & \color{red}x_{1j} & x_{1(j+1)} & \cdots & x_{1m}     \\
 			\vdots     &        &            & \color{red}\vdots &            &        & \vdots     \\
 			x_{(i-1)1} &        &            & \color{red}\vdots &            &        & x_{(i-1)m} \\
 			\color{red}x_{i1} 	&\color{red} \dots	& \color{red}\dots & \color{red}x_{ij} &
 			\color{red}\dots &\color{red} \dots 	& \color{red}x_{im} \\
 			x_{(i+1)1} &        &            & \color{red}\vdots &            &        & x_{(i+1)m} \\
 			\vdots     &        &            & \color{red}\vdots &            &        & \vdots     \\
 			x_{m1}     &        &            & \color{red}x_{mj} &            &        & x_{mm}
 		\end{pmatrix}
 	\]
 	Nämlich: Ist o.B.d.A. $ v_m=b_i $ in der Leibniz-Formel, also $ x_{km}=\delta_{ik} $, so erhält man
 	\begin{align*}
 		\det X & = \sum_{\sigma\in S_m}\sgn(\sigma)\prod_{j=1}^{m}x_{\sigma(j)j}                          \\
 		       & =\sum_{\sigma\in S_m}\sgn(\sigma)\Big(\prod_{j=1}^{m-1}x_{\sigma(j)j}\Big)x_{\sigma(m)m}
 		\intertext{nach Voraussetzung gilt: $ x_{\sigma(m)m} = 0 $ für $ \sigma(m)\neq i $ und $x_{\sigma(m)m} =1 $ für $ \sigma(m) = i $,}
 		       & = \sum_{\substack{\sigma\in S_m                                                          \\ \sigma(m)=i}}\sgn(\sigma)\prod_{j=1}^{m-1}x_{\sigma(j)j} \\
 		       & = \sum_{\sigma'\in S_{m-1}}(-1)^{m-i}\sgn(\sigma')\prod_{j=1}^{m-1}x_{\sigma'(j)j}       \\
 		       & = (-1)^{m-i}\det X_{im}.
 	\end{align*}
 	Im vorletzten Schritt werden die Transpositionen berücksichtigt die notwendig sind, um die nun "`gestrichene"' $i$-te Zeile ans Ende zu verschieben.

 	Ausmultiplizieren des o.B.d.A. $ m $-ten Eintrags in einer alternierenden $ m $-Form $ \omega\in \Lambda^mV^* $ liefert also
 	\begin{align*}
 		\omega(v_1,\dots,v_m) & = \sum_{i=1}^{m}\omega(v_1,\dots,v_{m-1},b_i)x_{im}              \\
 		                      & = \omega(b_1,\dots,b_m)\sum_{i=1}^{m}(-1)^{m-1}x_{im}\det X_{im}
 	\end{align*}
 	und damit die Behauptung, da $ \omega(b_1,\dots,b_m)\neq 0 $ angenommen werden kann (siehe unten).

 	Da wegen $ \sgn(\sigma^{-1})=(\sgn(\sigma))^{-1} = \sgn(\sigma) $
 	\begin{align*}
 		\sum_{\sigma\in S_m}\sgn(\sigma)\prod_{j=1}^{m}x_{j\sigma(j)}
 		  & =\sum_{\sigma\in S_m} \sgn(\sigma^{-1})\prod_{j=1}^{m}x_{\sigma^{-1}(j)j} \\
 		  & =\sum_{\sigma^{-1}\in S_m} \sgn(\sigma^{-1})\prod_{j=1}^{m}x_{\sigma(j)j} \\
 		  & =\sum_{\sigma\in S_m} \sgn(\sigma)\prod_{j=1}^{m}x_{\sigma(j)j}
 	\end{align*}
 	\begin{Definition}[Transponierte Matrix]
 		gleicht die Determinante einer Matrix $ X $ der ihrer \emph{Transponierten}:
 		\[
 			\det X^t = \det X \quad\text{mit}\quad X^t := (x_{ji})_{i,j\in \{i,\dots, m \}}
 		\]
 	\end{Definition}
 	Damit gilt der Laplacesche Entwicklungssatz auch für die Entwicklung nach einer Zeile von $ X $, anstelle nach einer Spalte, wie oben.

 	Eine andere Möglichkeit zur Bestimmung von $ \det X $ liefert das Gausssche Eliminationsverfahren (hier mit elementaren Spaltenumformungen; es wird von rechts multipliziert), da (vgl. oben)
 	\begin{align*}
 		\det XD_i    & = d\cdot \det X \Leftrightarrow \det D_iX^t = d\cdot \det X^t \\
 		\det XT_{ij} & = -\det X  \Leftrightarrow T_{ij}X^t = -\det X^t              \\
 		\det XS_{ij} & = \det X  \Leftrightarrow S_{ij}X^t = \det X^t
 	\end{align*}
 	\paragraph{Bemerkung}
 		Diese "`Rechenmethoden"' sind von historischer Bedeutung, manchmal sind sie theoretisch praktisch, aber von beschränkter praktischer Bedeutung (seit man Computer hat).

%VO23-2016-01-14
 \subsection{Beispiel \& Definition (Blockmatrix)}
 	\begin{Definition}[Blockmatrix]
 		Für eine Blockmatrix
 		\[
 			X =
 			\begin{pmatrix}
 				X_{11} & X_{12} \\ 0 & X_{22}
 			\end{pmatrix}
 		\]
 		mit
 		\[
 			X_{11}\in K^{m\times m},\ X_{12}\in K^{m\times n},\ X_{22}\in K^{n\times n}
 		\]
 		gilt
 		\[
 			\det X = \det X_{11} \cdot \det X_{22}
 		\]
 		Beweis in Übung.
 	\end{Definition}
 \subsection{Beispiel \& Definition (Vandermonde-Determinante)}
 	\begin{Definition}[Vandermonde-Determinante]
 		Für $ x_1,\dots, x_k\in K $ hat die \emph{Vandermonde-Matrix}
 		\[
 			X= \big(x_i^{k-j}\big)_{i,j\in \{1,\dots, k \}} =
 			\begin{pmatrix}
 				x_1^{k-1} & x_1^{k-2} & \cdots & x_1^0  \\
 				x_2^{k-1} & x_2^{k-2} & \cdots & x_2^0  \\
 				\vdots    & \vdots    & \ddots & \vdots \\
 				x_k^{k-1} & x_k^{k-2} & \cdots & x_k^0
 			\end{pmatrix}
 			\in K^{k\times k}
 		\]
 		die \emph{(Vandermonde-)Determinante}:
 		\[
 			\det X = \det \big(x_i^{k-j}\big)_{i,j = 1,\dots, k} = \prod_{i<j}x_i-x_j
 		\]

 		Denn:

 		Für $ k=2 $ gilt
 		\[
 			\det
 			\begin{pmatrix}
 				x_1 & 1 \\x_2&1
 			\end{pmatrix}
 			= x_1-x_2
 		\]
 		also ist die Induktionsvoraussetzung gegeben.

 		Für $ k>2 $ gilt
 		\begin{align*}
 			&XS_{21}(-x_k)\cdots S_{k(k-1)}(-x_k) \\
 			&=
 			\begin{pmatrix}
 			x_1^{k-1} &\cdots & x_1 & 1\\
 			\vdots & & \vdots & \vdots\\
 			x_k^{k-1} & \cdots & x_k & 1
 			\end{pmatrix}
 			\begin{pmatrix}
 			1 & 0 &   & \\
 			-x_k & 1 &  & \\
 			&  & \ddots & \\
 			& & & 1
 			\end{pmatrix}
 			\dots
 			\begin{pmatrix}
 			1 &  &   & \\
 			& \ddots&  &\\
 			&   &1 & 0 \\
 			& & -x_k & 1
 			\end{pmatrix}
 			\\
 			&=
 			\begin{pmatrix}
 			x_1^{k-1}-x_1^{k-2}x_k         & x_1^{k-2}-x_1^{k-3}x_k         & \cdots & x_1-x_k     & 1      \\
 			\vdots                         & \vdots                         &        & \vdots      & \vdots \\
 			x_{k-1}^{k-1}-x_{k-1}^{k-2}x_k & x_{k-1}^{k-2}-x_{k-1}^{k-3}x_k & \cdots & x_{k-1}-x_k & 1      \\
 			0                              & 0                              & \cdots & 0           & 1
 			\end{pmatrix}
 			\\
 			&=
 			\begin{pmatrix}
 			(x_1-x_k)x_1^{k-2}             & (x_1-x_k)x_1^{k-3}             & \cdots & x_1-x_k     & 1      \\
 			\vdots                         & \vdots                         &        & \vdots      & \vdots \\
 			(x_{k-1}-x_k)x_{k-1}^{k-2}     & (x_{k-1}-x_k)x_{k-1}^{k-3}     & \cdots & x_{k-1}-x_k & 1      \\
 			0                              & 0                              & \cdots & 0           & 1
 			\end{pmatrix}
 		\end{align*}
 		also ist (Laplacescher Entwicklungssatz nach letzter Zeile):
 		\begin{align*}
 			\det X &= 0+(-1)^{k+k}\det
 			\begin{pmatrix}
 			(x_1-x_k)x_1^{k-2}         & (x_1-x_k)x_1^{k-3}         & \cdots & x_1-x_k     \\
 			\vdots                     & \vdots                     &        & \vdots      \\
 			(x_{k-1}-x_k)x_{k-1}^{k-2} & (x_{k-1}-x_k)x_{k-1}^{k-3} & \cdots & x_{k-1}-x_k \\
 			\end{pmatrix}
 			\\
 			&= 1\cdot(x_1-x_k)\cdots (x_{k-1}-x_k)\det
 			\begin{pmatrix}
 			x_1^{k-2} & \cdots & 1 \\
 			\vdots &  & \vdots \\
 			x_{k-1}^{k-2} & \cdots & 1
 			\end{pmatrix}
 			\\
 			&=(x_1-x_k)\cdots(x_{k-1}-x_k)\prod_{\substack{i<j\\ i,j\in\{1,\dots, k-1 \}}}(x_i-x_j)\\
 			&= \prod_{\substack{i<j\\ i,j\in \{1,\dots,k\}}}(x_i-x_j)
 		\end{align*}
 		Also folgt die Behauptung mit Induktion.
 	\end{Definition}
 \subsection{Fortsetzungssatz für Determinantenformen}
 	\begin{Satz}[Fortsetzungssatz für Determinantenformen]
 		Ist $ (b_i)_{i\in \{1,\dots,n\}} $ eine Basis von $ V $ (also $ \dim V = n $) und $ d\in K $, so gilt:
 		\[
 			\exists! \omega\in\Lambda^nV^*:\omega(b_1,\dots,b_n) = d
 		\]
 	\end{Satz}
 	\paragraph{Beweis}
 		Eindeutigkeit folgt aus der Leibniz-Formel und der Tatsache, dass $ V=[(b_i)_{i\in\{1,\dots,n\}}] $.

 		Existenz: Gegeben sind eine Basis $ (b_i)_{i\in \{1,\dots, n\}} $ von $ V $ und $ d\in K $. Wir definieren $ \omega $ durch die Leibniz-Formel:
 		\[
 			\omega: \overbrace{V\times \dots \times V}^{n\text{-mal}} \to K,\ \omega(v_1,\dots,v_n):=d\cdot\det X
 		\]
 		wobei $ X\in K^{n\times n} $ die Koeffizientenmatrix für die Basisdarstellung der $ (v_i) $ ist,
 		\[
 			(v_1,\dots,v_n)=(b_1,\dots,b_n)\cdot X.
 		\]
 		Dann gilt:
 		\begin{itemize}
 			\item $ \omega $ ist wohldefiniert, da $ (b_i)_{i\in \{i,\dots,n\}} $ linear unabhängig ist, womit die Koeffizienten $ x_{ij},\ i=1,\dots,n $ für jedes $ j=1,\dots,n $ eindeutig sind.
 			\item $ \omega $ ist $ n $-Form, d.h.
 			      \[
 			      	\forall j=1,\dots,n: v_j\mapsto d\cdot\det X
 			      \]
 			      ist linear; offensichtlich!
 			\item $ \omega $ ist alternierend, d.h.
 			      \[
 			      	\omega(v_1,\dots,v_n)=0\text{ falls } v_i = v_j \text{ für }i\neq j;
 			      \]
 			      ist aber $ v_i = v_j $ für ein Paar $ (i,j) $ mit $ i\neq j $, so gilt:
 			      \[
 			      	\sgn(\sigma)x_{\sigma(1)1}\cdots x_{\sigma(n)n} + \sgn(\tau_{ij}\circ \sigma) x_{(\tau_{ij}\circ \sigma)(1)1}\cdots x_{(\tau_{ij}\circ \sigma)(n)n} = 0,
 			      \]
 			      denn
 			      \[
 			      	\sgn(\tau_{ij}\circ \sigma) = -\sgn(\sigma) \quad\text{und}\quad x_{(\tau_{ij}\circ\sigma)(1)1}\cdots x_{(\tau_{ij}\circ\sigma)(n)n} = x_{\sigma(1)1}\cdots x_{\sigma(n)n}
 			      \]
 			      da $ v_i = v_j $.

 			      Damit folgt:
 			      \[
 			      	\sum_{\sigma\in S_n}\sgn(\sigma)x_{\sigma(1)1}\cdots x_{\sigma(n)n}=0,
 			      \]
 			      da mit $ \sigma\in S_n $ auch $ \tau_{ij}\circ\sigma\in S_n $ in der Summe vorkommt.
 		\end{itemize}
 \subsection{Korollar}
 	\begin{Korollar}[Dimension der Determinantenform]
 		Ist $ \dim V = n $, so ist $ \dim\Lambda^nV^* = 1 $. Beweis in der Übung.
 	\end{Korollar}
 \subsection{Korollar}
 	Ist $ \omega(v_1,\dots,v_n)=0 $ für eine Determinantenform $ \omega\in\Lambda^nV^*\setminus\{0\}, $ so ist die Familie $ (v_i)_{i\in \{1,\dots, n\}} $ linear abhängig.
 	\paragraph{Beweis}
 		Ist $ \omega\neq 0 $, so existiert eine Basis $ (b_i)_{i=1,\dots,n} $ von $ V $ mit $ \omega(b_1,\dots,b_n)=d\neq 0 $.
 		Annahme: $ (v_i)_{i\in \{1,\dots,n\}} $ ist linear unabhängig, d.h. $ (v_i) $ ist Basis und damit existiert $ Y\in K^{n\times n} $ mit
 		\[
 			(b_1,\dots,b_n)=(v_1,\dots,v_n)Y,
 		\]
 		nach Leibniz-Formel ist damit
 		\[
 			0\neq \omega(b_1,\dots,b_n)=\omega(v_1,\dots,v_n)\cdot \det Y.
 		\]
 		Damit folgt
 		\[
 			\omega(v_1,\dots,v_n)\neq 0 \text{ (und }\det Y \neq 0)
 		\]

 	\paragraph{Bemerkung}
 		Ist also $ \dim V=n $ und sind $ \omega\in \Lambda^nV^* $ und $ (b_i)_{i\in \{1,\dots,n\}} $ eine Familie in $ V $, so folgt aus zwei der folgenden Aussagen die dritte:
 		\begin{enumerate}[(i)]
 			\item $ \omega\neq 0 $
 			\item $ \omega(b_1,\dots,b_n)\neq 0 $
 			\item $ (b_i)_{i\in\{1,\dots,n\}} $ ist Basis von $ V $.
 		\end{enumerate}
 	\paragraph{Bemerkung}
 		Weiter folgt damit: Sind $ f\in \End(V),\ (b_i)_{i\in \{1,\dots,n\}} $ Basis von $ V $ und $ \omega\in \Lambda^nV^*\setminus\{0\}, $ so gilt:
 		\[
 			f\in \mathrm{Gl}(V)\Leftrightarrow \omega(f(b_1),\dots,f(b_n))\neq 0
 		\]
 		bzw.
 		\[
 			\mathrm{Gl}(V)=\{f\in \End(V)\mid \omega(f(b_1),\dots,f(b_n))\neq 0\}
 		\]

%VO23-2016-01-14
\section{Äquiaffine Geometrie}
 \subsection{Definition}
 	\begin{Definition}[Parallelotop im affinen Raum]\index{Parallelotop im affinen Raum}
 		Seien $ (A,V,\tau) $ ein $ n $-dimensionaler affiner Raum und $ p_0\in A $; das von einer Familie $ (v_j)_{j\in \{1,\dots,n\}} $ in $ V $ aufgespannte \emph{Parallelotop} oder \emph{Spat} ist die (über dem \emph{abstrakten Würfel} $ \mathbb{Z}_2^n $ indizierte) Familie
 		\[
 			p: \mathbb{Z}_2^n\to A,\ \epsilon \mapsto p_\epsilon := p_0 +\sum_{j=1}^{n}v_j\epsilon_j.
 		\]
 		Ist $ \omega\in\Lambda^nV^*\setminus\{0\} $, so ist das zugehörige \emph{(Spat-)Volumen} von $ (p_\epsilon)_{\epsilon \in \mathbb{Z}_2^n} $
 		\[
 			\vol(p):= \omega(v_1,\dots,v_n).
 		\]
 	\end{Definition}
 	%%% Grafik Parallelogramm
 	\begin{figure}[H]
 		\centering
 		\definecolor{zzttqq}{rgb}{0.6,0.2,0.}
 		\definecolor{uuuuuu}{rgb}{0.26,0.26,0.26}
 		\definecolor{qqqqff}{rgb}{0.,0.,1.}
 		\begin{tikzpicture}[line cap=round,line join=round,>=triangle 45,x=5.0cm,y=3.0cm,]
 			\clip(0.6,0.7) rectangle (2.8,2.35);
 			\fill[color=zzttqq,fill=zzttqq,fill opacity=0.1] (1.,1.) -- (1.5,2.) -- (2.5,2.) -- (2.,1.) -- cycle;
 			\draw [->] (1.,1.) -- (1.5,2.);
 			\draw [->] (1.,1.) -- (2.,1.);

 			\draw [fill=qqqqff] (1.,1.) circle (2.5pt);
 			\draw[color=qqqqff] (0.96,0.94) node {p(0,0)};
 			\draw [fill=uuuuuu] (2.5,2.) circle (1.5pt);
 			\draw[color=uuuuuu] (2.6,2.0) node {p(1,1)};
 			\draw [fill=uuuuuu] (2.,1.) circle (1.5pt);
 			\draw[color=uuuuuu] (2.,1) node {p(1,0)};
 			\draw [fill=uuuuuu] (1.5,2.) circle (1.5pt);
 			\draw[color=uuuuuu] (1.58,2.05) node {p(0,1)};
 			\draw[color=black] (1.2,1.55) node {$v_2$};
 			\draw[color=black] (1.5,1) node {$v_1$};

 		\end{tikzpicture}
 	\end{figure}
 	%%%%% ENDE Grafik Parallelogramm %%%%%
 \subsection{Bemerkung \& Definition}
 	\begin{Definition}[Parallelogramm/Flächeninhalt]
 		Im Falle $ n=2 $ heißt ein Parallelotop $ p $ auch \emph{Parallelogramm}, sein Spatvolumen auch sein \emph{Flächeninhalt}. Der Flächeninhalt ist \emph{orientiert}:
 		Mit
 		\[
 			\epsilon = (\epsilon_j)_{j\in \{1,2\}}\cong (\epsilon_1,\epsilon_2)
 		\]
 		und
 		\[
 			v_1 = p_{(1,0)}-p_{(0,0)}\text{ und } v_2 = p_{(0,1)}-p_{(0,0)}
 		\]
 		ändert der Flächeninhalt das Vorzeichen, wenn man die Kantenvektoren vertauscht:
 		\[
 			\vol(p) = \omega(v_1,v_2) = -\omega(v_2,v_1) = \vol(p')
 		\]
 		mit $ p'_\epsilon = p_{\epsilon \circ \tau_{12}} $.
 	\end{Definition}

%VO23-2016-01-19
 	\paragraph{Bemerkung}
 		Für ein Dreieck $ \{a,b,c\} $ wählt man eine Orientierung, e.g. $ (a,b,c) $, und setzt den Flächeninhalt
 		\[
 			F(a,b,c) := \omega(b-a,c-a)\cdot\frac{1}{2}
 		\]
 		d.h. als halben Flächeninhalt des Parallelogramms
 		\[
 			p_{(0,0)} := a,\ p_{(1,0)} := b,\ p_{(0,1)} := c \text{ und } p_{(1,1)} := a(-1)+b\cdot 1+c\cdot 1.
 		\]
 		Dieser Flächeninhalt ist wohldefiniert, d.h. er hängt nur vom Dreieck und der gewählten Orientierung ab -- insbesondere ist für Permutationen $ \sigma $ von $ \{a,b,c\} $ mit $ \sgn(\sigma) = +1 $ der Flächeninhalt gleich dem ursprünglichen. (Siehe Aufgabe)
 	\paragraph{Bemerkung}
 		Ist $ A $ eine affine Ebene mit Flächenmessung $ \vol $ und $ \{a,b,c\} $ ein nicht-de"-ge"-nerier"-tes Dreieck, also ein baryzentrisches Bezugssystem, so gilt (vgl. Cramersche Regel)
 		\[
 			\forall s\in A: s=a\cdot \frac{F(s,b,c)}{F(a,b,c)}+ b\cdot\frac{F(a,s,c)}{F(a,b,c)} +c\cdot\frac{F(a,b,s)}{F(a,b,c)}
 		\]
 		Diese Flächenformel für die baryzentrischen Koordinaten eines Punktes $ s $ ist unabhängig von der (gewählten) Flächenmessung, da sich verschiedene Flächenmessungen nur um einen Faktor unterscheiden (der unabhängig vom Dreieck ist): $ \dim \Lambda^2V^*=1 $
 		%%% Grafik Dreieck mit Punkt s %%%%%
 		\begin{figure}[H]\centering
 			\definecolor{wwqqcc}{rgb}{0.4,0.,0.8}
 			\definecolor{ffzztt}{rgb}{1.,0.6,0.2}
 			\definecolor{qqqqff}{rgb}{0.,0.,1.}
 			\begin{tikzpicture}[line cap=round,line join=round,>=triangle 45,x=2.0cm,y=1.5cm]
 				\fill[color=ffzztt,fill=ffzztt,fill opacity=0.1] (1.,1.) -- (2.64,3.96) -- (3.92,0.42) -- cycle;
 				\fill[color=wwqqcc,fill=wwqqcc,fill opacity=0.1] (2.64,3.96) -- (2.7,1.6) -- (3.92,0.42) -- cycle;
 				\draw [color=wwqqcc] (2.64,3.96)-- (2.7,1.6);
 				\draw [color=wwqqcc] (2.7,1.6)-- (3.92,0.42);
 				\draw [color=wwqqcc] (3.92,0.42)-- (2.64,3.96);
 				\draw (1.,1.)-- (2.7,1.6);
 				\draw [fill=qqqqff] (1.,1.) circle (2.5pt);
 				\draw[color=qqqqff] (0.8,1) node {$a$};
 				\draw [fill=qqqqff] (2.64,3.96) circle (2.5pt);
 				\draw[color=qqqqff] (2.7,4.2) node {$b$};
 				\draw [fill=qqqqff] (3.92,0.42) circle (2.5pt);
 				\draw[color=qqqqff] (4.2,0.4) node {$c$};
 				\draw [fill=qqqqff] (2.7,1.6) circle (2.5pt);
 				\draw[color=qqqqff] (2.8,1.84) node {$s$};
 			\end{tikzpicture}
 		\end{figure}
 		%%%% Ende Grafik Dreieck mit Punkt s %%%%%
 \subsection{Definition}
 	\begin{Definition}[Äquiaffine Transformation]
 		Eine affine Abbildung $ \alpha:A\to A' $ zwischen AR $ A $ und $ A' $ mit Volumenmessungen $ \vol $ und $ \vol' $ heißt \emph{volumentreu}, falls für alle Parallelotope $ p $ in $ A $ gilt
 		\[
 			\vol'(\alpha\circ p)=\vol(p)
 		\]
 		Eine \emph{äquiaffine Transformation} ist eine volumentreue Affinität.
 	\end{Definition}
 	\paragraph{Bemerkung}
 		Ist $ \alpha:A\to A' $ affin mit linearem Anteil $ \lambda:V \to V' $ und $ p $ ein von einer Familie $ (v_j)_{j\in\{1,\dots,n\}} $ von Vektoren aufgespanntes Parallelotop in $ A $, so ist
 		\begin{align*}
 			p':= \alpha\circ p:\mathbb{Z}_2^n\to A',\ \epsilon \mapsto p'_\epsilon & = \alpha(p_0)+\lambda\Big(\sum_{j=1}^{n}v_j\epsilon_j\Big) \\
 			                                                                       & = \alpha(p_0)+\sum_{j=1}^{n}\lambda(v_j)\epsilon_j
 		\end{align*}
 		ein von der Familie $ (\lambda(v))_{j\in \{1,\dots,n\}} $ von Vektoren in $ V' $ aufgespanntes Parallelotop in $ A' $. Damit ist $ \vol'(\alpha\circ p) $ ein sinnvoller Ausdruck, also der Begriff "`volumentreu"' sinnvoll für affine Abbildungen.
 	\paragraph{Bemerkung}
 		Offenbar bilden die volumentreuen Affinitäten eine Gruppe: eine Untergruppe der affinen Gruppe (nach Untergruppenkriterium).
 \subsection{Äquiaffine Geometrie}
 	\begin{Definition}[Äquiaffine Geometrie]
 		Ist $ (A,V,\tau,\vol) $ ein mit einem Spatvolumen versehener Affiner Raum, so bestimmt die auf $ A $ operierende Gruppe der volumentreuen Affinitäten (der äquiaffinen Transformationen) eine \emph{äquiaffine Geometrie}.
 	\end{Definition}
 \subsection{Bemerkung \& Definition}
 	\begin{Definition}[Volumenverzerrung]\index{Volumenverzerrung}
 		Ist $ \alpha: A\to A $ Affinität eines AR $ A $ mit Volumenmessung $ \vol $, so gibt es genau eine Zahl $ \delta(\alpha)\in K^{\times} $, sodass für jedes Parallelotop $ p $ in $ A $ gilt
 		\[
 			\vol(\alpha\circ p) = \delta(\alpha)\vol(p),
 		\]
 		wobei $ \delta(\alpha) $ die \emph{Volumenverzerrung} genannt wird.

 		Die Volumenverzerrung $ \delta(\alpha) $ hängt nur vom linearen Anteil $ \lambda\in \End(V) $ ab: für ein von einer Basis $ (v_j)_{j\in \{1,\dots,n\}} $ von $ V $ aufgespanntes Parallelotop ist
 		\[
 			\delta(\alpha) = \frac{\omega(\lambda(v_1),\dots,\lambda(v_n))}{\omega(v_1,\dots,v_n)}
 		\]
 	\end{Definition}
 \subsection{Definition}
 	\begin{Definition}[Determinante eines Endomorphismus]\index{Determinante eines Endomorphismus}
 		Seien $ f\in\End(V) $, wobei $ \dim V=n $, und $ \omega\in \Lambda^nV^*\setminus \{0\} $. Dann heißt
 		\[
 			\det f:= \frac{f^*\omega}{\omega}
 		\]
 		die \emph{Determinante} von $ f $, wobei
 		\[
 			f^*\omega: V^n\to K,\ (v_1,\dots,v_n)\mapsto \omega(f(v_1),\dots,f(v_n)).
 		\]
 	\end{Definition}
 	\paragraph{Bemerkung}
 		Offenbar ist $ f^*\omega\in \Lambda^nV^*$; wegen $ \dim \Lambda^nV^*=1 $ ist $\Lambda^nV^* = [\omega]$, d.h.
 		\[
 			\exists! x\in K: f^*\omega = \omega \cdot x \quad\text{($(\omega) $ ist Basis von $ \Lambda^nV^* $)}
 		\]
 		dieses $ x $ ist die Determinante von $ f $, also $ \det f = x $.

 		Alternativ: Die Abbildung
 		\[
 			B\mapsto  \frac{f^*\omega(B)}{\omega(B)}\in K
 		\]
 		ist konstant ($ \equiv x $, unabhängig von der Basis $ B $).
 	\paragraph{Bemerkung}
 		Da $ f^*(\omega x) = (f^*\omega)x $ für $ x\in K $, liefert jedes $ \omega \in \Lambda^nV^*\setminus\{0\} $ die gleiche Determinante $ \det f $: die Determinante bzw. Volumenverzerrung ist unabhängig von der gewählten Volumenform $ \omega\in \Lambda^nV^*\setminus \{0\} $, d.h. der "`Referenzvolumenmessung"'.
 \subsection{Determinantenmultiplikationssatz}
 	\begin{Satz}[Determinantenmultiplikationssatz]\index{Determinantenmultiplikationssatz}
 		Für $ f,g\in \End(V) $ gilt
 		\[
 			\det(f\circ g) = \det f \cdot \det g.
 		\]
 	\end{Satz}
 	\paragraph{Beweis}
 		Mit $ \omega\in\Lambda^nV^*\setminus\{0\} $ berechnet man
 		\[
 			(f\circ g)^*\omega = g^*(f^*\omega) = g^*(\omega\cdot \det f) = (g^*\omega)\det f = \omega \cdot \det g\cdot \det f
 		\]

%VO24-2016-01-21
 	\paragraph{Wiederholung \& Bemerkung}
 		Sind $ \omega\in \Lambda^nV^*\setminus \{0\} $ und $ B $ eine Basis von $ V $, so gilt
 		\[
 			\mathrm{Gl}(V) = \{f\in \operatorname{End}(V)\mid 0\neq \omega(f(B))=f^*\omega(B)\}.
 		\]
 		Damit folgt also
 		\[
 			\mathrm{Gl}(V) = \{f\in\operatorname{End}(V)\mid \det f \neq 0\}
 		\]
 		Mit dem Determinantenmultiplikationssatz folgt dann:
 \subsection{Korollar}
 	\begin{Korollar}
 		\[
 			\det : \mathrm{Gl}(V)\to (K^\times,\cdot)
 		\]
 		ist Gruppenhomomorphismus.
 	\end{Korollar}
 \subsection{Korollar \& Definition}
 	\begin{Definition}[Spezielle lineare Gruppe]\index{Spezielle lineare Gruppe}
 		\[
 			\det\text{}^{-1}(\{1\}) = \{f\in \mathrm{Gl}(V)\mid \det f = 1\} \subset \mathrm{Gl}(V)
 		\]
 		liefert eine Untergruppe von $ \mathrm{Gl}(V) $, die \emph{spezielle lineare Gruppe}
 		\[
 			\mathrm{Sl}(V):= \{f\in \mathrm{Gl}(V)\mid \det f = 1 \}
 		\]
 	\end{Definition}
 	\paragraph{Beweis}
 		Für $ g\in \mathrm{Gl}(V) $ gilt nach DMS:
 		\begin{gather*}
 			1 = \det \id_V = \det(g^{-1}\circ g) = \det g^{-1}\cdot \det g \\
 			\Rightarrow \det g^{-1} = (\det g)^{-1}
 		\end{gather*}
 		damit folgt
 		\[
 			g\in \mathrm{Sl}(V)\Rightarrow g^{-1} \in \mathrm{Sl}(V),
 		\]
 		also mit DMS
 		\[
 			\forall f,g\in \mathrm{Sl}(V):g^{-1}\circ f\in \mathrm{Sl}(V),
 		\]
 		d.h. $ \mathrm{Sl}(V)\subset \mathrm{Gl}(V) $ ist eine Untergruppe nach Untergruppenkriterium.
 \subsection{Korollar}
 	Eine Affinität ist genau dann eine äquiaffine Transformation, wenn ihr linearer Anteil eine spezielle lineare Transformation ist.
 	\paragraph{Beweis}
 		Ist $ \lambda \in \End(V) $ linearer Anteil der Affinität $ \alpha:A\to A $ eines AR $ A  $ über $ V $, und ist $ p $ ein von einer Basis $ B=(b_j)_{j\in \{1,\dots,n\}} $ von $ V $ aufgespanntes Parallelotop, so gilt
 		\[
 			(\alpha\circ p)_\epsilon = \alpha(p_0)+\sum_{j=1}^{n}\lambda(b_j)\epsilon_j,
 		\]
 		also für eine durch $ \omega\in \Lambda^nV^*\setminus \{0\} $ gegebene Volumenmessung
 		\[
 			\frac{\vol(\alpha\circ p)}{\vol(p)} = \frac{\omega(\lambda(b_1),\dots,\lambda(b_n))}{\omega(b_1,\dots,b_n)}=\det \lambda,
 		\]
 		d.h. $ \alpha $ ist volumentreu genau dann, wenn $ \lambda \in \mathrm{Sl}(V) $. ($ \rightarrow $ vgl. Idee der Definition $ \det f $)
 	\paragraph{Bemerkung}
 		Dies liefert einen alternativen Beweis, dass die äquiaffinen Transformationen eine Gruppe (Untergruppe der Affinitäten) bilden:

 		Der lineare Anteil einer Komposition von Affinitäten ist die Komposition ihrer linearen Anteile -- und $ \mathrm{Sl}(V) $ ist eine Gruppe (Untergruppe von $ \mathrm{Gl}(V) $).
 	\paragraph{Beispiel}
 		Ist $ \lambda = \id_V + w\cdot \psi $ mit $ \psi\in V^* $ und $ w\in \ker \psi $ linearer Anteil einer Scherung
 		\[
 			\sigma:A\to A,\ o+v\mapsto \sigma(o+v) = o+\lambda(v),
 		\]
 		wobei $ w\cdot\psi\neq 0 $, so wähle eine Basis $ B = (b_j)_{j\in\{1,\dots,n\}}$ von $ V $ mit
 		\[
 			w=b_1 \text{ und }  \ker \psi = [(b_j)_{j\in \{1,\dots,n-1\}}].
 		\]
 		Dann ist
 		\[
 			\xi_B^B(\lambda) =
 			\begin{pmatrix}
 				1      & 0      & \cdots & \psi(b_n) \\
 				0      & 1      & \ddots & \vdots    \\
 				\vdots & \ddots & \ddots & 0         \\
 				0      & \cdots & 0      & 1
 			\end{pmatrix}
 			= S_{1n}(\psi(b_n))
 		\]
 		und mit einer Determinantenform $ \omega\in \Lambda^nV^*\setminus\{0\} $
 		\[
 			\det \lambda = \frac{\lambda^*\omega(B)}{\omega(B)} = \frac{\omega(\lambda(B))}{\omega(B)} = \frac{\omega(BS_{1n}(\psi(b_n)))}{\omega(B)} = \frac{\omega(B)}{\omega(B)} = 1.
 		\]
 		Jede Scherung ist also äquiaffine Transformation.

 		%-------------------Begin Scherung ist äquiaffine Transformation ----------------
 		\begin{figure}[H]\centering
 			\tdplotsetmaincoords{0}{0} %-27
 			\begin{tikzpicture}[yscale=1,tdplot_main_coords]

 				\def\xstart{0} %x Koordinate der Startposition der Grafik
 				\def\ystart{0} %y Koordinate der Startposition der Grafik
 				\def\myscale{1.1} %ändert die Größe der Grafik (Skalierung der Grafik)

 				\def\xstartdraw{(\xstart + 1.0)} %xKoordinate des Referenzstartpunktes (in dieser Zeichnung: a)
 				\def\ystartdraw{(\ystart + 0.6)}%yKoordinate des Referenzstartpunktes (in dieser Zeichnung: a)

 				\def\balkenhoehe{(5.3)}% Länge des vertikalen blauen Balkens
 				\def\balkenlaenge{(10)}% Länge des horizontalen blauen Balkens
 				\def\balkenbreite{0.4} %Balkenbreite

 				%---------Begin Balken----------
 				\def\drehwinkel{0}
 				\node (VekV) at ({\xstart+0.7*cos(\drehwinkel)-\balkenbreite*sin(\drehwinkel)},{\ystart+0.5*sin(\drehwinkel)+\balkenbreite*cos(\drehwinkel)})[right, xshift=-8,color=blue] {$V$};
 				\node (AffA) at ({\xstart+(\balkenlaenge-1)*cos(\drehwinkel)},{\ystart+(\balkenlaenge-1)*sin(\drehwinkel)+\balkenbreite*cos(\drehwinkel)})[color=red] {$A^2$};

 				\path[ shade, top color=white, bottom color=blue, opacity=.6]
 				({\xstart},{\ystart},0)  -- ({\xstart - \balkenbreite * cos(\drehwinkel)- (-\balkenbreite+0)*sin(\drehwinkel)},{\ystart - \balkenbreite * sin(\drehwinkel)+ (-\balkenbreite+0)*cos(\drehwinkel)},0)  -- ({\xstart - \balkenbreite * cos(\drehwinkel)- (\balkenhoehe+0.5)*sin(\drehwinkel)},{\ystart - \balkenbreite * sin(\drehwinkel)+ (\balkenhoehe+0.5)*cos(\drehwinkel)},0) -- ({\xstart - 0 * cos(\drehwinkel)- (\balkenhoehe+0)*sin(\drehwinkel)},{\ystart - 0 * sin(\drehwinkel)+ (\balkenhoehe+0)*cos(\drehwinkel)},0) -- cycle;

 				\path[ shade, right color=white, left color=blue, opacity=.6]
 				({\xstart},{\ystart},0)  -- ({\xstart - \balkenbreite * cos(\drehwinkel)- (-\balkenbreite+0)*sin(\drehwinkel)},{\ystart - \balkenbreite * sin(\drehwinkel)+ (-\balkenbreite+0)*cos(\drehwinkel)},0) --
 				({\xstart + (\balkenlaenge+0.5) * cos(\drehwinkel)- (-\balkenbreite+0)*sin(\drehwinkel)},{\ystart + (\balkenlaenge+0.5) * sin(\drehwinkel)+ (-\balkenbreite+0)*cos(\drehwinkel)},0) --
 				({\xstart + \balkenlaenge * cos(\drehwinkel)},{\ystart + \balkenlaenge * sin(\drehwinkel)},0)--
 				cycle;
 				%---------End Balken----------

 				%Punkte Definition

 				\node (pointol) at ({\xstartdraw},{\ystartdraw}) {};
 				\node (pointor) at ({\xstartdraw+(6.2 *\myscale)},{\ystartdraw+(1.8 *\myscale)}) {};

 				\node (pointo1) at ($(pointol)!0.2!(pointor)$) {};
 				\node (pointo2) at ($(pointol)!0.9!(pointor)$) {};

 				\node (offset) at ($(0.2*\myscale,2.5*\myscale) $) {}; %just an offset

 				\node (pointov) at ($(pointo1) + (offset) $) {};
 				\node (pointol2) at ($(pointol) + (offset) $) {};
 				\node (pointor2) at ($(pointor) + (offset) $) {};

 				\node (pointolambdav) at ($(pointov)!0.5!(pointor2)$) {};

 				\node (pointa1) at ({\xstartdraw},{\ystartdraw}) {};
 				\node (point02) at ({\xstartdraw},{\ystartdraw}) {};


 				%Gerade rot
 				\draw[-,line width=0.2pt,color=red,shorten <=-30pt] (pointor) -- (pointol);
 				\draw[-,line width=0.4pt,color=red,dotted,shorten >=-30pt] (pointor2) -- (pointol2);

 				%Vektoren blau
 				\draw[-{>[scale=1,length=10,width=6]},shorten >=2pt, shorten <=2pt,line width=0.5pt,color=blue] (pointo1) -- (pointo2);
 				\node (pointvekw) at ($(pointo1)!0.5!(pointo2)$) [below,color=blue]{$w$};

 				\draw[-{>[scale=1,length=10,width=6]},shorten >=2pt, shorten <=2pt,line width=0.5pt,color=blue] (pointo1) -- (pointov);
 				\node (pointvekv) at ($(pointo1)!0.5!(pointov)$) [left,color=blue]{$v$};

 				\draw[-{>[scale=1,length=10,width=6]},shorten >=2pt, shorten <=2pt,line width=0.5pt,color=blue] (pointov) -- (pointolambdav);
 				\node (pointvekws) at ($(pointov)!0.4!(pointolambdav)$) [above,color=blue,yshift=2]{$\stackrel{\text{mit } s=\psi(v)}{w\cdot s}$};

 				\draw[-{>[scale=1,length=10,width=6]},shorten >=2pt, shorten <=2pt,line width=0.5pt,color=blue] (pointo1) -- (pointolambdav);
 				\node (pointa1b1v) at ($(pointo1)!0.5!(pointolambdav)$) [right,color=blue]{$\lambda(v)$};

 				%Abbildung sigma
 				\draw [-{>[scale=1,length=10,width=6]},shorten >=8pt, shorten <=8pt,line width=0.4pt,color=blue!20!red!80] (pointov) to [bend right=15] (pointolambdav);
 				\node (pointveksigma) at ($(pointov)!0.5!(pointolambdav)$) [below,color=blue!20!red!80,yshift=-7]{$\sigma$};

 				%Punkte malen
 				\draw[fill,color=white] (pointo1) circle [x=1cm,y=1cm,radius=0.18];
 				\draw[fill,color=red] (pointo1) circle [x=1cm,y=1cm,radius=0.08]node[below, xshift=0, yshift=0]{$o$};
 				\draw[fill,color=red] (pointov) circle [x=1cm,y=1cm,radius=0.08]node[above, xshift=-20, yshift=-5]{$o+v$};

 				\draw[fill,color=white] (pointolambdav) circle [x=1cm,y=1cm,radius=0.18];
 				\draw[fill,color=red] (pointolambdav) circle [x=1cm,y=1cm,radius=0.08]node[below, xshift=55, yshift=5]{$\sigma(o+v)=o+\lambda(v)$};

 			\end{tikzpicture}
 		\end{figure}
 		%-------------------End Scherung ist äquiaffine Transformation ----------------

 	\paragraph{Bemerkung (Dreischerungssatz)}
 		Jede äquiaffine Transformation einer affinen Ebene (mit Flächenmessung) ist Komposition von (höchstens drei) Scherungen.

 		Mit dem Fortsetzungssatz für affine Abbildungen kann der Dreischerungssatz rein konstruktiv bewiesen werden.
 \subsection{Lemma}
 	Sind $ f\in \End(V) $ und $ B $ eine Basis von $ V $, so gilt
 	\[
 		\det f = \det \xi_B^B(f)
 	\]
 	\paragraph{Beweis}
 		Mit Leibniz-Formel (LF): Für $ \omega\in \Lambda^nV^*\setminus\{0\} $ mit $ n=\dim V $, gilt
 		\[
 			f^*\omega(B) = \omega(f(B)) \overset{\text{Def. }\xi}{=} \omega(B\cdot \xi_B^B(f))
 			\overset{\text{LF}}{=} \omega(B)\cdot \det \xi_B^B(f).
 		\]
 	\paragraph{Achtung:}
 		Im Allgemeinen ist für unterschiedliche Basen $ B,C $ von $ V $
 		\[
 			\det f \neq \det \xi_B^C(f).
 		\]
 		Zum Beispiel: Sei $ f\in \mathrm{Gl}(V) $ und $ B $ eine beliebige Basis, dann ist $ C:=f(B) $ auch eine Basis -- und
 		\[
 			\xi_B^C(f) = E_n \Rightarrow \det \xi_B^C(f) = 1
 		\]
 		Für $ f=2\cdot \id_V $ etwa gilt also
 		\[
 			\det f = 2^n \cdot 1\neq 1
 		\]
 \subsection{Buchhaltung}
 	Aus obigem Lemma folgt auch:
 	Für $ X\in K^{n\times n} $ gilt
 	\[
 		\xi_E^E(f_X) = X \Rightarrow \det X = \det \xi_E^E(f_X) = \det f_X.
 	\]
 	Der DMS für Endomorphismen liefert also einen \emph{Determinantenmultiplikationssatz für Matrizen}: Für $ X,Y\in K^{n\times n} $ gilt
 	\[
 		\det XY = \det f_{XY} = \det(f_X\circ f_Y) = \det f_X\cdot \det f_Y = \det X\cdot \det Y.
 	\]
 	Ebenso lassen sich nun andere Tatsachen auf die Determinante für Matrizen von der für Endomorphismen übertragen. Insbesondere definiert man
 	\[
 		\mathrm{Sl}(n) := \{X\in K^{n\times n}\mid \det X=1 \} = \{X\in K^{n\times n}\mid f_X\in \mathrm{Sl}(K^n)\},
 	\]
 	und nennt sie die \emph{spezielle lineare Gruppe in $n$ Variablen}. $ \mathrm{Sl}(n) $ ist Untergruppe von $ \mathrm{Gl}(n) $.
 	\paragraph{Neuer Begriff der Äquivalenz}
 		Definiert man zwei Matrizen $ X,X' \in K^{n\times n}$ als \emph{äquivalent},
 		\[
 			X\sim X' :\Leftrightarrow \exists P\in \mathrm{Gl}(n): X' =PXP^{-1},
 		\]
 		so erhält man:
 		\[
 			X\sim X' \Rightarrow \det X = \det X'
 		\]
 		Nämlich: Sind $ f\in \End(V) $ und $ B $ und $ B'=BP^{-1} $ (mit $ P\in \mathrm{Gl}(n) $) Basen von $ V $, so gilt
 		\[
 			\xi_{B'}^{B'}(f) = \xi_{B}^{B'}(\id_V)\cdot \xi_B^B(f)\cdot \xi_{B'}^{B}(\id_V) =
 			\xi_{B}^{B'}(\id_V)\cdot \xi_B^B(f)\cdot \left(\xi^{B'}_{B}(\id_V)\right)^{-1}
 		\]
 		mit
 		\[
 			\xi_B^{B'}(\id_V) = P \text{, da } B' = \id_V(B') = BP^{-1}
 		\]
 		Insbesondere gilt also für $ X,X' = PXP^{-1}\in K^{n\times n} $ mit $ P\in \mathrm{Gl}(n) $:
 		\begin{align*}
 			\det X' = \det f_{X'} & = \det f_P \circ f_X \circ f_{P^{-1}}                             \\
 			                      & =\det f_P \circ f_X \circ (f_P)^{-1}                              \\
 			                      & = \det f_P \cdot \det f_X \cdot \det f_P^{-1} = \det f_X = \det X
 		\end{align*}
 	\paragraph{Achtung}
 		Dies ist ein zweiter Begriff der Äquivalenz von Matrizen -- für Darstellungsmatrizen von Endomorphismen, im Gegensatz zur Äquivalenz für Darstellungsmatrizen von Homomorphismen (im Allgemeinen nicht quadratisch).
%VO1-2016-03-01
\section{Polynome \& Polynomfunktionen}
	Warum? (Vielleicht eher "`Algebra"' -- allgemein -- als "`lineare"' Algebra) Wichtig: das charakteristische Polynom eines Endomorphismus -- wichtiges Hilfsmittel im Kontext der Struktursätze.
\paragraph{Beispiel}
	Wir definieren Polynomfunktionen $ p,q: K\to K $ eines Körpers $ K $ in sich durch 
		\begin{align*}
		p:\ & K\to K,\ x\mapsto p(x):= 1+x+x^2\\
		q:\ & K\to K,\ x\mapsto q(x):= 1
		\end{align*}
	Falls $ K=\Z_2 $ so gilt dann
		\begin{align*}
		&\forall x\in K: x(x+1)=0\\
		\Rightarrow\ &\forall x\in K: p(x) = q(x)
		\end{align*}
	d.h., unterschiedliche "`Polynome"' liefern die gleiche Polynomfunktion: Koeffizientenvergleich funktioniert nicht.
\paragraph{Wiederholung}
	Auf dem Folgenraum $ K^\mathbb{N} $ betrachten wir die Familie $ (e_k)_{k\in \mathbb{N}} $ mit
		\[ e_k :\mathbb{N}\to K,\ j\mapsto e_k(j):= \delta_{jk} \]
	Wir wissen: $ (e_k)_{k\in \mathbb{N}} $ ist linear unabhängig, aber kein Erzeugendensystem\footnote{Die konstante 1-Folge ist nicht im Spann enthalten!}:
		\[ \forall k\in \mathbb{N}: e_k \notin [(e_j)_{j\neq k}] \text{ und }
		[(e_j)_{j\in\mathbb{N}}]\neq K^{\mathbb{N}}\]
	Insbesondere gilt:
		\[ \forall x\in [(e_j)_{j\in \mathbb{N}}]\ \exists n\in \mathbb{N}\ \forall k>n : x_k = 0 \]
\subsection{Idee \& Definition} \index{Polynom}\index{Cauchyprodukt}
	\begin{Definition}[Cauchyprodukt]
		Wir fassen ein Polynom als (endliche) Koeffizientenfolge auf,
		\[ \sum_{k=0}^{n} t^ka_k \cong
		\sum_{k\in \mathbb{N}}e_ka_k \text{ mit } a_k = 0 \text{ für } k>n \]
	und führen darauf das \emph{Cauchyprodukt} (vgl. Analysis) als Multiplikation ein:
		\[ (a_k)_{k\in \mathbb{N}} \odot (b_k)_{k\in \mathbb{N}} := (c_k)_{k\in \mathbb{N}} \]
	wobei
		\[ c_k := \sum_{j=0}^{k}a_jb_{k-j}. \]
	\end{Definition}
	Insbesondere gilt damit
		\begin{gather*}
		\forall j,k\in \mathbb{N}: e_j \odot e_k = e_{j+k}
		\Rightarrow \forall k\in \mathbb{N}:
			\begin{cases}
				e_0 \odot e_k = e_k\\
				e_1^k = \underset{k \text{ mal}}{\underbrace{e_1 \odot \cdots \odot e_1}} = e_k
			\end{cases}
		\end{gather*}
	Mit $ 1:= e_0,\ t:= e_1 $ und $ t^0 := 1 $, wie üblich, liefert dies:
		\[ \sum_{k=0}^{n}t^ka_k = \sum_{k\in \mathbb{N}}e_ka_k \in [(e_k)_{k\in \mathbb{N}}]\subset K^\mathbb{N} \]
		
\subsection{Definition}\index{Polynom!-algebra}\index{Polynom!Grad}\index{Polynom!normiertes}
		\begin{Definition}[Polynomalgebra]
			\[ K[t] := ([(e_k)_{k\in \mathbb{N}}],\odot) ,\]
	mit dem Cauchyprodukt $ \odot $, ist die \emph{Polynomalgebra} über dem Körper $ K $; die Elemente von $ K[t] $,
		\[ p(t) = \sum_{k=0}^{n}t^ka_k = \sum_{k\in\mathbb{N}}e_ka_k, \]
	heißen \emph{Polynome in der Variablen} $ t:= e_1 $.
	Der \emph{Grad} eines Polynoms ist
		\[  \deg\sum_{k=0}^{n}t^ka_k := \max \{k\in \mathbb{N}\mid a_k \neq 0\}  \quad \left( \text{bzw. } \deg 0 := -\infty \right) \]
	Ist (der "`höchste"' Koeffizient) $ a_n = 1 $ für $ \deg p(t) = n $, so heißt das Polynom $ p(t) $ \emph{normiert}.
		\end{Definition}
\paragraph{Notation}
	Mit $t^k = e_{k}$, also $ K[t] = [(e_k)_{k\in \mathbb{N}}] $
	wird das Cauchyprodukt auf $ K[t] $ eine "`normale"' Multiplikation, gefolgt von einer Sortierung nach den Potenzen der Variablen $ t $. Wir werden das "`$ \odot $"' daher oft unterdrücken, und z.B. $ p(t)q(t) $ schreiben, anstelle von $ p(t) \odot q(t) $.
\paragraph{Bemerkung (Koeffizientenvergleich)}
	Mit dieser Definition von "`Polynom"' gilt
		\begin{align*}
			p(t)=\sum_{k=0}^{n}t^ka_k = 0
			\quad\Rightarrow \forall k\in \mathbb{N}: a_k = 0,
		\end{align*}
	da $ (t^k)_{k\in \mathbb{N}} = (e_k)_{k\in \mathbb{N}}$ linear unabhängig ist.
	Koeffizientenvergleich funktioniert\footnote{$p(t)=q(t) \Leftrightarrow p(t)-q(t)=\sum_{k=0}^n t^k(a_k-b_k)=0$; $a_k-b_k=0\Leftrightarrow a_k=b_k$}!
\paragraph{Bemerkung}
	Die Polynomalgebra $ K[t] $ über $ K $ ist eine assoziative und kommutative $ K $-Algebra, weiters ist $ K[t] $ unitär mit Einselement $ 1=e_0 $.
	
\subsection{Definition}\index{Algebra}
	\begin{Definition}[Algebra]
		Eine $ K $-Algebra ist ein $ K $-VR mit einer \emph{bilinearen Abbildung},
		\[ \odot: V\times V \to V,\ (v,w)\mapsto v\odot w, \]
	d.h. es gilt
		\begin{enumerate}[(i)]
			\item $ \forall w\in V:\ V\ni v\mapsto v\odot w\in V $ ist linear;
			\item $ \forall v\in V: V\ni w\mapsto v\odot w\in V $ ist linear.
		\end{enumerate}
	Eine $ K $-Algebra heißt
		\begin{itemize}
			\item \emph{unitär} (mit \emph{Einselement} 1), falls \hfill$ \exists 1\in V^\times\forall v\in V: 1\odot v = v\odot 1 = v $;
			\item \emph{assoziativ}, falls \hfill $ \forall u,v,w\in V: (u\odot v)\odot w = u\odot (v\odot w) $;
			\item \emph{kommutativ}, falls\hfill $\forall v,w,\in V: v\odot w = w\odot v$.
		\end{itemize}
	\end{Definition}
\paragraph{Beispiel}
	Die additive Gruppe $ \End(V) $ ist (mit der Komposition) eine unitäre assoziative Algebra.
\paragraph{Bemerkung}
	In jeder Algebra $ (V,\odot) $ gilt:
		\[ \forall v\in V: 0\odot v = v\odot 0 = 0 \]
	da z.B. für $ v\in V $ gilt
		\[ v\odot 0 = v\odot (0+0) = v\odot 0 + v\odot 0 \ \Rightarrow\  0=v\odot 0\]
	Ist $ (V,\odot) $ unitär, so liefert $ [1]\subset V $ wegen $ 1\odot 1 = 1 $ einen Körper:
		\[ ([1], +\mid_{[1]\times [1]},\odot\mid_{[1]\times [1]} ) \cong K \]
	vermöge $ K\ni x \mapsto 1 \cdot x\in [1] $ (siehe Aufgabe 5).
	
\subsection{Definition}\index{Algebra!-Homomorphismus}
	\begin{Definition}[Algebra-Homomorphismus]
		Ein \emph{Algebra-Homomorphismus} zwischen $ K $-Algebren $ (V,\odot) $ und $ (W,*) $ ist eine lineare Abbildung $ \psi\in \hom(V,W) $, für die gilt:
		\[ \forall v,v' \in V: \psi(v\odot v')=\psi(v)*\psi(v'). \]
	\end{Definition}
\paragraph{Bemerkung}
	$ \hom(V,W) $ wird oft auch für den (Vektor-)Raum der Algebra-Homomorphismen verwendet. In dieser LVA bedeutet "`$ \hom(V,W) $"' immer VR-Homomorphismen, bei allen "`anderen"' Homomorphismen wird  erwähnt, was gemeint ist.

%VO2-2016-03-03

\subsection{Einsetzungssatz \& Definitionen}
	\begin{Satz}[Einsetzungssatz]\index{Einsetzungshomomorphismus}\index{Polynom!-funktion}
		Seien $ (V,\odot) $ eine unitäre assoziative Algebra und $ v\in V $. Dann ist\footnote{$ v^0 := 1 $ ist sinnvoll, da die Algebra unitär ist.}
			\[ \psi_v: K[t]\to V,\ \sum_{k=0}^{n}t^ka_k = p(t)\mapsto \psi_v(p(t)) := \sum_{k=0}^{n}v^ka_k \]
		ein Algebra-Homomorphismus; $ \psi_v $ heißt \emph{Einsetzungshomomorphismus}. 
			\[ p:V\to V,\ v\mapsto p(v) := \psi_v(p(t)) \]
		heißt die zu $ p(t)\in K[t] $ gehörige \emph{Polynomfunktion} auf $ V $.
	\end{Satz}
\paragraph{Bemerkung}
	Wie üblich: $ v^k := \underset{k-\text{mal}}{\underbrace{v\odot\cdots \odot v}} $ und $ v^0 := 1 $.
\paragraph{Beweis}
	\begin{enumerate}
		\item $ \psi_v $ ist linear:
			\begin{itemize}
				\item für $ p(t) = \sum_{k\in\mathbb{N}} t^ka_k $ und $ a\in K $ gilt:
					\[ \psi_v(p(t)a) = \psi_v\Big(\sum_{k\in\mathbb{N}} t^ka_ka\Big) = \sum_{k\in\mathbb{N}} v^ka_ka = \psi_v(p(t))a; \]
				\item für $ p(t) = \sum_{k\in\mathbb{N}} t^ka_k $ und $ q(t) = \sum_{k\in\mathbb{N}} t^k b_k $ gilt:
					\[ \psi_v(p(t)+q(t)) = \psi_v\Big(\sum_{k\in\mathbb{N}}t^k(a_k+b_k)\Big) = \sum_{k\in\mathbb{N}} v^k(a_k+b_k) = \psi_v(p(t))+\psi_v(q(t)) \]
			\end{itemize}
		\item $ \psi_v $ ist "`multiplikativ"', d.h. verträglich mit der beteiligten Multiplikation:
		
			Für die Vektoren der Basis $ (t^k)_{k\in\mathbb{N}} $ von $ K[t] $ gilt, da $ (V,\odot) $ assoziativ ist,
				\[ \psi_v(t^mt^n) = \psi_v(t^{m+n}) = v^{m+n} = v^m\odot v^n = \psi_v(t^m)\odot \psi_v(t^n). \]
			Da aber $ \psi_v $ linear und die Multiplikation in $ K[t] $ und in $ (V,\odot) $ bilinear sind, folgt die Behauptung\footnote{Siehe nächste Bemerkung!}.
	\end{enumerate}
\paragraph{Bemerkung (Fortsetzungssatz für bilineare Abbildungen)}\index{Fortsetzungssatz für bilineare Abbildungen}
	Im Beweis haben wir verwendet: Die Abbildungen
		\[ K[t]\times K[t]\to V,\ (p(t),q(t))\mapsto
			\begin{cases}
				\psi_v(p(t)q(t))               & \quad\text{(Cauchyprodukt)}       \\
				\psi_v(p(t))\odot \psi_v(q(t)) & \quad\text{(Produkt in $ (V,\odot) $)}
			\end{cases}
		\]
	sind bilinear (da $ \psi_v $ linear ist), sind also gleich, sobald sie auf einer Basis übereinstimmen.
	Dies ist die Eindeutigkeit eines Fortsetzungssatzes für bilineare Abbildungen:
	
	\begin{Satz}[Fortsetzungssatz für bilineare Abbildungen]
	Sind $ V,W\ K$-VR, $ (b_i)_{i\in I} $ eine Basis von $ V $ und $ (\beta_{ij})_{i,j\in I} $ eine Familie in $ W $, so gibt es eine eindeutige bilineare Abbildung
		\[ \beta: V\times V\to W \]
	mit
		\[ \forall i,j\in I: \beta(b_i,b_j) = \beta_{ij} \]
	\end{Satz}
	
	Dieser Fortsetzungssatz folgt direkt aus dem Fortsetzungssatz für lineare Abbildungen, da
		\[ \{\beta:V\times V\to W \text{ bilinear}\} \cong \hom(V,\hom(V,W))\]
	vermittels des Isomorphismus
		\[ \beta \mapsto \big(v\mapsto\underset{\in \hom(V,W)}{\underbrace{\beta(v,.)}}\big), \]
	d.h. durch Nacheinandereinsetzen der Argumente.
	
	Bew: Laut FSS für Homomorphismen existieren für alle $i\in I$ eindeutige Homomorphismen
		\[\beta_i:V\to W \quad\text{mit}\quad \beta_i(b_j)=\beta_{ij} \qquad\forall j\in J \]
	Dies bedeutet also $\beta(b_i,.)=\beta_i(.)$. Nun ist $(\beta_i)_{i\in I}$ eine Familie in $\Hom(V,W)$ und damit existiert wieder nach FSS für Homorphismen ein eindeutiger Homomorphismus
		\[\beta:V\to \Hom(V,W) \quad\text{mit}\quad \beta(b_i)=\beta_i \qquad\forall i\in I\]
	Nach Konstruktion ist der Ausdruck $\beta(b_i)(b_j)$ also sinnvoll und bildet auf ein Element $\beta_{ij}\in W$ ab. Wir können $\beta$ nun wie in der Beweisskizze oben isomorph identifizieren:
		\[\beta: V\times V \to W,\ (v,w)\mapsto \beta(v,w):=\beta(v)(w)\]
	Die Eindeutigkeit der bilinearen Abbildung ergibt sich aus der Kontruktion.
	
\paragraph{Bemerkung}
	Die Abbildung eines Polynoms auf seine Polynomfunktion auf dem Körper,
		\[ K[t]\ni p(t)\mapsto (x\mapsto p(x))=\psi_x(p(t))\in K^K \]
	ist für $ \Char K\neq 0 $ nicht injektiv\footnote{sonst wäre $K^K$ unendlich dimensional.}. Das heißt: Koeffizientenvergleich (für Polynomfunktionen) kann nur funktionieren, wenn $ \Char K = 0 $.
\paragraph{Beispiel \& Bemerkung}\index{Einsetzungshomomorphismus}
	Ist $ V\ K $-VR, so ist $ \End(V) $ eine $ K $-Algebra (mit Komposition $ \circ $). Man erhält also für $ f\in \End(V) $ einen Einsetzungshomomorphismus
		\[ \psi_f: K[t]\to \End(V),\ p(t) \mapsto \psi_f(p(t)) = p(f); \]
	und für jedes Polynom $ p(t)\in K[t] $ eine zugehörige Polynomfunktion
		\[ p: \End(V)\to\End(V),\ f\mapsto \psi_f(p(t))= p(f). \]
	Dieses Beispiel ist der Schlüssel zum Satz von Cayley-Hamilton (im nächsten Abschnitt).
\subsection{Lemma}
	\begin{Lemma}[Gradformel für Polynome]
		Für Polynome $ p(t), q(t)\in K[t] $ gilt:
			\begin{itemize}
				\item $ \deg p(t)\odot q(t) = \deg p(t)+\deg q(t) $,
				\item $ \deg p(t)+q(t) \leq \max\{\deg p(t), \deg q(t)\} $.
			\end{itemize}
	\end{Lemma}
\paragraph{Beweis}
	Für $ p(t) = \sum_{k\in\mathbb{N}}t^ka_k $ und $ q(t) = \sum_{k\in\mathbb{N}}t^kb_k $ ist
		\[ p(t)\odot q(t) = \sum_{k\in\mathbb{N}}t^kc_k \text{ mit } c_k = \sum_{j=0}^{k}a_jb_{k-j} \]
	Gilt nun $ \deg p(t) = n $ und $ \deg q(t) = m $, d.h.
		\[ a_n,b_m \neq 0 \land \forall k>n, k'>m:a_k = b_{k'}=0 \] 
	so folgt
		\[ \left.
		\begin{aligned}
		\forall k>m+n : c_k = 0\ \\
		        c_{m+n} = a_nb_m\ \\
		\end{aligned}
		 \right\}
		\Rightarrow \deg p(t)\odot q(t) = m+n \]
	Gilt andererseits $ \deg p(t) = -\infty $ oder $ \deg q(t) = -\infty $, also $ p(t) = 0 \lor q(t) = 0 $,
	so folgt
		\[ p(t)\odot q(t) = 0 \Rightarrow \deg p(t)\odot q(t) = -\infty. \]
	Die zweite Behauptung ist offensichtlich wahr.

%VO3-2016-03-08

\paragraph{Beispiel}
	Für $ p(t),q(t),d(t)\in K[t] $ mit $ d(t)\neq 0 $ gilt
		\[ d(t)p(t) = d(t)q(t)\Rightarrow p(t)=q(t). \]
	Nämlich: da $ \deg d(t) \geq 0$,
		\begin{align*}
			-\infty &= \deg d(t)\big(p(t)-q(t)\big)\\
			&= \deg d(t)+ \deg\big(p(t)-q(t)\big)\\
			\Rightarrow \deg\big(p(t)-q(t)\big) &= -\infty\\
			\Rightarrow p(t)&=q(t)
		\end{align*}
\subsection{Euklidischer Divisionsalgorithmus}\index{Polynom!-division}
	\begin{Satz}[Euklidischer Divisionsalgorithmus (Polynomdivision)]
	Seien $ p(t), d(t) \in K[t],\ d(t) \neq 0$. Dann existieren eindeutig $ q(t), r(t) \in K[t] $, sodass
		\[ p(t)= d(t)q(t) + r(t) \text{ und } \deg r(t) < \deg d(t). \]
	\end{Satz}
\paragraph{Bemerkung}
	Ist $ \deg p(t)\leq \deg d(t) $, so ist die Aussage trivial.
\paragraph{Beweis}
	Eindeutigkeit folgt wie im Beispiel; mit
		\begin{gather*}
		p(t) =
			\begin{cases}
				d(t)q(t)+r(t)\\
				d(t)\tilde{q}(t)+\tilde{r}(t)
			\end{cases}\\
		\Rightarrow d(t)\big(q(t)-\tilde{q}(t)\big) = \tilde{r}(t)-r(t)
		\end{gather*}
	erhält man
		\begin{align*}
			\deg d(t) + \deg \big(q(t)-\tilde{q}(t)\big) &= \deg (r(t)-\tilde{r}(t))\\
			&\leq \max \{\deg r(t), \deg \tilde{r} (t)\} < \deg d(t).
		\end{align*}
	Also folgt
		\[ \deg\big(q(t)-\tilde{q}(t)\big) = -\infty \Rightarrow \deg (r(t)-\tilde{r}(t)) = \deg d(t) -\infty = - \infty \]
	und damit
		\[ \tilde{q}(t)=q(t) \text{ und }\tilde{r}(t) = r(t). \]
	
	Existenz: Mit $ k := \deg d(t) \geq 0 $ und
		\[ K[t]_m := \{q(t)\in K[t]\mid \deg q(t)\leq m \} \text{ für }m\in \mathbb{N} \]
	betrachte man die Abbildung
		\[ K[t]_m \times K[t]_{k-1}\to K[t]_{k+m},\ \big(q(t), r(t)\big) \mapsto d(t)q(t)+r(t). \]
	Diese Abbildung ist linear (klar) und injektiv, denn:
	ist $ q(t) \neq 0 $, so folgt wegen
		\[ \deg r(t) < k = \deg d(t)\leq \deg d(t)q(t) \]
	dass %Anm.: da wegen des kleineren Grades der höchste Koeffizient durch die Addition nicht verschwinden kann
		\begin{gather*}
			\deg \big(d(t)q(t) + r(t)\big) = \deg d(t)q(t)\geq k > -\infty\\
			\Rightarrow d(t)q(t)+r(t)\neq 0,
		\end{gather*}
	also
		\[ d(t)q(t)+r(t)=0 \Rightarrow q(t)=0 \land r(t) = 0. \] %Damit ist der Kern der Abbildung \{0\} -> Injektivität
	Wegen
		\[ \dim K[t]_m \times K[t]_{k-1} = (m+1) + k = (k+m) + 1 = \dim K[t]_{k+m} \]
	liefert diese Abbildung dann für jedes $ m\in \mathbb{N} $ einen Isomorphismus
		\[ K[t]_m \times K[t]_{k-1} \to K[t]_{k+m}. \]
\subsection{Korollar \& Definition}
	\begin{Definition}[Nullstelle]\index{Nullstelle}
	Sei $ p(t)\in K[t] $ mit $ \deg p(t)\geq 1 $. Ist $ x\in K $ eine \emph{Nullstelle} von $ p(t) $, d.h.
		\[ p(x) = \psi_x(p(t)) = 0, \]
	\end{Definition}
	so folgt
	\begin{Korollar}[Abspalten von Linearfaktoren mit Hilfe von Nullstellen]
		\[ \exists! q(t)\in K[t]: p(t) = (t-x) q(t). \]
	\end{Korollar}
\paragraph{Beweis}
	Seien $ p(t)\in K[t] $ mit $ \deg p(t) \geq 1 $ und $ x\in K $ eine Nullstelle von $ p(t) $; dann gibt es eindeutig $ q(t),r(t) \in K[t] $ mit
		\[ p(t) = (t-x) q(t) + r(t) \text{ und } \deg r(t) < \deg (t-x) = 1, \]
	also
		\[ p(t) = (t-x)q(t)+r(t) = (t-x)q(t)+c_0. \]
	Einsetzen von $ x\in K $ liefert dann
		\[ 0 = p(x) = (x-x)q(x) + c_0 = c_0. \]
\paragraph{Bemerkung und Beispiel}\index{Linearfaktorisierung}
	Dies liefert eine Methode, um Polynome zu \emph{faktorisieren}: Für jede gefundene Nullstelle kann man einen \emph{Linearfaktor} abspalten.
		\[ p(t)= t^4-t^3+t^2-t =
			\begin{cases}
				t(t-1)(t-i)(t+i) \in \C[t]\\
				t(t-1)(t^2+1) \in \R[t].
			\end{cases} \]

\subsection{Definition}
	\begin{Definition}[Teiler]
		Sind $ p(t),d(t)\in K[t] $, so heißt $ d(t) $ \emph{Teiler} von $ p(t), d(t)\mid p(t) $, falls
			\[ \exists q(t)\in K[t]: p(t)= d(t)q(t). \]
	\end{Definition}

\subsection{Mehr zu Polynomen}
	Dies ist der Anfang einer der Teilbarkeitstheorie der natürlichen Zahlen ähnlichen Theorie für Polynome.
	
\paragraph{Primpolynome}\index{Primpolynome}
\begin{Definition}[Primpolynom]
	Nennt man $ p(t)\in K[t] $ mit $ \deg p(t) > 0$ ein \emph{Primpolynom} (oder \emph{irreduzibel}), falls für $ d(t),q(t)\in K[t] $ gilt:
		\[ p(t)= d(t)q(t) \Rightarrow \Big( \deg q(t)=0 \lor \deg d(t) = 0 \Big), \]
	so gilt der Satz über die \emph{Primfaktorzerlegung}:
\end{Definition}
	\begin{Satz}[Primfaktorzerlegung]
	\begin{addmargin}{1cm}
		\textit{Jedes Polynom $ p(t)\in K[t] $ mit $ \deg p(t)>0 $ zerfällt eindeutig in Primpolynome,
			\[ p(t) = a_n p_1(t) \cdots p_m(t), \]
		wobei $ a_n\in K $ und $ p_1(t),\dots,p_m(t)\in K[t] $ normierte Primpolynome sind.}
	\end{addmargin}
	\end{Satz}
	
\paragraph{Beweis}	
	Existenz ist einfach zu zeigen (Induktion über $ n $), die weniger leicht zu zeigende Eindeutigkeit benutzt die Existenz des \emph{größten gemeinsamen Teilers} $ d(t) = \ggT\big(p(t),q(t)\big) $ zweier Polynome $ p(t) $ und $ q(t) $:

	\begin{addmargin}{1cm}
		\textit{Zu $ p(t),q(t) \in K[t]\setminus \{0\} $ gibt es genau ein normiertes Polynom $ d(t)\in K[t] $ mit
			\begin{align*}
				d(t)&\mid p(t)\land d(t)\mid q(t) \text{ und}\\
				d'(t)&\mid p(t)\land d'(t)\mid q(t) \Rightarrow d'(t)\mid d(t).
			\end{align*}}
	\end{addmargin}
\paragraph{Lemma von B\'ezout}
	\label{Bezout}
	\begin{Lemma}[Lemma von B\'ezout]
	Für den $ \ggT $ gilt auch das Lemma von B\'ezout:
		\[ \exists p'(t),q'(t)\in K[t]: d(t)=p(t)p'(t)+q(t)q'(t) \]
	\end{Lemma}
\paragraph{Bemerkung}
	Aus der Gradformel,
		\[ \deg d(t)q(t) = \deg d(t)+\deg q(t) \]
	folgt direkt:
		\begin{addmargin}{1cm}
			\textit{Jedes Polynom $ p(t)\in K[t] $ mit $ \deg p(t)=1 $ ist Primpolynom.}
		\end{addmargin}

\paragraph{Fundamentalsatz der Algebra}
	Falls $ K = \C $, so sind die Polynome mit Grad 1 die einzigen Primpolynome:
		\begin{Satz}[Fundamentalsatz der Algebra]
		\begin{addmargin}{1cm}
			\textit{In $ \C $ zerfällt jedes Polynom (mit Grad $ \geq 1 $) in Linearfaktoren;
				\[ \forall p(t)\in \C[t],\ \deg \geq 1:\ \exists x_1,\dots,x_n \in \C \]
			mit
				\[ p(t) = a_n \prod_{j=1}^{n}(t-x_j). \]}
		\end{addmargin}
		\end{Satz}
	Ist $ K=\R $, so ist dies nicht der Fall; ein Primpolynom vom Grad $ \deg p(t) = 2 $ ist z.B.
		\[ p(t) = t^2+1\ \in \R[t], \]
	denn
		\[ t^2+1 = (t-x_1)(t-x_2) \Rightarrow
			\begin{cases}
				0 = x_1 + x_2\\
				1 = x_1 \cdot x_2
			\end{cases}
		\Rightarrow 1 = -x^2 \]
	Andererseits ist $ p(t)\in \R[t]\subset \C[t], $
	also existieren $ x_1,\dots,x_n\in \C $ mit
		\[ a_n\prod_{j=1}^{n}(t-x_j) = p(t) = \overline{p(t)} = \overline{a_n}\prod_{j=1}^{n}(t-\overline{x_j}), \]
	d.h. mit der Eindeutigkeit der Primfaktorzerlegung, $ a_n\in \R $ und die $ x_j $ sind entweder reell oder treten in komplex-konjugierten Paaren auf:
		\[ p(t) = a_n\prod_{j=1}^{m}(t^2-t(x_j+\overline{x_j})+x_j\overline{x_j})\prod_{j=2m+1}^{n}(t-x_j). \]
	Ist also $ p(t) \in \R[t] $ Primpolynom, so folgt $ \deg p(t)\leq 2 $ und
		\[ \deg p(t) = 2 \Rightarrow \exists x,y\in \R: p(t) = (t-x)^2+y^2 \text{ mit } y\neq 0\]
	In $ K=\mathbb{Q} $ gibt es noch "`mehr"' Primpolynome, wie z.B.:
		\[ p(t) = t^2-2 \text{ oder } p(t) = t^4+1 \]

\section{Das charakteristische Polynom}
\subsection{Definition}\index{Eigenwert,-vektor,-raum}
\begin{Definition}[Eigenwert,Eigenvektor,Eigenraum]
	Seien $ V $ ein $ K $-VR und $ f\in\End(V) $. Dann heißen
		\begin{enumerate}[(i)]
			\item $ x\in K $ ein \emph{Eigenwert} von $ f $, falls  $ \exists v\in V^\times: f(v)=vx; $
			\item $ v\in V^\times $ ein \emph{Eigenvektor} von $ f $, falls  $\exists x\in K:f(v)=vx; $
			\item $ \ker(f-\id_Vx) \subset V $ ein \emph{Eigenraum}, falls $\ker(f-\id_Vx) \neq \{0\}.$
		\end{enumerate}
	\end{Definition}
\paragraph{Bemerkung}
	Der Skalar $ x\in K $ ist genau dann ein Eigenwert von $ f\in \End(V) $, wenn $ \ker(f-\id_Vx)\neq \{0\} $, d.h., wenn ein Eigenvektor $ v\in V^\times $ zu $ x $ existiert.
\paragraph{Beispiel}
	Für $ \frac{d}{ds} \in \End(C^\infty(\R))$ ist jedes $ x\in \R $ ein Eigenwert, da
		\[ \Big(\frac{d}{ds}-\id_Vx\Big)v = 0 \text{ für } v:\R\to\R,\ s\mapsto v(s):= e^{xs}, \]
	wobei $ v\in C^\infty(\R)\setminus \{0\} $, d.h. $ s\mapsto v(s)=e^{xs} $ ist ein Eigenvektor zum Eigenwert $ x\in\R $.
\paragraph{Beispiel}
	Ist $ \dim V < \infty $, so kann die Determinante zur Bestimmung von Eigenwerten von Endomorphismen $ f\in\End(V) $ benutzt werden, da
		\[ \ker(f-\id_Vx)\neq \{0\} \Leftrightarrow (f-\id_Vx) \text{ nicht injektiv}\Leftrightarrow \det(f-\id_Vx) = 0, \]
	d.h. das Auffinden von Eigenwerten $ x\in K $ von $ f $ ist reduziert auf die Bestimmung der Nullstellen der Funktion
		\[ K\ni x\mapsto \det(f-\id_Vx)\in K. \]
		
\paragraph{Beispiel}	
	Ist z.B. $ (b_1,b_2) $ Basis von $ V $ und $ f\in \End(V) $ durch $ f(B)=BX $ gegeben, so liefern die Nullstellen der Polynomfunktion
		\begin{gather*}
		\det(f-\id_Vx) = \det(X-E_2 x)= \det \begin{pmatrix}
		x_{11}-x & x_{12}\\
		x_{21} & x_{22} -x
		\end{pmatrix}\\
	= (x_{11}-x)(x_{22}-x)-x_{12}x_{21}
	= x^2 - x(x_{11}+x_{22}) + (x_{11}x_{22}-x_{12}x_{21})
		\end{gather*}
	die Eigenwerte von $ f $ -- beispielsweise erhalten wir für
		\[ X = \begin{pmatrix} 2 &3\\1 & 0 \end{pmatrix}:\ 
			\det(f-\id_Vx) = x^2-2x-3 = (x+1)(x-3), \]
	also Eigenwerte $ x_1 = -1 $ und $ x_2 = 3 $ mit zugehörigen Eigenvektoren als Lösungen von
		\[ v_i \in \ker(f-\id_Vx_i), \]
	also durch Lösungen der linearen Gleichungssysteme % Die Rechtsmultiplikation der Koeffizienten der Linearkombination in der entsprechenden Basis ergeben die Koeffizenten des Bild des Vektors. Gesucht sind nun eben jene Koeffizenten, bei denen die Bildkoeffizienten alle 0 sind.
		\[ \begin{pmatrix}
		2-(-1) & 3\\ 1 & -(-1)
		\end{pmatrix}
		\begin{pmatrix}
		v_1^1\\v_1^2
		\end{pmatrix} = \begin{pmatrix}
		3 & 3\\ 1 & 1
		\end{pmatrix}
		\begin{pmatrix}
		v_1^1\\v_1^2
		\end{pmatrix} \text{ und} \]
		\[ \begin{pmatrix}
		2-3 & 3\\ 1 & -3
		\end{pmatrix}
		\begin{pmatrix}
		v_2^1\\v_2^2
		\end{pmatrix}=
		\begin{pmatrix}
		-1 & 3\\ 1 & -3
		\end{pmatrix}
		\begin{pmatrix}
		v_2^1\\v_2^2
		\end{pmatrix}  \]
	sodass
		\[ v_1 = b_1-b_2 \text{ und } v_2 = b_13+b_2 \]
	Eigenvektoren zu den Eigenwerten $ x_1,x_2 $ liefert.

\paragraph{Rechenbeispiel 1}
	Für $ X = \begin{pmatrix}2&-1\\1&0\end{pmatrix} $ erhält man
		\[ \det(f-\id_Vx) = \det\begin{pmatrix}2-x&-1\\1&-x	\end{pmatrix} =x^2-2x+1 \]
	und Eigenvektoren zum Eigenwert $ x = 1 $ durch Lösung der LGS
		\[ \begin{pmatrix}
		2-1&-1\\1&-1
		\end{pmatrix}\begin{pmatrix}
		v_1^1\\v_1^2
		\end{pmatrix} =  \begin{pmatrix}
		1&-1\\1&-1
		\end{pmatrix}\begin{pmatrix}
		v_1^1\\v_1^2
		\end{pmatrix} \]
	d.h. der Eigenraum zum Eigenwert $ x $,
		\[ \ker(f-\id_V) = [\{b_1+b_2\}]\quad \text{ hat }\quad\dim \ker(f-\id_V)<\dim V. \]
\paragraph{Rechenbeispiel 2}
	Ist $ K=\R $ und
		\[ \det(f-\id_Vx)=x^2+1, \]
	so hat $ f $ keine Eigenwerte: z.B., wenn
		$ X=\begin{pmatrix} 0&1\\-1&0 \end{pmatrix} $.
		
\subsection{Definition} \index{Charakteristisches Polynom}
\begin{Definition}[Charakteristisches Polynom]
	Sei $ V $ ein $ K $-VR, für $ f\in\End(V) $ ist das \emph{charakteristische Polynom} von $ f $:
		\[ \chi_f(t) := \det (\id_Vt-f)\in K[t]. \]
	Analog definiert man für $ X\in K^{n\times n} $ das charakteristische Polynom
		\[ \chi_X(t) := \det (E_nt-X)\in K[t]. \]
\end{Definition}
\paragraph{Bemerkung}
	Oft wird auch das andere Vorzeichen in der Determinante verwendet, also $ \det(f-\id_Vt) $ bzw. $ \det(X-E_nt) $.
\paragraph{Bemerkung}
	\emph{Diese Definition ist erklärungsbedürftig!}
	
	Da $ t\notin K $ ist $ \id_Vt-f\notin \End(V) $, sondern $ \id_Vt-f\in\End(V)[t] $. Zwei Lösungsstrategien bieten sich an:
		\begin{enumerate}
			\item Erweiterung der Determinante auf $ \End(V)[t] $.
			\item Benutzung von Darstellungsmatrizen.
		\end{enumerate}
	Beide führen schließlich zur Leibniz-Formel:
	
	Ist $ B $ eine Basis von $ V $ und $ \xi_B^B(f) = X = (x_{ij})_{i,j\in\{1,\dots,n\}}$, so erhält man 
		\[ \chi_f(t)=\sum_{\sigma\in S_n}\sgn(\sigma)\prod_{j=1}^{n}\underset{\in K[t]}{\underbrace{\left(\delta_{\sigma(j)j}t-x_{\sigma(j)j}\right)}} \in K[t]. \]
	Die Unabhängigkeit von der Basis $ B $ folgt aus der Transformationsformel für Darstellungsmatrizen und dem Determinanten-Multiplikationssatz (wie vorher für $ \det f = \det \xi_B^B(f) $).

% VO 2016-03-15

\subsection{Bemerkung \& Definition}\index{Spur}
\begin{Definition}[Spur]
	Ist $ \dim V=n $, so ist $ \chi_f(t) $ ein normiertes Polynom vom Grad $ \deg\left(\chi_f(t)\right)=n $,
		\[ \chi_f(t)=t^n-t^{n-1}\tr f + \dots + (-1)^n\det f,\] % = \det(-f) = \chi_f(0)
	wobei die \emph{Spur} $ \tr f $ (\glqq tr \grqq $\widehat{=}$ trace) von $ f $ durch diese Gleichung (wohl-)defininiert ist.
\end{Definition}	
	Ist $ (x_{ij})_{i,j\in\{1,\dots,n\}} = X = \xi_B^B(f) $ Darstellungsmatrix von $ f $, so gilt
		\[ \tr f = \sum_{j=1}^{n}x_{jj} = \sum_{j=1}^{n} b_j^*f(b_j). \]
	Oft wird $ \det(f-\id_Vt)=(-1)^n\chi_f(t) $ als charakteristisches Polynom definiert -- dieses Polynom ist dann nur für gerade $ n $ normiert.
\subsection{Korollar}
\begin{Korollar}[Eigenwerte sind Nullstellen des char. Polynoms]
	Ein $ x\in K $ ist genau dann Eigenwert von $ f $, wenn $ \chi_f(x)=0 $.\\
	Also: Die Eigenwerte von $ f $ sind genau die Nullstellen des charakteristischen Polynoms $ \chi_f(t) $.
\end{Korollar}
\paragraph{Beweis}
	Klar -- das war die Idee hinter der Definition des charakteristischen Polynoms.
\subsection{Korollar \& Definition}\index{Algebraische/geometrische Vielfachheit}
\begin{Korollar}[Eigenwert ist Nullstelle des charakteristischen Polynoms]
	Ist $ x\in K $ Eigenwert von $ f\in\End(V) $, so ist $ (t-x) $ Teiler des charakteristischen Polynoms. Insbesondere gilt:
		\[ \exists!k\in \mathbb{N}^\times:
			\begin{cases}
				(t-x)^k\mid \chi_f(t)\\
				(t-x)^{k+1}\nmid \chi_f(t)
			\end{cases} \]
\end{Korollar}
\begin{Definition}[algebraische Vielfachheit, geometrische Vielfachheit]
	Diese Zahl $ k $ heißt die \emph{algebraische Vielfachheit} von $ x $;
		\[ g:= \dfkt(\id_Vx-f) \leq k \]
	ist die \emph{geometrische Vielfachheit} von $ x $.
\end{Definition}
\paragraph{Beweis}
	Da $ x $ Eigenwert von $ f $ ist, ist die Existenz und Eindeutigkeit von $ k $ klar. Außerdem gilt analog auch $ g\geq 1 $.
	Zu zeigen bleibt: $ g\leq k $, d.h. $ (t-x)^g \mid \chi_f(t) $:\\
	Für eine Basis $ B = (b_1,\dots,b_n) $ von $ V $ mit
	$ \ker (\id_v x - f) = [(b_1,\dots,b_g)]$
	hat
		\[ \xi_B^B(f) =
		\begin{pmatrix}
			E_gx & Y\\
			0 & X
		\end{pmatrix}
		\text{ mit } Y\in K^{g\times (n-g)}, X\in K^{(n-g)\times(n-g)} \]
	Blockgestalt, also ist
		\[ \chi_f(t)=(t-x)^g\cdot \chi_X(t), \]
	d.h. $ (t-x)^g \mid \chi_f(t)$, da $ (t-x)^{k+1}\nmid \chi_f(t) $, gilt also $ g\leq k $.
\paragraph{Beispiel}
	Ist $ f\in\End(V) $ wie oben durch $ f(B)=BX $ gegeben, so haben die Eigenwerte
		\[ x_1 = -1 \text{ und } x_2 = 3 \text{ für }
		X=\begin{pmatrix} 2 &3\\1 & 0 \end{pmatrix} \]
	algebraische und geometrische Vielfachheiten 
		\[ 1 = g_i = k_i, \text{ da } 1\leq g_i \leq k_i \text{ und } k_1+k_2 \leq 2; \]
	der Eigenwert
		\[ x=1 \text{ für } X = \begin{pmatrix} 2&-1\\1&0 \end{pmatrix} \]
	hat algebraische und geometrische Vielfachheiten $ k = 2 \text{ und } g = 1 $, da
		\[ f\neq \id_V x = \id_V \]
	und $ \chi_f(t)=(t-x)^2 \in \R[t] $, da ein quadratisches Polynom zwei (relle oder komplex konjugierte) Nullstellen hat, oder aber eine doppelte reelle.

\subsection{Definition \& Lemma}\index{$ f $-invarianter Unterraum}
	Das Schlüsselargument im Beweis oben kann man verallgemeinern:\\
\begin{Definition}[$ f $-invarianter Unterraum]
	Sei $ f\in \End(V) $ und $ U\subset V $ ein \emph{$ f $-invarianter Unterraum}, d.h. $ f(U)\subset U $. 
\end{Definition}
\begin{Lemma}[]
	Ist dann $ V=U\oplus U' $ eine direkte Zerlegung und $ p,p'\in \End(V) $ die zugehörigen Projektionen, so gilt
		\[ \chi_f(t)=\chi_{f|_U}(t)\cdot \chi_{f'}(t), \]
	wobei
		\[ f':= p'\circ f|_{U'}\in \End(U'). \]

\end{Lemma}
\paragraph{Bemerkung}
	Man kann $ f|_U $ als Endomorphismus $ f|_U\in \End(U) $ auffassen, da $ f(U)\subset U $.
\paragraph{Beweis}
	Wie oben: Sei $ B=(b_1,\dots,b_n) $ Basis von $ V $, sodass
		\begin{itemize}
			\item $ C=(b_1,\dots,b_k) $ Basis von $ U $ und
			\item $ C'=(b_{k+1},\dots,b_n) $ Basis von $ U' $ ist.
		\end{itemize}
	Die Darstellungsmatrix von $ f $ bzgl. $ B $ hat dann Blockgestalt,
		\[ \xi_B^B(f) =
			\begin{pmatrix}
				X&Y\\0&X'
			\end{pmatrix}
		\text{ mit } X=\xi_C^C(f|_U), X' = \xi_{C'}^{C'}(f') \]
	Damit folgt die Behauptung (wie oben) mit der Leibniz-Formel.
\paragraph{Bemerkung}
	Alternativ kann man das Lemma mit der von $ f $ induzierten Quotientenabbildung $ f'\in \End(V/U) $ formulieren, wobei
		\[ f':V/U\to V/U, v+U\mapsto f'(v+U) := f(v)+U. \]
\subsection{Definition}\index{Diagonalisierbarkeit}\index{Triagonalisierbarkeit}
\begin{Definition}[Diagonalisierbarkeit, Triagonalisierbarkeit von Endomorphismen]
	Ein Endomorphismus $ f\in\End(f) $ heißt \emph{diagonalisierbar} bzw. \emph{trigonalisierbar}, falls es eine Basis $ B $ von $ V $ gibt, sodass $ \xi_B^B(f)=(x_{ij})_{i,j\in\{1,\dots,n\}} $ eine Diagonalmatrix 
		\[ i\neq j\Rightarrow x_{ij} = 0 \]
	bzw. obere Dreiecksmatrix ist,
		\[ i>j \Rightarrow x_{ij} = 0. \]
\end{Definition}
\paragraph{Bemerkung}
	Falls $ \dim V<\infty $, so ist $ f\in\End(V) $ genau dann diagonalisierbar, wenn $ V $ eine Basis aus Eigenvektoren von $ f $ besitzt. Damit kann man "`Diagonalisierbarkeit"' auch im Falle $ \dim V=\infty $ definieren.
	
	Bew: Existiert eine Eigenvektorbasis $B$, so ist $\xi_B^B$ klarerweise diagonal, mit den Eigenwerten als Diagonaleinträgen. %TODO
	
\paragraph{Bemerkung}
	Ist $ f $ trigonalisierbar (oder gar diagonalisierbar), so zerfällt $ \chi_f (t) $ in Linearfaktoren: für geeignete $ x_1,\dots,x_n\in K $ ist
		\[ \chi_f(t)=\prod_{j=1}^{n}(t-x_j). \]
\subsection{Bemerkung \& Definition}
\begin{Definition}[Diagonalisierbarkeit, Triagonalisierbarkeit von Matrizen]
	Man nennt eine Matrix $ X\in K^{n\times n} $ diagonalisierbar (bzw. trigonalisierbar), falls $ f_X\in \End(K^n) $ diagonalisierbar (bzw. trigonalisierbar) ist.
\end{Definition}	

	Dies ist genau dann der Fall, falls es $ P\in Gl(n) $ gibt, sodass $ PXP^{-1} $ Diagonalmatrix (bzw. obere Dreiecksmatrix) ist.

% VO 2016-03-17

\subsection{Lemma}
	Frage: Was sind hinreichende Kriterien dafür? Notwendigkeit kennen wir: $ \chi_f(t) $ zerfällt in Linearfaktoren.
	
	\begin{Lemma}[Lineare Unabhängigkeit von Eigenvektoren]
		Eigenvektoren $ v_1,\dots,v_m\in V $ zu paarweise verschiedenen Eigenwerten $ x_1,\dots,x_m $ eines Endomorphismus $ f\in\End(V) $ sind linear unabhängig.
	\end{Lemma}
\paragraph{Bemerkung}
	Anders gesagt: Die Summe von Eigenräumen zu paarweise verschiedenen Eigenwerten ist direkt.
\paragraph{Beweis}
	Zu zeigen: Ist $ \sum_{i=1}^m v_iy_i = 0 $ für Koeffizienten $ y_1,\dots,y_m\in K $, so folgt $ y_1 = \dots = y_m = 0 $.\\	
	Seien $ y_1,\dots,y_m \in K $ und $ w_i := v_iy_i $ und $ w:= \sum_{i=1}^{m}w_i = \sum_{i=1}^{m}v_iy_i$.
	Wiederholte Anwendung von $ f $ liefert, wegen $ f(w_i) = w_ix_i $
	
		\[ (f^{m-1}(w),\dots,f^2(w),f(w),w) = (w_1,\dots,w_m)
		\begin{pmatrix}
		 x_1^{m-1}&\cdots&x_1^2&x_1&1 \\
		 \vdots&\ddots&\vdots&\vdots&\vdots\\
		 x_m^{m-1}&\cdots&x_m^2&x_m & 1
		\end{pmatrix} \]
	mit der Vandermonde-Matrix $ X\in Gl(m) $, da
		\[ \det X = \prod_{i<j} (x_i - x_j)\neq 0 \]
	weil die Eigenwerte $ x_1,\dots,x_m $ paarweise verschieden sind.
	Damit folgt aus $ w=\sum_{i=1}^{m}v_iy_i = 0 $
		\[ (w_1,\dots,w_m)=(f^{m-1}(w),\dots,f(w),w)X^{-1} = (0,\dots,0) \]
	also
		\[ \forall i=1,\dots,m: 0 = w_i = v_iy_i \text{ und }v_i \neq 0 \Rightarrow y_i = 0.  \]
\subsection{Satz}
    \begin{Satz}[Diagonalisierbarkeit eines Endomorpismus]
		Ein Endomorphismus $ f\in \End(V) $ ist genau dann diagonalisierbar, wenn $ \chi_f(t) \in K[t] $ in Linearfaktoren zerfällt und die algebraischen und geometrischen Vielfachheiten aller Eigenwerte übereinstimmen,
			\[ \chi_f(t) = \prod_{i=1}^{m}(t-x_i)^{k_i} \text{ und } \forall i=1,\dots,m: k_i = g_i.\]
	\end{Satz}
\paragraph{Beweis}
	Ist $ f $ diagonalisierbar, so existiert eine Basis $ B $ aus Eigenvektoren von $ f $, also ist dann
		\[ \xi_B^B(f) =
			\begin{pmatrix}
				E_{g_1}x_1 &0& \cdots & 0 \\
				0 &E_{g_2}x_2& \ddots & \vdots\\
				\vdots & \ddots& \ddots & \vdots\\
				0 & 0 & \cdots & E_{g_m}x_m 
			\end{pmatrix} \]
	Damit ist
		\[ \chi_f(t)=\prod_{i=1}^{m}(t-x_i)^{g_i}. \]
	Hat andererseits das charakteristische Polynom diese Gestalt, so wähle man in jedem Eigenraum $ \ker(\id_Vx_i-f) $ eine Basis $ C_i,i=1,\dots,m $. Da Eigenvektoren zu verschiedenen Eigenwerten linear unabhängig sind, und wegen
		\[ g_1+\dots+g_m = k_1 + \dots + k_m = \dim V \]
	liefert $ B := \bigcup_{i=1}^mC_i $ eine Basis von $ V $.
\subsection{Korollar}
	\begin{Korollar}
		Ein Endomorphismus $ f\in\End(V) $ mit $ n=\dim V $ paarweise verschiedenen Eigenwerten ist diagonalisierbar.
	\end{Korollar}
\paragraph{Beweis}
	Für die geometrischen und algebraischen Vielfachheiten jedes Eigenwerts gilt
		\[ 1\leq g_i \leq k_i \text{ und } \sum_{i=1}^{n}k_i \leq n. \]
	Damit folgt
		\[ \forall i=1,\dots,n:k_i = 1 \text{ und } \sum_{i=1}^{n}k_i = n, \]
	d.h. das charakteristische Polynom zerfällt in Linearfaktoren und $ \forall i=1,\dots,n:k_i=g_i. $
\subsection{Satz}
\begin{Satz}[Trigonalisierbarkeit eines Endomorpismus]
	Ein Endomorphismus $ f\in\End(V) $ ist genau dann trigonalisierbar, wenn das charakteristische Polynom in Linearfaktoren zerfällt.
\end{Satz}
\paragraph{Bemerkung}
	Da Diagonalisierbarkeit bzw. Trigonalisierbarkeit durch die Existenz einer Darstellungsmatrix in spezieller Gestalt definiert wurde, wird in den Charakterisierungen immer (implizit) $ \dim V < \infty $ angenommen.
\paragraph{Beweis}
	Wir wissen schon: Ist $ f $ trigonalisierbar, so zerfällt $ \chi_f(t) $ in Linearfaktoren. Umkehrung: Beweis durch vollständige Induktion über $ n=\dim V $.\\
	Für $ n=1 $ ist nichts zu zeigen. Sei die Behauptung für $ n-1 $ bewiesen. Für $ n $ folgt dann:\\
	Da $ \chi_f(t) $ in Linearfaktoren zerfällt
		\[ \chi_f(t)=\prod_{i=1}^{n}(t-x_i) \]
	für geeignete $ x_1,\dots,x_n $, ist $ x_1 $ Eigenwert von $ f $. Nun seien
	\begin{itemize}
		\item $ b_1 $ ein Eigenvektor zum Eigenwert $ x_1 $ und $ U:= [\{b_1\}] $,
		\item $ U'\subset V $ ein zu $ U $ komplementärer Unterraum, und
		\item $ p,p'\in \End(V) $ die zur direkten Zerlegung $ V = U\oplus U' $ gehörenden Projektionen,
			\[ U = p(V) = \ker p' \text{ und } U' = p'(V) = \ker p, \]
		\item und $ f' := p'\circ f|_{U'}\in\End(U'). $
	\end{itemize}
	Da $ U (\neq \{0\}) $ $ f $-invarianter UR von $ V $ ist, faktorisiert das charakteristische Polynom
		\[ \chi_f(t)=\chi_{f|_U}(t)\cdot \chi_{f'}(t) = (t-x_1)\cdot \chi_{f'}(t); \]
	also zerfällt $ \chi_{f'}(t) $ in Linearfaktoren,
		\[ \chi_{f'}(t)=\prod_{i=2}^{n}(t-x_i). \]
	Nach Induktionsannahme existiert also eine Basis $ B' = (b_2,\dots,b_n) $ von $ U' $, sodass $ \xi_{B'}^{B'}(f) $ obere Dreiecksmatrix ist. Mit $ B=(b_1,\dots,b_n) $ als Basis von $ V $ gilt dann:
		\[ \xi_B^B (f) =
			\begin{pmatrix}
			x_1& Y\\
			0 & \xi_{B'}^{B'}(f')
			\end{pmatrix} \]
	ist obere Dreiecksmatrix.
\section{Der Satz von Cayley-Hamilton}
\subsection{Satz}
	Für $ f\in\End(V) $ gilt $ \chi_f(f) = 0$.
\paragraph{Unfug-Beweis}
	Durch direktes Einsetzen erhält man
		\[ \chi_f(f)=\det (\id_V f-f) = \det 0 = 0. \]
\paragraph{Zum Verständnis des Satzes}
	Ist $ V $ ein $ K $-VR mit $ n=\dim V < \infty $ und $ f\in \End (V) $, so ist
		\[ \chi_f(t) = \sum_{k=0}^{n}t^ka_k \in K[t] \]
	ein (abstraktes) Polynom in der Variablen $ t\ (= e_1\in K^\mathbb{N})$ und der Einsetzungshomomorphismus $ \psi_f:K[t]\to \End(V) $ (also ein Algebrahomomophismus) liefert
		\[ \chi_f(f) = \psi_f\left(\chi_f(t)\right) = \sum_{k=0}^{n}f^k a_k. \]
	Der Satz sagt, dass $ 0 = \chi_f(f)\in \End(V) $, d.h.
		\[ \forall v\in V: \chi_f(f)(v) = 0. \]
		
\subsection{Definition \& Lemma}\index{$ f $-zyklische Basis}
\begin{Definition}[$ f $-zyklische Basis]\label{fzykl}
	Seien $ f\in\End(V) $ und $ B $ eine \emph{$ f $-zyklische Basis} von $ V $, d.h. eine Basis der Form
		\[ B= (b_1,\dots,b_n) = \left(b,f(b),\dots,f^{n-1}(b)\right). \]		
\end{Definition}
\begin{Lemma}
	Dann existieren $ a_0,\dots,a_{n-1}\in K $ mit
		\[ f^n(b)+\sum_{k=0}^{n-1}f^k(b)a_k = 0, \]
	mit diesen Koeffizienten ist
		\[ \chi_f(t) = t^n+t^{n-1}a_{n-1}+\dots,+ta_1+a_0. \]
\end{Lemma}
\paragraph{Bemerkung}
	Im Allgemeinen existiert zu $ f\in \End(V) $ keine $ f $-zyklische Basis von $ V $, z.B. für $ f = \id_V $ und $ \dim V \geq 2 $.
\paragraph{Beweis}
	Da $ B= \left(b,f(b),\dots,f^{n-1}(b)\right) $ eine Basis ist, ist $ f^n(b)\in [B] $ und damit existieren die $ a_k $ mit
		\[ 0 = f^n(b) + \sum_{k=0}^{n-1}f^k(b)a_k. \]
	Damit ist die Darstellungsmatrix von $ f $
		\[ \xi_B^B(f) =
		\begin{pmatrix}
			0      & \cdots & \cdots & 0      & -a_0     \\
			1      & \ddots &        & \vdots & -a_1     \\
			0      & 1      & \ddots & \vdots & \vdots   \\
			\vdots & \ddots & \ddots & 0      & \vdots   \\
			0      & \cdots & 0      & 1      & -a_{n-1}
		\end{pmatrix} =: X
		\]
	und Entwicklung von $ \chi_f(t)=\det(E_nt-\xi_B^B(f)) $ nach der ersten Zeile (nach Laplaceschem Entwicklungssatz -- dieser Satz war "`nur"' eine Methode, die Terme in der Leibniz-Formel zu sortieren) liefert
	\begin{align*}
		\det(E_nt-X) &=\det 
			\begin{pmatrix}
				t      & 0      & \cdots & 0      & a_0       \\
				-1     & t      & \ddots & \vdots & a_1       \\
				0      & -1     & \ddots & 0	  & \vdots    \\
				\vdots & \ddots & \ddots & t      & \vdots    \\
				0      & \cdots & 0      & -1     & t+a_{n-1}
			\end{pmatrix}\\
		&= t\cdot\det
			\begin{pmatrix}
				t      & 0      & \cdots & 0      & a_1       \\
				-1     & t      & \ddots & \vdots & a_2       \\
				0      & -1     & \ddots & 0 	  & \vdots    \\
				\vdots & \ddots & \ddots & t      & \vdots    \\
				0      & \cdots & 0      & -1     & t+a_{n-1}
			\end{pmatrix}
		+ (-1)^{n+1} a_0 \underbrace{\det(X_{1n})}_{(-1)^{n-1}}\\
		\intertext{mittels vollständiger Induktion folgt}
		&= t (t(\dots(\underbrace{t(t+a_{n-1})+a_{n-2}}_{\det \begin{pmatrix}t&a_{n-2}\\-1&t+a_{n-1}\end{pmatrix}})\dots)+a_1)+a_0 \\
		&= t (t^{n-1}+t^{n-2}a_{n-1}+\dots+a_1)+a_0 \\
		&= t^n+t^{n-1}a_{n-1}+\dots,+ta_1+a_0,
	\end{align*}
	wie behauptet.
	
\paragraph{Beispiel}
	Zur Lösung des reellen \emph{Anfangswertproblems}
		\[ y'' + 2y' - 3y = 0,\ 
		\begin{cases}
			y(0)=4 \\
			y'(0)=0
		\end{cases} \]
	schreiben wir dieses als System erster Ordnung mit dem Ansatz $ y_1 = y $ und $y_2 = y' $:
	
	Daraus erhält man mit $ Y= (y_1,y_2) $
	  \begin{align*}
		 Y' &= (y_1',y_2') = (y',y'') = (y',-2y'+3y)\\
		 &= (y,y') \begin{pmatrix}
		 	0 & 3\\ 1 & -2
		 \end{pmatrix} = YX	\text{ mit }
		 X= \begin{pmatrix}
			0 & 3\\ 1 & -2
		\end{pmatrix},
		\end{align*}
	d.h. wir suchen eine $ \frac{d}{ds} $-zyklische Basis $ (y,\frac{d}{ds}y) = (y,y') $ eines 2-$ \dim $ UVR $ [(y,y')]\subset C^\infty(\R) $ bezüglich derer $ \frac{d}{ds}\in \End(C^\infty(\R)) $ Darstellungsmatrix $ X $ hat.
	
	Der Ansatz $ y(s) = e^{xs} (v_0,v_1)$ reduziert das AWP auf ein Eigenwertproblem.
		\[ 0 = \left(Y'-YX\right)(s) = \left(\frac{d}{ds}Y - YX\right)(s) = \underset{Y}{\underbrace{e^{xs}(v_0,v_1)}} \{E_2x-X\}\]
	bzw. (vgl. Abschnitt 3.1) mit dem zur transponierten Matrix $ X^t $ assoziierten Endomorphismus $ f_{X^t}\in \End(\R^2) $
		\[ f_{X^t}(v) = vx \text{ für }x\in\R \text{ und }v\in \R^2. \]
	Nach obigem Lemma sind die Eigenwerte Lösungen der Gleichung
		\[ 0 = \chi_{X^t}(x) = \chi_X(x) \overset{\text{Lemma}}{=} x^2+2x-3 = (x-1)(x+3). \]
	Also sind $ x_1 = 1 $ und $ x_2 = -3 $ die Eigenwerte; zugehörige Eigenvektoren erhält man als Lösungen der linearen Gleichungssysteme
		\[ (0,0) = (v_0,v_1)(E_2x_i-X)= (v_0,v_1)\begin{pmatrix}
		x_i&-3\\-1&x_i+2
		\end{pmatrix} = \begin{cases}
		(v_0,v_1)\begin{pmatrix}
		1&-3\\-1&3
		\end{pmatrix}& \text{ für } i = 1\\
		(v_0,v_1)\begin{pmatrix}
		-3&-3\\-1&-1
		\end{pmatrix}& \text{ für } i = 2
		\end{cases} \]
	Damit bekommt man Eigenvektoren $ (v_0,v_1) = (1,1) $ zum Eigenwert $ x=1 $ und $ (v_0,v_1) = (1,-3) $ zum Eigenwert $ x = -3 $.
	
	Die allgemeine, durch \emph{Superposition} (Linearkombination) erhaltene Lösung der Differentialgleichung ist also
		\[ s\mapsto Y(s) = e^s(1,1)c_1 + e^{-3s}(1,-3)c_2 \]
	mit Koeffizienten $ c_1,c_2 \in \R $. Abgleich der "`Integrationskonstanten"' $ c_1 $ und $ c_2 $ mit den Anfangsbedingungen liefert dann die Lösung
		\[ s \mapsto y(s) = 3e^s+e^{-3s}. \]
\paragraph{Bemerkung}
	Man bemerke: $ (y,y') $ ist linear unabhängig für die Lösung, ist also tatsächlich $ \frac{d}{ds} $-zyklische Basis eines 2-$ \dim $ URs $ [(y,y')]\subset C^\infty(\R) $ -- obwohl die den gleichen Raum aufspannenden "`Basislösungen"'
		\[ s\mapsto e^s \text{ und } s\mapsto e^{-3s} \]
	keine $ \frac{d}{ds} $-zyklischen Basen erzeugen, da sie lineare Differentialgleichungen erster Ordnung (mit konstanten Koeffizienten) lösen.
	
\subsection{Korollar}
\begin{Korollar}\label{korcay}
	Besitzt $ V $ eine $ f $-zyklische Basis für $ f\in\End(V) $, so gilt $ \chi_f(f)=0 $.
\end{Korollar}
\paragraph{Beweis}
	Sei also $ B=(b_1,\dots,b_n) =(b,f(b),\dots,f^{n-1}(b)) $ $ f $-zyklische Basis von $ V $ und $ a_0,\dots,a_{n-1}\in K $ so, dass
		\[ 0 = f^n(b)+\sum_{k=0}^{n-1}f^k(b)a_k. \]
	Dann gilt
		\[ \chi_f(f)(b_1) = \chi_f(f)(b) = \left(f^n+\sum_{k=0}^{n-1}f^ka_k\right)(b) = f^n(b)+\sum_{k=0}^{n-1}f^k(b)a_k. \]
	Damit folgt für $ i=2,\dots,n $
		\[ \chi_f(f)(b_i) = \chi_f(f)\left(f^{i-1}(b)\right) \stackrel{\footnotemark}{=} f^{i-1}\left(\chi_f(f)(b) \right) = 0. \]
		\footnotetext{Aufgrund der Linearität der Endomorphismen $ \End(V) $ als unitäre Algebra.}
	Da also $ V=[B] \subset \ker {\chi_f(f)}$, folgt $ \chi_f(f) = 0. $
\paragraph{Bemerkung}
	Damit ist der Satz von Cayley-Hamilton bewiesen, sofern $ V $ eine $ f $-zyklische Basis besitzt.
	
\subsection{Lemma}
\begin{Lemma}[ $ f $-invarianter UVR endlicher Dimension besitzen eine $ f $-zyklische Basis]
	Für $ f\in\End(V) $ und $ v\in V^\times  $ sei
		\[ U := \left[\left(f^k(v)\right)_{k\in{\mathbb{N}}}\right]. \]
	Damit ist $ U $ ein $ f $-invarianter UVR von $ V $. Ist $ \dim V < \infty $, so besitzt $ U $ eine $ f $-zyklische Basis $ \left(v,f(v),\dots,f^{r-1}(v)\right) $.
\end{Lemma}
\paragraph{Beweis}
	Offenbar ist $ U $ $ f $-invarianter UR:
	\begin{itemize}
		\item $ U $ ist (als lineare Hülle einer Familie) ein UVR von $ V $;
		\item da gilt
			\[ \forall k\in \mathbb{N}: f\left(f^k(v)\right)=f^{k+1}(v)\in U \]
		folgt, dass
			\[f(U) = f\left(\left[\left(f^k(v)\right)_{k\in{\mathbb{N}}}\right]\right) = \left[\left(f^{k+1}(v)\right)_{k\in{\mathbb{N}}}\right]\subset U. \]
	\end{itemize}
	Ist $ \dim V < \infty $ und $ v\neq 0 $, so existiert $ r\in \mathbb{N} $, sodass
		\[ \left(v,\dots,f^{r-1}(v)\right) \text{ linear unabhängig und }f^r(v)\in\left[\left(v,\dots,f^{r-1}(v)\right)\right]; \]
	damit ist $ \left(v,f(v),\dots, f^{r-1}(v)\right) $ $ f $-zyklische Basis von $ U $:
	\begin{enumerate}
		\item $ \left(v,\dots,f^{r-1}(v) \right) $ ist linear unabhängig.
		\item $ f^r(v) \in \left[\left(v,\dots,f^{r-1}(v)\right)\right]$, damit gilt
		\[ \forall k\in \mathbb{N}: k\geq r \Rightarrow f^k(v)\in \left[\left(v,\dots,f^{r-1}(v)\right)\right] \]
		wie man z.B. mit Induktion sehen kann: ist
			\[ f^{k-1}(v) = \sum_{j=0}^{r-1}f^j(v)x_j \in \left[\left(v,\dots,f^{r-1}(v)\right) \right], \]
		so folgt
			\[ f^k(v) = \sum_{j=1}^{r}f^j(v)x_{j-1}=f^{r}(v)x_{r-1}+\sum_{j=1}^{r-1}f^j(v)x_{j-1}\in \left[\left(v,\dots,f^{r-1}(v)\right)\right] \]
		und damit
			\[ U = \left[\left(f^k(v)\right)_{k\in \mathbb{N}}\right]\subset \left[\left(v,\dots,f^{r-1}(v)\right)\right]. \]
	\end{enumerate}
		
\subsection{Beweis vom Satz von Cayley-Hamilton}
	Zu zeigen: für $ f\in \End(V) $ gilt $ \chi_f(f)=0 $, d.h.
		\[ \forall v\in V:\chi_f(f)(v) = 0. \]
	Seien also $ v\in V^\times $ und
		\[ U := \left[\left( f^k(v)_{k\in \mathbb{N}}\right)\right]\subset V. \]
	Mit einem zu $ U $ komplementären UVR $ U'\subset V $, $ V=U\oplus U' $ und den zugehörigen Projektionen
		\[\begin{aligned}
		&p: V\to V,\ p(V) = U, &&\ker p = U' \\
		&p':V\to V,\ p'(V) = U', &&\ker p' = U
		\end{aligned}\]
	ist dann (nach \ref{finvchar}) $ \chi_f(t) = \chi_{f'}(t)\cdot \chi_{f\mid_U}(t) \text{ mit } f' := p'\circ f\mid_{U'}\in \End(U') $.
	
	Damit folgt
		\[ \chi_f(f)(v) = \chi_{f'}(f) \left(\chi_{f\mid_U}(f)(v) \right) = \chi_{f'}(f)(0)=0 \]
	nach Korollar \ref{korcay}, da $ U $ eine $ f $-zyklische Basis besitzt und $ v\in U $.
	
	%------------------ Projektion ----------------
	\begin{figure}[h]\centering
		\include{Chap4/Projektion.tikz}
		\caption{Geometrische Veranschaulichung der Projektionen}
	\end{figure}\noindent
	%------------------ Projektion ----------------#
	
% % VO-12-04-2016 % % 
\subsection{Definition}\index{Annulatorpolynom}\index{Minimalpolynom}
\begin{Definition}[Annulatorpolynom, Minimalpolynom]
	Sei $ V $ ein $ K $-VR und $ f\in\End(V) $. Dann heißt $ p\in K[t] $
		\begin{itemize}
			\item \emph{Annulatorpolynom von $ f $}, falls $ p(f)=0 $;
			\item \emph{Minimalpolynom von $ f $}, falls $ p(t) $ normiertes Annulatorpolynom minimalen Grades ist.
		\end{itemize}
\end{Definition}
\paragraph{Bemerkung}
	Jedes (polynomiale) Vielfache 
		\[ p(t) = q(t)\mu_f(t)\in K[t] \]
	eines Minimalpolynoms $ \mu_f(t) $ von $ f $ ist ein Annulatorpolynom, da
		\[ \forall v\in V: p(f)(v) = \left(q(f)\circ \mu_f(f)\right)(v) = q(f)\left(\mu_f(f)(v)\right) = q(f)(0) = 0 \]
\paragraph{Bemerkung}
	Nach dem Satz von Cayley-Hamilton hat jeder Endomorphismus $ f\in\End(f) $ ein Annulatorpolynom, also auch ein Minimalpolynom -- wenn $ \dim V < \infty $.
	
\subsection{Lemma}
\begin{Lemma}[Jedes Minimalpolynom ist Teiler des Annulatorpolynom von $ f\in\End(V) $] \label{minpol}
	Ist $ p(t)\in K[t] $ Annulatorpolynom von $ f\in\End(V) $, so ist jedes Minimalpolynom $ \mu_f(t)\in K[t] $ Teiler von $ p(t) $. 
\end{Lemma}

\paragraph{Beweis}
	Seien $ q(t),r(t)\in K[t] $ die (nach dem euklidischen Divisionsalgorithmus) eindeutigen Polynome mit
		\[ p(t) = q(t)\mu_f(t)+r(t) \text{ und }\deg r(t)<\deg \mu_f(t). \]
	Dies liefert
		\[ r(f) = p(f)-q(f)\circ \mu_f(f) = 0-q(f)(0) = 0, \]
	also $ r(t) = 0 $, denn andernfalls wäre $ \mu_f(t) $ nicht normiertes Annulatorpolynom minimalen Grades.
	
\subsection{Korollar}
\begin{Korollar}
	Das Minimalpolynom $ \mu_f(t)\in K[t] $ eines Endomorphismus $ f\in \End(V) $ ist eindeutig.
\end{Korollar}
\paragraph{Beweis}
	Sind $ \mu_f(t),\tilde{\mu}_f(t)\in K[t] $ Minimalpolynome von $ f\in \End(V) $, so gilt (siehe \ref{minpol})
		\[ \exists! q(t)\in K[t] : \tilde{\mu}_f(t) = q(t)\mu_f(t) \] % Nach Lemma 4.5.7
	wobei
		\begin{itemize}
			\item $ \deg q(t) = 0 $, da $ \deg \tilde{\mu}_f(t) \leq \deg \mu_f(t) $,
			\item $ q(t) = 1$, da $ \tilde{\mu}_f(t) $ und $ \mu_f(t) $ normiert sind.
		\end{itemize}
	Daher ist
		\[ \tilde{\mu}_f(t) = 1\cdot \mu_f(t) = \mu_f(t). \]
\paragraph{Bemerkung}
	Wie für Endomorphismen kann man Annulatorpolynome, Minimalpolynome, usw. auch für Matrizen $ X\in K^{n\times n} $ definieren:
		\begin{itemize}
			\item mithilfe der assoziierten Endomorphismen $ f_X\in \End(K^n) $, oder 
			\item mithilfe des Einsetzungshomomorphismus $ \psi_X: K[t] \to K^{n\times n}. $ % in der Algebra der quadratischen Matrizen
		\end{itemize}
	Beide Methoden liefern das gleiche Ergebnis durch den Algebrahomomorphismus zwischen den Endomorphismen und den quadratischen Matrizen.

\paragraph{Bemerkung \& Beispiel}
	Zerfällt das charakteristische Polynom in Linearfaktoren, so zerfällt auch das Minimalpolynom in dieselben Linearfaktoren:
		\[ \chi_f(t)= \prod_{i=1}^{m}(t-x_i)^{k_i} \Rightarrow \mu_f(t) = \prod_{i=1}^{m}(t-x_i)^{m_i}, \]
	wobei für $ i= 1,\dots,m $ gilt $ 1\leq m_i\leq k_i $.
	
	Zum Beispiel: 
		\begin{itemize}
			\item $ X = \begin{pmatrix}
			1&0\\0&0
			\end{pmatrix} $: $ \chi_{f_X} = t(t-1) = \mu_{f_X}(t)$
			\item $ X = \begin{pmatrix}
			1&0\\0&1
			\end{pmatrix} $: $ \chi_{f_X} = (t-1)^2 \Rightarrow \mu_{f_X}(t) = (t-1) $
			\item $ X = \begin{pmatrix}
			1&1\\0&1
			\end{pmatrix} $: $ \chi_{f_X} = (t-1)^2 = \mu_{f_X}(t)$.
		\end{itemize}

\paragraph{Bemerkung}
	Die Definition des charakteristischen Polynoms ist etwas problematisch:
		\[ \chi_f(t) := \det (\id_Vt-f) \]
	ist "`gut"' für Polynomfunktionen, aber "`nicht korrekt"' für abstrakte Polynome; die Definition 
		\[ \chi_f(t) := \sum_{\sigma\in S_n}\sgn(\sigma)\prod_{i=1}^{n}\left(\delta_{\sigma(j)j}-x_{\sigma(j)j} \right)\in K[t] \]
	mithilfe der Darstellungsmatrix $ X = (x_{ij})_{i,j\in \{1,\dots,n\}} = \xi_B^B(f) $
	von $ f $ bzgl. einer Basis $ B $ und der Leibniz-Formel ist nicht sehr übersichtlich. Vergleiche auch [Axler, Kap. 8] zum Thema.
	
	Im Gegensatz dazu: Definitionen von "`Annulatorpolynom"' und "`Minimalpolynom"' etc. sind einfach (konzeptionell).
	
	Frage: Braucht man das charakteristische Polynom überhaupt?
	Man kommt auch ohne das charakteristische Polynom "`recht weit"':
		\begin{itemize}
			\item Für $ \dim V <\infty $ folgt die Existenz eines Annulatorpolynoms, und damit des Minimalpolynoms recht einfach wegen $ \dim \End(V) <\infty $.
			\item Durch Einsetzen: Jeder Eigenwert eines Endomorphismus ist Nullstelle seines Minimalpolynoms.
			\item Umgekehrt ist auch jede Nullstelle des Minimalpolynoms Eigenwert -- ist $ \mu_f(x) = 0 $, so existiert $ q(t)\in K[t] $ mit
				\[ \mu_f(t) = q(t)(t-x); \]
			wäre $ x $ kein Eigenwert, also\footnote{Zur Erinnerung: Ist ein Endomorphismus injektiv oder surjektiv so ist er bijektiv.} $ f-\id_V x \in Gl(V) $, so gälte für den Linearfaktor unter Einsetzungshomomorphismus:
				\[ (f-\id_Vx)(V) = V \]
			also, da $\mu_f(V)=\{0\}$, müsste
				\[\{0\} = q(f)(V)\]
			d.h. $ \mu_f(t) $ wäre nicht Minimal-Polynom.
			\item Ein Endomorphismus ist diagonalisierbar, wenn sein Minimal-Polynom in paarweise verschiedene Linearfaktoren zerfällt.
		\end{itemize}
	
	Nachteil des Minimal-Polynoms: schwierig berechenbar?

% VO 14-04-2016 %
\chapter{Längen- und Winkelmessung}
Plan: 
	Längen und Winkel (in "`Punkträumen"' $ \cong $ affinen Räumen) verstehen.\\
Algebraisch:
	via Produkte (bilineare -- oder fast bilineare -- Abbildungen).
	
%TODO schönere Grafik...; dann auslagern.
%	\definecolor{qqwuqq}{rgb}{0.,0.39215686274509803,0.}
%	\definecolor{qqqqff}{rgb}{0.,0.,1.}
%	\begin{tikzpicture}[line cap=round,line join=round,>=triangle 45,scale=1.8]
%	\clip(0,0) rectangle (10,3);
%	\draw [shift={(5.635,1.07)},color=qqwuqq,fill=qqwuqq,fill opacity=0.1] (0,0) -- (34.85:0.14) arc (34.85:143.6:0.14) -- cycle;
%	\draw [->] (2.58,1) -- (0.56,2.3);
%	\draw [->] (6.6,0.35) -- (4.25,2.1);
%	\draw [->] (4.5,0.26) -- (6.8,1.9);
%
%	\draw [fill=qqqqff] (2.58,1) circle (1pt);
%	\draw[color=qqqqff] (2.6,1) node[below] {$A$};
%	\draw [fill=qqqqff] (0.56,2.3) circle (1pt);
%	\draw[color=blue] (0.5,2.3) node[above] {$B$};
%	\draw (1.5,0.1) node {Abstand a bis b $ \cong $ Länge b-a};
%	\draw (5.4,0.1) node {Winkel $ \cong $ Winkel zwischen Richtungsvektoren};
%	\end{tikzpicture}

\section{Bilinearformen \& Sesquilinearformen}
\paragraph{Zur Erinnerung}
	Sind $ V $ und  $W$ $ K $-VR, so nennt man eine Abbildung
		\[ \beta: V\times V\to W \]
	\emph{bilinear} oder ein \emph{Produkt}, wenn sie in jedem Argument linear ist:
		\begin{enumerate}[(i)]
			\item $ \forall w\in V :V\ni v \mapsto \beta(v,w)\in W $ ist linear;
			\item $ \forall v\in V: V\ni w\mapsto \beta(v,w)\in W $ ist linear.
		\end{enumerate}
	Zu vorgegebenen Werten $ \beta_{ij} \in W$ auf einer Basis $ (b_i)_{i\in I} $ von $ V $ existiert dann eine eindeutige Bilinearform $ \beta $ (Fortsetzungssatz, Bemerkung zu \ref{FSSBIL}):
		\[ \exists! \beta:V\times V\to W \text{ bilinear}: \forall i,j\in I: \beta(b_i,b_j) = \beta_{ij}. \]
\paragraph{Bemerkung}
	Man kann auch bilineare Abbildungen $ V\times V'\to W $ betrachten und, zum Beispiel, auch einen Fortsetzungssatz beweisen.
	
	Wir benötigen eine Verallgemeinerung in eine andere Richtung:
\subsection{Definition} \index{Sesquilinearform}\index{Semilinearität}
\begin{Definition}[Sesquilinearform]
Seien $ V $ ein $ K $-VR und $ K\ni x\mapsto \overline{x}\in K $ ein (Körper-) Automorphismus, d.h. eine bijektive Abbildung mit
		\[ \overline{x+y} = \overline{x}+\overline{y} \text{ und } \overline{xy} = \overline{x}\cdot \overline{y} \]
	für alle $ x,y\in K $. Eine Abbildung $ \sigma: V\times V \to K $ heißt dann \emph{Sesquilinearform} (bzgl. $ \bar{.} $), falls
		\begin{enumerate}[(i)]
			\item $ \forall v\in V: V\ni w \mapsto \sigma(v,w)\in K $ ist linear, d.h. $ \sigma(v,.)\in V^* $;
			\item $ \forall w\in V: V\ni v \mapsto \sigma(v,w)\in K $ ist \emph{semilinear}, d.h.
				\begin{enumerate}[(a)]
					\item $ \forall v,v' \in V: \sigma(v+v',w) = \sigma(v,w)+\sigma(v',w) $ und
					\item $ \forall v\in V\ \forall x\in K: \sigma(vx,w) = \overline{x}\sigma(v,w) $.
				\end{enumerate}
		\end{enumerate}
\end{Definition}

\paragraph{Beispiel}
	Die Identität $ K\ni x\mapsto \overline{x}:= x\in K $ ist offensichtlich ein Körperautomorphismus für jeden Körper $ K $. \emph{Bilinearformen} sind genau die Sesquilinearformen bezüglich $ \id_K $.
\paragraph{Beispiel}
	Für $ K = \C $ liefert \emph{komplexe Konjugation} einen Körperautomorphismus (keinen VR-Automorphismus, vgl. Abschnitt 1.4):
		\[ \C\ni x+iy \mapsto \overline{x+iy}:= x-iy \in \C. \]
	Dieses Beispiel ist unser Grund für die Einführung des Begriffs der Sesquilinearform.
\paragraph{Bemerkung}
	Ist $ \sigma $ Bilinearform und Sesquilinearform bezüglich $ \bar{.} $, so ist $ \sigma $ oder $ \bar{.} $ trivial:
		\[ \forall x\in K\ \forall v,w\in V: 0 = \sigma(vx,w) - \sigma(vx,w) = (x-\overline{x})\sigma(v,w)  \]
		\[ \Rightarrow \begin{cases}
		\forall v,w\in V: \sigma(v,w) = 0 \text{ oder}\\
		\exists v,w\in V: \sigma(v,w)\neq 0 \land \forall x\in K: \overline{x} = x.
		\end{cases} \]
\paragraph{Bemerkung}
	In $ \mathbb{Z}_p, \mathbb{Q} $ und $ \R $ gibt es nur \emph{einen} Körperautomorphismus: $ \id_K $. Ein Automorphismus $ \bar{.} $ von $ \C $ mit $ \overline{\R} = \R $ ist trivial, $ \bar{.} = \id_\C $ oder die komplexe Konjugation.
	
\subsection{Fortsetzungssatz für Sesquilinearformen}
\begin{Satz}[Fortsetzungssatz für Sesquilinearformen]
	Sind $ V $ ein $ K $-VR und $ K\ni x\mapsto \overline{x}\in K $ ein Körperautomorphismus, $ (b_i)_{i\in I} $ Basis von $ V $ und $ (s_{ij})_{i,j\in I} $ eine Familie in $ K $, so existiert eine eindeutige Sesquilinearform $ \sigma $ mit
		\[ \forall i,j\in I:\sigma(b_i,b_j) = s_{ij}. \]
\end{Satz}

% VO 19-04-2016 % 
\paragraph{Beweis}
	Wir imitieren den Beweis unseres ersten Fortsetzungssatzes für lineare Abbildungen:
	
	{Eindeutigkeit:}
	Sei $ \sigma $ eine Sesquilinearform mit der gewünschten Eigenschaft oben; gilt
		\[ v = \sum_{i\in I}b_ix_i \text{ und }w = \sum_{i\in I}b_i y_i \]
	so folgt
		\[ \sigma(v,w) = \sum_{i,j\in I}\overline{x_i}\sigma(b_i,b_j)y_j = \sum_{i,j\in I}\overline{x_i}s_{ij}y_j \]
	d.h. $ \sigma $ ist durch die Familie $ (s_{ij})_{i,j\in I} $ eindeutig bestimmt.
	
	{Existenz:}
	Da jeder Vektor $ v\in V $ eine eindeutige Basisdarstellung $ v=\sum_{i\in I}b_ix_i $ hat, wird durch
	\[ \sigma:V\times V \to K,\ (v,w)= \left(\sum_{i\in I}b_ix_i, \sum_{j\in I}b_jy_j\right) \mapsto \sigma(v,w) := \sum_{i,j\in I}\overline{x_i}s_{ij}y_j \]
	eine Abbildung wohldefiniert. Offenbar (nachrechnen) ist $ \sigma $ dann sesquilinear. 

\paragraph{Bemerkung}
	Jede Sesquilinearform $ \sigma: V\times V\to K $ liefert eine semi-lineare Abbildung
		\[ V\ni v\mapsto \sigma(v,.)\in V^*. \]
	Mit einem "`Fortsetzungssatz für semi-lineare Abbildungen"' (Aufgabe 34) hätte man auch den früher skizzierten Beweis für bilineare Abbildungen imitieren können.

\subsection{Buchhaltung}\index{Gramsche Matrix}
\paragraph{Gramsche Matrix}
	Ist $ n=\dim V < \infty $ und $ B=(b_1,\dots,b_n) $ Basis von $ V $, so kann man eine Sesquilinearform $ \sigma: V\times V\to K $ durch eine Matrix $ S $ beschreiben:
	\[ \begin{array}{c|ccc}
		\sigma &  b_1   & \dots  &  b_n   \\ \hline
		 b_1   & s_{11} &        & s_{1n} \\
		\vdots &        & \ddots &  	  \\
		 b_n   & s_{n1} &        & s_{nn}
	\end{array}  \]
	Diese Matrix
		\[ \Gamma_B(\sigma) = S = \left(\sigma(b_i,b_j)\right)_{i,j\in \{1,\dots,n\}} \]
	heißt die Darstellungsmatrix oder \emph{Gramsche Matrix} von $ \sigma $ bezüglich $ B $. Für Vektoren
		\[ v = \sum_{i=1}^{n}b_ix_i = BX \quad\text{und}\quad w = \sum_{j=1}^{n}b_jy_j = BY \quad\text{mit}\ X,Y \in K^{n\times 1} \]
	ist dann
	\begin{align*}
		 \sigma(v,w) = \sum_{i,j=1}^{n}\overline{x_i}s_{ij}y_j &= \overline{X}^tSY \\
		 &= (\overline{x_1},\dots,\overline{x_n})
			\begin{pmatrix}
				\sum_{i=1}^{n}s_{1j}y_j\\ \vdots\\ \sum_{j=1}^{n}s_{nj}y_j
			\end{pmatrix}\\
		 &= \sum_{i=1}^{n}\overline{x_i}\sum_{j=1}^{n}s_{ij}y_j.
	\end{align*}
\paragraph{Transformationsformel}
	Ein Basiswechsel $ B' = BP $ mit $ P = \xi_{B'}^B(\id_V) \in Gl(n)$ liefert dann
		\[ v = BX = (B'P^{-1})X = B' \underbrace{(P^{-1}X)}_{X'} \text{ und } w = B' \underbrace{(P^{-1}Y)}_{Y'} \]
	und damit für $ X,Y \in K^{n\times 1} $, wegen $BX=B(PX')$ und $BY=B(PY')$
		\[ \overline{X}^tSY =  \overline{(PX)}^tS(PY') 
		=(\overline{X'}^t\overline{P}^t)S(PY')
		=\overline{X'}^t \underbrace{(\overline{P}^tSP)}_{S'} Y' \]
	woraus die \emph{Transformationsformel für Gramsche Matrizen} folgt
		\[ S' = \overline{P}^tSP, \]
	wobei $ \overline{P}^t $ die Transponierte der Matrix mit Einträgen $ \overline{p_{ij}} $ ist.
\paragraph{Äquivalenz von Matrizen}
Dies liefert einen weiteren Äquivalenzbegriff für quadratische Matrizen $ S\in K^{n\times n} $:
		\[ S' \sim S :\Leftrightarrow \exists  P\in Gl(n): S' = \overline{P}^tSP. \]
	Die verschiedenen Begriffe der Äquivalenz von Matrizen (vgl. 3.1 \& 4.2) spiegeln die verschiedenen Funktionen/Bedeutungen von Matrizen wieder.
	
\paragraph{Bemerkung}
	Die Menge der Sesquilinearformen auf einem $ K $-VR ist selbst ein $ K $-VR. Ist $ n=\dim V< \infty $ und $ B $ Basis von $ V $, so erhält man (Fortsetzungssatz) einen Isomorphismus
		\[ K^{V\times V}\supset \{\sigma:V\times V\to K \text{ Sesquilinearform}\}\ni \sigma \mapsto \Gamma_B(\sigma)\in K^{n\times n}. \]
\subsection{Beispiel \& Definition} \index{Sesquilinearform!kanonische}\index{Sesquilinearform!assoziierte}
\begin{Definition}[assoziierte Sesquilinearform]
	Sei $ \bar{.}:K\to K $ Körperautomorphismus; jedes $ S\in K^{n\times n} $ liefert dann eine eindeutige Sesquilinearform
		\[ \sigma_S:K^n\times K^n \to K \text{ mit } (e_i,e_j)\mapsto \sigma_S(e_i,e_j):= s_{ij}, \] 
	die zu \emph{$ S $ assoziierte Sesquilinearform}.

	Für $ S = E_n $ bezeichnet man $ \sigma_S $ auch als \emph{kanonische Sesquilinearform}.
\end{Definition}

\subsection{Definition}\index{Sesquilinearform!(schief-)symmetrische}
\begin{Definition}[symmetrische, schiefsymmetrische und alternierende Sesquilinearformen]
	Eine Sesquilinearform $ \sigma:V\times V\to K $ auf einem $ K $-VR bzgl. eines Automorphismus $ \bar{.}:K\to K $ nennen wir
		\begin{enumerate}[(i)]
			\item \emph{symmetrisch}, falls $\forall v,w\in V: \sigma(w,v) = \overline{\sigma(v,w)} $;
			\item \emph{schiefsymmetrisch}, falls $ \forall v,w\in V: \sigma(w,v) = - \overline{\sigma(v,w)}$;
			\item \emph{alternierend}, falls $\forall v \in V: \sigma(v,v) = 0.$
		\end{enumerate}
\end{Definition}

\begin{Definition}[Hermitesche Sesquilinearform]\index{Sesquilinearform!Hermitesche}
	Falls $ K =\C $ und $ \bar{.} $ komplexe Konjugation sind, so nennt man eine symmetrische Sesquilinearform auch \emph{Hermitesche Sesquilinearform}.
\end{Definition}

\paragraph{Bemerkung}
	Ist $ \sigma $ nicht-trivial und (schief-)symmetrisch, so muss $ \bar{.} $ eine Involution sein.
	
	Nämlich: Wähle $ v,w\in V $ mit $ \sigma(v,w) = 1$; dann gilt
		\[ \forall x\in K: \overline{\overline{x}} = \overline{\sigma(vx,w)} = \pm \sigma(w,vx) = \overline{\overline{x}\sigma(v,w)} = \pm \sigma(w,v)x = \overline{\sigma(v,w)}x = x. \]
% VO 21-04-2016 %
	Ist $ \Char(K) \neq 2 $ und $ \bar{.}  $ Involution, so kann jede Sesquilinearform in einen symmetrischen und einen schiefsymmetrischen Anteil zerlegt werden:
		\[ \forall v,w\in V: \sigma(v,w) = \frac{1}{2}\left(\sigma(v,w)+\overline{\sigma(w,v)}\right) +\frac{1}{2}\left(\sigma(v,w)-\overline{\sigma(w,v)} \right).\]
\paragraph{Bemerkung}
	Ist $ \Char(K)\neq 2 $ und $ \bar{.} = \id_K $, so sind "`alternierend"' und "`schiefsymmetrisch"' äquivalent für eine Sesquilinearform $ \sigma $.
	
	Andererseits ist jede alternierende Sesquilinearform bilinear, d.h. $ \bar{.} = \id_K $ oder $ \sigma = 0 $.
	
\paragraph{Buchhaltung}
	Unter den folgenden Annahmen:
		\begin{itemize}
			\item $ \Char(K)\neq 2 $ und $ \bar{.} $ Involution;
			\item $ n=\dim V <\infty $ und $ B $ ist Basis von $ V $;
		\end{itemize}
	gilt für die Gramsche Matrix $ S = \Gamma_B(\sigma) $ einer Sesquilinearform $ \sigma $ auf $ V $:
		\begin{itemize}
			\item $ 0 = \overline{S}^t-S\Leftrightarrow \sigma $ symmetrisch;\footnote{bis auf Faktor 2: Gramsche Matrix des schiefsymmetrischen Anteils von $ \sigma $}
			\item $ 0 = S + \overline{S}^t \Leftrightarrow \sigma $ schiefsymmetrisch.
		\end{itemize}
	Nämlich:
		\[ \overline{S}^t = \begin{pmatrix}
		\overline{\sigma(b_1,b_1)} & \overline{\sigma(b_1,b_2)} &\cdots& \overline{\sigma(b_1,b_n)} \\ 
		\overline{\sigma(b_2,b_1)} &  & & \vdots \\ 
		\vdots &  & & \vdots \\ 
		\overline{\sigma(b_n,b_1)} & \cdots & & \overline{\sigma(b_n,b_n)}
		\end{pmatrix}^t =  \begin{pmatrix}
		\overline{\sigma(b_1,b_1)} & \overline{\sigma(b_2,b_1)} &\cdots& \overline{\sigma(b_n,b_1)} \\ 
		\overline{\sigma(b_1,b_2)} &  & & \vdots \\ 
		\vdots &  & & \vdots \\ 
		\overline{\sigma(b_1,b_n)} & \cdots & & \overline{\sigma(b_n,b_n)}
		\end{pmatrix} \]
		\[ S = \begin{pmatrix}
		\sigma(b_1,b_1) & \sigma(b_1,b_2) &\cdots& \sigma(b_1,b_n) \\ 
		\sigma(b_2,b_1) &  & & \vdots \\ 
		\vdots &  & & \vdots \\ 
		\sigma(b_n,b_1) & \cdots & & \sigma(b_n,b_n)
		\end{pmatrix} \]
		
\subsection{Definition} \index{Orthogonal!-raum}\index{Orthogonal}
\begin{Definition}[orthogonal, Orthogonalraum]
	Sei $ \sigma $ symmetrische Sesquilinearform auf einem Vektorraum $ V $. Zwei Vektoren $ v,w\in V $ heißen \emph{orthogonal} (bzgl. $ \sigma $),
		\[ w \perp v, \text{ falls } \sigma(v,w) = 0. \]
	Der \emph{Orthogonalraum} einer Menge $ \emptyset \neq S\subset V $ ist der UVR
		\[ S^\perp := \bigcap_{s\in S} \ker \underset{\in V^*}{\underbrace{\sigma(s,.)}}. \]
\end{Definition}
\paragraph{Bemerkung}
	Wegen der Symmetrie von $ \sigma $ ist die \emph{Orthogonalitätsrelation} symmetrisch,
		\[ w \perp v \Leftrightarrow v \perp w. \]
\paragraph{Bemerkung}
	Da $ \forall v\in V: \sigma(v,.) \in V^* $, ist der Orthogonalraum wohldefiniert und (als Schnitt von UVR) ein UVR. Offenbar gilt
		\[ \tilde{S} \subset S \Rightarrow \tilde{S}^\perp \supset S^\perp. \]
	Damit folgt direkt $ S^\perp \supset [S]^\perp $, sind andererseits $ w\in S^\perp $ und $ v\in [S] $, so gilt
		\[ v = \sum_{s\in S}sx_s \Rightarrow \sigma(v,w)= \sum_{s\in S}\overline{x_s}\sigma(s,w) = 0, \text{ da } \forall s\in S: w\perp s \]
	d.h. $ w\in S^\perp \Rightarrow w\in [S]^\perp. $ Insgesamt ist also
		\[ \forall S \subset V: [S]^\perp=S^\perp.\]
	Ähnlich zeigt man für jede Familie $ (U_i)_{i\in I} $ von UVR $ U_i\subset V $:
		\[ \left(\sum_{i\in I}U_i \right)^\perp= \bigcap_{i\in I} U_i^\perp. \]
\paragraph{Bemerkung \& Beispiel}
	Für $ S\subset V $ kann man $ S^{\perp\perp} := \left(S^\perp\right)^\perp $ betrachten; im Allgemeinen gilt
		\[ S\subset S^{\perp\perp} \text{ aber } S\neq S^{\perp\perp}. \]
	Ist etwa $ \sigma = 0 $, so ist $ S^\perp = V $ für jede Menge $ \emptyset \neq S\subsetneq V $; also ist
		\[ S^{\perp\perp} = V^\perp = V \neq S. \]

\subsection{Definition}\index{Radikal!-raum}\index{Radikal!-frei}\index{Sesquilinearform!(nicht-)degenerierte}
\begin{Definition}[Radikal(-raum),radikalfrei,nicht-degeneriert,degeneriert]
$ V^\perp $ ist der \emph{Radikal(-raum)} eines VR mit symmetrischer Sesquilinearform $ \sigma $; ist $ V^\perp = \{0 \} $, so heißt $ \sigma $ \emph{radikalfrei} oder \emph{nicht-degeneriert}, andernfalls \emph{degeneriert}.
\end{Definition}
\paragraph{Beispiel}
	Betrachte $ V=\R^2 $ mit Standardbasis $ (e_1,e_2) $.
	
	Ist für eine symmetrische Sesquilinearform (Bilinearform) $ \sigma $ auf $ V $
		\[ \sigma(e_1,e_1) = 0, \sigma(e_1,e_2) = 1, \sigma(e_2,e_2) = 0 \]
	so ist $ \sigma $ nicht-degeneriert, $ V^\perp = \{0\} $, da
		\[ v = e_1x_1 + e_2x_2 \perp e_1,e_2 \Rightarrow
		\begin{cases}
		0 = \sigma(e_1,v) = x_2\\
		0 = \sigma(e_2,v) = x_1
		\end{cases} \]
		\[ \Rightarrow x_1 = x_2 = 0, \]
	also $ V^\perp = \{0\} $, d.h. $ \sigma $ ist nicht-degeneriert.
	
	% Grafik R^2
	
	Ist aber
		\[ \sigma(e_1,e_1) = 1, \sigma(e_1,e_2) = 1, \sigma(e_2,e_2) = 1, \]
	so ist $ V^\perp = [e_1-e_2] $, d.h. $ \sigma $ ist degeneriert.
	
	% Grafik R^2 mit UVR [e_1-e_2]-> Gerade y = -x

\subsection{Lemma}
\begin{Lemma}[]
	Ist $ U\subset V $ ein zum Radikal von $ (V,\sigma) $ komplementärer UVR, $ V = V^\perp \oplus U $, so ist
		\[ \sigma\big|_{U\times U}:U\times U \to K \]
	radikalfrei.
\end{Lemma}
\paragraph{Beweis}
	Sei $ u\in U $ im Radikal von $ (U,\sigma\big|_{U\times U}) $, d.h. es gelte $ \forall v\in U: \sigma(v,u) = 0 $.
	Weil
		\[ \forall v\in V^\perp\forall w\in V: v\perp w \Rightarrow \forall v\in V^\perp: v\perp u \]
	erhalten wir $ u\in U\cap V^\perp = \{0\} $.
\paragraph{Beispiel}
	Die Einschränkung von $ \sigma $ mit (wie oben)
		\[ \forall i,j \in \{1,2\}: \sigma(e_i,e_j) = 1 \]
	auf jeden UVR $ U = [e_1x_1 + e_2x_2] $ mit $ x_1 + x_2 \neq 0 $ ist radikalfrei, denn
			\[ \sigma(e_1x_1+e_2x_2, e_1x_1+e_2x_2) = (x_1+x_2)^2 \neq 0. \]
	
% VO 26-04-2016 %
\section{Der Satz von Sylvester}
\paragraph{Beispiel}
	Ist $ \sigma $ symmetrische Sesquilinearform auf $ V=\mathbb{Z}^2_2 $ mit
		\[ \sigma(e_1,e_1):=0,\ \sigma(e_1,e_2) := 1,\ \sigma(e_2,e_2) := 0, \]
	so ist $ \sigma $ (wie vorher) nicht-degeneriert, $ V^\perp =\{0\}$; trotzdem gilt
		\[ \forall v\in V: \sigma(v,v) = 0. \]
	Das folgende Lemma zeigt, dass dies ein degenerierter Fall ist:
	
\subsection{Lemma \& Definition (Polarisation)}\index{quadratische Form}
\begin{Lemma}[Polarisationslemma]	
	Ist $ \sigma $ symmetrische Bilinearform auf einem $ K $-VR $ V $ über einem Körper $ K $ mit $ \Char K \neq 2 $, so gilt
		\[ \forall v,w\in V: \sigma(v,w)=\frac{1}{2}\left(q(v+w)-q(v)-q(w)\right), \]
	wobei
\end{Lemma}
\begin{Definition}[quadratische Form]
		\[ q:V\to K,\ v\mapsto q(v):= \sigma(v,v) \]
	die zu $ \sigma $ gehörige \emph{quadratische Form} bezeichnet.
\end{Definition}
\paragraph{Beweis}
	Ausrechnen: sind $ v,w\in V $, so gilt
	\begin{align*}
	q(v+w) &= \sigma(v+w,v+w)\\
			&= \sigma(v,v) + \sigma(v,w)+\sigma(w,v)+\sigma(w,w)\\
			&= q(v)+2\sigma(v,w)+q(w)
	\end{align*}
	diese Gleichung kann (da $ \Char K\neq 2 $) nach $ \sigma(v,w) $ aufgelöst werden.
\paragraph{Bemerkung}
	Ist $ \Char K=0 $ so kann man statt
		\[ q(v+w)=q(v)+2\sigma(v,w)+q(w) \]
	auch
		\[ q(v+w)-q(v-w) = 4 \sigma(v,w) \]
	für die Polarisation verwenden.
	
\subsection{Lemma}
\begin{Lemma}[]
	Ist $ \sigma $ symmetrische Sesquilinearform auf einem $ K $-VR $ V $ über einem Körper $ K $ mit $ \Char K \neq 2 $, so gilt
		\[ \sigma = 0 \Leftrightarrow \forall v\in V: \sigma(v,v) = 0. \]
\end{Lemma}
\paragraph{Bemerkung}
	Im Falle einer Bilinearform folgt dies direkt mit Polarisation.
	
	Im Falle eines nicht-trivialen Körperautomorphismus $ \bar{.} $ liefert $ v\mapsto \sigma(v,v) $ wegen
		\[ K\ni x \mapsto \sigma(vx,vx)-x^2\sigma(v,v) = (\overline{x}x-x^2)\sigma(v,v)\neq 0 \]
	im Allgemeinen \emph{keine} quadratische Form:
		\[ \exists x\in K: \exists v\in V: \sigma(vx,vx) = \overline{x}\sigma(v,v)x \neq x^2\sigma(v,v). \]
\paragraph{Beweis}
	Ist $ \sigma = 0 $, so folgt trivialerweise
		\[ \forall v\in V: \sigma(v,v) = 0. \]
	Sei nun $ \sigma \neq 0 $, d.h.
		\[ \exists v,w\in V: \sigma(v,w)\neq 0. \]
	Wie vorher berechnet man für $ v,w\in V $
		\[ \sigma(v+w,v+w) = \sigma(v,v)+\sigma(v,w)+\overline{\sigma(v,w)}+\sigma(w,w). \]
	Wähle nun $ v,w\in V $ mit $ \sigma(v,w)\neq 0 $, o.B.d.A. $ \sigma(v,w) = 1 $.
	\footnote{Ggf. ersetzt man $ w $ durch $ \frac{w}{\sigma(v,w)} $.}
	Ist $ \sigma(v,v) \neq 0 $ oder $ \sigma(w,w)\neq 0 $, so sind wir fertig.
	
	Gilt jedoch $ \sigma(v,v) = \sigma(w,w) = 0 $, so liefert
		\[ \sigma(v+w,v+w) = 0 + 1 + 1 + 0 \neq 0 \]
	wieder die Behauptung, da $ \Char K \neq 2 $.
\paragraph{Vereinbarung}
	Im Folgenden schließen wir $ \Char K = 2 $ aus.
	
\subsection{Lemma}
\begin{Lemma}[]
	Für eine symmetrische Sesquilinearform $ \sigma $ auf $ V $ und $ b\in V $ mit $ \sigma(b,b)\neq 0 $ gilt
		\[ V = [b]\oplus \{b\}^\perp. \]
\end{Lemma}
\paragraph{Beweis}
	Es gilt $ V = [b]+\{b\}^\perp $, da für $ v\in V $
		\[ v = u + b\frac{\sigma(b,v)}{\sigma(b,b)} \text{ mit } u := v-b\frac{\sigma(b,v)}{\sigma(b,b)}\perp b;\footnote{denn $ \sigma(b,u) = \sigma(b,v-b\frac{\sigma(b,v)}{\sigma(b,b)}) = \sigma(b,v)-\sigma(b,b)\frac{\sigma(b,v)}{\sigma(b,b)} = 0 $} \]
	Ist außerdem $v=bx$, also $ v\in [b]\cap \{b\}^\perp $, so gilt
		\[ 0 = \sigma(b,v) = \sigma(b,bx) = \underbrace{\sigma(b,b)}_{\neq 0}x \]
			\[ \Rightarrow x = 0 \Rightarrow v = 0, \]
	d.h. $ [b]\cap \{b\}^\perp = \{0\}$ und damit folgt die Behauptung.
\paragraph{Bemerkung}
	Ist $ \sigma(b,b) = 0 $ für $ b \in V $, so gilt
		\[ b\in [b]\cap \{b\}^\perp, \]
	d.h. ist $ b\neq 0 $, so ist $ [b]\cap \{b\}^\perp \neq \{0\}$. Außerdem ist dann $ \sigma\big|_{U\times U} $ für $ U:= \{b\}^\perp $ degeneriert, da
		\[ \exists v = b \in U^\times \forall u\in U: u\perp b.\]
		
\subsection{Diagonalisierungslemma}
\begin{Lemma}[Diagonalisierungslemma]
	Zu jeder symmetrischen Sesquilinearform $ \sigma $ auf einem endlichdimensionalen VR $ V $, also $ n = \dim V < \infty $,  gibt es eine Basis $ B = (b_1,\dots,b_n) $ von $ V $, die $ \sigma $ \emph{diagonalisiert}, d.h. für die gilt
		\[ \sigma(b_i,b_j)=0, \text{ falls } i\neq j. \]
\end{Lemma}

\paragraph{Beweis}
	Durch Induktion über $ n $.
	
	Für $ n = 1 $ ist die Behauptung trivial (denn $ i\neq j $ existiert nicht).
	
	Sei die Behauptung also für $\dim V = n $ bewiesen. Ist $ \sigma $ symmetrische Sesquilinearform auf $ V $ mit $ \dim V = n+1 $ und o.B.d.A. $ \sigma \neq 0 $, also
		\[ \exists b\in V: \sigma(b,b) \neq 0 \]
	nach obigem Lemma lässt sich also $ V $ aufspalten in 
		\[ V = [b]\oplus U \text{ mit } U:=\{b\}^\perp \]
	und $ \dim U = n $. Nach Annahme existiert eine Basis $ (b_1,\dots,b_n) $ von $ U $, die $ \sigma|_{U\times U} $ diagonalisiert. Da $ b \perp b_1,\dots,b_n \in U$ liefert $ B := (b,b_1,\dots,b_n) $ eine $ \sigma $-diagonalisierende Basis von $ V $. 
\paragraph{Bemerkung}
	Ist $ B = (b_1,\dots,b_n) $ eine $ \sigma $-diagonalisierende Basis, also
		\[ s_{ij} = \sigma(b_i,b_j) = 0 \text{ für } i\neq j \]
	so ist
		\[ \sigma(v,v) = \sum_{i=1}^{n}\overline{x_i}s_{ii}x_i \text{ für } v = \sum_{i=1}^{n}b_ix_i. \]
	Sind $ a_1,\dots,a_n\in K^\times $ und $ b_i' = b_ia_i $, so zeigt
		\[ s_{ij}' = \sigma(b_i',b_j') = \overline{a_i}\sigma(b_i,b_j)a_j = \overline{a_i}s_{ij}a_j, \]
	dass $ B' = (b_1',\dots,b_n') $ eine weitere $ \sigma $-diagonalisierende Basis ist.
	Man kann also die $ s_{ii} $ "`adjustieren"', sofern man die (unabhängigen) Gleichungen
		\[ s_{ii}' = \overline{a_i}s_{ii}a_i \]
	für gegebene $ s_{ii}' $ (nach den $ a_i $) lösen kann. Zum Beispiel:
\subsection{Korollar}
\begin{Korollar}[]\label{silkor1}
	Ist $ \sigma $ symmetrische Bilinearform auf einem $ \C $-VR $ V $ mit $ \dim V < \infty $, so besitzt $ V $ eine Basis $ B = (b_1,\dots,b_n) $, sodass
		\[ \exists r\in \mathbb{N}: s_{ij} = \sigma(b_i,b_j) = \begin{cases}
		1 & \text{für } i = j \leq r\\
		0 & \text{sonst}.
		\end{cases} \]
\end{Korollar}

% VO 28-04-2016 %
\paragraph{Bemerkung}
	D.h.
		\[ \Gamma_B(\sigma) = \begin{pmatrix}
		E_r & 0 \\ 0 & 0
		\end{pmatrix}. \]
\paragraph{Beweis}
	Sei (nach Diagonalisierungslemma) $ B' = (b_1',\dots,b_n') $ eine $ \sigma$-diagonalisierende Basis von $ V $ -- durch Umsortierung der Basisvektoren kann man erreichen, dass
		\[ s_{11}' ,\dots, s_{rr}' \neq 0 \text{ und } s_{r+1,r+1}' = \dots = s_{nn}' = 0 \]
	für ein $ r\in \{0,\dots,n\} $. Mit einer Wahl der Wurzel bilden die Vektoren 
		\[ b_i := \begin{cases}
		{b_i'}\cdot \frac{1}{\sqrt{s_{ii}'}} & \text{ für } i = 1,\dots,r\\
		b_i' = 0 & \text{ für } i = r+1,\dots,n 
		\end{cases} \]
	dann eine Basis $ B $ mit der gewünschten Eigenschaft:
		\[ \sigma(b_i,b_i) = \underset{s_{ii}'}{\underbrace{\sigma(b_i',b_i')}} \cdot \left(\frac{1}{\sqrt{s_{ii}'}}\right)^2 = 1 \text{ für } i = 1,\dots,r\]
		\[ \sigma(b_i,b_i) = \sigma(b_i',b_i') = 0 \text{ für } i = r+1,\dots, n. \]
		
\subsection{Korollar}
\begin{Korollar}[]\label{silkor2}
	Ist $ V $ ein $ K $-VR mit $ \dim V <\infty $ und $ \sigma $ entweder
		\begin{itemize}
			\item symmetrische Bilinearform, wenn $ K=\R $, oder
			\item Hermitesche Sesquilinearform, wenn $ K = \C $,
		\end{itemize}
	so besitzt $ V $ eine Basis $ B = (b_1,\dots,b_n) $, sodass
		\[ \exists r\in \mathbb{N}: s_{ij} = \sigma(b_i,b_j) =
		\begin{cases}
			\pm 1 & \text{ für }i = j \leq r\\
			0 & \text{ sonst.}
		\end{cases} \]
\end{Korollar}
\paragraph{Beweis}
	Wie oben -- aber:
	In diesen beiden Fällen gilt für eine diagonalisierende Basis $ B'=(b_1',\dots,b_n') $ und $ b_i = b_i'\cdot \frac{1}{a_i} $ mit $ a_i\in K $ für $ i=1,\dots,n $:
		\[ s_{ii}' = \sigma(b_i',b_i')\in \R \text{ und }
		\begin{cases}
			a_i^2 \geq 0 & \text{falls } K = \R,\\
			\overline{a_i}a_i \geq 0 & \text{falls } K =\C.
		\end{cases} \]
	Also kann man die $ s_{ii}' $ (nur) positiv reskalieren und so $ s_{ii} = 0 $ oder $ s_{ii} = \pm 1 $ erreichen.
\paragraph{Notation}
	Im Folgenden bezeichnet $ \K $ entweder $ \R $ oder $ \C $.
\paragraph{Motivation}
	Für die obige Basis $ B $ von $ V $ mit den Eigenschaften des Korollars gilt offenbar:
		\[ v\perp b_1,\dots,b_r \Rightarrow v\in [\{b_{r+1},\dots,b_{n}\}] \]
	und
		\[ b_{r+1},\dots,b_n \perp V, \]
	also ist $ (b_{r+1},\dots,b_n) $ Basis des Radikalraums $ V^\perp $ von $ (V,\sigma) $,
		\[ V^\perp = [\{b_{r+1},\dots,b_n\} ] \Rightarrow r = \dim V-\dim V^\perp. \]
	Insbesondere ist $ \dim V^\perp $ und damit $ r $ unabhängig von der Basis $ B $.

\subsection{Satz von Sylvester}\index{Sesquilinearform!Signatur}
\begin{Satz}[Satz von Sylvester (Trägheitssatz von Sylvester)]	
	Sei $ V $ ein $ \K $-VR, $ \dim V <\infty $, und $ \sigma $
		\begin{itemize}
			\item symmetrische Bilinearform, wenn $ \K=\R $, oder
			\item Hermitesche Sesquilinearform, wenn $ \K=\C $.
		\end{itemize}
	Dann gibt es eine direkte Zerlegung von $ V $ mit UVR $ V_{\pm}\subset V $,
		\[ V= V_+ \oplus_\perp V_- \oplus_\perp V^\perp,  \]
	wobei
		\[ V_+ \perp V_- \text{ und } \forall v\in V^\times_\pm: \pm \sigma(v,v) > 0. \]
\end{Satz}
\begin{Definition}[Signatur]
	Die \emph{Signatur} $ \sgn(\sigma):= (\dim V_+,\dim V_-,\dim V^\perp) $ von $ \sigma $ ist unabhängig von der direkten Zerlegung von $ V $.
\end{Definition}
\paragraph{Bemerkung \& Definition}\index{Trägheitsindex}\index{Positivitätsindex}\index{Negativitätsindex}
\begin{Definition}[Signatur, Trägheitsindex,Positivitäts- ,Negativitätsindex  ]
	Ist $ \sigma $ nicht-degeneriert, $ V^\perp = \{0\} $, so bezeichnet man auch\footnote{Die Reihenfolge kann bei verschiedenen Autoren auch jeweils $ - $ vor $ + $ sein.}
		\begin{itemize}
			\item das Paar $ \sgn (\sigma) =(\dim V_+,\dim V_-)$ als Signatur von $ \sigma $, und
			\item die Differenz $ \dim V_+ - \dim V_- $ als \emph{Trägheitsindex} von $ \sigma $.
		\end{itemize}
	Die Dimension $ \dim V_\pm $ ist auch der \emph{Positivitäts-} bzw. \emph{Negativitätsindex} von $ \sigma $.
\end{Definition}	
	Der Satz von Sylvester wird auch "`Trägheitssatz von Sylvester"' genannt.
\paragraph{Beweis}
	Sei $ B=(b_1,\dots,b_n) $ eine Basis von $ V $ und $ p,r\in \mathbb{N} $, sodass (siehe Korollar \ref{silkor2})
		\[ \sigma(b_i,b_j) =
		\begin{cases}
			+1 & \text{ für } 0 < i=j\leq p\\
			-1 & \text{ für } p < i=j\leq r\\
			0 & \text{ sonst. }
		\end{cases} \]
	Mit
		\[ V_+ := [\{b_1,\dots,b_p \}] \text{ und } V_- := [\{b_{p+1},\dots,b_r\}] \]
	erhält man die gewünschte direkte orthogonale Zerlegung von $ V $,
		\[ V = V_+ \oplus_\perp V_- \oplus_\perp V^\perp. \]
	Zur Eindeutigkeit der Signatur $ \sgn(\sigma) = (p,r-p,n-r) $:
	
	Seien
	\[ V=V_+ \oplus_\perp V_- \oplus_\perp V^\perp
	= \tilde{V}_+\oplus_\perp\tilde{V}_-\oplus_\perp\tilde{V}^\perp \]
	direkte orthogonale Zerlegungen von $ V $ mit
		\[ \pm \sigma(v,v)>0 \text{ für }
			\begin{cases}
				v\in V_\pm^\times\\
				v\in \tilde{V}_\pm^\times.
			\end{cases} \]
	Nun gilt
	\begin{gather*}
		\forall v\in V_-^\times: \sigma(v,v)< 0 \\
		\Rightarrow \forall v\in V_- \oplus V^\perp: \sigma(v,v) \leq 0
	\end{gather*}
	und damit, da $ \sigma(v,v)>0 $ für $ v\in \tilde{V}_+^\times $,
		\[ v\in (V_-\oplus V^\perp)\cap \tilde{V}_+ \Rightarrow v= 0. \]
	Es folgt, mit dem Dimensionssatz, $ \tilde{p}\leq p $, da
		\[ \tilde{p}+(n-p) = \dim \tilde{V}_+ + \dim(V_-\oplus V^\perp) \leq \dim V = n. \]
	Vertauscht man die Rollen der Zerlegungen, so erhält man die Ungleichung $ p\leq \tilde{p} $ und damit also
		\[ p = \tilde{p}. \]
\paragraph{Bemerkung}
	Diese Zerlegung $ V= V_+ \oplus_\perp V_- \oplus_\perp V^\perp $ ist im Allgemeinen \emph{nicht} eindeutig!
\paragraph{Beispiel}
	Betrachte eine durch ihre Werte auf der Standardbasis $ E=(e_1,e_2) $ gegebene symmetrische Bilinearform $ \sigma: \R^2\times \R^2 \to \R $.
		\begin{enumerate}
			\item $ S=(\sigma(e_i,e_j))_{i,j\in \{1,2\}} =
			\begin{pmatrix}
				0&1\\1& 0
			\end{pmatrix} $. Mit $ P:=\begin{pmatrix}
			1&1\\ 1& -1
			\end{pmatrix}\in Gl(2) $ liefert ein Basiswechsel $ B=EP $
				\[ (\sigma(b_i,b_j))_{i,j\in \{1,2\}} = P^tSP = \begin{pmatrix}
				2 & 0 \\ 0 & -2
				\end{pmatrix} \]
			die Signatur $ \sgn(\sigma) = (1,1,0)\cong (1,1) $.
			Jeder weitere Basiswechsel
				\[ \tilde{B}=BQ \quad\text{mit}\quad Q = \begin{pmatrix}
				\cosh(s) & \sinh(s)\\ \sinh(s)& \cosh(s) 
				\end{pmatrix}, s\in \R, \]
			liefert eine andere Zerlegung, ohne die Gramsche Matrix zu ändern.
			\item $ S=(\sigma(e_i,e_j))_{i,j\in \{1,2\}}= \begin{pmatrix}
			1 & 1\\ 1 & 1
			\end{pmatrix}. $ Der Basiswechsel $ B=EP $ wie oben liefert hier
				\[ (\sigma(b_i,b_j))_{i,j\in \{1,2\}} = P^tSP = \begin{pmatrix}
				4 & 0 \\ 0 & 0
				\end{pmatrix}, \]
			also die Signatur $ \sgn(\sigma) = (1,0,1) $ von $ \sigma $. Hier ist $ V^\perp = [\{b_2\}]$ durch $ \sigma $ festgelegt, aber jeder Basiswechsel
				\[ \tilde{B} = BQ \quad\text{mit}\quad Q= \begin{pmatrix}
				1 & 0 \\ s & 1
				\end{pmatrix}, s\in \R \]
			ändert die der Basis zugeordnete Zerlegung -- wieder ohne Änderung der Gramschen Matrix.
		\end{enumerate}

% VO 03-05-2016 %
\subsection{Bemerkung \& Definition}\index{Äquivalenz von Sesquilinearformen}
	Zur geometrischen Analyse der Lösungsmengen von quadratischen Gleichungen (Quadriken), ist es hilfreich, eine \emph{Äquivalenz} für symmetrische Bilinearformen/Sesquilinearformen $ \sigma $ und $ \sigma' $ auf einem $ \K $-VR $ V $ einzuführen:
		\[ \sigma' \sim \sigma :\Leftrightarrow \exists f\in Gl(V)\forall v,w\in V: \sigma'(v,w) = \sigma(f(v),f(w)). \]
	Ist $ \dim V <\infty $, so liefert der Satz von Sylvester im Falle
		\begin{itemize}
			\item symmetrische Bilinearform auf $ \R $-VR, oder
			\item Hermitesche Sesquilinearform auf $ \C $-VR:
		\end{itemize}
\paragraph{Satz:}
	Zwei symmetrische Sesquilinearformen sind genau dann äquivalent, wenn ihre Signaturen übereinstimmen,
		\[ \sigma' \sim \sigma \Leftrightarrow  \sgn(\sigma') = \sgn(\sigma). \]
		
\subsection{Definition}\index{Skalarprodukt}\index{Orthonormal!-system}\index{Orthonormal!-basis}
	Ein \emph{Skalarprodukt} auf einem $ K $-VR $ V $ ist eine nicht-degenerierte symmetrische Sesquilinearform
		\[ \langle.,.\rangle:V\times V \to K,\ (v,w)\mapsto \langle v,w\rangle. \]
	Eine Familie $ (e_i)_{i\in I} $ in einem VR $ (V,\langle.,.\rangle) $ mit Skalarprodukt heißt \emph{Orthonormalsystem (ONS)}, falls
		\[ \forall i,j\in I: \langle e_i,e_j\rangle = \pm \delta_{ij}; \]
	\emph{Orthonormalbasis (ONB)}, falls $ (e_i)_{i\in I} $ zusätzlich Basis ist.
\paragraph{Bemerkung}
	Ein ONS ist linear unabhängig:

	Für $ v=\sum_{i\in I}e_ix_i $ gilt
		\[ 0 = v \quad\Rightarrow \forall i\in I: 0 = \langle e_i,v\rangle = \sum_{j\in I}\langle e_i,e_j\rangle x_j = \pm x_i \]
	Ist $ \dim V < \infty $, so hat $ (V,\langle.,.\rangle) $ jedenfalls eine ONB, wenn das Skalarprodukt (siehe \ref{silkor1} und \ref{silkor2})
		\begin{itemize}
			\item symmetrische Bilinearform auf einem $ \K $-VR ist, oder
			\item Hermitesche Sesquilinearform auf einem $ \C $-VR ist.
		\end{itemize}
	Ist $ K\neq \K $, so kann die "`Normierung"' problematisch sein.
\paragraph{Beispiel}
	Auf dem $ \R $-VR der beschränkten Zahlenfolgen:
		\[ V=\{(x_n)_{n\in \mathbb{N}}\in \R^\mathbb{N}, \exists c\in \R\forall n\in \mathbb{N}: |x_n|<c \}, \]
	führen wir ein Skalarprodukt, durch Angabe seiner quadratischen Form (Polarisation!) ein:
		\[ \langle (x_n)_{n\in \mathbb{N}},(x_n)_{n\in \mathbb{N}}\rangle := \sum_{n\in \mathbb{N}}\left(\frac{x_n}{2^n}\right)^2.  \]
	%PERSONAL ADDITION:
		Mittels Polarisation ist für $(x_n)_{n \in \N},(y_n)_{n \in \N} \in V$ also
			\[\sigma((x_n)_{n \in \N},(y_n)_{n \in \N})=\sum_{n \in \N} \frac{x_ny_n}{2^{2n}}\]
	Man erhält ein ONS $ (e_n)_{n\in \mathbb{N}} $ aus skalierten Standardvektoren
		\[ e_m:\mathbb{N}\to \R,\ n\mapsto e_m(n):= 2^n\delta_{mn}. \]
	Dieses ONS kann zu einer Basis ergänzt werden (nach BES), nicht jedoch zu einer ONB (in unserem Sinne)\footnote{$v$ bezeichnet einen Vektor mit dem wir unser ONS zu erweitern versuchen, mit der Anforderung, dass die Erweiterung durch $v$ wieder ein ONS ist!}:
		\[ \langle e_m, v\rangle = \frac{x_m}{2^m} \quad\text{für } v=(x_{n})_{n\in \mathbb{N}}, \]
	also gilt die Implikation
		\[ \forall m\in \mathbb{N}: v\perp e_m \Rightarrow v = 0. \]
\paragraph{Bemerkung}
	Später wird der Begriff "`Basis"' modifiziert, z.B. in der Funktionalanalysis würde man $ (e_m)_{m\in \mathbb{N}} $ aus dem Beispiel als "`Orthonormalbasis"' bezeichnen.
\section{Euklidische \& unitäre Vektorräume}

% GRAFIK-MOTIVATION %
	% (A,V,\tau) reelle affine Ebene
	% (o; e_1,e_2) affines Bezugssystem
	% Abstand/Länge von b-a ist (falls e_1 \perp e_2 und |e_1| = |e_2| = 1):
	% d(a,b) = \|b-a\| = \sqrt{(y_1-x_1)^2+(y_2-x_2)^2}

\paragraph{Bemerkung}
	Die folgende Definition ist nur für Skalarprodukte $ \Skl{.}{.} $ sinnvoll, für die
		\[ v\mapsto \Skl{ v}{v} \in T \]
	mit einem angeordneten Teilkörper $ T \subset K $ des Körpers $ K $ (vgl. Abschnitt 1.2).
	Ein nicht-triviales Beispiel, mit $ T=\R\subset \C = K $, ist ein Hermitesches Skalarprodukt:
		\[ \forall v\in V: \Skl{ v}{v} = \overline{\Skl{ v}{v}} \Rightarrow \forall v\in V: \Skl{ v}{v} \in \R\subset \C. \]
\paragraph{Vereinbarung}
	Im Folgenden beschränken wir uns bis auf Weiteres auf $ \K $-VR mit $ \Skl{.}{.} $ Hermitsche Sesquilinearform, falls $ \K = \C $ (vgl. Satz von Sylvester).
	
\subsection{Definition}\index{positiv definit} \index{Norm!induzierte} \index{Vektorraum!Euklidischer} \index{Vektorraum!unitärer}
\begin{Definition}[positiv definit, euklidischer-, unitärer Vektorraum ]
	Ein Skalarprodukt $ \Skl{.}{.} $ auf einem $ \K $-VR $ V $ heißt \emph{positiv definit}, falls
		\[ \forall v\in V^\times:\Skl{ v}{v} >0; \]
	die \emph{induzierte Norm} eines positiv-definiten Skalarprodukts $ \Skl{.}{.} $ ist die Abbildung
		\[ \|.\|: V\to \R,\ v\mapsto \|v\| := \sqrt{\Skl{ v}{v}}\geq 0. \]
	Ein $ \K $-VR $ (V,\Skl{.}{.} ) $ mit positiv-definitem Skalarprodukt ist
		\begin{itemize}
			\item ein \emph{Euklidischer Vektorraum}, falls $ \K=\R $, und
			\item ein \emph{unitärer Vektorraum}, falls $ \K = \C $ und $ \Skl{.}{.} $ Hermitesche Sesquilinearform ist. 
		\end{itemize}
\end{Definition}

\subsection{Bemerkung \& Definition}
\begin{Definition}[negativ definit, indefinit]
	Ebenso definiert man ein Skalarprodukt als \emph{negativ definit}, falls
		\[ \forall v\in V^\times: \Skl{ v}{v} < 0; \]
	$ \Skl{.}{.} $ heißt \emph{indefinit}, falls es weder positiv, noch negativ definit ist.
	Die Definition der induzierten Norm ist nur im positiv definiten Fall sinnvoll.
\end{Definition}

% VO 10-05-2016 %

\paragraph{Beispiel}	
	Der \emph{Betrag} einer komplexen Zahl
		$ z=x+iy\in \C\cong_\R \R^2 $
			% Vektorraum-Isomorphismus - R^2 ist kein Körper! %
	ist 
		\[ |z| = \sqrt{\overline{z}z} = \sqrt{x^2+y^2} \]
	die vom Standardskalarprodukt auf $ \R^2 $ induzierte Norm.
	Insbesondere gilt
		\[ \forall z\in \C: |\Re z| \leq |z| \]	
\subsection{Bemerkung zum Zusammenhang von reellen und komplexen VR}
	Da $ \R\subset \C $ ein Teilkörper ist, kann jeder $ \C $-VR $ V $ auch als $ \R $-VR aufgefasst werden (Einschränkung der Skalarmultiplikation).
	
	Ist nun $ S\subset V $ linear unabhängig über $ \C $ (in $ V $ als $ \C $-VR), so ist
		\[ S' := S\cup Si = S \cup \{si\mid s\in S\} \]
	linear unabhängig über $ \R $, denn
		\begin{align*}
			0 &= \sum_{s\in S}sx_s + \sum_{s\in S}siy_s = \sum_{s\in S} s (x_s+iy_s)\\
			&\Rightarrow \forall s\in S: x_s + iy_s = 0\\
			&\Rightarrow \forall s\in S: x_s = y_s = 0,
		\end{align*}
	d.h. $ S' = S\cup S_i $ ist linear unabhängig über $ \R $.
	Insbesondere folgt
		\[ \dim_\R V = 2\dim_\C V.  \]

	Weiters definiert für ein Hermitesches Skalarprodukt $ \Skl{ .}{. } $ auf $ V $ (als $ \C $-VR)
		\[ \Skl{ .}{. } :V\times V\to \R,\ (v,w)\mapsto \Skl{ v}{w }_\R := \Re \Skl{v}{w } \]
	ein reelles Skalarprodukt auf $ V $ (als $ \R $-VR), das genau dann positiv definit ist, wenn $ \Skl{.}{. } $ positiv definit ist:
		\[ \forall v\in V: \Skl{v}{v }_\R = \Skl{v}{v }. \]
	Damit kann man jeden unitären Vektorraum als Euklidischen Vektorraum auffassen:
		\begin{itemize}
			\item mit verschiedenen Skalarprodukten $ \Skl{.}{.} $ bzw. $ \Skl{.}{.}_\R $, aber
			\item mit gleichen induzierten Normen. % denn diese sind ohnehin immer reell
		\end{itemize}
\index{Komplexifizierung}
\paragraph{Komplexifizierung}
	Fasst man einen $ \C $-VR $ V $ als $ \R $-VR auf, so liefert Multiplikation mit $ i\in \C $ einen Endomorphismus
		\[ J:V\to V,\ v\mapsto J(v):= vi \] % Skalarmultiplikation in $ V $ als $ \C $-VR.
	mit
		\[ J^2 = -\id_V. \]
	Insbesondere besitzt $ J $ keine reellen Eigenwerte;
	ist $ \dim V < \infty $, so folgt damit\footnote{Nach dem Fundamentalsatz (siehe \ref{fund}) treten komplexe Nullstellen immer in konjugierten Paaren auf}
		\[ \dim V = \deg{\chi_J}(t) = 0\ \text{mod}\ 2. \]
	Umgekehrt: Ist $ V $ ein $ \R $-VR und $ J\in \End(V) $ mit $ J^2 = -\id_V $ gegeben, so erhält man eine komplexe Skalarmultiplikation
		\[ \cdot :\C\times V \to V,\ (z,v)\mapsto vz := vx+J(v)y, \]
	für $ z = x+iy $.
	Ist weiter $ \Skl{.}{.} $ ein (reelles) Skalarprodukt auf $ V $, das von $ J $ erhalten wird, 
		\[ \forall v,w\in V: \Skl{Jv}{Jw} = \Skl{v}{w}, \]
	so definiert
		\[ \Skl{v}{w}_\C := \Skl{v}{w} -i \Skl{v}{Jw} \]
	ein Hermitesches Skalarprodukt auf dem so konstruierten $ \C $-VR (siehe Aufgabe 53).

\paragraph{Beispiel}\label{JDrehung}
	Ist $ \Skl{.}{.} $ das kanonische Skalarprodukt auf $ \R^2 $, mit der Standardbasis $ (e_1,e_2) $ als ONB, so definiert (Fortsetzungssatz)
		\[ J(e_1) = e_2 \text{ und } J(e_2) = -e_1, \]
	einen Endomorphismus $ J\in\End(\R^2) $ mit
		\[ J^2 = -\id_{\R^2} \text{ und } \Skl{Je_i}{Je_j} = \Skl{e_i}{e_j}. \]
	Vermöge
		\[ e_1i := J(e_1) = e_2\ \text{ und }\ e_2i := J(e_2) = J^2(e_1) = -e_1 = e_1 i^2 \]
	wird $ \R^2 $ zu einem eindimensionalen $ \C $-VR, $ \R^2 = [\{e_1\}]_\C $, da
		\[ e_1x+e_2y = e_1x+J(e_1)y = e_1 (x+iy); \]
	und\footnote{$\Skl{e_1}{e_2}_\C = \Skl{e_1}{e_1}-i\Skl{e_1}{e_2}=1$}
		\[ \Skl{e_1x+e_2y}{e_1x'+e_2y'}_\C = \Skl{e_1(x+iy)}{e_1(x'+iy')}_\C = \overline{(x+iy)}(x'+iy') \]
	liefert das kanonische Skalarprodukt auf $ \C $, mit dem kanonischen Euklidischen Skalarprodukt von $ \R^2 $ als Realteil.

\subsection{Komplexifizierungslemma}
\begin{Lemma}[Komplexifizierungslemma]
	Ist $ (V,\Skl{.}{.}) $ ein Euklidischer Vektorraum, so liefert
		\[ (v,w)(x+iy) := (vx-wy,wx+vy) \]
	eine komplexe Skalarmultiplikation auf $ V_\C := V\times V $, und
		\[ \SSkl{((v,w))}{(v',w')}_\C := \left(\Skl{v}{v'}+\Skl{w}{w'} \right)+i\left(\Skl{v}{w'}-\Skl{w}{v'} \right) \]
	ein Hermitesches Skalarprodukt, das $ (V_\C,\SSkl{.}{.}_\C) $ zu einem unitären VR macht.
\end{Lemma}

\paragraph{Beweis}
	Auf dem Euklidischen VR $ (V^2,\SSkl{.}{.}) $, wobei
		\[ \SSkl{.}{.}:V^2\times V^2 \to \R, \left((v,w),(v',w')\right)\mapsto \SSkl{(v,w)}{(v',w')} := \Skl{v}{v'}+\Skl{w}{w'}, \]
	definiere $ J\in \End(V^2) $ durch
		\[ J:V^2 \to V^2, (v,w)\mapsto J\left((v,w)\right):= (-w,v). \]
	Offenbar gilt $ J^2 = -\id_{V^2} $ und
		\[ \SSkl{J(v,w)}{J(v',w')} = \Skl{w}{w'}+\Skl{v}{v'} = \SSkl{(v,w)}{(v',w')}, \]
	sodass
		\[ (v,w)(x+iy) = (v,w)x+J(v,w)y= (vx-wy,wx+vy) \]
	und
		\[ \SSkl{(v,w)}{(v',w')}_\C = \SSkl{(v,w)}{(v',w')}-i\SSkl{(v,w)}{J(v',w')}  \]
		\[ = \left(\Skl{v}{v'}+\Skl{w}{w'}\right) -i\left(-\Skl{v}{w'}+\Skl{w}{v'}\right) \]
	$ (V^2,\SSkl{.}{.}) $ zu einem unitären VR machen, wie vorher.
\paragraph{Bemerkung}
	Mit dem "`Komplexifizierungslemma"' kann man jeden Euklidischen VR in einen unitären VR gleicher (komplexer) Dimension einbetten:
		\[ \dim_\C V_\C = \frac{1}{2}\dim_\R V^2 = \dim_\R V. \]
	Wichtig für den Zusammenhang zwischen unitären und Euklidischen VR:
	Die induzierte Norm des Hermitschen Skalarprodukts kann als die eines Euklidischen Skalarprodukts aufgefasst werden.
	
% VO 12-05-2016 %
\subsection{Definition}\index{Norm}
\begin{Definition}[Norm , normierter Vektorraum ]
	Eine Abbildung $ \|.\|:V\to \R $ auf einem $ \K $-VR $ V $ heißt \emph{Norm}, falls
	\begin{enumerate}[(i)]
		\item $ \forall v\in V^\times: \|v\| > 0 $, d.h. $ \|.\| $ ist \emph{positiv definit};
		\item $ \forall v\in V \forall x\in \K: \|vx\| = \|v\|\cdot |x| $, d.h. $ \|.\| $ \emph{positiv homogen};
		\item $ \forall v,w\in V: \| v+w\|\leq \|v\| + \|w\| $, d.h. $ \|.\| $ erfüllt die \emph{Dreiecksungleichung}.
	\end{enumerate}
	Ein Vektorraum mit Norm, $ (V,\|.\|) $ heißt \emph{normierter Vektorraum}.
\end{Definition}

\paragraph{Bemerkung}
	Die von einem positiv definiten Skalarprodukt $ \Skl{.}{.} $ induzierte Norm $ \|.\| $ erfüllt offenbar (i) und (ii); die Dreiecksungleichung zeigen wir unten.
\paragraph{Cauchy-Schwarzsche Ungleichung}
\begin{Satz}[Cauchy-Schwarzsche Ungleichung]
	Ist $ (V,\Skl{.}{.}) $ Euklidisch oder unitär, so gilt\footnote{Im Euklidischen Fall ist der Betrag offenbar überflüssig.}
		\[ \forall v,w\in V: |\Skl{v}{w}|^2 \leq \Skl{v}{v}\Skl{w}{w} \]
\end{Satz}
\paragraph{Beweis}
	Seien $ v,w\in V $, o.B.d.A $ v\neq 0 $. Wir bestimmen das Minimum der Funktion im Euklidischen Fall (unitärer Fall in der Übung)
		\[ \R \ni s\mapsto g(s):= \Skl{vs-w}{vs-w}. \]
	Einsetzen des kritischen Punktes,
		\[ 0 = g'(s) = 2\Skl{v}{v}s -(\Skl{v}{w}+\Skl{w}{v}) = 2(\Skl{v}{v}s-\Re \Skl{v}{w}) \]
		\[ \Rightarrow s = \frac{\Skl{v}{w}}{\Skl{v}{v}} \]
	liefert
		\[ 0\leq g(s) = \Skl{v}{v}\frac{\Skl{v}{w}^2}{\Skl{v}{v}^2}-2\Skl{v}{w}\frac{\Skl{v}{w}}{\Skl{v}{v}}+\Skl{w}{w} \]
		\[ = \frac{1}{\Skl{v}{v}}\left(-\Skl{v}{w}^2+\Skl{v}{v}\Skl{w}{w} \right) \Leftrightarrow 0\leq -\Skl{v}{w}^2+\Skl{v}{v}\Skl{w}{w}. \ \]
		
\subsection{Korollar}
\begin{Korollar}[]
	Die induzierte Norm in $ (V,\Skl{.}{.}) $ erfüllt die Dreiecksungleichung.
\end{Korollar}
\paragraph{Beweis}
	Ist $ (V,\Skl{.}{.}) $ Euklidischer VR, so gilt für $ v,w\in V $:
		\begin{align*}
		\|v+w\|^2 = \Skl{v+w}{v+w} &= \|v\|^2+2\Skl{v}{w}+\|w\|^2 \\
		&\overset{C.S.}{\leq} \|v\|^2+2\|v\|\cdot \|w\|+\|w\|^2 = (\|v\|+\|w\|)^2.
		\end{align*}
\paragraph{Bemerkung}
	Das Skalarprodukt eines Euklidischen VR $ (V,\Skl{.}{.}) $ kann (Polarisation) aus seiner induzierten Norm rekonstruiert werden.
	
	Nicht jede Norm ist jedoch von einem Skalarprodukt induziert (vgl. Aufgabe 56).
	Hinreichende (Satz von Jordan-von Neumann) und notwendige Bedingung ist die Parallelogrammgleichung:
	
\subsection{Parallelogrammgleichung}
\begin{Satz}[Parallelogrammgleichung]
	Für die induzierte Norm $ \|.\| $ von $ (V,\Skl{.}{.}) $ gilt:
		\[ \forall v,w\in V: \|v+w\|^2+\|v-w\|^2=2\|v\|^2+2\|w\|^2 \]
\end{Satz}		

	%------------------ Parallelogrammgleichung ----------------
 	\begin{figure}[H]\centering
 		\include{Chap5/Parallelogrammgleichung.tikz}
	\end{figure}
	%------------------ Parallelogrammgleichung ----------------
		
\paragraph{Beweis}
	Rechnung, wie bei Polarisation.
\paragraph{Beispiel}
	Für die induzierte Norm des kanonischen Skalarprodukts auf $ \R^n $ gilt die Parallelogrammgleichung:
		\[ \sum_{i=1}^{n}(x_i+y_i)^2 + \sum_{i=1}^{n}(x_i-y_i)^2 = 2\sum_{i=1}^{n}x_i^2+2\sum_{i=1}^{n}y_i^2. \]
	Für die durch
		\[ \|(x_i)_{i\in \{1,\dots,n\}}\|_1 = \sum_{i=1}^{n}|x_i| \]
	auf $ \R^n $ definierte Norm $ \|.\|_1 $ gilt sie nicht; diese Norm ist also nicht induzierte Norm eines Euklidischen Skalarprodukts auf $ \R^n $.
\paragraph{Beispiel}
	Auf dem Raum $ C^0([0,1]) $ der stetigen Funktionen auf $ [0,1] $ definiert
		\[ \|.\|_\infty: C^0([0,1]) \to \R, f\mapsto \|f\|_\infty := \max_{x\in [0,1]}|f(x)| \]
	die \emph{Maximumsnorm} (vgl. gleichmäßige Konvergenz).
	
	Für $ f,g\in C^0([0,1]) $,
		\[ f(x):= 1-x \text{ und } g(x) = x \]
	ist dann
		\[ \|f\|_\infty = \|g\|_\infty = \|f+g\|_\infty = \|f-g\|_\infty = 1 \]
	womit die Parallelogrammgleichung offenbar nicht erfüllt, und die Norm keine induzierte Norm eines Skalarprodukts ist.
\section{Euklidische Geometrie}
\subsection{Definition}\index{Euklidischer Raum}\index{Länge}\index{Abstand}\index{Winkel}
\begin{Definition}[Euklidischer Raum, Abstand, Länge, Winkel ]
	Ein \emph{Euklidischer Raum} ist eine affiner Raum $ (A,V,\tau) $ über einem Euklidischen Vektorraum $ (V,\Skl{.}{.}) $ mit induzierter Norm $ \|.\| $.
		\begin{itemize}
			\item Die \emph{Länge} eines Vektors $ v\in V $ ist seine Norm, der \emph{Abstand} zweier Punkte $ a,b\in A $ ist die Länge ihres Verbindungsvektors,
				\[ d(a,b) := \|b-a\| = \sqrt{\Skl{b-a}{b-a}}. \]
			\item Der \emph{Winkel} $ \alpha\in [0,\pi] $ zweier Vektoren $ v,w\in V^\times $ ist durch die Gleichung
				\[ \Skl{v}{w} = \|v\|\cdot \|w\|\cdot \cos \alpha \]
			definiert; der \emph{Winkel} (am Punkt $ a $) in einem nicht-degenerierten Dreieck $ \{a,b,c\} \subset A$ ist der Winkel der beiden Seitenvektoren $ v=b-a $ und $ w = c-a $.
		\end{itemize}
\end{Definition}

\begin{figure}[h]
\centering	
\definecolor{qqwuqq}{rgb}{0.,0.39215686274509803,0.}
\definecolor{qqqqff}{rgb}{0.,0.,1.}
\begin{tikzpicture}[line cap=round,line join=round,>=triangle 45,x=1.0cm,y=1.0cm]
\draw[->,color=black] (-1.9743290273232554,0.) -- (5.524000640987365,0.);
\foreach \x in {-1.,1.,2.,3.,4.,5.}
\draw[shift={(\x,0)},color=black] (0pt,2pt) -- (0pt,-2pt) node[below] {\footnotesize $\x$};
\draw[->,color=black] (0.,-1.133095836146613) -- (0.,4.526265243390068);
\foreach \y in {-1.,1.,2.,3.,4.}
\draw[shift={(0,\y)},color=black] (2pt,0pt) -- (-2pt,0pt) node[left] {\footnotesize $\y$};
\draw[color=black] (0pt,-10pt) node[right] {\footnotesize $0$};
\clip(-1.9743290273232554,-1.133095836146613) rectangle (5.524000640987365,4.526265243390068);
\draw [shift={(1.,1.)},color=qqwuqq,fill=qqwuqq,fill opacity=0.1] (0,0) -- (13.366930696316846:0.38851449058604254) arc (13.366930696316846:70.38212059188172:0.38851449058604254) -- cycle;
\draw [->] (1.,1.) -- (1.72,3.02);
\draw [->] (1.,1.) -- (3.02,1.48);
\begin{scriptsize}
\draw [fill=qqqqff] (1.,1.) circle (2.5pt);
\draw[color=qqqqff] (0.9654306181111331,0.848328065842202) node {$A$};
\draw [fill=qqqqff] (1.72,3.02) circle (2.5pt);
\draw[color=qqqqff] (1.8072120143808919,3.257117907475663) node {$B$};
\draw [fill=qqqqff] (3.02,1.48) circle (2.5pt);
\draw[color=qqqqff] (3.115210799353902,1.7160104281510296) node {$C$};
\draw[color=black] (1.1726383464236891,2.130425884776141) node {$v$};
\draw[color=black] (2.014419742693448,1.1202882092524316) node {$w$};
\draw[color=qqwuqq] (1.5093509049315927,1.495852216818939) node {$\alpha$};
\end{scriptsize}
\end{tikzpicture}
\end{figure}

\paragraph{Bemerkung}
	Nach der Cauchy-Schwarzschen Ungleichung ist für $ v,w\in V^\times $
		\[ \frac{\Skl{v}{w}}{\|v\|\cdot \|w\|}\in [-1,1]; \]
	andererseits ist 
		\[ \cos:[0,\pi]\to [-1,1]\text{ bijektiv} \]
	Damit ist der Winkel von Vektoren bzw. im Dreieck wohldefiniert.

\subsection{Definition}\index{Kongruenzabbildung}\index{Isometrie}\index{Ähnlichkeits!transformation}
\begin{Definition}[Kongruenzabbildung, Ähnlichkeitstransformation]
    Eine affine Transformation eines Euklidischen Raumes heißt
		\begin{itemize}
			\item \emph{Kongruenzabbildung} oder \emph{Isometrie}, falls sie Abstandstreu ist,
			\item \emph{Ähnlichkeitstransformation}, falls sie winkeltreu ist.
		\end{itemize}
\end{Definition}
\paragraph{Bemerkung}
	Jede Kongruenzabbildung ist Ähnlichkeitstransformation (Polarisation).
\paragraph{Bemerkung}
	Offenbar bilden die Kongruenz- bzw. Ähnlichkeitsabbildungen eines Euklidischen Raumes $ A $ auf $ A $ operierende (Transformations-)Gruppen.
	
\subsection{Definition (Geometrie)}\index{Euklidische Geometrie}\index{Ähnlichkeits!geometrie}
\begin{Definition}[Euklidische Geometrie, Ähnlichkeitsgeometrie]
	Die auf einem Euklidischen Raum operierende Gruppe der Kongruenzabbildungen bestimmt eine Euklidische Geometrie.
	
	Die Gruppe der Ähnlichkeitstransformationen eines Euklidischen Raumes $ A $ bestimmt eine Ähnlichkeitsgeometrie.
\end{Definition}

% VO 19-05-2016 %

\paragraph{Beispiel}
Jede Translation $ \tau_v:A\to A $ ist eine Isometrie:
	Für $ a,b\in A $ gilt
		\[ \exists!w\in V: b=\tau_w(a) \]
	d.h. $ w=b-a $; also
		\[ \tau_v(b) = \tau_v(\tau_w(a)) = \tau_{v+w}(a) = \tau_w(\tau_v (a)) \]
	d.h. $ w = \tau_v(b)-\tau_v(a) $.
	Damit folgt:
	\[ \|\tau_v(b)-\tau_v(a)\| = \|w\| = \|b-a\| \]
	d.h. $ \tau_v $ ist abstandstreu, da $ a,b\in A $ beliebig waren.
	
	%------------------ IsometrieTranslation ----------------
 	%\begin{figure}[h]\centering
 	%	\include{Chap5/IsometrieTranslation.tikz}
	%\end{figure}
	%------------------ IsometrieTranslation ----------------

	

\paragraph{Beispiel}
	Die Streckung mit Zentrum $ o\in A $ um den Faktor $ s\in \mathbb{R}^\times $,
		\[ o+v=a\overset{\delta_s}{\mapsto}\delta_s(a) = \delta_s(o+v):= o+vs \]
	ist winkeltreu, denn für $ a=o+v, b=o+w $ gilt
		\[ \delta_s(b)-\delta_s(a) = (o+ws)-(o+vs) = \dots = (w-v)s \]
	und damit für drei paarweise verschiedene Punkte $ a,b,c\in A $
		\[ \cos \alpha = \frac{\Skl{\delta_s(b)-\delta_s(a)}{\delta_s(c)-\delta_s(a)}}{\|\delta_s(b)-\delta_s(a)\|\|\delta_s(c)-\delta_s(a)\|} = \frac{\Skl{(b-a)s}{(c-a)s}}{\|(b-a)s\|\|(c-a)s\|} =\frac{s^2}{|s^2|} \cdot \frac{\Skl{b-a}{c-a}}{\|b-a\|\|c-a\|} \]
	d.h. $ \delta_s $ ist winkeltreu; andererseits ist $ \delta_s $ für $ s\neq \pm 1 $ nicht abstandstreu.
	Ist $ a \neq b $, so gilt dann
	\[ \|\delta_s(b)-\delta_s(a)\| = \|b-a\|\cdot |s| \neq \|b-a\|. \] 
\begin{minipage}[t]{0.45\linewidth}
	%------------------ IsometrieTranslation ----------------
		\centering
 		\tdplotsetmaincoords{0}{0} %-27
 	\begin{tikzpicture}[yscale=1,xscale=0.9,tdplot_main_coords]

 		\def\xstart{0} %x Koordinate der Startposition der Grafik
 		\def\ystart{0} %y Koordinate der Startposition der Grafik
 		\def\myscale{1.0} %ändert die Größe der Grafik (Skalierung der Grafik)
        \def\myscalex{(\myscale)}
        \def\myscaley{(\myscale)}
                
 		\def\xstartdraw{(\xstart + 2.0)} %xKoordinate des Referenzstartpunktes (in dieser Zeichnung: a)
 		\def\ystartdraw{(\ystart + 1.5)}%yKoordinate des Referenzstartpunktes (in dieser Zeichnung: a)

 		\def\balkenhoehe{(3.5)}% Länge des vertikalen blauen Balkens
 		\def\balkenlaenge{(5.5)}% Länge des horizontalen blauen Balkens
 		\def\balkenbreite{0.4} %Balkenbreite

 		%---------Begin Balken----------
 		\def\drehwinkel{0}
 		\node (VekV) at ({\xstart+0.2*cos(\drehwinkel)-\balkenbreite*sin(\drehwinkel)},{\ystart+0.5*sin(\drehwinkel)+\balkenbreite*cos(\drehwinkel)})[right, xshift=1,color=blue] {$V$};
 		\node (AffA) at ({\xstart+(\balkenlaenge-0.5)*cos(\drehwinkel)},{\ystart+(\balkenlaenge-0.5)*sin(\drehwinkel)+\balkenbreite*cos(\drehwinkel)})[color=red] {$A$};

 		\path[ shade, top color=white, bottom color=blue, opacity=.6]
 		({\xstart},{\ystart},0)  -- ({\xstart - \balkenbreite * cos(\drehwinkel)- (-\balkenbreite+0)*sin(\drehwinkel)},{\ystart - \balkenbreite * sin(\drehwinkel)+ (-\balkenbreite+0)*cos(\drehwinkel)},0)  -- ({\xstart - \balkenbreite * cos(\drehwinkel)- (\balkenhoehe+0.5)*sin(\drehwinkel)},{\ystart - \balkenbreite * sin(\drehwinkel)+ (\balkenhoehe+0.5)*cos(\drehwinkel)},0) -- ({\xstart - 0 * cos(\drehwinkel)- (\balkenhoehe+0)*sin(\drehwinkel)},{\ystart - 0 * sin(\drehwinkel)+ (\balkenhoehe+0)*cos(\drehwinkel)},0) -- cycle;

 		\path[ shade, right color=white, left color=blue, opacity=.6]
 		({\xstart},{\ystart},0)  -- ({\xstart - \balkenbreite * cos(\drehwinkel)- (-\balkenbreite+0)*sin(\drehwinkel)},{\ystart - \balkenbreite * sin(\drehwinkel)+ (-\balkenbreite+0)*cos(\drehwinkel)},0) --
 		({\xstart + (\balkenlaenge+0.5) * cos(\drehwinkel)- (-\balkenbreite+0)*sin(\drehwinkel)},{\ystart + (\balkenlaenge+0.5) * sin(\drehwinkel)+ (-\balkenbreite+0)*cos(\drehwinkel)},0) --
 		({\xstart + \balkenlaenge * cos(\drehwinkel)},{\ystart + \balkenlaenge * sin(\drehwinkel)},0)--
 		cycle;
 		%---------End Balken----------
 		\def\lightoffset{0.2*\myscale} %offeset der Vektoren

 		% rote Punkte Definition
 		
 		\node (offsetx) at ({(2.5*\myscalex},{0.0}) {}; %just an offset
 		\node (offsety) at ({0.0},{1.5*\myscaley}) {}; %just an offset
 		
 		\node (pointintersection) at ({\xstartdraw},{\ystartdraw}) {};
 		
 		
 	%	\draw[red] (fov) -- ++(295:2cm);
    %\draw[red] (fov) -- ++(335:2cm);
       %\coordinate (B) at (45:2cm) ;
        
        \node (pointa2) at ($(pointintersection) + (70:2.5)$) {};
 		\node (pointb2) at ($(pointintersection) + (25:4)$) {};
 		
 		\node (pointa1) at ($(pointintersection) + (250:1.25)$) {};
 		\node (pointb1) at ($(pointintersection) + (205:2)$) {};
 		
 	%	\node (pointa2) at ($(pointa1) - 0.15*(offsetx) + 1.0*(offsety)$) {};
 		
 	
 		\node[ xshift=3mm, yshift=0mm,color=red] (labela1) at (pointa1) {$a$};
 		\node[ xshift=6mm, yshift=-1mm,color=red] (labela2) at (pointa2) {$\delta_{\text{\tiny  -2}} (a)$};
 		\node[ xshift=1mm, yshift=4mm,color=red] (labelataub) at (pointb2) {$\delta_{\text{\tiny  -2}} (b)$};
 		\node[ xshift=0mm, yshift=4mm,color=red] (labelataua) at (pointb1) {$b$};
 	
 	%    \draw[name path=line 1] (0,0) -- (2,2);
     %   \draw[name path=line 2] (2,0) -- (0,2);
%\fill[red,name intersections={of=line 1 and line 2,total=\t}]
 %   \foreach \s in {1,...,\t}{(intersection-\s) circle (2pt) node {\footnotesize\s}};
    
    
 		%Vektoren blau
 	    %waagrecht
 		\draw[name path=a--da,{<[scale=1,length=6,width=6]}-{>[scale=1,length=6,width=6]},shorten >=2pt, shorten <=2pt,line width=0.2pt,color=blue] (pointa1) -- (pointa2);
 		\draw[name path=b--db,{<[scale=1,length=6,width=6]}-{>[scale=1,length=6,width=6]},shorten >=2pt, shorten <=2pt,line width=0.2pt,color=blue] (pointb1) -- (pointb2);
 		
 		\draw[line width=0.2pt,color=red] ($(pointintersection) + (28:0.7)$) arc[radius=0.7, start angle=28, end angle=67] ($(pointintersection) + (67:0.7)$);
 		\draw[line width=0.2pt,color=red] ($(pointintersection) + (28:0.62)$) arc[radius=0.62, start angle=28, end angle=67] ($(pointintersection) + (67:0.62)$);
 		
 		\draw[line width=0.2pt,color=red] ($(pointintersection) + (208:0.7)$) arc[radius=0.7, start angle=208, end angle=247] ($(pointintersection) + (247:0.7)$);
 		\draw[line width=0.2pt,color=red] ($(pointintersection) + (208:0.62)$) arc[radius=0.62, start angle=208, end angle=247] ($(pointintersection) + (247:0.62)$);
 	
 		%\path [name intersections={of=a--da and b--db,by=E}];
 		
 	
 		%Beschriftung der Vektoren
 		\node [color=blue] (pointlabelvu) at ($(pointintersection)!0.5!(pointa1)$) [ xshift=2mm, yshift=0mm] {\small $v$} ;
 		\node [color=blue] (pointlabelvo) at ($(pointintersection)!0.5!(pointb1)$) [above, xshift=0, yshift=0mm] {\small $w$} ;
 		
 		\node [color=blue] (pointlabelwl) at ($(pointintersection)!0.5!(pointa2)$) [ xshift=-8mm, yshift=0mm] {\small $v \cdot (-2)$} ;
 		\node [color=blue] (pointlabelwr) at ($(pointintersection)!0.5!(pointb2)$) [ xshift=7mm, yshift=-2mm] {\small $w \cdot (-2)$} ;
 		
 	

 		%Punkte malen
 		\draw[fill,color=red] (pointa1) circle [x=1cm,y=1cm,radius=0.08]node[above, xshift=0, yshift=0]{};
 		\draw[fill,color=red] (pointb1) circle [x=1cm,y=1cm,radius=0.08]node[above, xshift=0, yshift=0]{};
 		\draw[fill,color=red] (pointa2) circle [x=1cm,y=1cm,radius=0.08]node[below, xshift=5, yshift=0]{};
 		\draw[fill,color=red] (pointb2) circle [x=1cm,y=1cm,radius=0.08]node[below, xshift=5, yshift=0]{};
 		
 		\draw[fill,color=white] (pointintersection) circle [x=1cm,y=1cm,radius=0.18];
 		
 		\draw[fill,color=red] (pointintersection) circle [x=1cm,y=1cm,radius=0.08]node[below, xshift=5, yshift=0]{};
 		
 		
 		
\end{tikzpicture}
	%------------------ IsometrieTranslation ----------------
\end{minipage}
\hfill
\begin{minipage}[t]{0.45\linewidth}
	%------------------ WinkeltreuAberNichtAbstandstreu ----------------
		\centering
		\tdplotsetmaincoords{0}{0} %-27
 	\begin{tikzpicture}[yscale=0.9,xscale=0.9,tdplot_main_coords]

 		\def\xstart{0} %x Koordinate der Startposition der Grafik
 		\def\ystart{0} %y Koordinate der Startposition der Grafik
 		\def\myscale{1.0} %ändert die Größe der Grafik (Skalierung der Grafik)
        \def\myscalex{(\myscale)}
        \def\myscaley{(\myscale)}
                
 		\def\xstartdraw{(\xstart + 2.5)} %xKoordinate des Referenzstartpunktes (in dieser Zeichnung: a)
 		\def\ystartdraw{(\ystart + 2.0)}%yKoordinate des Referenzstartpunktes (in dieser Zeichnung: a)

 		\def\balkenhoehe{(4.0)}% Länge des vertikalen blauen Balkens
 		\def\balkenlaenge{(6.5)}% Länge des horizontalen blauen Balkens
 		\def\balkenbreite{0.4} %Balkenbreite

 		%---------Begin Balken----------
 		\def\drehwinkel{0}
 		\node (VekV) at ({\xstart+0.2*cos(\drehwinkel)-\balkenbreite*sin(\drehwinkel)},{\ystart+0.5*sin(\drehwinkel)+\balkenbreite*cos(\drehwinkel)})[right, xshift=1,color=blue] {$V$};
 		\node (AffA) at ({\xstart+(\balkenlaenge-0.5)*cos(\drehwinkel)},{\ystart+(\balkenlaenge-0.5)*sin(\drehwinkel)+\balkenbreite*cos(\drehwinkel)})[color=red] {$A$};

 		\path[ shade, top color=white, bottom color=blue, opacity=.6]
 		({\xstart},{\ystart},0)  -- ({\xstart - \balkenbreite * cos(\drehwinkel)- (-\balkenbreite+0)*sin(\drehwinkel)},{\ystart - \balkenbreite * sin(\drehwinkel)+ (-\balkenbreite+0)*cos(\drehwinkel)},0)  -- ({\xstart - \balkenbreite * cos(\drehwinkel)- (\balkenhoehe+0.5)*sin(\drehwinkel)},{\ystart - \balkenbreite * sin(\drehwinkel)+ (\balkenhoehe+0.5)*cos(\drehwinkel)},0) -- ({\xstart - 0 * cos(\drehwinkel)- (\balkenhoehe+0)*sin(\drehwinkel)},{\ystart - 0 * sin(\drehwinkel)+ (\balkenhoehe+0)*cos(\drehwinkel)},0) -- cycle;

 		\path[ shade, right color=white, left color=blue, opacity=.6]
 		({\xstart},{\ystart},0)  -- ({\xstart - \balkenbreite * cos(\drehwinkel)- (-\balkenbreite+0)*sin(\drehwinkel)},{\ystart - \balkenbreite * sin(\drehwinkel)+ (-\balkenbreite+0)*cos(\drehwinkel)},0) --
 		({\xstart + (\balkenlaenge+0.5) * cos(\drehwinkel)- (-\balkenbreite+0)*sin(\drehwinkel)},{\ystart + (\balkenlaenge+0.5) * sin(\drehwinkel)+ (-\balkenbreite+0)*cos(\drehwinkel)},0) --
 		({\xstart + \balkenlaenge * cos(\drehwinkel)},{\ystart + \balkenlaenge * sin(\drehwinkel)},0)--
 		cycle;
 		%---------End Balken----------
 		\def\lightoffset{0.2*\myscale} %offeset der Vektoren

 		% rote Punkte Definition
 		
 		\node (offsetx) at ({(2.5*\myscalex},{0.0}) {}; %just an offset
 		\node (offsety) at ({0.0},{1.5*\myscaley}) {}; %just an offset
 		
 		\node (pointintersection) at ({\xstartdraw},{\ystartdraw}) {};
 		
 		
  		\node (pointa2) at ($(pointintersection) + (80:2)$) {};
 		\node (pointb2) at ($(pointintersection) + (35:3)$) {};
 		\node (pointc2) at ($(pointintersection) + (5:2.25)$) {};
 		
 		\node (pointa1) at ($(pointintersection) + (260:1.25)$) {};
 		\node (pointb1) at ($(pointintersection) + (215:2)$) {};
 		\node (pointc1) at ($(pointintersection) + (185:1.5)$) {};
 
 		
 	
 		\node[ xshift=3mm, yshift=0mm,color=red] (labela1) at (pointa1) {$a$};
 		\node[ xshift=-6mm, yshift=-1mm,color=red] (labela2) at (pointa2) {$\delta_{\text{\tiny  -2}} (a)$};
 		\node[ xshift=1mm, yshift=-4mm,color=red] (labelataub) at (pointb2) {$\delta_{\text{\tiny  -2}} (b)$};
 		\node[ xshift=1mm, yshift=-4mm,color=red] (labelatauc) at (pointc2) {$\delta_{\text{\tiny  -2}} (c)$};
 		\node[ xshift=0mm, yshift=4mm,color=red] (labelb) at (pointb1) {$b$};
 		\node[ xshift=-1mm, yshift=2mm,color=red] (labelc) at (pointc1) {$c$};
 		
 	
 		%Vektoren blau
 		\draw[name path=a--da,-{>[scale=1,length=6,width=6]},shorten >=2pt, shorten <=2pt,line width=0.2pt,color=blue] (pointa2) -- (pointb2);
 		\draw[name path=b--db,-{>[scale=1,length=6,width=6]},shorten >=2pt, shorten <=2pt,line width=0.2pt,color=blue] (pointa2) -- (pointc2);
 		
 		\draw[name path=a--da,-{>[scale=1,length=6,width=6]},shorten >=2pt, shorten <=2pt,line width=0.2pt,color=blue] (pointa1) -- (pointb1);
 		\draw[name path=b--db,-{>[scale=1,length=6,width=6]},shorten >=2pt, shorten <=2pt,line width=0.2pt,color=blue] (pointa1) -- (pointc1);
 	
 	    %punktierte Linien	
 		\draw[line width=0.2pt,color=blue,dotted] (pointa1) -- (pointa2);
 		\draw[line width=0.2pt,color=blue,dotted] (pointb1) -- (pointb2);
 		\draw[line width=0.2pt,color=blue,dotted] (pointc1) -- (pointc2);
 		
 		\draw[line width=0.2pt,color=red] ($(pointintersection) + (38:0.6)$) arc[radius=0.6, start angle=38, end angle=77] ($(pointintersection) + (77:0.6)$);
 		\draw[line width=0.2pt,color=red] ($(pointintersection) + (38:0.52)$) arc[radius=0.52, start angle=38, end angle=77] ($(pointintersection) + (77:0.52)$);
 		
 		\draw[line width=0.2pt,color=red] ($(pointintersection) + (218:0.6)$) arc[radius=0.6, start angle=218, end angle=257] ($(pointintersection) + (257:0.6)$);
 		\draw[line width=0.2pt,color=red] ($(pointintersection) + (218:0.52)$) arc[radius=0.52, start angle=218, end angle=257] ($(pointintersection) + (257:0.52)$);
 	
 	    \draw[line width=0.2pt,color=red] ($(pointintersection) + (8:0.9)$) arc[radius=0.9, start angle=8, end angle=33] ($(pointintersection) + (33:0.9)$);
 		\draw[line width=0.2pt,color=red] ($(pointintersection) + (8:0.82)$) arc[radius=0.82, start angle=8, end angle=33] ($(pointintersection) + (33:0.82)$);
 		
 		\draw[line width=0.2pt,color=red] ($(pointintersection) + (188:0.9)$) arc[radius=0.9, start angle=188, end angle=213] ($(pointintersection) + (213:0.9)$);
 		\draw[line width=0.2pt,color=red] ($(pointintersection) + (188:0.82)$) arc[radius=0.82, start angle=188, end angle=213] ($(pointintersection) + (213:0.82)$);
 		
 		%Beschriftung der Vektoren
 		
 		\node [color=blue] (pointlabelvr) at ($(pointa1)!0.5!(pointb1)$) [ xshift=0mm, yshift=-3mm] {\footnotesize $b- a$} ;
 		
 		\node [color=blue] (pointlabelwr) at ($(pointa2)!0.5!(pointb2)$) [ xshift=0mm, yshift=4mm] {\footnotesize $ \delta_{\text{\tiny  -2}} (b) - \delta_{\text{\tiny  -2}} (a) $} ;
 	
 	
 		%Punkte malen
 		\draw[fill,color=red] (pointa1) circle [x=1cm,y=1cm,radius=0.08]node[above, xshift=0, yshift=0]{};
 		\draw[fill,color=red] (pointb1) circle [x=1cm,y=1cm,radius=0.08]node[above, xshift=0, yshift=0]{};
 		\draw[fill,color=red] (pointa2) circle [x=1cm,y=1cm,radius=0.08]node[below, xshift=5, yshift=0]{};
 		\draw[fill,color=red] (pointb2) circle [x=1cm,y=1cm,radius=0.08]node[below, xshift=5, yshift=0]{};
 		\draw[fill,color=red] (pointc1) circle [x=1cm,y=1cm,radius=0.08]node[above, xshift=0, yshift=0]{};
 		\draw[fill,color=red] (pointc2) circle [x=1cm,y=1cm,radius=0.08]node[above, xshift=0, yshift=0]{};
 		
 		\draw[fill,color=white] (pointintersection) circle [x=1cm,y=1cm,radius=0.18];
 		
 		\draw[fill,color=red] (pointintersection) circle [x=1cm,y=1cm,radius=0.08]node[below, xshift=5, yshift=0]{};
 		\node[ xshift=-2mm, yshift=3mm,color=red] (label0) at (pointintersection) {\small $0$};
 		
 		
 		
\end{tikzpicture}
	%------------------ WinkeltreuAberNichtAbstandstreu ----------------
\end{minipage}		

\paragraph{Zur Erinnerung}
	Jede affine Abbildung $ \alpha:A\to A' $ besitzt einen (eindeutigen) \emph{linearen Anteil} $ \lambda:V\to V' $, sodass
		\[ \forall a\in A\forall v\in V: \alpha(a+v) = \alpha(a)+\lambda(v); \]
	ist $ \alpha $ eine affine Transformation, so ist $ \lambda \in Gl(V) $.
\paragraph{Bemerkung}
	Jede Ähnlichkeitstransformation ist Komposition einer Streckung und einer Kongruenzabbildung.
	
	Nämlich: Ist $ \alpha $ Ähnlichkeitstransformation mit linearem Anteil $ \lambda\in Gl(V) $, so erhält $ \lambda $ Winkel von Vektoren, insbesondere also Orthogonalität.
	Nun wähle $ w\in V^\times $ und setze
		\[ s := \frac{\|w\|}{\|\lambda w\|}. \]
	Ist dann $ v\in V $ mit $ \|v\|=\|w\| $, so folgt
		\[ v+w \perp v-w \Rightarrow \lambda(v+w)\perp \lambda(v-w) \Rightarrow \|\lambda(v)\| = \|\lambda(w)\|, \]
	also 
		\[ \forall v\in V^\times: \frac{\|\lambda(v)\|}{\|v\|} = \|\lambda(v\frac{\|w\|}{\|v\|})\|\frac{1}{\|w\|} = \frac{\|\lambda (w)\|}{\|w\|} = \frac{1}{s}. \]
	Mit einem beliebigen Streckungszentrum $ o\in A $ erhält man also eine Isometrie durch
		\[ \delta_s\circ \alpha :A\to A. \]
\paragraph{Beispiel}
	Eine \emph{nicht-triviale} Scherung ist \emph{keine} Ähnlichkeitstransformation. Beweis in der Übung. % 3 Zeilen Rechnung, 5 Zeilen Begründung!

\subsection{Lemma \& Definition}
\begin{Lemma}
	Eine affine Transformation $ \alpha:A\to A $ eines Euklidischen Raumes $ A $ ist genau dann eine Kongruenzabbildung, wenn ihr linearer Anteil $ \lambda $ \emph{orthogonal} ist:
\end{Lemma}
\begin{Definition}[orthonogale Gruppe]
		\[ \lambda\in O(V):= \{f\in Gl(V)\mid \forall v,w\in V: \Skl{f(v)}{f(w)} = \Skl{v}{w}\}. \]
	$ O(V) $ heißt die \emph{orthonogale Gruppe} von $ (V,\Skl{.}{.}) $.
\end{Definition}

\paragraph{Bemerkung}
	$ O(V)\subset Gl(V) $ ist eine Gruppe. Beweis in der Übung.
\paragraph{Bemerkung}
	Ist $ f\in \End(V) $, so folgt die Injektivität von $ f $ aus
		\[ \forall v,w\in V: \Skl{f(v)}{f(w)} = \Skl{v}{w}. \]
	Aus $ f(v) = 0 $ folgt nämlich
		\[ 0 = \|f(v)\| = 0 = \|v\| \Rightarrow v = 0, \text{ da } \Skl{.}{.}\text{ pos. definit.} \]
	Ist $ \dim V <\infty $, so folgt mit dem Rangsatz, $ \dim V = \rg f + \dfkt f = \rg $, dass $ f\in Gl(V) $.\\
	Im Fall $ \dim V = \infty $ ist $ f $ nicht notwendigerweise surjektiv, wie der \emph{Shiftoperator}
		\[ f\in \End(\R^\N), \forall n\in \N: f(e_n) = e_{n+1} \]
	zeigt.
\paragraph{Beweis (Lemma)}
	Sei $ (A,V,\tau) $ Euklidischer Raum über einem Euklidischen VR $ (V,\Skl{.}{.}) $ und $\alpha:A\to A $ Affinität mit linearem Anteil $ \lambda\in Gl(V) $. Dann ist $ \alpha $ genau dann Isometrie, wenn
		\[ \forall a,b\in A: \|\lambda(b-a)\| = \|\alpha(b)-\alpha(a)\| = \|b-a\|,  \]
	also (Polarisation), wenn $ \lambda\in O(V) $.
\subsection{Definition}
\begin{Definition}[unitäre Gruppe]
	Ist $ (V,\Skl{.}{.}) $ unitärer VR, so heißt $ f\in Gl(V) $ mit
		\[ \forall v,w\in V: \Skl{f(v)}{f(w)} = \Skl{v}{w} \]
	\emph{unitär}; die \emph{unitäre Gruppe} von $ (V,\Skl{.}{.}) $ ist die Gruppe
		\[ U(V) := \{f\in Gl(V)\mid \forall v,w\in V: \Skl{f(v)}{f(w)} = \Skl{v}{w} \}. \]
\end{Definition}

% VO 24-05-2016 %

\subsection{Schulgeometrie}
	Betrachte eine Euklidische Ebene $ A^2 $ über Euklidischem VR $ (\R^2,\Skl{.}{.}) $ mit kanonischem Skalarprodukt $ \forall i,j\in \{1,2\}:\Skl{e_i}{e_j} = \delta_{ij}. $\\
	Weiter (vgl. Abschnitt \ref{JDrehung}) bezeichne $ J\in \End(\R^2) $ den durch
		\[ J(e_1) = e_2 \text{ und } J(e_2) = -e_1 \]
	definierten Endomorphismus, also eine "`$ 90^{\circ}$-Drehung"',
	bzw. die $ \R^2 $ mit $ \C $ identifizierende komplexe Multiplikation mit $ i $,
		\[ v(x+iy) = vx+J(v)y \text{ für }
		\begin{cases}
			v\in \R^2\\ (x+iy)\in \C.
		\end{cases} \]
	Man bemerke: Für $ v\in \R^2\setminus \{0\} $ ist $ \{Jv\}^\perp = [v] $ und damit
		\[ \forall w\in \R^2: w\perp Jv \Leftrightarrow w\parallel v. \]
    So ermöglicht $ J $ einen einfachen Wechsel zwischen \emph{parametrischer} und \emph{impliziter Darstellung} (e.g. \emph{Hessesche Normalform}) einer Geraden
   		\begin{align*}
		    g &= \{p = o + vx\mid x\in \R\} \Leftrightarrow \\
		    g &= \{p\in A^2\mid \Skl{p-o}{Jv} = 0\} 
		\end{align*} 

    	%------------------ ImpliziteDarstellungGerade ----------------
     	\begin{figure}[h]\centering
     		\include{Chap5/ImpliziteDarstellungGerade.tikz}
     		\caption{Implizite Darstellung einer Geraden}
    	\end{figure}
    	%------------------ ImpliziteDarstellungGerade ----------------

\subsection{Definition}\index{Kreis}
\begin{Definition}[Kreis, Mittelpunkt]
    Ein \emph{Kreis} mit \emph{Mittelpunkt} $ z\in A^2 $ und \emph{Radius} $ r\geq 0 $ ist die Menge
		\[ k = \{p\in A^2\mid \|p-z\| = r\}. \]
\end{Definition}
\paragraph{Bemerkung}
	Es ist mitunter sinnvoll, Punkte als Kreise mit Radius $ r=0 $ zu betrachten.
	
\subsection{Umkreissatz}\index{Umkreis}\index{Streckensymmetrale}
\begin{Satz}[Umkreissatz]
	Sei $ \{a,b,c\} \subset A^2 $ ein nicht-degeneriertes Dreieck. Dann gibt es genau einen Kreis $ k\subset A^2 $, den \emph{Umkreis} des Dreiecks, der die Eckpunkte $ a,b $ und $ c $ des Dreiecks enthält.
\end{Satz}	
	Sein Mittelpunkt ist der Schnittpunkt der drei \emph{Streckensymmetralen}/\emph{Mittelsenkrechten} $ m_{ab}, m_{bc}$ und $ m_{ca} $ des Dreiecks, wobei

    		\[ m_{ab} = \{p\in A^2\mid \Skl{p-s_{ab}}{b-a} = 0\} \text{ mit } s_{ab} = a\frac{1}{2}+b\frac{1}{2} \text{ etc}. \]
 
    	%------------------ Umkreis ----------------
     	\begin{figure}[h]\centering
     		\include{Chap5/Umkreis.tikz}
    	\end{figure}
    	%------------------ Umkreis ----------------

	
\paragraph{Beweis}
	Definiere
		\[ g_{ab} : A^2\to \R, p\mapsto g_{ab}(p):= 2\Skl{p-s_{ab}}{b-a} \]
	und analog $ g_{bc} $ und $ g_{ca} $ (zyklische Vertauschung).
	Für $ p\in A^2 $ gilt dann mit
		\[ g_{ab}(p) \overset{!}{=} \Skl{(p-a)+(p-b)}{(p-a)-(p-b)}  = \|p-a\|^2-\|p-b\|^2 \tag{$ * $} \]
	damit folgt
		\[ \forall p\in A^2: (g_{ab}+g_{bc}+g_{ca})(p) = 0, \]
	also
		\[ p\in m_{ab}\cap m_{bc} \Rightarrow p\in m_{ca}. \]
	Nun ist
		\[ m_{ab} = \{p(x) = s_{ab}+J(b-a)x\mid x\in \R\} \]
	mit $ J(b-a)\not\perp b-c $, da das Dreieck $ \{a,b,c\} $ nicht-degeneriert ist. Dies liefert einen eindeutigen Schnittpunkt $ z\in p(x)\in m_{ab}\cap m_{bc} $ als Lösung der linearen Gleichung
		\[ 0 = g_{bc}(p(x)) = 2\Skl{s_{ab}+J(b-a)x-s_{bc}}{c-b} \]
		\[ = 2\Skl{J(b-a)}{c-b}x+\Skl{a-c}{c-b}. \]
	Wegen $ (*) $ gilt nun für diesen Schnittpunkt $ z $
		\[ \|z-a\| = \|z-b\| = \|z-c\| \tag{$ ** $} \]
	d.h. $ a,b $ und $ c $ liegen auf einem Kreis mit Mittelpunkt $ z $.
	Andererseits: Wegen $ (*) $ impliziert $ (**) $, dass $ z\in m_{ab}\cap m_{bc} $, womit die Eindeutigkeit von $ z $ und damit des Umkreises folgt.

\subsection{Höhensatz}\index{Höhen!-schnittpunkt}\index{Höhen}
\begin{Satz}[Höhensatz]
	Die \emph{Höhen} $ h_a,h_b $ und $ h_c $ eines nicht-degenerierten Dreiecks $ \{a,b,c\} \subset A^2 $ schneiden sich in einem Punkt, dem \emph{Höhenschnittpunkt}, wobei
		\[ h_a = \{p\in A^2\mid \Skl{p-a}{b-c} = 0 \},\text{ etc.} \]
\end{Satz}

	%------------------ Hoehensatz ----------------
     	\begin{figure}[h]\centering
     		\include{Chap5/Hoehensatz.tikz}
    	\end{figure}
   %------------------ Hoehensatz ----------------


	Beweis in der Übung, analog zum Umkreissatz.

\subsection{Euler-Gerade}\index{Euler-Gerade}
\begin{Satz}[Euler-Gerade]
	Seien $ s$, $h$ und $ z $ Schwerpunkt, Höhenschnittpunkt und Umkreismittelpunkt eines nicht-degenerierten Dreiecks $ a,b,c\subset A^2 $.
	Dann gilt
		\[ s=z\frac{2}{3}+h\frac{1}{3}. \]
	Ist $ s\neq z $, so liegen die drei Punkte also auf einer eindeutig bestimmten Geraden, \emph{Euler-Geraden}, mit einem Teilverhältnis $ (zs:hs)=-\frac{1}{2} $.
\end{Satz}
	Beweis in der Übung.

\subsection{Satz von Pythagoras}
\begin{Satz}[Satz von Pythagoras]
	 In einem Dreieck $ \{a,b,c\}\subset A^2 $ mit einem rechten Winkel $ \alpha = \frac{\pi}{2} $ bei $ a $ gilt stets
		 \[ \|c-a\|^2+\|a-b\|^2 = \|c-b\|^2. \]
\end{Satz}
\paragraph{Beweis}
	Offenbar gilt $ c-b = (c-a)+(a-b) $, daher
		\[ \|c-b\|^2 = \|c-a\|^2 + 2\Skl{c-a}{a-b}+\|a-b\|^2 = \|c-a\|^2+\|a-b\|^2. \]
\paragraph{Bemerkung}
	Für allgemeine Dreiecke liefert die gleiche Rechnung den Cosinussatz:
		\[ \|b-c\|^2 = \|c-a\|^2+\|a-b\|^2- 2\|c-a\|\|a-b\|\cos \alpha. \]
\paragraph{Bemerkung}
	Ist $ (o;e_1,e_2) $ ein affines Bezugssystem in $ A^2 $ mit
		\[ e_1 \perp e_2 \text{ und } \|e_1\| = \|e_2\| = 1, \]
	so ist jeder Punkt $ a\in A^2 $ Eckpunkt eines \emph{rechtwinkligen} Dreiecks
		\[ \{o,o+e_1x_1,o+e_1x_1+e_2x_2\} \text{ für } a = o+e_1x_1+e_2x_2; \]
	der Abstand vom Ursprung ist also (Pythagoras)
		\[ \|a-o\| = \sqrt{x_1^2+x_2^2}. \]
	Wegen seiner Translationsinvarianz kann der Abstand zwischen beliebigen Punkten genau so berechnet werden.
	
\subsection{Definition}\index{Kartesisches Bezugssystem}
\begin{Definition}[Kartesisches Bezugssystem]
	Ein \emph{kartesisches Bezugssystem} $ (o,E) $ eines Euklidischen Raumes $ (A,V,\tau) $ über einem Euklidischen VR $ (V,\Skl{.}{.}) $ besteht aus einem Ursprung $ o\in A $ und einer ONB $ E $ von $ (V,\Skl{.}{.}) $.
\end{Definition}
\paragraph{Bemerkung}
	In jedem endlichdimensionalen Euklidischen Raum gibt es ein kartesisches Bezugssystem, im Allgemeinen ist dies nicht so (vgl. Übung).
	
\subsection{Lemma}
\begin{Lemma}[]
	Ist $ (o;E) $ mit $ E=(e_i)_{i\in I} $ kartesisches Bezugssystem eines Euklidischen Raumes $ (A,V,\tau) $ über $ (V,\Skl{.}{.}) $, so ist 
		\[ \forall a\in A: a = o + \sum_{i\in I} e_i \Skl{e_i}{a-o} \]
\end{Lemma}
\paragraph{Beweis}
	Da $ E $ Basis ist, existiert zu $ a\in A $ eine Familie $ (x_i)_{i\in I} $ in $ \R $ mit
		\[ a = o + \sum_{i\in I}e_ix_i, \]
	wobei
		\[ \forall i\in I: \Skl{e_i}{a-o} = \Skl{e_i}{\sum_{j\in I}e_jx_j} = \sum_{j\in I}\delta_{ij}x_j = x_i. \]
\section{Orthogonalprojektion}
\subsection{Definition}
\begin{Definition}[Orthogonalprojektion]\index{Orthogonalprojektion}
	Sei $ (A,V,\tau) $ ein Euklidischer Raum über einem Euklidischen VR $ (V, \Skl{.}{.}) $. Dann heißt
		\begin{itemize}
			\item $ p\in \End(V) $ \emph{Orthogonalprojektion}, falls $ p $ Projektion ist, $ p^2 = p $, mit
				\[ \ker p \perp p(V) \]
			\item $ \pi: A\to A $ \emph{Orthogonalprojektion}, falls $ \pi $ Parallelprojektion ist, mit einer Orthogonalprojektion $ p\in End(V) $ als linearem Anteil.
		\end{itemize}
\end{Definition}
\paragraph{Bemerkung}
	Ist $ p\in \End(V) $ Orthogonalprojektion, so ist auch die komplementäre Projektion $ p' = \id_V-p $ Orthogonalprojektion, denn
		\[ \ker p' = p(V)\perp \ker p = p'(V.) \]
\paragraph{Bemerkung}
	Ist $ (o;E) $ mit $ E=(e_i)_{i\in I} $ kartesisches Bezugssystem eines Euklidischen Raumes $ A $ und $ J\subset I $, so liefert
		\[ \pi: A\to A,\ a = o+v \mapsto o+p(v) := o+\sum_{i\in J}e_i\Skl{e_i}{v} \]
	eine Orthogonalprojektion von $ A $ auf
		\[ \pi(A) = o + p(V) = o + [(e_i)_{i\in J}]. \]

% VO 31-05-2016 %
\subsection{Gram-Schmidtsches Orthogonalisierungsverfahren}\index{Gram-Schmidtsches Orthogonalisierungsverfahren}
	Sei $ (V,\Skl{.}{.}) $ ein Euklidischer VR und $ (v_1,\dots,v_n) $ linear unabhängig in $ V $; dann existiert ein ONS $ (e_1,\dots,e_n) $ mit
		\[ [(v_1,\dots,v_k)] = [(e_1,\dots,e_k)] \quad\text{und}\quad \Skl{e_k}{v_k} > 0 \text{ für } k = 1,\dots,n \tag{$ * $} \]
\paragraph{Beweis}
	Induktion über $ n $. Ist $ n=1 $, so liefert $ e_1 := v_1\cdot\frac{1}{\|v_1\|} $ das gewünschte Orthonormalsystem. 
	
	Ist $ (v_1,\dots,v_{n+1}) $ linear unabhängig und (nach Induktions-Annahme) $ (e_1,\dots,e_n) $ ONS mit
		\[ V_k := [(v_1,\dots,v_k)] = [(e_1,\dots,e_k)] \text{ und } \Skl{e_k}{v_k} > 0 \text{ für } k = 1,\dots,n, \]
	so setzen wir
		\[ p:V\to V,\ v\mapsto p(v):= v-\sum_{i=1}^{n}e_i\Skl{e_i}{v}\in \{e_1,\dots,e_n\}^\perp; \]
	da $ v_{n+1}\notin V_n $ ist\footnote{da nur $e_1,\dots,e_n$ skaliert abgezogen werden!}  $p(v_{n+1}) \neq 0$ und 
		\[e_{n+1} := p(v_{n+1})\frac{1}{\|p(v_{n+1})\|} \]
	ergänzt dann $ (e_1,\dots,e_n) $ zum gesuchten Orthonormalsystem\footnote{da $p$ Orthogonalprojektion ist, $v_{n+1}\notin \ker p$ und $e_1,\dots,e_n\in \ker p$}.
\paragraph{Bemerkung}
	Das ONS $ (e_1,\dots,e_n) $ im Gram-Schmidtschen Verfahren ist durch die Bedingungen $ (*) $ eindeutig festgelegt.
\paragraph{Bemerkung}
	Der Beweis lässt sich wörtlich auf unitäre VR übertragen.

\subsection{Korollar \& Definition}\index{orthogonales Komplement}
\begin{Korollar}
	Ist $ U\subset V $ UVR eines Euklidischen VR (oder unitären VR) $ (V,\Skl{.}{.}) $ mit $ \dim V<\infty $, so gilt
		\[ V = U\oplus U^\perp. \]
\end{Korollar}
\begin{Definition}[orthogonale Komplement]
	Der UVR $ U^\perp $ heißt dann das \emph{orthogonale Komplement} von $ U $ (in $ (V,\Skl{.}{.}) $).
\end{Definition}
\paragraph{Beweis}
	Für $ v\in U\cap U^\perp $ ist $ \Skl{v}{v} = 0 $, also $ v = 0 $, da das Skalarprodukt positiv definit ist. Sei $ (e_1,\dots,e_k) $ ONB von $ U $ (Gram-Schmidt) und
		\[ p:V\to V,\ v\mapsto p(v) := \sum_{i=1}^{k}e_i\Skl{e_i}{v}\in U. \]
	Wegen
		\[ \Skl{e_j}{v-p(v)} = \Skl{e_j}{v}-\sum_{i=1}^{k}\delta_{ij}\Skl{e_i}{v} = 0 \]
	für $ j = 1,\dots, k $ ist dann
		\[ \forall v\in V:v=p(v) + (v-p(v))\in U+U^\perp, \]
	also $ V = U+U^\perp $.
\paragraph{Bemerkung}
	Die Einschränkung $ \dim V < \infty $ wurde nur benutzt, um die Orthogonalprojektion $ p\in \End(V) $ zu definieren/konstruieren. Insbesondere reicht es, $ \dim U < \infty $ anzunehmen.
\paragraph{Bemerkung}
	Ist $ \dim V < \infty $, so ist $ U^{\perp\perp} = U$.

\subsection{Beispiel \& Definition}\index{Spiegelung}
	Ist $ p\in \End(V) $ eine Projektion und $ p' = \id_V -p $, so erhält man eine Involution
		\[ s := p-p' \in \End(V) \]
% TODO Grafik %
	Im Falle einer Orthogonalprojektion $ p $ nennt man die zugehörige Transformation	
		\[ \sigma: A\to A,\ o+v\mapsto \sigma(o+v):= o+s(v) \]
	eines Euklidischen Raumes eine \emph{Spiegelung}: $ \sigma $ ist eine Isometrie, da
		\[ \forall v\in V: \|p(v)\pm p'(v)\|^2 =
			\begin{cases}
				\|v\|^2 &\text{für }+\\ \|s(v)\|^2 &\text{für } -
			\end{cases}
			= \|p(v)\|^2+\|p'(v)\|^2 \pm \underset{ = 0\text{, da }p(V)\perp p'(V)}{2\underbrace{\Skl{p(v)}{p'(v)}}} \]
\paragraph{Bemerkung}
	Jede Kongruenzabbildung eines endlichdimensionalen Euklidischen Raumes ist Komposition von Spiegelungen.

\subsection{Beispiel \& Definition}\index{Drehung}
	Ist $ A^2 $ Euklidische Ebene mit kartesischem Bezugssystem $ (o;e_1,e_2) $ und $ J\in \End(\R^2) $ wie oben,
	$ J(e_1) = e_2 \text{ und } J(e_2) = -e_1 $
	so liefert 
		\[ \rho_\theta : A^2\to A^2,\ o+v\mapsto \rho_\vartheta(o+v):= o+v\cos \vartheta + J(v)\sin\vartheta \]
	eine \emph{Drehung} mit \emph{Zentrum} $ o\in A^2 $ und \emph{Drehwinkel} $ \vartheta \in \R $. Die affine Abbildung $ \rho_\vartheta $ ist dann Komposition zweier Spiegelungen, 
		\[ \rho_\vartheta= \sigma'\circ \sigma, \]
	die durch ihre Fixpunktgeraden festgelegt sind:
		\[ g = o + [e_1] \text{ und }g' = o+[e_1'] \text{ mit } e_1' = e_1 \cos\frac{\vartheta}{2}+e_s\sin \frac{\vartheta}{2}. \]
	
% TODO Grafik Drehung % 

\subsection{Lemma}
\begin{Lemma}\label{sadj}
	Eine Projektion $ p\in \End(V) $ ist genau dann Orthogonalprojektion, wenn
		\[ \forall v,w\in V: \Skl{p(v)}{w} = \Skl{v}{p(w)} \]
\end{Lemma}
\paragraph{Beweis}
	Sei $ p\in \End(V) $ Projektion und $ p' = \id_V - p $ die komplementäre Projektion mit
		\[ \ker p = p'(V) \text{ und } p(V) = \ker p'. \]
	Ist $ p $ Orthogonalprojektion, $ p(V)\perp \ker p = p'(V) $, so ist für $ v,w\in V $
		\[ \Skl{p(v)}{w}-\Skl{v}{p(w)} = \Skl{p(v)}{p(w)}+\underset{0}{\underbrace{\Skl{p(v)}{p'(w)}}} - \Skl{p(v)}{p(w)}-\underset{0}{\underbrace{\Skl{p'(v)}{p(w)}}} = 0. \]
	Gilt andererseits für $ v,w\in V $, also insbesondere für $ v\in p(V), w\in \ker p $, stets
		\[ 0 = \Skl{p(v)}{w}-\Skl{v}{p(w)} = \Skl{v}{w}, \]
	so ist $ p(V)\perp \ker p $, also $ p $ Orthogonalprojektion.

\chapter{Struktursätze für Endomorphismen}
\section{Adjungierte \& duale Abbildungen}
	Zunächst: Ziel dieses Kapitels ist besseres, strukturelles Verständnis der Bedingungen für orthogonale Transformationen bzw. Orthogonalprojektionen:
		\[ \forall v,w\in V: \Skl{f(v)}{f(w)} = \Skl{v}{w} \text{ bzw. } \Skl{p(v)}{w} = \Skl{v}{p(w)}. \]
\paragraph{Motivation}
	Jede Sesquilinearform $ \sigma: V\times V\to K $ liefert  eine (semi-)lineare Abbildung
		\[ V\ni v\mapsto \sigma(v,.)\in V^*. \]
	Ist die Sesquilinearform $ \sigma $ symmetrisch und nicht-degeneriert, d.h. ein Skalarprodukt auf $ V $, so ist die Abbildung injektiv:
		\[ \sigma(v,.) = 0 \Rightarrow v = 0 \Rightarrow v\in V^\perp = \{0\}. \]
	Ist die Abbildung auch surjektiv, so kann man sie benutzen, um $ V^* $ und $ V $ zu identifizieren,
		\[ V^* \mathop{\overline{\simeq}} V. \]
5\subsection{Rieszsches Darstellungslemma}
\begin{Lemma}[Rieszsches Darstellungslemma]
	Sei $ (V,\Skl{.}{.}) $ ein $ K $-VR mit Skalarprodukt. Die \emph{kanonische Injektion}
		\[ \phi : V\to V^*,\ v\mapsto \phi(v):= \Skl{v}{.}. \]
	ist semi-linear und injektiv. Ist $ \dim V < \infty $, so ist $ \phi $ auch surjektiv; wir nennen dann
		\begin{itemize}
			\item $\nabla w := \phi^{-1}(w) $ den \emph{Gradienten} von $ w\in V^* $, und
			\item $ \phi: V\to V^* $ die \emph{kanonische Identifikation} von $ (V,\Skl{.}{.}) $ mit $ V^* $.
		\end{itemize}
\end{Lemma}
% VO 02-06-2016 %
\paragraph{Beweis}
	Semi-Linearität und Injektivität folgen sofort aus den Eigenschaften des Skalarprodukts.\footnote{Semi-Linearität in der linken Komponente; Nicht-Degeneriertheit impliziert Injektvität.}
	
	Ist $ \dim V <\infty $, so ist $ \phi:V\to V^* $ wegen $ \dim V^* = \dim V $ und der Injektivität auch surjektiv.
\paragraph{Bemerkung}
	Dies ist eine "`kleine"' Version des Rieszschen Darstellungssatzes
	\[ \forall \omega\in V^*\exists! w\in V: \omega = \Skl{w}{.}. \]
	Der "`richtige"' Satz schränkt die Dimension nicht ein, und ist ein wichtiges Hilfsmittel in der Funktional-Analysis.
\paragraph{Bemerkung}
	Ist $ E = (e_1,\dots,e_n) $ ONB von $ (V,\Skl{.}{.}) $, so gilt für die Vektoren der dualen Basis $ E^* = (e_1^*,\dots,e_n^*) $
		\[ \nabla e_i^* = e_i\frac{1}{\Skl{e_i}{e_i}} = \pm e_i, \]
	da für $ j=1,\dots,n $ gilt:
		\[ \Skl{e_i\frac{1}{\Skl{e_i}{e_i}}}{e_j} = \delta_{ij} = e_i^*(e_j), \]
	also
		\[ \phi(e_i\frac{1}{\Skl{e_i}{e_i}}) = e_i^*. \]
	Insbesondere gilt im Falle eines Euklidischen VR
		\[ \forall i=1,\dots,n: \nabla e_i^* = e_i \Leftrightarrow \phi(e_i)= e_i^*, \]
	d.h. $ \phi $ realisiert den früher diskutierten (vgl. Abschnitt 1.4) durch duale Basen gegebenen
	Isomorphismus -- im Falle von ONB.

\subsection{Korollar \& Definition}\index{Adjungierte}
\begin{Definition}[Adjungierte]
	Sind $ (W,\SSkl{.}{.}) $ und $ (V,\Skl{.}{.}) $ Vektorräume mit Skalarprodukten, $ \dim W < \infty $ und $ f\in \Hom(W,V) $, so hat $ f $ eine eindeutige \emph{Adjungierte} $ f^* \in \Hom(V,W) $;
	dabei ist $ f^* $ \emph{adjungiert zu $ f $}, falls
		\[ \forall v\in V\forall w\in W: \SSkl{f^*(v)}{w} = \Skl{v}{f(w)}. \]
\end{Definition}
\paragraph{Achtung:}
$ V $ und $ W $ sind VR über dem gleichen Körper $ K $; die Skalarprodukte sind sesquilinear bzgl. des gleichen Körperautomorphismus!
\paragraph{Beweis}
	Für jedes $ v\in V $ definiert
		\[ \omega_v:W\to K,\ w\mapsto \omega_v(w) := \Skl{v}{f(w)} \]
	eine Linearform $ \omega_v\in W^* $; nach Rieszschem Darstellungslemma erhält man daher eine eindeutige Abbildung 
		\[ f^*:V\to W,\ v\mapsto f^*(v):= \nabla \omega_v. \]
	%PERSONAL:
	Genauer: Wir haben eine Gleichung $W^*\ni\SSkl{s}{.} = \Skl{v}{f(.)}\in W^*$ und wollen das Bild $s\in W$ der Adjungierten von $v$ bestimmen (also $s\in W$ wird unbestimmt als Bildvektor unter der Adjungierten von $v$ angesetzt).
	Nach Rieszschen Darstellungslemma gibt es, da $\dim W<\infty$ und damit die kanonische Injektion surjektiv, einen Vektor $s\in W$ dessen kanonische Injektion die Linearform der rechten Seite ergibt.
	Damit ist $s(v)=\nabla\Skl{v}{f(.)}$.
	
	Die Linearität von $ f^* $ folgt aus der dualen Abbildung, (siehe Bemerkung bei \ref{adjhom}).
\paragraph{Bemerkung}
	Offenbar (Symmetrie) ist $ f^{**} = f $, wenn $ f^{**} := (f^*)^* $ existiert.

\subsection{Definition \& Lemma}
\begin{Definition}[transponiert, dual]
	Ist $ f\in \Hom(W,V) $, so heißt $ f^t\in \Hom(V^*,W^*) $,
		\[ f^t:V^*\to W^*,\ \nu\mapsto f^t(\nu):= \nu\circ f \]
	zu $ f $ \emph{transponiert} oder \emph{dual}. Sind $ \SSkl{.}{.} $ und $ \Skl{.}{.} $ Skalarprodukte auf $ W $ bzw. $ V $ und
		\[ \psi:W\to W^*\text{ und }\phi:V\to V^* \]
	die zugehörigen kanonischen Injektionen, und ist $ f^*\in \Hom(V,W) $ adjungiert zu $ f $, so gilt
	% TODO: Diagramm
	\begin{equation}
		\psi \circ f^* = f^t\circ \phi. \tag{$\star$}
		\label{lemtrans}
	\end{equation}

\end{Definition}
\paragraph{Beweis}
	Für $ v\in V $ und $ w\in W $ gilt:
		\begin{align*}
			\left((\psi\circ f^*)(v)\right)(w) &= \SSkl{f^*(v)}{w}\\
			\left((f^t\circ \phi)(v)\right)(w) &= (\phi(v)\circ f)(w) = \Skl{v}{f(w)}
		\end{align*}
	und nach Definition der Adjungierten folgt die Gleichheit.
\paragraph{Bemerkung}
	Ist $ \dim W<\infty $, so ist $ \psi $ bijektiv und (\ref{lemtrans}) kann als Definition dienen:
		\[ f^* := \psi^{-1}\circ f^t\circ \phi. \]
	Wegen $ f^t\in \Hom(V^*,W^*) $ folgt damit auch $ f^*\in \Hom(V,W) $.\label{adjhom}
\paragraph{Bemerkung}
	$ f\in \Hom(W,V) $ hat \emph{immer} eine Transponierte, eine Adjungierte aber nur unter bestimmten Voraussetzungen, z.B. wenn $ \dim W < \infty $.
\paragraph{Bemerkung}
	Oft wird die Transponierte/Duale $ f^t $ auch mit $ f^* $ bezeichnet.
\paragraph{Buchhaltung}
	Sind $ B=(b_1,\dots,b_n) $ und $ C=(c_1,\dots,c_m) $ Basen von $ V $ bzw. $ W $ und $ f\in \Hom(W,V) $, so gilt (siehe Aufgabe 75):
		\[ \xi_{B^*}^{C^*}(f^t) = \left(\xi_C^B(f)\right)^t. \]
	Sind $ (V,\Skl{.}{.}) $ und $ (W,\SSkl{.}{.}) $ unitär (oder Euklidisch) und $ B$ und $C $ ONB, so gilt
		\[ \xi_B^C(f^*) = \left(\xi_C^B(f)\right)^*, \]
	wobei 
		\[ X^*:= \overline{X}^t \text{ für } X\in K^{n\times m}. \]
	In diesem Falle gilt nämlich $ b_j^*=\phi(b_j) $ und $ c_i^* = \psi(c_i) $ und damit
		\[ x_{ij}^* = c_i^*(f^*(b_j)) = \SSkl{c_i}{f^*(b_j)} = \overline{\Skl{b_j}{f(c_i)}} =\overline{b_j^*(f(c_i))}= \overline{x_{ji}}. \]
\paragraph{Bemerkung}
	Sind $ f,g \in \Hom(W,V)$ und $ x\in K $, so gilt
		\[ (f+gx)^t = f^t+g^tx \quad\text{und}\quad (f+gx)^* = f^*+g^*\overline{x}. \]
		
\subsection{Lemma}
\begin{Lemma}
	Sind $ f\in \Hom(W,V) $ und $ g\in \Hom(V,U) $, so gilt
		\[ (g\circ f)^t = f^t \circ g^t \quad\text{und}\quad (g\circ f)^* = f^* \circ g^*. \]
\end{Lemma}
\paragraph{Beweis}
	Über ein kommutatives Diagramm:
	%------------------ KommutativesDiagramm ----------------
	% https://de.wikipedia.org/wiki/Kommutatives_Diagramm
     	\begin{figure}[ht]\centering
     		\include{Chap6/KommutativesDiagramm.tikz}
     		\caption{Kommutatives Diagramm}
    	\end{figure}\noindent
   %------------------ KommutativesDiagramm ----------------	
	oder nachrechnen:
	Für beliebige $ \nu \in U^* $ gilt
		\[ (g\circ f)^t(\nu) = \nu \circ (g\circ f) = (\nu \circ g)\circ f = f^t(g^t(\nu)). \]	
	damit folgt für die Adjungierten (sofern sie existieren)
		\[ \psi \circ (g\circ f)^* = (g\circ f)^t \circ \mu = f^t\circ g^t \circ \mu = \psi \circ f^*\circ g^* \]
	mit den kanonischen Injektionen.
% VO 07-06-2016 %

\subsection{Lemma}\label{fruh}
\begin{Lemma}[]
	Sind $ f\in \Hom(W,V) $ und $ f^*\in \Hom(V,W) $ adjungiert, so gilt
		\[ \ker f^* = f(W)^\perp \]
	und $ f^* $ ist injektiv, wenn $ f $ surjektiv ist.
\end{Lemma}
\paragraph{Beweis}
	Da die Skalarprodukte $ \SSkl{.}{.} $ und $ \Skl{.}{.} $ auf $ W $ bzw. $ V $ nicht-degeneriert sind, gilt
		\begin{align*}
			v\in \ker f^* &\Leftrightarrow \forall w\in W: \SSkl{f^*(v)}{w} = 0 \\
			&\Leftrightarrow \forall w\in W: \Skl{v}{f(w)} = 0 \\
			&\Leftrightarrow v\in f(W)^\perp,
		\end{align*}
	also die erste Behauptung; ist $ f(W)  = V$, so folgt damit $ \ker f^* = f(W)^\perp = V^\perp = \{0\} $.

\subsection{Korollar}
\begin{Korollar}[]
	Sind $ (W,\SSkl{.}{.}) $ und $ (V,\Skl{.}{.}) $ Euklidisch oder unitär mit $ \dim V, \dim W < \infty $ und $ f\in \Hom(W,V) $ und $ f^*\in \Hom(V,W) $ adjungiert, so gilt
		\[ \rg f^* = \rg f. \]
\end{Korollar}
\paragraph{Beweis}
	In der Situation hier (endlich-dimensional, positiv definite Skalarprodukte) sind $ f(W) $ und $ f(W)^\perp $ komplementäre UR und damit
		\[ V = f(W)\oplus f(W)^\perp = f(W)\oplus \ker f^*, \]
	also
		\[ \rg f = \dim V-\dfkt f^* = \rg f^* \]
	nach Rangsatz für $ f^* $.
\paragraph{Bemerkung}
	Ist $ \dim V, \dim W <\infty $, so folgt dann mit der Gleichheit der Abbildungen über die kanonischen Injektionen auch
		\[ \rg f^t = \rg f \text{ für } f\in \Hom(V,W). \]
	Daher gilt $ \rg f^* = \rg f $ dann auch für allgemeine Skalarprodukte auf $ \R $- oder $ \C $-VR $ V $ und $ W $.

\subsection{Buchhaltung}
	Mit den zu $ X\in K^{n\times m} $ und $ Y\in K^{k\times n} $ assoziierten Homomorphismen $ f_X\in \Hom(K^m,K^n) $ und $ f_Y\in \Hom(K^n,K^k) $ zeigt man also
		\[ (YX)^t = X^tY^t \quad\text{und}\quad (YX)^* = X^*Y^*. \]
	Weiters folgt wegen $ \rg f_X^* = \rg f_X^t = \rg f_X $
		\[ \rg X^t = \rg X^* = \rg X; \]
	anders ausgedrückt: der Zeilenrang einer Matrix $ X\in K^{n\times m} $ stimmt mit ihrem (Spalten-)Rang überein.
	
	Insbesondere gilt:
		\[ X\in Gl(n) \Rightarrow X^t \in Gl(n). \]
		
\subsection{Lemma}
\begin{Lemma}[]
	Sei $ (V,\Skl{.}{.}) $ Euklidisch (unitär); $ f\in Gl(V) $ ist genau dann orthogonal (unitär), wenn $ f $ und $ f^{-1} $ adjungiert sind, also $  f^*=f^{-1} $.
\end{Lemma}
\paragraph{Beweis}
	Ist $ f^{-1}=f^* $ zu $ f $ adjungiert, so ist $ f\in O(V) $ (bzw. $ f\in U(V) $), da
		\[ \forall v,w\in V: \Skl{f(v)}{f(w)} = \Skl{(f^*\circ f)(v)}{w} = \Skl{v}{w}; \]
	ist umgekehrt $ f\in O(V) $ (bzw. $ U(V) $), so ist
		\[ \forall v,w\in V: \Skl{v}{f(w)} = \Skl{f(f^{-1}(v))}{f(w)} = \Skl{f^{-1}(v)}{w}, \]
	d.h. $ f^{-1} $ ist zu $ f $ adjungiert.
\paragraph{Bemerkung}
	$ f\in Gl(V) $ ist also genau dann orthogonal/unitär, wenn
		\[ f^*\circ f = f\circ f^* = \id_V. \]
\paragraph{Bemerkung}
	Das Lemma lässt sich auf Isomorphismen $ f\in \Iso(W,V) $ verallgemeinern:
	$ f $ ist genau dann \emph{isometrisch} (längentreu), wenn $ f^{-1} = f^* $.

\subsection{Buchhaltung}\index{Orthogonale Gruppe}\index{Unitäre Gruppe}
	Ist $ (V,\Skl{.}{.}) $ Euklidischer VR, $ \dim V <\infty $,
	und $ X=\xi_B^B(f) $ Darstellungsmatrix von $ f\in \End(V) $ bzgl. einer ONB $ B $,
	so ist
		\[ f\in O(V)\Leftrightarrow X^*X = E_n, \]
	und analog für einen unitären VR $ V $.
	Daher definiert man die \emph{orthogonale (unitäre) Gruppe in $ n $ Variablen}:
		\[ O(n) := \left\{X\in \R^{n\times n}\mid X^*X = X^tX = E_n \right\}, \]
		\[ U(n) := \left\{X\in \C^{n\times n}\mid X^*X = \overline{X}^tX = E_n\right\}.\]
	$ X $ ist also orthogonal/unitär, wenn die Spalten von $ X $ eine ONB von $ \K^{n\times 1} $ mit dem kanonischen Skalarprodukt bilden.

\subsection{Definition}\index{selbstadjungiert}\index{schiefadjungiert}
\begin{Definition}[selbstadjungiert, symmetrisch ]
	Sei $ (V,\Skl{.}{.}) $ Euklidisch oder unitär; $ f\in \End(V) $ heißt dann
		\begin{itemize}
			\item \emph{selbstadjungiert} oder \emph{symmetrisch}, falls
				\[ \forall v,w\in V: \Skl{f(v)}{w} = \Skl{v}{f(w)}; \]
			\item \emph{schiefadjungiert} oder \emph{schiefsymmetrisch}, falls
				\[ \forall v,w\in V: \Skl{f(v)}{w} + \Skl{v}{f(w)} = 0. \]
		\end{itemize}
\end{Definition}
\paragraph{Bemerkung}
	$ f\in \End(V) $ ist also genau dann (schief-)symmetrisch, wenn $ f $ eine Adjungierte $ f^* \in \End(V)$ besitzt und $ f^* = \pm f $. Dies ist genau dann der Fall, wenn
		\[ (v,w)\mapsto \SSkl{v}{w} := \Skl{v}{f(w)} \]
	eine (schief-)symmetrische Sesquilinearform definiert.

\subsection{Korollar}
\begin{Korollar}[]
	Sei $ (V,\Skl{.}{.}) $ Euklidischer VR; eine Projektion $ p\in \End(V) $ ist genau dann Orthogonalprojektion, wenn sie selbstadjungiert ist, $ p^* = p $.
\end{Korollar}
\paragraph{Beweis}
	Lemma \ref{sadj}
\section{Normale Endomorphismen}
\paragraph{Motivation}
	Für orthogonale/unitäre selbst- und schiefadjungierte Endomorphismen $ f\in \End(V) $ gilt stets
		\[ f^*\circ f = f\circ f^*. \]
	Dies ist eine "`gute"' Eigenschaft: sie liefert viele/wichtige strukturelle Aussagen über Endomorphismen.
\paragraph{Generalvoraussetzung}
	In diesem Abschnitt ist $ (V,\Skl{.}{.}) $ Euklidisch oder unitär.

\subsection{Definition}\index{normal (Endomorphismen)}
\begin{Definition}[normal]
	$ f\in \End(V) $ heißt \emph{normal}, wenn $ f $ eine Adjungierte $ f^*\in \End(V) $ besitzt und 
		\[ f^*\circ f = f\circ f^*. \]
\end{Definition}
\subsection{Buchhaltung}
	Ist $ \dim V < \infty $ und $ X=\xi_B^B(f) $ Darstellungsmatrix von $ f\in \End(V) $ bzgl. einer ONB $ B $ von $ (V,\Skl{.}{.}) $, so gilt
		\[ f \text{ normal}\Leftrightarrow X^*X=XX^*, \]
	d.h. wenn $ X\in \K^{n\times n} $ \emph{normal} ist.
	
\subsection{Lemma}
\begin{Lemma}[]
	Ist $ f\in \End(V) $ normal, so gilt:
		\begin{enumerate}[(i)]
			\item $ \ker f = \ker f^* = f(V)^\perp $;
			\item $ \forall v,w\in V: \Skl{f^*(v)}{f^*(w)}=\Skl{f(v)}{f(w)} $;
			\item $ \forall x\in \K\forall v\in V: f(v) = vx \Rightarrow f^*(v) = v\overline{x} $.
			\item Sind $ v,w\in V $ Eigenvektoren zu EW $ x,y\in \K $ von $ f $, so gilt
				\[ x = y \text{ oder } v\perp w. \]
		\end{enumerate}
\end{Lemma}
\paragraph{Beweis}
	Sei $ f\in \End(V) $ normal.
		\begin{enumerate}
			\item[(ii)] Wegen $ f^{**} = f $ gilt für $ v,w\in V $:
				\begin{align*}
					&\Skl{f^*(v)}{f^*(w)}-\Skl{f(v)}{f(w)} \\
					= &\Skl{(f^{**}\circ f^*)(v)}{w}-\Skl{(f^*\circ f)(v)}{w}\\
					= &\Skl{(f\circ f^*-f^*\circ f)(v)}{w} = 0
				\end{align*}
			\item[(i)] Wegen (ii) gilt für $ v\in V $
				\[ f^*(v) = 0 \Rightarrow 0 = \|f^*(v)\|^2 = \|f(v)\|^2\Rightarrow f(v) = 0 \]
				und umgekehrt, und damit
				\[ \ker f^* = \ker f. \]
				Nach Lemma \ref{fruh} ist
					\[ \ker f^* = f(V)^\perp \]
			\item[(iii)] Nach (i) ist für $ x\in \K $
				\[ \ker(f-\id_Vx) = \ker(f-\id_Vx)^* = \ker(f^*-\id_V\overline{x}), \]
				da $ f-\id_Vx $ mit $ f $ normal ist.
			\item[(iv)] Mit (iii) folgt für $ v,w\in V $ mit $ f(v) = vx $ und $ f(w) = wy $	\[ (x-y)\Skl{v}{w}=\Skl{v\overline{x}}{w}-\Skl{v}{wy} = \Skl{f^*(v)}{w}-\Skl{v}{f(w)} = 0. \] 
		\end{enumerate}

% VO 09-06-2016 %
\subsection{Lemma}
\begin{Lemma}[]
	Ist $ f\in \End(V) $ normal und $ U\subset V $ UVR, so gilt
		\begin{enumerate}[(i)]
			\item Ist $ U\ f $-invariant, so ist $ U^\perp\ f^* $-invariant;
			\item Ist $ U\ f $- und $ f^* $-invariant, so liefert Einschränkung normale Endomorphismen
				\[ f\big|_U \in \End(U) \text{ und } f\big|_{U^\perp}\in \End(U^\perp). \]
		\end{enumerate}
\end{Lemma}
\paragraph{Beweis}
	Für (i) wird nur die Existenz der Adjungierten benutzt.
		\begin{enumerate}[(i)]
			\item Sei $ v\in U^\perp $, dann gilt:
				\[ \forall u\in U: \Skl{f^*(v)}{u} = \Skl{v}{f(u)} = 0 \]
			\item Da $ U\ f $- und $ f^* $-invariant ist, ist (nach(i)) $ U^\perp\ f^*$- und $ f^{**} = f $-invariant.		
			Damit ist es sinnvoll
				\[ f\big|_U \in \End(U), f\big|_{U^\perp}\in \End(U^\perp) \]
				\[ f^*\big|_U\in \End(U), f^*\big|_{U^\perp}\in \End(U^\perp) \]
			zu betrachten. Nun gilt:
				\[ \forall u,v\in U: \Skl{f\big|_U^*(u)}{v} = \Skl{u}{f\big|_U(v)} = \Skl{u}{f(v)} = \Skl{f^*(u)}{v} = \Skl{f^*\big|_U(u)}{v} \]
			und analog für $ v,w\in U^\perp $. Damit folgt: $ f\big|_U^* = f^*\big|_U $ und $ f\big|^*_{U^\perp} = f^*\big|_{U^\perp}$ und also
				\[ f\big|_U^* \circ f\big|_U = f^*\circ f\big|_U = f\circ f^*\big|_U = f\big|_U \circ f\big|_U^*. \]
		\end{enumerate}
\paragraph{Bemerkung \& Beispiel}
	 Eine Orthogonalprojektion ist selbstadjungiert, $ p\in \End(V) $ mit $ p^2=p, p^*=p $ und damit normal. Ist $ p\neq \id_V, 0 $, so ist
		 \[ V=U\oplus_\perp U^\perp \text{ mit }
			 \begin{cases}
			 U:= p(V),\\ U^\perp = \ker p.
			 \end{cases} \]
	Wir definieren:
		\[ \pi:V\to U, v\mapsto \pi(v):= p(v) \text{ und } \iota : U\to V, u\mapsto \iota(u):= u; \]
	man nennt die isometrische Abbildung $ \iota $ auch die \emph{Inklusion} von $ U $ in $ V $. Dann sind $ \pi $ und $ \iota $ adjungiert:
		\[ \forall u\in U\forall v\in V: \Skl{\iota(u)}{v} = \Skl{u}{\pi(v)}\big|_U \]
	Insbesondere ist die "`Projektionsabbildung"' $ \pi $ \emph{nicht} selbstadjungiert.
\paragraph{Achtung:}
	Die Adjungierte hängt von Definitions- und Wertebereich ab!
	
\subsection{Spektralsatz (unitärer Fall)}\index{Spektral!-satz}
\begin{Satz}[Spektralsatz]
	Sei $ (V,\Skl{.}{.}) $ unitär, $ \dim V<\infty $, und sei $ f\in \End(V) $ normal. Dann besitzt $ V $ eine ONB aus Eigenvektoren von $ f $.
\end{Satz}
\paragraph{Bemerkung}
	Es gilt auch die Umkehrung: Ist $ (e_1,\dots,e_n) $ ONB mit
		\[ f(e_i) = e_ix_i, \text{ also } f^*(e_i) = e_i\overline{x_i},\ i=1,\dots,n, \]
	so gilt
		\[ (f^*\circ f)(e_i) = e_i\overline{x_i}x_i = e_ix_i\overline{x_i} = (f\circ f^*)(e_i),\ i=1,\dots,n, \]
	d.h. $ f $ ist normal.
\paragraph{Beweis}
	Induktion über $ n=\dim V $.
	
	Für $ n=1 $ ist die Aussage trivial. Sei die Aussage für $ n\in\N $ wahr. Für $ n+1 $ gilt dann:
	
	$ f $ hat einen Eigenwert $ x\in \C $, da das charakteristische Polynom $ \chi_f(t)\in \C[t] $ nach Fundamentalsatz der Algebra in Linearfaktoren zerfällt.
	
	Sei $ e\in V^\times $ ein zugehöriger Eigenvektor,
		\[ f(e)=ex, \text{ o.B.d.A. } \|e\| = 1. \]
	Wegen $ f^*(e)=e\overline x $ ist $ [e] $ dann $ f $- und $ f^* $- invariant und damit
		\[ V=[e] \oplus_\perp [e]^\perp, \]
	wobei $ f\big|_{[e]}\in \End([e]) $ und $ f\big|_{[e]^\perp}\in \End([e]^\perp) $ normal sind (Lemma).
	
	Da $ \dim [e]^\perp = n$ liefert die Induktions-Annahme eine ONB $ (e_1,\dots,e_n) $ von $ [e]^\perp $ aus Eigenvektoren von $ f\big|_{[e]^\perp} $. Damit ist $ (e,e_1,\dots,e_n) $ eine ONB aus Eigenvektoren von $ f $.

\subsection{Buchhaltung}
	Ein normaler Endomorphismus $ f\in \End(V) $ eines unitären VR $ (V,\Skl{.}{.}) $ mit $ \dim V < \infty $ ist also \emph{orthogonal diagonalisierbar}, d.h. es existiert eine ONB aus Eigenvektoren von $ f $.
	
	Also, bezüglich einer solchen ONB $ B $ ist
		\[ \xi_B^B(f) = \operatorname{diag}(x_1,\dots,x_n). \]
	Da $ \xi_B^B(f^*) = (\xi_B^B(f))^* $ gilt:
		\begin{itemize}
			\item ist $ f $ selbstadjungiert, so sind alle Eigenwerte reell, $ x_i = \overline{x_i} $;
			\item ist $ f $ schiefadjungiert, so sind alle Eigenwerte imaginär, $ x_i = -\overline{x_i} $
			\item ist $ f $ unitär, so sind alle Eigenwerte \emph{unitär}, d.h. für $ j=1,\dots,n $ ist
				\[ x_j\in S^1 := \{x\in \C:\overline{x}x = 1\} = \{e^{iy}\mid y\in \R\}. \]
		\end{itemize}

\subsection{Korollar \& Definition}	
\begin{Korollar}[]
	Ist $ X\in \C^{n\times n} $ normal, $ X^*X = XX^* $, so gilt
		\[ \exists P\in U(n): P^{-1}XP=\operatorname{diag}(x_1,\dots,x_n). \]
	Dabei gilt:
		\begin{itemize}
			\item ist $ X $ \emph{selbstadjungiert}, $ X^*=X $, so sind $ x_1,\dots,x_n \in \R $;
			\item ist $ X $ \emph{schiefadjungiert}, $ X^*=-X $, so sind $ x_1,\dots,x_n\in i\R $;
			\item ist $ X $ \emph{unitär}, $ X\in U(n)= \{Y\in \C^{n\times n}\mid Y^*Y=E_n\} $, so gilt $ |x_1|,\dots,|x_n| = 1. $
		\end{itemize}
\end{Korollar}
\paragraph{Beweis}
	Sei $ X\in \C^{n\times n} $, betrachte $ \C^n $ als unitären VR mit Standardbasis $ E $ als ONB.
		\[ \Gamma_E(\Skl{.}{.}) = E_n, \]
	und den assoziierten Endomorphismus $ f_X\in \End(\C^n) $. Orthonormale Basiswechsel $ B=EP $ sind dann durch unitäre Matrizen $ P\in U(n) $ gegeben:
		\[ E_n = \Gamma_B(\Skl{.}{.}) = P^*\Gamma_E(\Skl{.}{.})P = P^*P \Leftrightarrow P\in U(n). \]
	Anwendung des Spektralsatzes liefert also die Behauptung.

\paragraph{Beispiel}
	Für  $ X = \begin{pmatrix}
	\cos s & -\sin s\\ \sin s & \cos s
	\end{pmatrix}\in \C^{2\times 2} $ mit $ s\in \R $ ist $ X^*=X^t = X^{-1} $, d.h. $ X $ ist unitär, also normal, und damit
		\[ \exists P\in Gl(2): P^{-1}\begin{pmatrix}
		\cos s & -\sin s\\ \sin s & \cos s
		\end{pmatrix}P = \begin{pmatrix}
		x_1 & 0 \\ 0 & x_2
		\end{pmatrix} \]
	wobei $ x_i $ die Eigenwerte von $ f_X $ sind, d.h. Nullstellen des charakteristischen Polynoms
		\[ \chi_X(t) = (t-\cos s)^2 + \sin^2 s = t^2-2t\cos s + 1 = (t-e^{is})(t-e^{-is}). \]
	Bemerke: $ |e^{\pm is}| = 1 $, d.h. $ x_{1,2} $ sind unitär. Eigenvektoren bzw. P:
		\[ P = \frac{1}{\sqrt{2}}\begin{pmatrix} i & -i \\ 1 & 1 \end{pmatrix} \]

\subsection{Spektralzerlegung (unitärer Fall)}\index{Spektral!-zerlegung (unitär)}
\begin{Lemma}[]
	Sei $ (V,\Skl{.}{.}) $ unitär, $ \dim V <\infty $, und sei $ f\in \End(V) $ normal; dann zerfällt $ V $ als orthogonale direkte Summe der Eigenräume von $ f $,
		\[ V = \bigoplus_{x\in \chi_f^{-1}(\{0\})} \ker(\id_Vx-f). \]
\end{Lemma}
\paragraph{Beweis}
	Folgt direkt aus dem Spektralsatz.
\paragraph{Bemerkung}
	Mit gewissen Voraussetzungen gilt der Satz auch für $ \dim V = \infty $.
% VO 14-06-2016 %

\subsection{Definition \& Lemma}\index{komplexe Erweiterung}
\begin{Definition}[komplexe Erweiterung]
	Seien $ (V,\Skl{.}{.}) $ Euklidischer VR und $ f\in \End(V) $ normal. Die \emph{komplexe Erweiterung} 
		\[ f_\C:V_\C \to V_\C, (v,w)\mapsto f_\C(v,w) := (f(v),f(w)) \]
\end{Definition}
\begin{Lemma}[]
	von $ f $ ist dann ein normaler Endomorphismus von $ (V_\C,\SSkl{.}{.}_\C) $.
\end{Lemma}
\paragraph{Bemerkung}
	Die komplexe Erweiterung für $ f\in\Hom(V,W) $ definiert man analog.
\paragraph{Bemerkung}
	Auf $ V_\C = V\times V $ ist die (komplexe) Skalarmultiplikation
		\[ (v,w)(x+iy) = (v,w)x+J(v,w)y \]
	wobei $ J(v,w) = (-w,v) $; damit ist $ f_\C $ komplex linear, da 
		\[ f_\C \circ J = J\circ f_\C. \]
\paragraph{Beweis}
	Nach Komplexifizierungslemma ist $ (V_\C,\SSkl{.}{.}) $ unitär, wobei
		\[ \SSkl{(v,w)}{(v',w')} = \left(\Skl{v}{v'}+\Skl{w}{w'}\right)+i\left(\Skl{v}{w'}-\Skl{w}{v'}\right); \]
	offenbar gilt für $ v,w,v',w'\in V_\C $
		\begin{align*}
		\SSkl{\left(f^*(v),f^*(w)\right)}{(v',w')}_\C &= \SSkl{(v,w)}{\left(f(v'),f(w')\right)}_\C\\
		&= \SSkl{(v,w)}{f_\C(v',w')}_\C
		\end{align*}
	also $ (f_\C)^*=(f^*)_\C $ und damit
		\[ f_\C^* \circ f_\C = \left(f^* \circ f\right)_\C = \left(f_\C \circ f^*_\C \right), \]
	d.h. $ f_\C $ ist normal.
\paragraph{Bemerkung}
	Ist $ (v,w)\in V_\C^\times $ Eigenvektor zum Eigenwert $ (x-iy) \in \C $ von $ f_\C $,
		\[ f_\C(v,w) = \left(f(v),f(w)\right) = (v,w)(x-iy) = (vx+wy,-vy+wx) = (v,w)
			\begin{pmatrix}
				x & -y\\ y & x
			\end{pmatrix} \]
	so können zwei Fälle eintreten:
		\begin{enumerate}
			\item $ y=0 $ und $ [\{v,w\}] \subset \ker (\id_Vx-f) $, oder
			\item $ y\neq 0 $ und $ \dim [\{v,w\}] = 2 $ und\footnote{$ [\{v,w\}] $ ist $ f $-invarianter UVR, also $ f|_{[\{v,w\}]} \in \End([\{v,w\}])$} $ f|_{[\{v,w\}]} $ ist \emph{Drehstreckung}.
		\end{enumerate}
	Im zweiten Fall $ (y\neq 0) $ ist $ (v,-w) $ ebenfalls Eigenvektor zum Eigenwert $ (x+iy) $ von $ f_\C $, d.h. komplexe Eigenwerte/-vektoren treten in "`komplex konjugierten Paaren"' auf.
	
\subsection{Spektralzerlegung (Euklidischer Fall)}\index{Spektral!-zerlegung (Euklidisch)}
\begin{Satz}[Spektralzerlegung (Euklidischer Fall)]
	Seien $ (V,\Skl{.}{.}) $ Euklidischer VR mit $ \dim V <\infty $ und $ f\in \End(V) $ normal; dann zerfällt $ V $ als orthogonale direkte Summe $ f $- und $ f^* $-invarianter UVR $ V_i $
		\[ V = \bigoplus_{i=1}^mV_i \text{ mit } V_i\perp V_j \text{ für } i\neq j \]
	und
		\[ \begin{cases}
			\dim V_i = 1& \text{und } f|_{V_i} \text{ Streckung für } i\leq k\leq m\\
			\dim V_i = 2& \text{und } f|_{V_i} \text{Drehstreckung für }k < i\leq m,
		\end{cases} \]
	für geeignetes $ k\in\{0,\dots,m\} $. Beweis nach Lemma.
\end{Satz}
\subsection{Buchhaltung}
	Zu einem normalen $ f\in \End(V) $ eines Euklidischen VR $ (V,\Skl{.}{.}) $ mit $ \dim V < \infty $
	gibt es also eine ONB $ E $ von $ (V,\Skl{.}{.}) $, sodass 
		\[ \xi_E^E(f) = \begin{pmatrix}
		x_1& 0&\cdots & & &\\
		0 & \ddots & \ddots& & 0& \\
		\vdots & \ddots & x_k& & &\\
		& & & X_{n+1} & \ddots & \vdots\\
		&0 & & \ddots & \ddots & 0\\
		& & & \cdots&0 & X_m
		\end{pmatrix}
		\text{ mit } X_i = \begin{pmatrix}
		x_i & -y_i\\ y_i & x_i
		\end{pmatrix} \text{ für } i=k+1,\dots,m \]
	und $ x_1,\dots,x_m,y_{k+1},\dots,y_m\in \R $. Da $ \xi_E^E(f^*)  = (\xi_E^E(f))^* $ gilt
		\begin{itemize}
			\item ist $ f $ selbstadjungiert, so sind alle Eigenwerte reell, $ k=m $;
			\item ist $ f $ schiefadjungiert, so sind alle Eigenwerte imaginär, $ k=0 $ und $ x_1=\dots=x_m=0 $;
			\item ist $ f $ orthogonal, so sind alle Eigenwerte unitär, $ x_i^2+y_i^2=1 $ (insbesondere $ x_1^2=\dots=x_k^2=1 $).
		\end{itemize}
	Entsprechendes gilt für normale Matrizen. Insbesondere erhält man den Satz über die
\subsection{Hauptachsentransformation}\index{Hauptachsentransformation}
\begin{Satz}[Hauptachsentransformation]
	Ist $ (V,\Skl{.}{.}) $ Euklidischer VR mit $ \dim V<\infty $ und $ f\in \End(V) $ selbstadjungiert, so ist $ f $ orthogonal diagonalisierbar.
\end{Satz}
\paragraph{Bemerkung}
	Die Hauptachsentransformation kann zur Bestimmung der Signatur einer symmetrischen Bilinearform $ \sigma:V\times V\to \R $ auf einem $ \R $-VR $ V $ mit $ \dim V <\infty $ dienen:
		\begin{itemize}
			\item Wähle (beliebig) ein Euklidisches (Referenz-) Skalarprodukt $ \Skl{.}{.}:V\times V\to \R $;
			\item Definiere $ b\in \End(V) $ (Rieszsches Darstellungslemma) durch
				\[ \forall v,w\in V: \sigma(v,w) = \Skl{v}{b(w)}; \]
			da $ \sigma $ symmetrisch ist, ist $ b $ selbstadjungiert.
			\item Bestimme Eigenwerte $ x_i\in \R $ von $ b $ mit Vielfachheiten\footnote{Geometrische und algebraische Vielfachheiten sind gleich, da $ b $ diagonalisierbar ist.} $ k_i\in \N $ (Hauptachsentransformation).
			\item Dann ist
				\[ \sgn(\sigma) = \left(\sum_{x_i>0}k_i, \sum_{x_i<0}k_i,\dfkt b\right). \]
		\end{itemize}
	Ist $ E $ ONB aus Eigenvektoren von $ b $, so gilt\footnote{Gleichheit der Einträge; sonst sinnlos!}
		\[ \Gamma_E(\sigma) = \xi_E^E(b). \]
\subsection{Quadratwurzelsatz}\index{Quadratwurzel (Endomorphismus)}
\begin{Satz}[Quadratwurzelsatz]
	Ist $ (V,\Skl{.}{.}) $ Euklidischer VR und $ f\in \End(V) $ selbstadjungiert, so heißt $f$
		\begin{enumerate}[(i)]
			\item \emph{positiv semi-definit} $ (f\geq 0) $, falls
				\[ \forall v\in V :\Skl{v}{f(v)}\geq 0;\]
			\item \emph{positiv definit} $ (f>0) $, falls
				\[ \forall v\in V^\times: \Skl{v}{f(v)}>0. \]
		\end{enumerate}
	Ist $ \dim V < \infty $ und $ f $ positiv semi-definit, so gilt
		\[ \exists! g\in \End(V): \begin{cases}
		g\geq 0\\f = g\circ g =g^2.
		\end{cases} \]
\end{Satz}
\paragraph{Beweis}
	Mit Hauptachsentransformation: Ist $ E = (e_1,\dots,e_n) $ ONB aus Eigenvektoren von $ f $,
		\[ f(e_i) = e_ix_i \text{ mit } x_i = \Skl{e_i}{f(e_i)}\geq 0 \]
	für $ i=1,\dots,n $, dann liefert
		\[ g\in\End(V) \text{ mit } \forall i\in \{1,\dots,n\}: g(e_i) = e_i\sqrt{x_i} \]
	eindeutig die gesuchte "`Quadratwurzel"' von $ f $.
\subsection{Polarzerlegung}
\begin{Satz}[Polarzerlegung]
	Ist $ (V,\Skl{.}{.}) $ Euklidischer VR mit $ \dim V <\infty $, so gilt
		\[ \forall f\in Gl(V) \exists! h > 0 \exists! k\in O(V) : f = h\circ k. \]
\end{Satz}
\paragraph{Beweis}
	
	\begin{itemize}
		\item \emph{Eindeutigkeit:\quad} Ist $ f=h\circ k $ mit $ k\in O(V), h>0 $, so gilt
		\[ H:= f\circ f^* = h\circ \underbrace{k \circ k^*}_{=\id_V} \circ\, h^* = h^2 \]
		nach Quadratwurzelsatz ist also $ h $, und damit $ k $ eindeutig bestimmt.
		\item \emph{Existenz:\quad} Wegen $ \ker f^* = f(V)^\perp = V^\perp = \{0\} $ gilt für $ H:= f\circ f^* $
			\[ \forall v\in V^\times: \Skl{v}{H(v)} = \Skl{v}{(f\circ f^*)(v)} = \Skl{f^*(v)}{f^*(v)}>0  \]
		also $ H>0 $. Definiere (Quadratwurzelsatz)
			\[ h:= \sqrt{H}>0 \text{ und } k:= h^{-1}\circ f; \]
		dann ist
			\[ \forall v\in V: \Skl{k(v)}{k(v)} = \Skl{(h^{-1}\circ f)(v)}{(h^{-1}\circ f)(v)} = \Skl{(H^{-1}\circ f)(v)}{f(v)} \]
			\[= \Skl{\left((f^*)^{-1}\circ f^{-1}\circ f\right)(v)}{f(v)} = \Skl{v}{\left(f^{-1}\circ f\right)(v)} = \Skl{v}{v}, \]
		also ist $ k\in O(V) $.
	\end{itemize}
\paragraph{Bemerkung}
	Quadratwurzelsatz und Polarzerlegung gelten auch in unitären VR -- "`positiv (semi-)definit"' ist auch im unitären Fall sinnvoll:
		\[ \forall v\in V: \overline{\Skl{v}{f(v)}} = \Skl{f(v)}{v}=\Skl{v}{f(v)} \]
	für selbstadjungierte $ f $, also $ \forall v\in V: \Skl{v}{f(v)}\in \R $.
% VO 16-06-2016 %
\section{Nilpotente Endomorphismen und Jordansche Normalform}
\paragraph{Generalvoraussetzung}
	In diesem Abschnitt werden nur endlichdimensionale Vektorräume behandelt.
\subsection{Lemma}
\begin{Lemma}[]
	Sei $ f\in \End(V) $ und $ p_1(t),p_2(t)\in K[t] $ normiert und teilerfremd.
	Ist 
		\[ p(f) = 0, \text{ wobei }p(t) = p_1(t)p_2(t), \]
	so gilt für $ V_i := \ker p_i(f)\, (i=1,2)$
		\[ V=V_1\oplus V_2 \text{ und }f(V_i)\subset V_i \]
\end{Lemma}
\paragraph{Beweis}
	Für jedes Polynom $ q(t)\in K[t] $ ist $ \ker q(f)\subset V $ ein $ f $-invarianter UVR, da
		\begin{align*}
			\forall v\in \ker q(f): q(f)(f(v)) &= \left(q(f)\circ f\right)(v)\\	
			&=\left(f\circ q(f)\right)(v) = f\big(\underbrace{q(f)(v)}_{=0}\big) = 0 
		\end{align*}
	Wegen $ p(f) = 0 $ gilt
		\[ \{0\} = p(f)(V) = 
		\begin{cases}
			\Big(p_1(f)\circ p_2(f)\Big)(V) \Rightarrow p_2(f)(V)\subset \ker p_1(f)\\
			\Big(p_2(f)\circ p_1(f)\Big)(V) \Rightarrow p_1(f)(V)\subset \ker p_2(f).
		\end{cases} \]
	Da $ p_1(t),p_2(t) $ teilerfremd sind, gilt nach Lemma von B\'ezout (vgl. Abschnitt \ref{Bezout})
		\[ \exists q_1(t),q_2(t)\in K[t]: 1= q_1(t)p_1(t)+q_2(t)p_2(t), \]
	und damit
		\[ V= \Big(p_1(f)\circ q_1(f)+p_2(f)\circ q_2(f) \Big)(V)\subset p_1(f)(V)+p_2(f)(V)\subset V_2+V_1; \]
	andererseits gilt für $ v\in V_1\cap V_2 $
		\begin{align*}
		v&=\Big(q_1(f)\circ p_1(f)+q_2(f)\circ p_2(f) \Big)(v)\\
		 &=q_1(f)\Big(\underbrace{p_1(f)(v)}_0\Big)+q_2(f)\Big(\underbrace{p_2(f)(v)}_0\Big) = 0
		\end{align*}
	Also ist $ V=V_1\oplus V_2 $.
\subsection{Hauptraumzerlegung}\index{Hauptraumzerlegung}
\begin{Satz}[Hauptraumzerlegung]
	Ist das Minimalpolynom $ \mu_f(t)\in K[t] $ eines Endomorphismus $ f\in \End(V) $ Produkt von Linearfaktoren,
		\[ \mu_f(t) = (t-x_1)^{r_1}\cdots (t-x_m)^{r_m},\, x_i\neq x_j \text{ für }i\neq j \]
	so ist $ V $ direkte Summe der \emph{Haupträume} zu den Eigenwerten $ x_i $ von $ f $:
		\[ V=\bigoplus_{i=1}^m V_i \quad\text{mit}\ V_i := \ker (\id_Vx_i-f)^{r_i}. \]
\end{Satz}
\paragraph{Beweis}
	Folgt direkt mit dem Lemma (Induktion).

\paragraph{Bemerkung}
	Der Wert $ k_i = \dim V_i $ ist die algebraische Vielfachheit des Eigenwerts $ x_i $. Formuliert man die Hauptraumzerlegung mit dem charakteristischen Polynom,
		\[ \chi_f(t) = (t-x_1)^{k_1}\cdots (t-x_m)^{k_m},\, x_i \neq x_j \text{ für } i\neq j, \]
	so folgt dies leicht, da wegen $ f(V_i)\subset V_i $ und $ x_i\neq x_j $ für $ i\neq j $
		\[ \chi_{f|_{V_i}}(t) = (t-x_i)^{k_i}\Rightarrow \dim V_i = \deg \chi_{f|_{V_i}} = k_i. \]
\subsection{Buchhaltung}\index{Mikrostruktur}\index{Makrostruktur}
	Ist also
		\[ \mu_f(t) = (t-x_1)^{r_1}\cdots (t-x_m)^{r_m},\, x_i\neq x_j \text{ für }i\neq j, \]
	so hat $ f $ eine Darstellungsmatrix in \emph{Block-Diagonalgestalt},
		\[ \xi_B^B(f) = \begin{pmatrix}
		X_1 & & 0 \\
		& \ddots & \\
		0 & & X_m
		\end{pmatrix} \text{ mit }X_i \in K^{k_i\times k_i}, \]
	wobei $ k_i\geq r_i $ die algebraischen Vielfachheiten der Eigenwerte $ x_i $ sind.
	
	Dies liefert die "`Makrostruktur"' eines Endomorphismus mit zerfallendem Minimal- oder charakteristischem Polynom -- eine weitere Strukturanalyse der $ f|_{V_i} \in \End(V_i) $ liefert dann die "`Mikrostruktur"' (mögliche Form der $ X_i $'s).
\paragraph{Bemerkung}
	Für $ V_i = \ker (\id_Vx_i-f)^{r_i} $ und $ g_i := (f-\id_Vx_i)|_{V_i} \in \End(V_i) $ gilt offenbar $ g_i^{r_i} = 0 $; andererseits ist $ g_i^{r_i-1}\neq 0 $, denn sonst wäre
		\[ p(t) = (t-x_1)^{r_1}\cdots (t-x_{i-1})^{r_{i-1}}(t-x_i)^{r_i-1}(t-x_{i+1})^{r_{i+1}}\cdots (t-x_m)^{r_m} \]
	normiertes Annulatorpolynom mit
		\[ \deg p(t) = \deg \mu_f(t)-1 < \deg \mu_f(t). \]
	Also wäre $ \mu_f(t) $ nicht Minimalpolynom.

\subsection{Definition}\index{nilpotent}
\begin{Definition}[nilpotent]
	Eine Abbildung $ f\in \End(V) $ heißt \emph{nilpotent}, falls $ f^r=0 $ für ein $ r\in \N $.
\end{Definition}
\paragraph{Bemerkung}
	Die weitere Strukturanalyse eines Endomorphismus mit in Linearfaktoren zerfallendem Minimalpolynom reduziert sich also auf die nilpotenter Endomorphismen
		\[ g_i = (f-\id_Vx_i)|_{V_i}\in \End(V_i). \]
	Dies liefert dann die "`Mikrostruktur"'.
\paragraph{Zur Erinnerung}
	Ist $ U\subset V $ ein $ f $-invarianter UVR $ f(U)\subset U $, so ist $ f|_U\in \End(U) $; eine $ f $-zyklische Basis von $ U $ ist dann eine Basis der Form (vgl. \ref{fzykl})
		\[ \left(v,f(v),\dots,f^{r-1}(v)\right). \]
	Insbesondere besitzt für jedes $ v\in V^\times $ der von $ f $ erzeugte \emph{$ f $-zyklische Unterraum}
		\[ \mathcal{Z}_v := \left[\left(f^k(v)\right)_{k\in \N}\right] \]
	eine $ f $-zyklische Basis; ist für $ v\in V $ und $ r\in \N $
		\[ f^r(v) = 0 \text{ und }f^{r-1}(v)\neq 0, \]
	so ist $ (v,\dots,f^{r-1}(v)) $ eine $ f $-zyklische Basis von $ \mathcal{Z}_v $.
\subsection{Lemma}
\begin{Lemma}[]
	Seien $ f\in \End(V) $ nilpotent, $ f^r = 0 $, und $ v\in V $ so, dass $ f^{r-1}(v)\neq 0 $. Damit existiert ein UVR $ U\subset V $ mit
		\[ f(U)\subset U \text{ und } V = \mathcal{Z}_v \oplus U. \]
	Die Einschränkung $ f|_U \in \End(U) $ ist dann nilpotent,
		\[ {f|_U}^q = 0 \text{ mit } q\leq r. \]
\end{Lemma}
\paragraph{Beweis}
	Sei $U\subset V$ ein UVR mit
		\[f(U)\subset U \text{\quad und \quad} \{0\}=\mathcal{Z}_v\cap U\]
	Es gibt solche Unterräume, e.g. $U=\{0\}$. Zu zeigen: Es gibt solch einen Unterraum $U$ mit
		\[V=\mathcal{Z}_v+U\]
	Strategie: Wir zeigen, dass $U$ vergrößert werden kann, wenn $V\neq \mathcal{Z}_v+U$, sei das also der Fall.
	Da $f^r(V)=\{0\}\subset \mathcal{Z}_v+U$ existiert $s\in\{1,\dots,r\}$ mit
		\[f^s(V)\subset \mathcal{Z}_v+U \text{\quad und \quad} W:=f^{s-1}(V)\not\subset \mathcal{Z}_v+U. \]
	Wegen $\mathcal{Z}_v\cap U  = \{0\}$ hat $f(w)$ für $w\in W$ eine eindeutige Zerlegung
		\[f(w)=\sum_{k=0}^{r-1}f^k(v)x_k+u \in \mathcal{Z}_v \oplus U\]
	wobei
		\[0=f^r(w)=f^{r-1}(v)x_0+f^{r-1}(u) \quad\Rightarrow x_0=0\]
	und damit
		\[f(u')=u\in U \quad\text{für}\quad u':=w-\sum_{k=1}^{r-1} f^{k-1}(v)x_k. \]
	Folglich ist $U':=[u']+U$ ein $f$-invarianter Unterraum, $f(U')\subset U \subset U'$.
	Weiters ist $u'-w\in \mathcal{Z}_v$.
	Wählt man also $w\in W\setminus(\mathcal{Z}_v+U)$, so erhält man $u'\notin \mathcal{Z}_v+U$ und damit
		\[U'\neq U \quad\text{und}\quad \mathcal{Z}_v\cap U'=\{0\}.\]
	Da $f^r=0$ gilt dies offenbar auch für jede Einschränkung von $f$.

\subsection{Struktursatz für nilpotente Endomorphismen}
\begin{Satz}[Struktursatz für nilpotente Endomorphismen]
	Ist $ f\in \End(V) $ nilpotent, so ist $ V $ direkte Summe $ f $-zyklischer UVR $ \mathcal{Z}_{v_j} $,
		\[ V = \bigoplus_{j=1}^d \mathcal{Z}_{v_j} =
		\bigoplus_{j=1}^d \left[\left(f^k(v_j)\right)_{k\in \N}\right].  \]
	Die Familie der Dimensionen $ (r_1,\dots,r_d) $ der Dimensionen $ r_j = \dim \mathcal{Z}_{v_j} $ ist bis auf Permutationen eindeutig\footnote{Beispielsweise folgt die Eindeutigkeit, wenn man aufsteigende Dimensionen fordert.}.
\end{Satz}
\paragraph{Bemerkung}
	Die Zerlegung in $ f $-zyklische UVR ist \emph{nicht} eindeutig!

	Da $ \ker f\cap \mathcal{Z}_{v_j} = \left[f^{r_j-1}(v_j) \right] $ für $ j=1\dots,d $, ist $ d = \dfkt f $.
\paragraph{Buchhaltung}
	Ist also $ f $ nilpotent, so hat $ f $ eine Darstellungsmatrix in Block-Diagonalgestalt,
		\[ \xi_B^B(f) = \begin{pmatrix}
		J_1 & & 0 \\
		& \ddots & \\
		0 & & J_d
		\end{pmatrix} \text{ mit } J_j = \begin{pmatrix}
		0 & & & 0\\1 & \ddots & &\\ & \ddots &\ddots &\\ 0 & & 1 & 0
		\end{pmatrix} \]
\paragraph{Beweis}
	Die Existenz der Zerlegung folgt induktiv aus dem Lemma.
	
	Zur Eindeutigkeit der Dimensionsfamilie: Es bezeichne
		\[ n_k := \#\{r_j = k\mid j = 1,\dots,d \} \]
	die Anzahl der $ f $-zyklischen UR $ \mathcal{Z}_{v_j} $ mit $ \dim \mathcal{Z}_{v_j} = k $ für $ k=1,\dots,n=\dim V $. Dann gilt
		\begin{align*}
		\rg f^0 &= \sum_{k=1}^{n}kn_k\\
		\rg f^1 &= \sum_{k=2}^{n}(k-1)n_k\\
		&\vdots\\
		\rg f^s &= \sum_{k=s+1}^{n}(k-s)n_k
		\end{align*}
	für $ s=0,\dots,n-1 $. Also erfüllen die $ n_k $'s ein eindeutig lösbares lineares Gleichungssystem 
		\[
		\begin{pmatrix}
			1      & 2      & \cdots & n      \\
			0      & 1      & \cdots & n-1    \\
			\vdots & \ddots & \ddots & \vdots \\
			0      & \cdots & 0      & 1
		\end{pmatrix} 
		\begin{pmatrix}
		n_1\\
		\vdots\\
		\vdots\\
		n_n
		\end{pmatrix} = \begin{pmatrix}
		\rg f^0\\
		\vdots\\
		\vdots\\
		\rg f^{n-1}
		\end{pmatrix} \]
\subsection{Jordansche Normalform}\index{Jordansche Normalform}
\begin{Satz}[Jordansche Normalform]
	Ist das Minimalpolynom $ \mu_f(t)\in K[t] $ eines Endomorphismus $ f\in \End(V) $ Produkt von Linearfaktoren,
		\[ \mu_f(t) = (t-x_1)^{r_1}\cdots(t-x_m)^{r_m},\, x_i\neq x_j \text{ für }i\neq j \]
	so besitzt $ f $ eine Darstellungsmatrix in \emph{Jordanscher Normalform}, d.h.
		\begin{itemize}\index{Mikrostruktur}\index{Makrostruktur}
			\item \emph{Makrostruktur}: $ \xi_B^B = \operatorname{diag}(X_1,\dots,X_m) $, wobei $ X_i \in K^{k_i\times k_i} $ mit $ k_i\geq r_i $ die algebraischen Vielfachheiten der Eigenwerte $ x_i $ sind, und
			\item \emph{Mikrostruktur:} jedes $ X_i = \operatorname{diag}(J_{i1}(x_i),\dots,J_{id}(x_i)) $, mit \emph{Jordanblöcken}
				\[ J_{ij}(x) = \begin{pmatrix}
				x &  &  &  \\ 
				1 & x &  &  \\ 
				& \ddots & \ddots &  \\ 
				&  & 1 & x
				\end{pmatrix}.  \]
		\end{itemize}
	Dabei ist $ \xi_B^B(f) $ eindeutig, bis auf Anordnung der Blöcke.
\end{Satz}
\paragraph{Bemerkung}
	Die Basis $ B $ ist \emph{nicht} eindeutig!
\paragraph{Beweis}
	Folgt direkt aus den vorigen beiden Sätzen.

% VO 21-06-2016 %
\section{Quadriken}
\paragraph{Generalvoraussetzung}
	In diesem Abschnitt ist $ (A,V,\tau) $ ein \emph{reeller} affiner Raum über einem $ \R $-VR $ V $; $ \Skl{.}{.} $ ist ein Euklidisches Skalarprodukt.
\subsection{Definition}\index{Quadrik}
\begin{Definition}[Quadrik]
	Eine \emph{Quadrik} $ Q\subset A $ ist die Lösungsmenge einer quadratischen Gleichung
		\[ Q = \{q=o+v\mid \beta(v,v)+2\lambda(v)+\rho = 0 \} \]
	wobei $ o\in A $ ein Ursprung ist und
	\begin{itemize}
		\item $ \beta: V\times V\to \R $ eine symmetrische Bilinearform, $ \beta \neq 0 $;
		\item $ \lambda: V\to \R $ eine Linearform; und
		\item $ \rho \in \R$ sind.
	\end{itemize}
	Diese Definition hängt nicht vom Ursprung $ o\in A $ ab:
\end{Definition}
\subsection{Lemma}
\begin{Lemma}[]
	Ist $ Q=\{o+v\in A \mid \beta(v,v)+2\lambda(v)+\rho=0 \} $ eine Quadrik und $ o' = o+w\in A $ ein anderer Ursprung, so ist
		\[ Q=\{o'+v\in A\mid \beta'(v,v)+2\lambda'(v)+\rho'=0 \} \]
	mit $ \beta'=\beta $, $ \lambda'=\lambda+\beta(w,.) $, $ \rho' = \rho+2\lambda(w)+\beta(w,w) $.
\end{Lemma}
\paragraph{Beweis}
	Mit $ q=o'+v=o+(w,v) $ nachrechnen, vgl. Aufgabe 91.
\paragraph{Bemerkung}
	Insbesondere ist $ \beta' = \beta $ unabhängig vom gewählten Ursprung, der lineare Term ändert sich mit $ \beta(w,.) $ -- unter "`guten Umständen"' kann man also $ \lambda $ durch geeignete Wahl von $ o' $ verschwinden lassen (quadratische Ergänzung).
\paragraph{Beispiel}
	Für $ \lambda\in V^*\setminus\{0\} $ liefert
		\[ \beta:V\times V\to \R,\ (v,w)\mapsto \beta(v,w):=\lambda(v)\lambda(w) \]
	eine symmetrische Bilinearform $ \beta\neq 0 $ und daher
		\[ Q=\{o+v\in A\mid \lambda^2(v)=\beta(v,v) = 0 \} \]
	eine Quadrik; andererseits ist $ Q = o+\ker \lambda $ eine Hyperebene (AUR mit $ \dim = \dim A-1 $).
\subsection{Bemerkung \& Definition}
\begin{Definition}[echte Quadriken]
	Im Folgenden betrachten wir nur \emph{echte Quadriken}, d.h. Quadriken $ Q\subset A $, die nicht in einer affinen Hyperebene enthalten sind.
	Insbesondere schließen wir $ Q=\emptyset $ aus. 
\end{Definition}

	Nach Wahl des Ursprungs $ o\in A $ bestimmt eine echte Quadrik die zugehörige Gleichung bis auf Vielfache: Das Tupel $ (\beta,\lambda,\rho) $ ist bis auf (gemeinsame) Skalarmultiplikation mit $ x\in\R^\times  $ eindeutig bestimmt.
\subsection{Definition}\index{Mittelpunkt}\index{Spitze}
\begin{Definition}[Mittelpunkt, Spitze]
	Ein Punkt $ z\in A $ heißt \emph{Mittelpunkt} einer Quadrik $ Q\subset A $, falls
		\[ \forall q=z+v\in A: q\in Q\Rightarrow z-v\in Q, \]
	ein Mittelpunkt $ z $ einer Quadrik $ Q $ heißt \emph{Spitze}, falls $ z\in Q $.
\end{Definition}	
\begin{Definition}[Mittelpunktsquadrik, Kegel, Paraboloid]

	Eine Quadrik $ Q $ heißt
	\begin{itemize}
		\item \emph{Mittelpunktsquadrik}, falls sie einen Mittelpunkt $ z\in A $ hat;
		\item \emph{Kegel}, falls sie eine Spitze hat; 
		\item \emph{Paraboloid} (oder Parabel für $ \dim A = 2 $), falls sie keinen Mittelpunkt hat.
	\end{itemize}
\end{Definition}	
\paragraph{Bemerkung \& Beispiel}
	Eine Quadrik $ Q $ kann mehr als einen Mittelpunkt oder eine Spitze haben.
	Beispielsweise liefert für $ \lambda\in V^*\setminus\{0\} $
		\[ Q = \{q=o+v\in A\mid \beta(v,v) = \lambda^2(v) = 1\} \]
	ein Paar paralleler Hyperebenen, eine Quadrik, für die jeder Punkt $ z = o+w \in o+\ker \lambda $ ein Mittelpunkt ist, da für $ o+v = (o+w)+(v-w) = z+(v-w)\in Q $ gilt
		\[ \lambda\left((z-(v-w))-o\right) = \lambda(2w-v) = -\lambda(v)\implies o+v\in Q\Rightarrow z-(v-w)\in Q. \]
\subsection{Lemma}
\begin{Lemma}[]
	Seien $ Q\subset A $ eine echte Quadrik und $ z\in A $. Dann ist
	\begin{itemize}
		\item $ z\in A $ Mittelpunkt von $ Q $, falls
			\[ \exists c\in \R: Q=\{q\in A\mid \beta(q-z,q-z)=c \}; \]
		\item $ z\in A $ Spitze von $ Q $, falls
			\[ Q=\{q\in A\mid \beta(q-z,q-z) = 0\}. \]
	\end{itemize}
\end{Lemma}
\paragraph{Beweis}
	Da eine Spitze ein Mittelpunkt auf $ Q $ ist, folgt die zweite Aussage direkt aus der ersten.
	Sei $ z\in A $ Mittelpunkt von $ Q $; mit $ z $ als Ursprung und geeigneten $ (\beta,\lambda,\rho) $ ist dann
		\[ Q=\{q=z+v\mid \beta(v,v)+2\lambda(v)+\rho = 0\}. \]
	Da $ z $ Mittelpunkt von $ Q $ ist, gilt
		\[ \forall q=z+v\in Q: \begin{cases}
		0 = \beta(v,v)+2\lambda(v)+\rho\\
		0 = \beta(v,v)-2\lambda(v)+\rho
		\end{cases} \]
	mithin
		\[ \forall q=z+v\in Q: \lambda(v) = 0,\quad \text{ also }\quad Q\subset z +\ker \lambda. \]
	Da $ Q $ echte Quadrik ist, folgt also $ \ker \lambda = V $ bzw. $ \lambda = 0 $. Die Behauptung folgt dann mit der Wahl von $ c=-\rho $. Umgekehrt: Ist für ein $ c\in \R $
		\[ Q=\{q=z+v\in A\mid \beta(v,v)=c \}, \]
	so ist $ z $ offenbar Mittelpunkt von $ Q $.
\subsection{Bemerkung \& Definition}\index{Erzeugende}
	Ist $ Q\subset A $ ein Kegel mit Spitze $ z\in Q $, so ist für $ q\in Q\setminus\{z\} $ und $ v:= q-z $
		\[ \forall x\in \R: \beta(vx,vx)= \beta(v,v)x^2 = 0,  \]
	also ist mit $ q $ auch die gesamte Gerade $ [\{z,q\}] = \{z+vx\mid x\in \R\}\subset Q $. Diese in $ Q $ enthaltenen Geraden heißen auch \emph{Erzeugende} des Kegels.
\subsection{Affine Klassifikation der Mittelpunktsquadriken}
\begin{Lemma}[]
	Ist $ Q\subset A $ echte Mittelpunktsquadrik eines affinen Raumes $ A $, so existieren
	\begin{itemize}
		\item affines Bezugssystem $ (o,e_1,\dots,e_n) $ von $ A $ und
		\item $ c\in \{0,1\} $ und $ p,r\in \N $ mit $ 1\leq p\leq r \leq n $, 
	\end{itemize}
	sodass 
		\[ Q=\left\{q=o+\sum_{i=1}^{n}e_ix_i\mid \sum_{i=1}^{p}x_i^2-\sum_{i=p+1}^{r}x_i^2 = c \right\} \]
	und $ p $ ist der Positivitätsindex von $ \beta $, $ r-p $ der Negativitätsindex ($ n-r $ Radikaldimension). 
\end{Lemma}
\paragraph{Beweis}
	Folgt direkt aus dem Satz von Sylvester.
\subsection{Bemerkung \& Definition}
	Zwei echte Mittelpunktsquadriken $ Q,Q' $ sind also genau dann \emph{affin äquivalent}, d.h. $ Q' = \alpha(Q) $ für eine Affinität $ \alpha:A\to A $, wenn $ \sgn(\beta')=\sgn(\beta) $, bzw. $ \sgn(\beta')=\sgn(\pm \beta) $ im Fall eines Kegels.
% VO 23-06-2016 %
\subsection{Euklidische Klassifikation der Mittelpunktsquadriken}
\begin{Lemma}[]
	Ist $ Q\subset A $ eine echte Mittelpunktsquadrik eines Euklidischen Raumes, $ \dim A <\infty $, so existierten 
	\begin{itemize}
		\item ein kartesisches Bezugssystem $ (o;e_1,\dots,e_n) $ von $ A $,
		\item $ c\in \{0,1\} $ und $ p,r\in \N $ mit $ 1\leq p\leq r\leq n $ und
		\item $ a_i\in (0,\infty) $ für $ i=1,\dots,r $
	\end{itemize}
	sodass
		\[ Q = \left\{o+\sum_{i=1}^{n}e_ix_i\in A \mid \sum_{i=1}^{p}\left(\frac{x_i}{a_i}\right)^2-\sum_{i=p+1}^{r}\left(\frac{x_i}{a_i}\right)^2 = c \right\}. \]
\end{Lemma}
\paragraph{Beweis}
	Folgt mit der Hauptachsentransformation (daher ihr Name).
	Sei $ b\in \End(V) $ so, dass
		\[ \forall v,w\in V: \beta(v,w) = \Skl{v}{b(w)}, \]
	nach Riesz existiert ein eindeutiges solches $ b $.
	
	Da $ \beta $ symmetrisch ist, gilt $ b^* = b $. Da $ Q $ Mittelpunktsquadrik ist, kann sie mithilfe des Mittelpunkts $ z\in A $ geschrieben werden als
		\[ Q=\left\{z+v\in A\mid \Skl{v}{b(v)} = c \right\} \]
	mit o.B.d.A. (Multiplikation der Gleichung mit $ c^{-1} $ im Fall $ c\neq 0 $) $ c\in \{0,1\} $. Nach Hauptachsentransformation existiert eine ONB $ (e_1,\dots,e_n) $ aus Eigenvektoren von $ b $, wobei o.B.d.A die Eigenwerte zu $ e_1,\dots,e_p $ positiv, $ e_{p+1},\dots,e_r $ negativ und zu $ e_{r+1},\dots,e_n $ gleich 0 sind, für $ 0\leq p\leq r\leq n $.
	
	Also existieren $ a_1,\dots,a_r\in (0,\infty) $, sodass
		\[ b(e_i)=\begin{cases}
			e_i\frac{1}{a_i^2} &\text{für } i = 1,\dots,p\\
			-e_i\frac{1}{a_i^2} &\text{für } i = p+1,\dots,r\\
			0 & \text{für } i = r+1,\dots,n.
		\end{cases} \]
	Damit gilt mit dem kartesischen Bezugssystem $ (z;e_1,\dots,e_n) $
		\begin{align*}
		Q=\left\{z+\sum_{i=1}^{n}e_ix_i\in A\mid c\right.&=\left\langle{\sum_{i=1}^{n}e_ix_i},{b\left(\sum_{j=1}^{n}e_jx_j\right)}\right\rangle \\
		 &= \left.\sum_{i,j=1}^{n}x_i\Skl{e_i}{b(e_j)}x_j = \sum_{i=1}^{p}\left(\frac{x_i}{a_i}\right)^2-\sum_{i=p+1}^{r}\left(\frac{x_i}{a_i}\right)^2 \right\}.
		\end{align*}
	Da $ \#Q \leq 1 $ für $ p=0 $, muss $ p\geq 1 $ sein, denn $ Q $ war als echt vorausgesetzt.
\paragraph{Bemerkung}
	Diese beiden Sätze liefern "`Klassifikationen"' in den jeweiligen Geometrien, d.h. eine Einteilung der Menge der Quadriken in Äquivalenzklassen, wobei zwei Quadriken $ Q,Q'\subset A $ äquivalent sind, wenn $ Q $ durch eine Transformation der jeweiligen Geometrie auf $ Q' $ abgebildet werden kann, d.h. $ \alpha $ ist affine Transformation oder Kongruenzabbildung:
		\[ Q\sim Q' :\Leftrightarrow \exists \alpha:A\to A\: Q' = \alpha(Q). \]
	Also: Haben (nach Klassifikationssätzen) zwei Quadriken $ Q,Q'$ die gleiche Gleichung (bzgl. affiner/kartesischer Bezugssysteme $ (z;e_1,\dots,e_n) $ bzw. $ (z',e_1',\dots,e_n') $), so existiert eine affine Transformation $ \alpha:A\to A $, definiert durch
		\[  \alpha(z) = z' \text{ und } \alpha(z+e_i) = z+e_i' \text{ für } i=1,\dots,n \]
	mit $ \alpha(Q) = Q' $. Die Umkehrung folgt ähnlich (Vorsicht bei Kegeln!).
\subsection{Beispiel \& Definition}\index{Ellipse}\index{Hyperbel}\index{Asymptotenkegel}
	In einer Euklidischen Ebene $ E^2 $ ergeben sich als echte Mittelpunktsquadriken:
		\begin{itemize}
			\item $ p=r=2 : c=1 $ ($ c=0 $ liefert nur einen Punkt) und
				\[ Q=\left\{z+e_1x_1+e_2x_2\in E^2\mid \left(\frac{x_1}{a_1}\right)^2+\left(\frac{x_2}{a_2}\right)^2 = 1\right\} \]
				liefert eine \emph{Ellipse} mit \emph{Halbachsenlängen} $ a_1,a_2 > 0$.
			\item $ p=1 $ und $ r=2 $
				\begin{itemize}
					\item mit $ c=1 $ liefert
						\[ Q=\left\{z+e_1x_1+e_2x_2\in E^2\mid \left(\frac{x_1}{a_1}\right)^2-\left(\frac{x_2}{a_2}\right)^2 = 1\right\} \]
						eine \emph{Hyperbel} durch die Scheitel $ z=\pm e_1a_1 $ und sich in $ z $ schneidenden Asymptoten.
    		%------------------ Hyperbel.tikz ----------------
	        \begin{figure}[h]\centering
     		    \include{Chap6/Hyperbel.tikz}
    	    \end{figure}\noindent
            %------------------ Hyperbel.tikz ----------------
   
					\item mit $ c=0 $ liefert
						\[ Q=\left\{z+e_1x_1+e_2x_2\in E^2\mid \left(\frac{x_1}{a_1}\right)^2=\left(\frac{x_2}{a_2}\right)^2 \right\} \]
						den \emph{Asymptotenkegel} der obigen Hyperbel.
				\end{itemize}
			\item $ p=r=1 : c=1 $ ($ c=0 $ liefert eine Gerade, also keine echte Mittelpunktsquadrik) liefert
				\[ Q=\left\{z+e_1x_1+e_2x_2\in E^2\mid \left(\frac{x_1}{a_1}\right)^2 = 1\right\} \]
			zwei parallele Geraden im Abstand $ 2a_1 > 0$.
		\end{itemize}
	

\small
\printindex
\end{document}
