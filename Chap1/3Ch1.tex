%VO06-2015-10-22
\section{Basis und Dimensionen}

\subsection{Definition (Basis)}
	\begin{Definition}[Basis]
		Eine Teilmenge $S\subset V$ oder eine Familie $(v_i)_{i\in I}$ in $ V $ heißt
	\begin{itemize}
		\item \emph{Erzeugendensystem} von $ V $, falls \[[S] = V \text{ bzw. } [(v_i)_{i\in I}] = V\]
		\item \emph{linear unabhängig}, falls \[\forall v\in S: v \notin [S\setminus\{{v\}}] \text{ bzw. } \forall i\in I: v_i \notin [(v_j)_{j\in I\setminus\{{i\}}}]\] und sonst \emph{linear abhängig}.
	\end{itemize}
        Eine \emph{Basis} ist ein linear unabhängiges Erzeugendensystem.
	\end{Definition}

\paragraph{Bemerkung}
	Man kann jede (Teil-)Menge $S\subset V$ als Familie in $V$ auffassen mit
		\[v: S \to V,\ v\mapsto id_S(v) = v.\]
	Andererseits gilt für eine Familie $(v_i)_{i\in I} $:
		\[(v_i)_{i\in I} \text{ linear unabhängig } \Rightarrow \{v_i \mid i\in I\} \text{ linear unabhängig.}\]
	Die Umkehrung gilt im Allgemeinen nicht: Eine Familie (in $ V $) enthält mehr Information als eine Teilmenge von $ V $.
	
\subsection{Beispiel und Definition (Standardbasis)}
	\begin{Definition}[Standardbasis]
		Für $V = K^n$ ist $(e_1, ... , e_n)$,
	\begin{equation*}
		e_i:\{{1, ... ,n\}} =: I\to K,\ j\mapsto e_i(j)= \delta_{ij}=
		\begin{cases}
			1,& \text{falls } i=j\\
			0,& \text{sonst}
		\end{cases}
	\end{equation*}
	für $i=1,\dots,n$, eine Basis -- die \emph{Standardbasis} des (Standard-)Vektorraumes $K^n$.
	\end{Definition}

\paragraph{Beweis}
	z.z.: $ (e_i)_{i\in I} $ ist ein linear unabhängiges Erzeugendensystem. Wir wissen bereits $ [(e_i)_{i\in I}] = K^n $. Andererseits gilt für jedes $i\in I$ und jede Familie $(x_j\mid j\in I)$ in $ K $
	\begin{gather*}
		\Big(\sum_{j\in I\setminus\{i\}}e_jx_j\Big)(i) = \sum_{j\in I\setminus\{i\}}e_j(i)x_j = 0 \neq 1 = e_i(i) \Rightarrow \sum_{j\in I\setminus\{i\}} e_jx_j \neq e_i
	\end{gather*}
	also gilt
	\begin{equation*}
		\forall i\in I: e_i \notin [(e_j)_{j\in I\setminus\{i\}}] = \Big\{\sum_{j=I\setminus\{i\}} e_jx_j\mid (x_j)_{ j\in I}\Big\} \text{ mit } \#\{j\in I\mid x_j \neq 0\}<\infty
	\end{equation*}
	
\subsection{Lemma}
	\begin{Lemma}
		Eine Familie $(v_i)_{i\in I}$ ist linear unabhängig gdw. für jede Linearkombination
		\[0 = \sum_{i\in I} v_ix_i \Rightarrow \forall i\in I: x_i = 0.\]
	\end{Lemma}

\paragraph{Beweis}
	Wir zeigen zwei Richtungen der Äquivalenz der Negationen: 
		\[(v_i)_{i\in I} \text{ linear abhängig } \Leftrightarrow \exists(x_i)_{i\in I} \neq (0)_{i\in I}: \sum_{i\in I} v_ix_i = 0.\]
        
	"$\Leftarrow$":
	Wir nehmen an, es gäbe eine \emph{nicht-triviale}\footnote{d.h. $(x_i)_{i\in I}\neq 0$} Linearkombination der Null,
		\[0 = \sum_{i\in I} v_ix_i, \text{ wobei } \exists j\in I: x_j \neq 0.\]
	Für $(y_i)_{i\in I}, y_i := - \frac{x_i}{x_j}$ ist dann
	\begin{gather*}
		0 = v_jx_j + \sum_{i\in I\setminus\{j\}} v_ix_i \\
		\Rightarrow v_j = -\Big(\sum_{i\in I\setminus\{j\}}v_ix_i\Big)x_j^{-1} = \sum_{i\in I\setminus\{j\}} v_iy_i \in [(v_i)_{i\in I\setminus\{j\}}]
	\end{gather*}
	insbesondere ist also $(v_i)_{i\in I}$ linear abhängig.
	
	"$\Rightarrow$": siehe Aufgabe.
	
\subsection{Korollar}
	\begin{Korollar}
		Ist $(v_i)_{i\in I}$ Basis von $ V $, so ist jeder Vektor $v\in V$ eindeutig in den $v_i$ darstellbar:
		\[\forall v\in V \exists! (x_i)_{i\in I}: v = \sum_{i\in I} v_ix_i\]
	\end{Korollar}

\paragraph{Beweis}
	Sei $v\in V$ beliebig, dann gilt:
		\[V = [(v_i)_{i\in I}] \Rightarrow \exists (x_i)_{i\in I}: v = \sum_{i\in I} v_ix_i\]
	liefern $(x_i)_{i\in I}$ und $(y_i)_{i\in I}$
	\begin{align*}
		v = \sum_{i\in I} v_ix_i = \sum_{i\in I}v_iy_i \Rightarrow 0 &= \sum_{i\in I} v_i(x_i-y_i)\\
                &\Rightarrow \forall i\in I: x_i = y_i \Rightarrow (x_i)_{i\in I} = (y_i)_{i\in I}
	\end{align*}
	Damit ist die Basisdarstellung $v = \sum_{i\in I} v_ix_i$ von $v$ auch eindeutig.

%VO07-2015-10-27
\subsection{Basislemma}
    \begin{Lemma}[Basislemma]
    	Sei $S\subset V$ lin. unabh. und $E\subset V$ ein Erzeugendensystem mit $S\subset E$. Dann existiert eine Basis $B$ von $V$ mit $S\subset B\subset E$.
    \end{Lemma}

\paragraph{Beweis}
    Wir gehen für den Beweis davon aus, dass $\#E<\infty$. Betrachte alle Teilmengen $X\subset V$ mit $S\subset X\subset E$ und $X$ lin. unabh. Sei $B$ eine solche Menge, die maximal ist, d.h.
        \[\forall X\subset E: ((B\subset X\land X\text{ lin. unabh.}) \Rightarrow X= B)\]
    Nach Konstruktion ist $B=\{b_1,...,b_n\}$ lin. unabh. Zu zeigen: $V=[B]$.\\
    Ist $B=E$, so folgt $[B]=[E]=V$.\\
    Ist $B\neq E$, so ist $B\cup \{v\} $ für (jedes) $v\in E\setminus B$ lin. abh., da $B$ maximal lin. unabh. ist; also existiert eine nicht-triviale Linearkombination des Nullvektors.
        \[\exists x,x_1,...,x_n \in K: 0=vx+\sum^n_{i=1}b_ix_i\]
    Wäre $x=0$, so würde folgen $x_1=...=x_n=0$, da $B$ lin. unabh. ist. 
    Also ist $x\neq 0$ und 
    	\[v=-\sum^n_{i=1} b_i\frac{x_i}{x} \in [B].\]
    Da dies für beliebiges $v\in E\setminus B$ gilt, folgt
    	\[E\subset [B] \Rightarrow V=[E]\subset [[B]] = [B],\]
    d.h., $ B $ ist Erzeugendensystem und damit eine Basis mit $S\subset B\subset E$.

\paragraph{Bemerkung}
    Ist $\#E = \infty$, so kann man einen analogen Beweis führen, falls man die Existenz einer maximalen Menge voraussetzt: Dies garantiert das \emph{Zornsche Lemma} bzw. \emph{Auswahlaxiom}.
    Wir werden das Lemma auch im Falle $\#E = \infty$ benutzen!

\paragraph{Beispiel}
    Für $V=K^3=K^I$ mit $I=\{1,2,3\}$ betrachte die Standardbasisvektoren 
    \begin{align*}
        e_i &:I\to K,\ j\mapsto e_i(j) = \delta_{ij}\text{, und}\\
        f_i &: I\to K,\ j\mapsto f_i(j):= 1-\delta_{ij}
    \end{align*}
    dann sind $S:= \{e_1,f_1\}$ und $E:= \{e_i,f_i\mid i\in I\}$ lin. unabh. bzw. Erzeugendensystem von $K^3$. Ergänzung von $S$ durch einen Vektor $e_i$ oder $f_i$ mit $i = 2,3$ liefert eine Basis $B$ mit $S\subset B\subset E$.
    
    Zum Beispiel ist $B=\{e_1,f_1,f_2\}$ eine Basis, da sich jede Funktion $v\in K^3$ aus den Funktionen $e_1,f_1$ und $f_2$ linear kombinieren lässt:
    \begin{gather*}
        v=e_1x_1+f_1y_1 + f_2y_2\Leftrightarrow \left\{
            \begin{array}{l}
                v(2)=y_1\\
                v(3) - v(2) = y_1 + y_2 - y_1 = y_2\\
                v(1) + v(2) - v(3) = x_1 + y_2 - y_2 = x_1
            \end{array}
    	\right.
    \end{gather*}
    Dass $B$ lin. unabh. folgt dann: Wäre $B$ lin. abh., so würde folgen $f_2\in [\{e_1,f_1\}]\Rightarrow [B] \subset [\{e_1,f_1\}] \neq K^3$, was nicht der Fall ist.

\subsection{Basisergänzungssatz}
    \begin{Satz}[Basisergänzungssatz]
    	Jede lin. unabh. Menge $S\subset V$ kann zu einer Basis $B$ von $V$ ergänzt werden: Es existiert eine Basis $B$ von $V$ mit $S\supset B$.
    \end{Satz}

\paragraph{Beweis}
    Sei $E\subset V$ ein Erzeugendensystem von $V$ (z.B. $E=V$). Dann ist $S\cup E$ ein Erzeugendensystem von $V$ mit $S\subset S\cup E$, das Basislemma liefert dann die\footnote{nicht eindeutig!} gesuchte Basis.

\paragraph{Bemerkung}
    Strikt genommen haben wir den Basisergänzungssatz (BES) nur unter der Annahme bewiesen, dass $V$ \emph{endlich erzeugt} sei, d.h. $V$ ein endliches Erz. Syst. $E$ besitzt, $V=[E]$ und $\#E<\infty$.

\paragraph{Bemerkung}
    Wir haben für den BES die (in diesem Falle einfachere) Mengenschreibweise (anstelle der Familienschreibweise) verwendet.

\paragraph{Bemerkung}
    Ähnlich kann man einen Verkürzungssatz beweisen: Jedes Erzeugendensystem eines Vektorraums $V$ kann zu einer Basis verkürzt werden.

\subsection{Austauschlemma}
    \begin{Lemma}[Austauschlemma]
    	Seien $B,B' \subset V$ Basen von $V$. Dann gilt:
        \[\forall b\in B\ \exists b' \in B': (B\setminus\{b\})\cup\{b'\} \text{ ist Basis}\]
    \end{Lemma}
    
\paragraph{Beweis}
    Sei $b\in B$ beliebig gewählt und $S:= B\setminus \{b\}$. Da $B$ lin. unabh. ist, gilt 
        \[b\notin [S] \Rightarrow \emptyset \neq V\setminus [S] = [B']\setminus [S] \Rightarrow B' \not\subset [S]\]
    d.h. es existiert $b' \in B'$ mit $b' \notin [S]$. Wir zeigen, dass $B'' := S\cup \{b'\} = (B\setminus\{b\})\cup \{b'\}$ Basis ist.
    
    \begin{enumerate}
	    \item $B''$ ist Erzeugendensystem:\\
	    Da $b'\in [B]$ existiert $(x_j)_{j\in B}$ mit
		    \[b' = \sum_{j\in B} jx_j \text{ mit } x_b \neq 0, \text{ da } b' \notin [S].\]
	    Damit ist $b=(b'-\sum_{j\in S} jx_j)\frac{1}{x_b} \in [B''] \Rightarrow V = [B] \subset [B'' \cup \{b\}] =  [B'']$.
    
	    \item $B''$ ist linear unabhängig:\\
		    $B''$ ist Erz. Syst. und $S\subset B'' = S \cup \{b'\}$ lin unabh., kann also (nach Basislemma) ergänzt werden zu einer Basis $\tilde{B}$ mit $S\subset \tilde{B}\subset B''$.
		    Da $[S] \neq V$ gilt $\tilde{B} \neq S$ und damit $\tilde{B} = B''$ Basis, insbesondere linear unabhängig.
    \end{enumerate}
    
\paragraph{Bemerkung}
    Hier haben wir die Familienschreibweise (mit $B$ bzw. $S$ als Indexmenge) verwendet, um Linearkombinationen darzustellen.
    
\subsection{Basissatz}
	\begin{Satz}[Basissatz]
	Sei $V$ ein endlich erzeugter $K$-VR, $V=[E]$ mit $\#E < \infty$. Dann gilt:
	\begin{enumerate}[(i)]
		\item $V$ besitzt eine endliche Basis $B$ mit $n:= \#B \leq \#E$.
		\item Ist $B'\subset V$ eine Basis von $V$, so ist $\#B' = \#B = n$.
	\end{enumerate}
	\end{Satz}
    
\paragraph{Beweis}
    \begin{enumerate}[(i)]
        \item  Dies folgt direkt aus dem Basislemma (mit $S=\emptyset$).
        \item Seien $B,B'$ Basen von V, $B = (b_1,...,b_n)$.\\
        Annahme: $\#B' < n,\ B' = (b'_1,...,b'_k)$ mit $k < n$. Wiederholte Anwendung des Austauschlemmas auf die Basen $B$ und $B'$ liefert nach (spätestens) $k+1\leq n$ Schritten einen Widerspruch zur linearen Unabhängigkeit der neuen Basis $B''$, da Vektoren $b'_i$ doppelt vorkommen müssen.\\
        Annahme: $\#B' > n,\ B' = (b'_1,...,b'_n,b'_{n+1})$: Das gleiche Argument mit vertauschten Rollen der Basen führt wieder zum Widerspruch.
     \end{enumerate}

\subsection{Definition (Dimension)}
    \begin{Definition}[Dimension]
    	Sei $V$ ein $K$-VR, die \emph{Dimension} von $ V $ ist dann:
        \begin{itemize}
            \item $\dim V:= \#B$, falls $ V $ endlich erzeugt und $B$ eine Basis von $V$ ist;
            \item $\dim V:= \infty$, falls $V$ nicht endlich erzeugt ist.
        \end{itemize}
    \end{Definition}
    
\paragraph{Bemerkung}
    Nach dem Basissatz hängt $\dim V = \#B$ (falls $V$ endlich erz.) nicht von der Basis $B$ ab, d.h. $\dim V$ ist wohldefiniert.
    
\paragraph{Beispiel}
    $\dim K^n = \#\{e_1,...,e_n\} = n$ (Standardbasis).

%VO08-2015-10-29
\subsection{Korollar (Dimension und Teilmengen)}
	\begin{Korollar}[Dimension und Teilmengen]
		Sei $ V $ ein $ K $-VR mit $\dim V =: n\in \mathbb{N}$. Dann gilt:
    \begin{enumerate}[(i)]
    	\item Ist $S \subset V$ linear unabhängig, so ist $\# S \leq n$ und $\# S = n$ gdw. $ S $ Basis ist.
    	\item Ist $E \subset V$ Erzeugendensystem, so ist $\#E \geq n$, bzw. $\#E = n$ gdw. $ E $ eine Basis ist.
    \end{enumerate}
	\end{Korollar}
    
\paragraph{Bemerkung}
	Insbesondere: Ist $U\subset V$ UVR mit $\dim U=\dim V < \infty$, so gilt $ U=V $.
   
\paragraph{Beweis}
    \begin{enumerate}[(i)]
        \item Ist $ S $ linear unabhängig, so existiert (nach BES) eine Basis $ B $ von $ V $ mit 
            \begin{gather*}
                S\subset B\Leftrightarrow \left\{
                \begin{array}{l}
                    \#S \leq \#B\\
                    \#S = \#B \Leftrightarrow S = B
                \end{array}\right.
            \end{gather*}
        \item Analog (mit Basislemma), siehe Aufgabe 23.
        \end{enumerate} 
