%VO04-2015-10-15
\section{Translationen und Vektoren}

\begin{tikzpicture}[scale=1.5,>=triangle 45]
	\draw[->,color=black] (-0.1,0) -- (10,0);
	\draw[->,color=black] (0,-0.1) -- (0.,4);
	
	\coordinate[label=left:$x$] (x) at (1,2);
	\coordinate[label=below:$y\equal\tau_v(x)$] (y) at (5,1.5);
	\coordinate[label=above:$y'\equal\tau_w(x)$] (y') at (2,3.5);
	\coordinate (z) at (6,3);
	
	\draw [fill] (x) circle (.5pt);
	\draw [fill] (y') circle (.5pt);
	\draw [fill] (y) circle (.5pt);
	\draw [fill] (z) circle (.5pt);
	
	\draw [->] (x) to node[below left]{$ v $} (y);
	\draw [->] (x) --node[above left]{$ w $} (y');
	\draw [->] (y) --node[below right]{$ w $} (z);
	\draw [->] (y') --node[above right]{$ v $} (z);
	\draw [->] (x) --node[above]{$ v+w $} node[below]{$ w+v $} (z);
	
	\draw (z) node[above right] {$z = \tau_w(y)=(\tau_w\circ\tau_v)(x)=\tau_{w+v}(x)$};
	\draw (z) node[below right] {$z' = \tau_v(y')=(\tau_v\circ\tau_w)(x) = \tau_{v+w}(x)$};
	\draw (4,0.5) node[] {Translationen \glqq der\grqq{} Ebene bilden eine abelsche Gruppe};
\end{tikzpicture}

\begin{tikzpicture}[scale=1.5, >=triangle 45]
	\draw[->,color=black] (-0.1,0) -- (9,0);
	\draw[->,color=black] (0,-0.1) -- (0,4);
	
	\coordinate[label=left:$ x $] (x) at (1,3);
	\coordinate[label=right:$ y \equal \tau_v(x) \equal (\tau_{\frac{v}{2}} \circ \tau_{\frac{v}{2}})(x) \equal \tau_{\frac{v}{2} + \frac{v}{2}}(x) $] (y) at (3,1);
	
	\draw [fill] (x) circle (.5pt);
	\draw [fill] (y) circle (.5pt);
	
	\draw [->] (1,2.8) --node[left]{$v$} (3,.8);
	\draw [->] (1,3.2) --node[above] {$\frac{v}{2}$} (2,2.2);
	\draw [->] (2,2.2) --node[above] {$\frac{v}{2}$} (3,1.2);
	
	\draw (5,3) node[text width = 7cm] {Translationen kann man \glqq strecken\grqq{}, sodass gewisse Rechengesetze gelten.};
\end{tikzpicture}

\subsection{Definition (Vektorraum)}
	\begin{Definition}[Vektorraum]
		Sei $K$ ein Körper. Eine Menge $V$ mit zwei Abbildungen
	\begin{align*}
		 +&: V \times V \to V:(v,u)\mapsto v+w,\\
		 \cdot &: K \times V \to V:(x,v)\mapsto vx,
	\end{align*}
	
	heißt \emph{Vektorraum über $K$} ($K$-VR), falls gilt:
	\begin{enumerate}[(i)]
		\item $(V,+)$ ist eine abelsche Gruppe
		\item $\forall v\in V: v\cdot 1=v$ und\\
                      $\forall x,y \in K\ \forall v\in V: (v\cdot x)\cdot y = v\cdot (x\cdot y)$
		\item $\forall x,y \in K\ \forall v\in V: v(x+y) = vx + vy$\\
                      $\forall x\in K\ \forall v,w\in V: (v+w)x = vx + wx$
	\end{enumerate}
	\end{Definition}

\paragraph{Bemerkung}
	Wir notieren die Skalarmultiplikation als Rechtsmultiplikation (Skalar steht rechts):
		\[ \cdot: K \times V \to V : (x,v) \mapsto vx \]

\paragraph{Beispiel}
	Die Translationen eines affinen Raumes bilden einen Vektorraum (vgl. mit der Skizze oben): Dieses Beispiel wird im nächsten Kapitel repräsentiert.
	
\paragraph{Beispiel}
	Jeder Körper $ K $ ist ein $ K $-VR (Vektorraum über sich selbst): das ist ein (trivialer) Spezialfall des folgenden:
	
\subsection{Beispiel und Definition (Standardvektorraum)}
	\begin{Definition}[Standardvektorraum]
		Ist $ I $ eine Menge und $ K $ ein Körper, so bilden die $ K $-wertigen Abbildungen
		\[ v: I \to K: i \mapsto v_i \]
	einen Vektorraum mit der punktweise definierten Addition und Skalarmultiplikation:
	\begin{align*}
		I\ni i \mapsto (v+w)_i &:= v_i+w_i\in K\\
		I\ni i \mapsto (vx)_i &:= v_ix \in K
	\end{align*}

	Dieser Vektorraum wird mit $K^{I}$ bezeichnet und \emph{Standardvektorraum} (über I und K) genannt. Im Falle $ I=\{1,...,n\} $ schreibt man auch $K^{n} := K^{\{1,...,n\}}$
	\end{Definition}

\subsection{Bemerkung und Definition (Familienschreibweise)}
	\begin{Definition}
		Anstelle der normalen Schreibweise
		\[ I\ni i \mapsto v(i) \in K \]
	für die Auswertung einer Abbildung  $v: I \to K$ um einen Punkt $i\in I$ haben wir die \emph{Indexschreibweise} verwendet:	
		\[ I\ni i \mapsto v_i \in K \]
	Wir haben damit eine Abbildung $v: I \to K$ als \emph{Familie} $ (v_i)_{i\in I} $ über der Indexmenge $ I $ aufgefasst -- Familie ist ein \glqq alternativer\grqq{} Begriff für Abbildung.
	\end{Definition}
	
\paragraph{Beispiel}
	Sei $i$ eine \glqq Zahl\grqq{} mit $i^2=-1$ ($i$ entspricht nicht dem Element der Indexmenge aus dem vorherigen Abschnitt). Die Menge der komplexen Zahlen
		\[ \mathbb{C}:=\{{x+iy\mid x,y\in \mathbb{R}}\} \]
	bildet mit der Addition und Multiplikation einen Körper:
	\begin{align*}
		+&:\mathbb{C}\times \mathbb{C} \to \mathbb{C}: ((x+y),(x'+y')) \mapsto (x+iy)+(x'+iy') := (x+x')+i(y+y')\\
		\cdot &:\mathbb{C}\times \mathbb{C} \to \mathbb{C}: ((x+iy),(x'+iy'))\mapsto (x+iy)\cdot (x'+iy') :=(xx'-yy')+i(xy'+x'y)
	\end{align*}

	Der Körper $\mathbb{C}$ bildet einen $\mathbb{R}$-VR mit
		\[ +:\mathbb{C}\times\mathbb{C}\to\mathbb{C} \]
	wie oben und der Skalarmultiplikation
		\[ \cdot:\mathbb{R}\times\mathbb{C}\to\mathbb{C}:(x',(x+iy))\mapsto(x+iy)x':=xx'+iyx' \]
	Diese Skalarmultiplikation ist also gerade die Einschränkung der komplexen Multiplikation auf $\mathbb{R}\times\mathbb{C}$ wobei die Identifikation
		\[ \mathbb{R}\cong \{{x+iy\in\mathbb{C}\mid y=0}\} \]
	verwendet wird.
	
%VO05-2015-10-20
\subsection{Definition (Untervektorraum)}
	\begin{Definition}[Untervektorraum]
		Eine Teilmenge $U\subset V$ eines $K$-VR $V$ heißt \emph{Unter(vektor)raum} (UVR), falls $U$ mit der eingeschränkten Addition und Skalarmultiplikation
	\begin{align*}
		 ^+    & \mid_{U\times U}: U\times U \to V,(v,w) \mapsto v+w \\
		 \cdot & \mid_{K\times U}: K\times U \to V,(x,v) \mapsto vx
	\end{align*}

	selbst ein Vektorraum ist, d.h. wenn insbesondere
	\begin{align*}
		&\forall v,w \in U: v+w\in U \text{ und}\\
		&\forall x\in K\ \forall v\in U: vx\in U.
	\end{align*}
	\end{Definition}

\paragraph{Bemerkung}
	Eine nicht-leere Teilmenge $U\subset V, U\neq\emptyset$, ist genau dann ein UVR, wenn die auf U eingeschränkten Operationen wohldefiniert sind, d.h. wenn $ U $ bzgl. $ + $ und $ \cdot $ abgeschlossen ist.

	Dies kann zum \emph{Unterraumkriterium} zusammengefasst werden:
	\begin{equation*}
		U\subset V \text{ ist UVR }\Leftrightarrow 
 		 \begin{cases}
 		 	U\neq\emptyset\\
 		 	\forall v,w\in U\ \forall x\in K: vx+w\in U
 		 \end{cases}
	\end{equation*}

\paragraph{Beispiel}
	Sei $I=\{1,...,n\}$. Für jedes (feste) $i\in I$ ist
		\[ U_i := \{v:I\to K\mid v_i =0\} \]
	ein UVR von $K^n$, denn
	\begin{enumerate}
		\item $v = 0 \in U_i\text{, also } U_i \neq \emptyset$
		\item Seien $v,w\in U_i$, d.h. $v,w\in K^n$ mit $v_i =w_i =0$, und $x\in K$; dann gilt $(vx+w)_i = v_ix+ w_i = 0\cdot x + 0 = 0$, also $vx+w\in U_i$ und damit ist $U_i$ UVR nach Unterraumkriterium.  
	\end{enumerate}
	
	Kein UVR von $K^n, n\geq 2$, ist jedoch die Menge
		\[ N:=\{v:I\to K\mid v_1\cdot v_2 = 0\}, \]
	denn 
	\begin{enumerate}
		\item $N$ ist zwar nicht-leer, $N\neq \emptyset$, aber
		\item $^+\mid_{N\times N}: N\times N\to N$ nicht wohldefiniert: seien $v,w\in N$, so dass
			\begin{gather*}
				v_1=0, v_2=1\text{ }(v_3 ... v_n \text{ irrelevant})\\
				w_1=1, w_2 = 0\text{ }(w_3 ... w_n \text{ irrelevant})
			\end{gather*}
	\end{enumerate}
	dann gilt:
	\begin{gather*}
		(v+w)_1 = v_1 + w_1 = 0+1=1\\
		(v+w)_2 = v_2 + w_2 = 1+0 = 1
	\end{gather*}
	und damit
		\[ (v+w)_1(v+w)_2 = 1 \Rightarrow v+w\notin N. \]

\paragraph{Bemerkung und Beispiel}
	In analoger Weise definiert man die Begriffe
	\begin{itemize}
		\item einer \emph{Untergruppe} $H\subset G$ einer Gruppe $(G,\cdot)$, bzw.
		\item eines \emph{Unter-} oder \emph{Teilkörpers} $T\subset K$ eines Körpers $(K,+,\cdot )$
	\end{itemize}
	
	z.B. bildet jeder UVR $U\subset V$ eines $K$-VR $V$ (mit der Addition) eine Untergruppe der Gruppe $(V,+)$.\\
        In gleicher Weise ist eine nicht-leere Teilmenge eine \emph{Untergruppe} bzw. einen \emph{Unterkörper}, falls die eingeschränkten Operationen wohldefiniert sind.
    
        z.B. ist $H\subset G$ eine Untergruppe, falls (Untergruppenkriterium):
        \begin{enumerate}
            \item $H\neq \emptyset$
            \item $\forall g,h\in H: g\circ h^{-1} \in H$
        \end{enumerate}
            
	Achtung: Inversenbildung muss im Kriterium explizit formuliert werden, sonst würde z.B.: $\mathbb{N}\subset\mathbb{Z}$ als Teilmenge von $(\mathbb{Z}, +)$ als Gruppe ein Gegenbeispiel liefern.
        
        Weitere Beispiele:
        \begin{itemize}
            \item die Translationen bilden eine Untergruppe der Bewegungsgruppe
            \item $\mathbb{Q}\subset\mathbb{R}$ und $\mathbb{R}\cong \{x+iy\mid y=0\}\subset\mathbb{C}$ bilden Teilkörper von $\mathbb{R}$ bzw. $\mathbb{C}$.
        \end{itemize}

\subsection{Lemma (Schnitt von UVR)}
    \begin{Lemma}[Schnitt von UVR]
    	Ist $(U_i)_{i\in I}$ eine Familie von UVR $U_i\subset V$ eines $K$-VR $V$, so ist ihr Schnitt
        \[ U:= \bigcap_{i\in I}U_i =\{ u\in V\mid \forall i\in I: u\in U_i\} \]
	ein UVR von $V$. (Beweis durch UR-Krit. in Aufgabe 17)
    \end{Lemma}
    
\subsection{Definition (Lineare Hülle)}
	\begin{Definition}
		Die \emph{lineare Hülle} $[S]$ einer Teilmenge $S\subset V$ eines $ K $-VR $ V $ ist der Schnitt aller $S$ enthaltenden UVR $U\subset V$:
		\[ [S] := \bigcap_{S\subset U \text{ UVR}} U \]
	Die lineare Hülle einer Familie $(v_i)_{i\in I}$ von Vektoren $v_i\in V$ in einem $ K $-VR $ V $ ist
		\[ [(v_i)_{i\in I}] := [\{v_i\mid i\in I\}] \]
	\end{Definition}

\paragraph{Bemerkung}
    $[S]$ ist ein UVR (nach Lemma), der \glqq kleinste\grqq{} UVR, der $S$ enthält, d.h. ist $U\subset V$ UVR mit $S\subset U$, so gilt $[S]\subset U$; da aber $[S] = \bigcap_{S\subset \tilde{U}  \text{ UVR}}\tilde{U}\subset U$,
    da $S\subset U$, also $U$ am Schnitt beteiligt ist.

\paragraph{Bemerkung}
	$[\emptyset ] = \{0\}$ und $[V] = V$.

\paragraph{Beispiel}
	Ist $U\subset V$ UVR, so gilt $[U] = U$.

\paragraph{Beispiel}
	$N=\{v:I\to K\mid v_1v_2=0\} \subset K^n,I=\{1,...,n\},n\geq 2$, hat die lineare Hülle $[N]=K^n$.

\paragraph{Beispiel}
	Für $I=\{1,...,n\}$ und $i\in I$ definiere
	$e_i:I\to K , j\mapsto e_i(j):= \delta_{ij}$, wobei 
	\begin{equation*}
		\delta_{ij} :=
		\begin{cases}
			1,& \text{falls }i=j\\
			0,& \text{sonst}
		\end{cases}
	\end{equation*}
	das \emph{Kroneckersymbol} bezeichnet.
	
	Damit ist die lineare Hülle der Familie $(e_i)_{i\in I}$
		\[ [(e_i)_{i\in I}] = K^n. \]
	
	Nämlich: Da $[(e_i)_{i\in I}]\subset K^n$ ist, gilt für beliebige $x_1,...,x_n\in K$
	\begin{gather*}
		\underbrace{e_1x_1\underbrace{+...+\underbrace{e_nx_n + 0}_{\in [(e_i)_{i\in I}]}}_{\in [(e_i)_{i\in I}]}}_{\in [(e_i)_{i\in I}]}\in [(e_i)_{i\in I}]
	\end{gather*}
	da $[(e_i)_{i\in I}] \subset K^n$ UVR ist. 
	Andererseits gilt für beliebiges $v\in K^n$:
		\[ v=\sum^n_{i=1}e_iv(i): I\to K, \]	
	denn
		\[ \forall j\in I: \left(\sum^n_{i=1} e_iv(i)\right)(j) = \sum^n_{i=1}e_i(j)v(i) = v(j) \]
	Damit ist gezeigt, dass die beiden Abbildungen übereinstimmen; da $v\in K^n$ beliebig war, folgt $K^n \subset [(e_i)_{i\in I}]$
	
\subsection{Definition (Linearkombination)}
	\begin{Definition}
		Seien $(v_i)_{i\in I}$ und $(x_i)_{i\in I}$ Familien in einem $ K $-VR bzw. dem Körper $ K $, wobei
	\begin{align*}
		\# &\{i\in I\mid x_i \neq 0\} < \infty\text{ , also}\\
		   &\{i\in I \mid x_i \neq 0\} = \{i_1,...,i_n\}
        \end{align*}
        für ein geeignetes  $n\in \mathbb{N}$;
    	Dann heißt die endliche Summe
            \[\sum_{i\in I} v_ix_i:= \sum^n_{j=1}v_{i_j}x_{i_j}\]
        eine \emph{Linearkombination}.
	\end{Definition}
	
\paragraph{Bemerkung}
	Die Bedingung $\#\{i\in I \mid x_i\neq 0\} <\infty$
	garantiert, dass die Summe wohldefiniert ist $\rightarrow$ vgl. Reihen in der Analysis.

%VO06-2015-10-22
\subsection{Lemma (Lineare Hülle und Linearkombinationen)}
	\begin{Lemma}[Lineare Hülle und Linearkombinationen]
		Ist $(v_i)_{i\in I}$, $I \neq \emptyset$, Familie in einem $K$-VR, so gilt: 
		\[ [(v_i)_{i\in I}] = \left\{\sum_{i\in I} v_ix_i\mid x: I\to K: \# \{i\in I \mid x_i \neq 0\}< \infty\right\}, \]
	d.h. die lineare Hülle der Familie ist die Menge aller Linearkombinationen der Familie.
	\end{Lemma}

\paragraph{Beweis}
	Wir zeigen (wie üblich) zwei Inklusionen:	

	"$\supseteq$":
	
	Sei also $(x_i)_{i\in I}$ eine geeignete Familie in $ K $, dann gilt:
		\[ \sum_{i\in I} v_i x_i = \underbrace{v_{i_1} x_{i_1} + ... + \underbrace{(v_{i_n}x_{i_n}+0)}_{\in [(v_i)_{i\in I}]}}_{\mathrlap{ \in [(v_i)_{i\in I}] \text{ nach UR-Krit. (nach n Schritten)}}} \]

	"$\subseteq$":
	
	Setze $U := \{{\sum_{i\in I} v_ix_i\mid x: I\to K \text{ mit } \#\{{i\in I\mid x_i \neq 0\}} < \infty\}}$, offenbar gilt:
		\[ \forall i\in I: v_i\in U \]
	Wir zeigen, dass $U$ ein Untervektorraum ist. Das heißt:
	\begin{align*}
	^+    & \mid_{U\times U}: U\times U \to U \subset V\\
	\cdot & \mid_{K\times U}: K\times U \to U \subset V,
	\end{align*}
	
	also die Addition und Skalarmultiplikation vererben sich auf $ U $.

Zur Skalarmultiplikation:
        \begin{addmargin}[25pt]{0pt}
	Sind $(x_i)_{i\in I}$ mit $\#\{i\in I \mid x_i \neq 0\}<\infty$ eine Familie in $ K $ und $x\in K$, so gilt für ein geeignetes $n\in \mathbb{N}$
		\[ \{i\in I\mid x_i \neq 0\} = \{i_1, ... , i_n\} \]

	und damit
	\begin{equation*}
		\{i\in I\mid x_ix\neq 0\} =
		\begin{cases}
			\{{i_1,...,i_n\}}& \text{falls }x \neq 0\\
			\emptyset& \text{falls }x = 0.
		\end{cases}
	\end{equation*}

	Also folgt
	\begin{align*}
		(\sum_{i\in I}v_i x_i) x &= (\sum_{j=1}^{n} v_{i_j}x_{i_j})x\\
		&= \sum_{j=1}^{n} v_{i_j}(x_{i_j}x) = \sum_{i\in I} v_i(x_ix) \in U_i,
	\end{align*}

	da $\sum_{i\in I} v_i(x_ix)$ Linearkombination (mit der Familie $(x_ix)_{i\in I}$ in K) ist.
	\end{addmargin}

Zur Addition:
        \begin{addmargin}[25pt]{0pt}
            Ähnlich (Vereinigung zweier Mengen, ist endlich), siehe Aufgabe.
	\end{addmargin}
	
\paragraph{Bemerkung}
	Um triviale Diskussionen zu vermeiden, setzt man $\sum_{i\in \emptyset} ...:=0$.
