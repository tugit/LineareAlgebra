% VO 21-06-2016 %
\section{Quadriken}
\paragraph{Generalvoraussetzung}
	In diesem Abschnitt ist $ (A,V,\tau) $ ein \emph{reeller} affiner Raum über einem $ \R $-VR $ V $; $ \Skl{.}{.} $ ist ein Euklidisches Skalarprodukt.
\subsection{Definition}
	Eine \emph{Quadrik} $ Q\subset A $ ist die Lösungsmenge einer quadratischen Gleichung
		\[ Q = \{q=o+v\mid \beta(v,v)+2\lambda(v)+\rho = 0 \} \]
	wobei $ o\in A $ ein Ursprung ist und
	\begin{itemize}
		\item $ \beta: V\times V\to \R $ eine symmetrische Bilinearform, $ \beta \neq 0 $;
		\item $ \lambda: V\to \R $ eine Linearform; und
		\item $ \rho \in \R$ sind.
	\end{itemize}
	Diese Definition hängt nicht vom Ursprung $ o\in A $ ab:
\subsection{Lemma}
	Ist $ Q=\{o+v\in A \mid \beta(v,v)+2\lambda(v)+\rho=0 \} $ eine Quadrik und $ o' = o+w\in A $ ein anderer Ursprung, so ist
		\[ Q=\{o'+v\in A\mid \beta'(v,v)+2\lambda'(v)+\rho'=0 \} \]
	mit $ \beta'=\beta $, $ \lambda'=\lambda+\beta(w,.) $, $ \rho' = \rho+2\lambda(w)+\beta(w,w) $.
\paragraph{Beweis}
	Mit $ q=o'+v=o+(w,v) $ nachrechnen, vgl. Aufgabe 91.
\paragraph{Bemerkung}
	Insbesondere ist $ \beta' = \beta $ unabhängig vom gewählten Ursprung, der lineare Term ändert sich mit $ \beta(w,.) $ -- unter "`guten Umständen"' kann man also $ \lambda $ durch geeignete Wahl von $ o' $ verschwinden lassen (quadratische Ergänzung).
\paragraph{Beispiel}
	Für $ \lambda\in V^*\setminus\{0\} $ liefert
		\[ \beta:V\times V\to \R, (v,w)\mapsto \beta(v,w):=\lambda(v)\lambda(w) \]
	eine symmetrische Bilinearform $ \beta\neq 0 $ und daher
		\[ Q=\{o+v\in A\mid \lambda^2(v)=\beta(v,v) = 0 \} \]
	eine Quadrik; andererseits ist 
		\[ Q = o+\ker \lambda \]
	eine Hyperebene (AUR mit $ \dim = \dim A-1 $).
\subsection{Bemerkung \& Definition}
	Im Folgenden betrachten wir nur \emph{echte Quadriken}, d.h. Quadriken $ Q\subset A $, die nicht in einer affinen Hyperebene enthalten sind.
	Insbesondere schließen wir $ Q=\emptyset $ aus. 
	
	Nach Wahl des Ursprungs $ o\in A $ bestimmt eine echte Quadrik die zugehörige Gleichung bis auf Vielfache: $ (\beta,\lambda,\rho) $ ist bis auf (gemeinsame) Skalarmultiplikation mit $ x\in\R^\times  $ eindeutig bestimmt.
\subsection{Definition}
	Ein Punkt $ z\in A $ heißt \emph{Mittelpunkt} einer Quadrik $ Q\subset A $, falls
		\[ \forall q=z+v\in A: q\in Q\Rightarrow z-v\in Q, \]
	ein Mittelpunkt $ z $ einer Quadrik $ Q $ heißt \emph{Spitze}, falls $ z\in Q $.
	
	Eine Quadrik $ Q $ heißt
	\begin{itemize}
		\item \emph{Mittelpunktsquadrik}, falls sie einen Mittelpunkt $ z\in A $ hat,
		\item \emph{Kegel}, falls sie eine Spitze hat; 
		\item \emph{Paraboloid} (oder Parabel für $ \dim A = 2 $), falls sie keinen Mittelpunkt hat.
	\end{itemize}
\paragraph{Bemerkung \& Beispiel}
	Eine Quadrik $ Q $ kann mehr als einen Mittelpunkt oder eine Spitze haben.
	Beispielsweise liefert für $ \lambda\in V^*\setminus\{0\} $
		\[ Q = \{q=o+v\in A\mid \beta(v,v) = \lambda^2(v) = 1\} \]
	ein Paar paralleler Hyperebenen, eine Quadrik, für die jeder Punkt $ z = o+w \in o+\ker \lambda $ ein Mittelpunkt ist, da für $ o+v = (o+w)+(v-w) = z+(v-w)\in Q $ gilt
		\[ \lambda\left((z-(v-w))-o\right) = \lambda(2w-v) = -\lambda(v)\implies o+v\in Q\Rightarrow z-(v-w)\in Q. \]
\subsection{Lemma}
	Seien $ Q\subset A $ eine echte Quadrik und $ z\in A $. Dann ist
	\begin{itemize}
		\item $ z\in A $ Mittelpunkt von $ Q $, falls
			\[ \exists c\in \R: Q=\{q\in A\mid \beta(q-z,q-z)=c \}; \]
		\item $ z\in A $ Spitze von $ Q $, falls
			\[ Q=\{q\in A\mid \beta(q-z,q-z) = 0\}. \]
	\end{itemize}
\paragraph{Beweis}
	Da eine Spitze ein Mittelpunkt auf $ Q $ ist, folgt die zweite Aussage direkt aus der ersten.
	Sei $ z\in A $ Mittelpunkt von $ Q $; mit $ z $ als Ursprung und geeigneten $ (\beta,\lambda,\rho) $ ist dann
		\[ Q=\{q=z+v\mid \beta(v,v)+2\lambda(v)+\rho = 0\}. \]
	Da $ z $ Mittelpunkt von $ Q $ ist, gilt
		\[ \forall q=z+v\in Q: \begin{cases}
		0 = \beta(v,v)+2\lambda(v)+\rho\\
		0 = \beta(v,v)-2\lambda(v)+\rho
		\end{cases} \]
	mithin
		\[ \forall q=z+v\in Q: \lambda(v) = 0, \]
	also 
		\[ Q\subset z +\ker \lambda. \]
	Da $ Q $ echte Quadrik ist, folgt also $ \ker \lambda = V $ bzw. $ \lambda = 0 $. Die Behauptung folgt dann mit $ c=-\rho $. Umgekehrt: Ist für ein $ c\in \R $
		\[ Q=\{q=z+v\in A\mid \beta(v,v)=c \}, \]
	so ist $ z $ offenbar Mittelpunkt von $ Q $.
\subsection{Bemerkung \& Definition}
	Ist $ Q\subset A $ ein Kegel mit Spitze $ z\in Q $, so ist für $ q\in Q\setminus\{z\} $ und $ v:= q-z $
		\[ \forall x\in \R: \beta(vx,vx)= \beta(v,v)x^2 = 0,  \]
	also ist mit $ q $ auch die gesamte Gerade $ [\{z,q\}] = \{z+vx\mid x\in \R\}\subset Q $. Diese in $ Q $ enthaltenen Geraden heißen auch \emph{Erzeugende} des Kegels.
\subsection{Affine Klassifikation der Mittelpunktsquadriken}
	Ist $ Q\subset A $ echte Mittelpunktsquadrik eines affinen Raumes $ A $, so existieren
	\begin{itemize}
		\item affines Bezugssystem $ (o,e_1,\dots,e_n) $ von $ A $ und
		\item $ c\in \{0,1\} $ und $ p,r\in \N $ mit $ 1\leq p\leq r \leq n $, 
	\end{itemize}
	sodass 
		\[ Q=\{q=o+\sum_{i=1}^{n}e_ix_i\mid \underbrace{\sum_{i=1}^{p}x_i^2}-\sum_{i=p+1}^{r}x_i^2 = c \} \]
	und $ p $ ist der Positivitätsindex von $ \beta $, $ r-p $ der Negativitätsindex ($ n-r $ Radikaldimension). 
\paragraph{Beweis}
	Folgt direkt aus dem Satz von Sylvester.
\subsection{Bemerkung \& Definition}
	Zwei echte Mittelpunktsquadriken $ Q,Q' $ sind also genau dann \emph{affin äquivalent}, d.h. $ Q' = \alpha(Q) $ für eine Affinität $ \alpha:A\to A $, wenn $ \sgn(\beta')=\sgn(\beta) $, bzw. $ \sgn(\beta')=\sgn(\pm \beta) $ im Fall eines Kegels.