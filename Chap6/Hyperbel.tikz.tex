\tdplotsetmaincoords{0}{0} %-27
 	\begin{tikzpicture}[yscale=1,tdplot_main_coords]

 		\def\xstart{0} %x Koordinate der Startposition der Grafik
 		\def\ystart{0} %y Koordinate der Startposition der Grafik
 		\def\myscale{1.0} %ändert die Größe der Grafik (Skalierung der Grafik)
        \def\myscalex{(\myscale)}
        \def\myscaley{(\myscale)}
                
 		\def\xstartdraw{(\xstart + 3.5)} %xKoordinate des Referenzstartpunktes (in dieser Zeichnung: a)
 		\def\ystartdraw{(\ystart + 2.0)}%yKoordinate des Referenzstartpunktes (in dieser Zeichnung: a)

 		\def\balkenhoehe{(3.5)}% Länge des vertikalen blauen Balkens
 		\def\balkenlaenge{(7.5)}% Länge des horizontalen blauen Balkens
 		\def\balkenbreite{0.4} %Balkenbreite

 		%---------Begin Balken----------
 		\def\drehwinkel{0}
 		\node (VekV) at ({\xstart+0.2*cos(\drehwinkel)-\balkenbreite*sin(\drehwinkel)},{\ystart+0.5*sin(\drehwinkel)+\balkenbreite*cos(\drehwinkel)})[right, xshift=1,color=blue] {$\mathbb{R}^2$};
 		\node (AffA) at ({\xstart+(\balkenlaenge-0.5)*cos(\drehwinkel)},{\ystart+(\balkenlaenge-0.5)*sin(\drehwinkel)+\balkenbreite*cos(\drehwinkel)})[color=red] {$A$};

 		\path[ shade, top color=white, bottom color=blue, opacity=.6]
 		({\xstart},{\ystart},0)  -- ({\xstart - \balkenbreite * cos(\drehwinkel)- (-\balkenbreite+0)*sin(\drehwinkel)},{\ystart - \balkenbreite * sin(\drehwinkel)+ (-\balkenbreite+0)*cos(\drehwinkel)},0)  -- ({\xstart - \balkenbreite * cos(\drehwinkel)- (\balkenhoehe+0.5)*sin(\drehwinkel)},{\ystart - \balkenbreite * sin(\drehwinkel)+ (\balkenhoehe+0.5)*cos(\drehwinkel)},0) -- ({\xstart - 0 * cos(\drehwinkel)- (\balkenhoehe+0)*sin(\drehwinkel)},{\ystart - 0 * sin(\drehwinkel)+ (\balkenhoehe+0)*cos(\drehwinkel)},0) -- cycle;

 		\path[ shade, right color=white, left color=blue, opacity=.6]
 		({\xstart},{\ystart},0)  -- ({\xstart - \balkenbreite * cos(\drehwinkel)- (-\balkenbreite+0)*sin(\drehwinkel)},{\ystart - \balkenbreite * sin(\drehwinkel)+ (-\balkenbreite+0)*cos(\drehwinkel)},0) --
 		({\xstart + (\balkenlaenge+0.5) * cos(\drehwinkel)- (-\balkenbreite+0)*sin(\drehwinkel)},{\ystart + (\balkenlaenge+0.5) * sin(\drehwinkel)+ (-\balkenbreite+0)*cos(\drehwinkel)},0) --
 		({\xstart + \balkenlaenge * cos(\drehwinkel)},{\ystart + \balkenlaenge * sin(\drehwinkel)},0)--
 		cycle;
 		%---------End Balken----------
 	
 	\node (pointz1)[color=red] at ({\xstartdraw},{\ystartdraw}) {};
 		
    \def\cradius{2.5}
 	\def\cwinkel{14}
 	
 		
 	\node (pointlo1) at ($(pointz1) + (135+\cwinkel:\cradius)$) {};
 	\node (pointro1) at ($(pointz1) + (45-\cwinkel:\cradius)$) {};
 	\node (pointlu1) at ($(pointz1) + (225-\cwinkel:\cradius)$) {};
 	\node (pointru1) at ($(pointz1) + (315+\cwinkel:\cradius)$) {};
 	
 
 	\draw[name path=lo--ru,-,shorten >=0pt, shorten <=0pt,line width=0.1pt,color=blue!50!white] (pointlo1.center) -- (pointru1.center);
 	\draw[name path=lu--ro,-,shorten >=0pt, shorten <=0pt,line width=0.1pt,color=blue!50!white] (pointro1.center) -- (pointlu1.center);

 	
    \draw [red, thick, domain=-1.1:1.1, samples=50] plot ({\xstartdraw + sqrt( ((\x*\x)*2.3 +1)*1.2 ) },{\ystartdraw +\x});
    \draw [red, thick, domain=-1.1:1.1, samples=50] plot ({\xstartdraw - sqrt( ((\x*\x)*2.3 +1)*1.2 ) },{\ystartdraw +\x});
   
   \draw [green!70!black, thick, domain=-1.3:1.3, samples=50] plot ({\xstartdraw + sqrt( ((\x*\x)*2.3 )*1.2 ) },{\ystartdraw +\x});
    \draw [green!70!black, thick, domain=-1.3:1.3, samples=50] plot ({\xstartdraw - sqrt( ((\x*\x)*2.3 )*1.2 ) },{\ystartdraw +\x});
    
 	
 	%Vektoren blau
 	\draw[name path=pe2,-{>[scale=1,length=6,width=5]},shorten >=0pt, shorten <=0pt,line width=0.2pt,color=blue] (pointz1)  -- ($(pointz1) + (90:1)$);
 	\draw[name path=pe1,-{>[scale=1,length=6,width=5]},shorten >=0pt, shorten <=0pt,line width=0.2pt,color=blue] (pointz1)  -- ($(pointz1) + (0:1)$);
 	 
 	\node (pointScheitel1) at ({\xstartdraw + sqrt( ((1.2 ) },{\ystartdraw }) {};
 	 
 	\path[name path=line1] (pointScheitel1) -- +(0,3);
 	\draw[thick,color=red,line width=0.3pt, name intersections={of=lu--ro and line1,by={Int1}}, dotted] (pointScheitel1) -- (Int1);
 	 
 	\path[name path=line2] (Int1) -- +(-3,0);
 	\node(helppoint) at ($(pointz1.center) + (90:1)$){};
 	\draw[thick,color=red,line width=0.3pt, dotted] (Int1) -- ($(pointz1)!(Int1)!(helppoint)$);

 	%Punkte malen

 	\node [color=blue] (pointle1) at ($(pointz1) + (-15:0.7)$) {\small $e_1$};
 	\node [color=blue] (pointle1) at ($(pointz1) + (105:0.7)$) {\small $e_2$};
 
 	%Punkte weiss
 	%\draw[fill,color=white] (pointz1.center) circle [radius=0.11] node[below, xshift=0, yshift=0]{};
    %Punkte rot
 	\draw[fill,color=green!70!black] (pointz1.center) circle [radius=0.06]node[below, xshift=0, yshift=0]{};
 		
 	
 	\draw[fill,color=red]  (pointScheitel1) circle [radius=0.06]node[below, xshift=0, yshift=0]{};
 	\draw[fill,color=red]  ({\xstartdraw - sqrt( ((1.2 ) },{\ystartdraw }) circle [radius=0.06]node[ xshift=-8mm, yshift=0mm]{\small Scheitel};
 		
 	\node [color=red] (pointla1) at ($(pointz1) + (20:1.0)$) {\small $a_2$};
 	\node [color=red] (pointla2) at ($(pointz1) + (55:1.0)$) {\small $a_1$};
 		
 	\node [color=red] (pointlaHaupt) at ($(pointz1) + (90:1.3)$) {\small Q für $c=1$};
 	\node [color=green!70!black] (pointlaHaupt) at ($(pointz1) + (-90:1.3)$) {\small Q für $c=0$};
 		
 	\node [color=red] (pointlz1) at ($(pointz1) + (-90:0.25)$) {\small $z$};
\end{tikzpicture}