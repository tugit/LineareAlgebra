\documentclass[a4paper, fontsize=11pt, DIV=14, parskip=half]{scrreprt}
\usepackage{microtype}
\usepackage[utf8]{inputenc}
\usepackage[T1]{fontenc}
\usepackage{lmodern}%schoeneres Schriftbild
\usepackage[ngerman]{babel}%deutsche Silbentrennung
\usepackage[onehalfspacing]{setspace}
\usepackage{tikz}%Zeichnungen
\usetikzlibrary{arrows,positioning,arrows.meta}
\usepackage{tikz-3dplot}
\newcommand{\equal}{=}
\usepackage{amsmath,amsfonts,amssymb}%Mathematik-Pakete
\usepackage{graphicx}
\usepackage{paralist}%enumerate mit roemischen Zahlen
\usepackage{float}%fuer H Positionierung
\usepackage[colorlinks]{hyperref} %Verlinktes Inhaltsverzeichnis
\usepackage{mathtools} %für \mathrlap command
\usepackage{makeidx} %Stichwortverzeichnis
\makeindex

\author{Studierendenmitschrift}
\title{Skript Lineare Algebra \& Geometrie 1, Hertrich-Jeromin}

\setcounter{chapter}{-1}%Grundlagen bei 0
\setcounter{tocdepth}{1}

\newenvironment{Satz}[1][]{\index{#1}}{\par\addvspace{\baselineskip}}
\newenvironment{Lemma}[1][]{\index{#1}}{\par\addvspace{\baselineskip}}
\newenvironment{Definition}[1][]{\index{#1}}{\par\addvspace{\baselineskip}}
\newenvironment{Korollar}[1][]{\index{#1}}{\par\addvspace{\baselineskip}}

% Eigene Operatoren:
\let\hom\relax
\DeclareMathOperator{\Char}{Char}
\DeclareMathOperator{\End}{End}
\DeclareMathOperator{\Aut}{Aut}
\DeclareMathOperator{\Iso}{Iso}
\DeclareMathOperator{\hom}{Hom}
\DeclareMathOperator{\rg}{rg}
\DeclareMathOperator{\dfkt}{def}
\DeclareMathOperator{\id}{id}
\DeclareMathOperator{\sgn}{sgn} % Signum
\DeclareMathOperator{\vol}{vol} % Volumen 
\begin{document}
\maketitle
\tableofcontents
% % % %Kapitel 0 - Grundlagen % % % %
\chapter{Grundlagen}

\section*{Vorbereitungen, logische Symbolik}
	Es existieren zwei Methoden zur präzisen Formulierung:
	\begin{itemize}
	\item Funktion einer Formulierung wird präzisiert durch:
		\begin{itemize}
			\item Definition: Begriffsklärung
			\item Satz (Lemma, Proposition, Korollar): Aussage über einen (mathematischen) Sachverhalt
			\item Beweis: eine (logische) Argumentationskette, die erklärt, warum ein Satz/Lemma wahr ist
			\item Bemerkung, Beispiel: zusätzliche Information/Illustration, die oft Eigenarbeit (Beweis) erfordert
		\end{itemize}
	\item Formeln und (logische) Symbole werden verwendet:
		\begin{itemize}
			\item $\forall$ -- All-Quantor: \glqq für alle\grqq
			\item $\exists(!)$ -- Existenz-Quantor: \glqq es existiert (genau) ein\grqq
			\item $\lnot$ -- logische Verneinung: $\lnot A$ ist wahr, wenn $A$ falsch ist
			\item $\land ,\lor$ -- logisches \glqq und\grqq{} und \glqq oder\grqq
			\item $\Rightarrow ,\Leftrightarrow$ -- Implikation und Äquivalenz
		\end{itemize}
	\end{itemize}

	\begin{figure}[H]\centering
		\begin{tabular}{c|c|c|c|c|c|c}
			$A$ & $B$ & $\lnot A$ & $A\land B$ &$A\lor B$&$A \Rightarrow B$ & $A\Leftrightarrow B$\\\hline
			w & w & f & w & w & w & w\\
			w & f & f & f & w & f & f\\
			f & w & w & f & w & w & f\\
			f & f & w & f & f & w & w\\
		\end{tabular}
	\caption{Wahrheitstafel}
	\end{figure}

	Beispiele:
	\begin{itemize}
		\item Implikation: Für $x,y\in\mathbb{R}: xy = 0 \Rightarrow (x = 0\lor y = 0)$
		\item Für Aussagen $ A $ und $ B $ gilt: $(A\Rightarrow B)\Leftrightarrow (\lnot A \lor B)$, Beweis durch Wahrheitstafel
	\end{itemize}
	
	\begin{figure}[H]\centering
		\begin{tabular}{c|c|c|c|c|c}
			$A$ & $B$ & $\lnot A$ & $\lnot A\lor B$ & $A \Rightarrow B$ & $(A\Rightarrow B)\Leftrightarrow (\lnot A \lor B)$\\\hline
			w & w & f & w & w & w \\
			w & f & f & f & f & w \\
			f & w & w & w & w & w \\
			f & f & w & w & w & w \\
		\end{tabular}
	\caption{Beweis durch Wahrheitstafel}
	\end{figure}

\paragraph{Bemerkung (Kommutativität):}
	$\land$, $\lor$, und $\Leftrightarrow$ sind kommutativ (symmetisch), $\Rightarrow$ jedoch nicht, d.h.:
	\begin{gather*}
		(A\land B)\Leftrightarrow (B\land A)\\
		(A\lor B)\Leftrightarrow (B\lor A)\\
		(A\Leftrightarrow B)\Leftrightarrow (B\Leftrightarrow A)\\
		(A\Rightarrow B)\nLeftrightarrow (B\Rightarrow A)\\
	\end{gather*}
	
	weil beispielsweise formal gilt: $x,y\in\mathbb{R}: x = 0 \Rightarrow xy = 0$, aber nicht $xy = 0 \Rightarrow x = 0$.

\paragraph{Bemerkung (Beweisformen der Implikation):}
	Um eine Implikation $A\Rightarrow B$ zu zeigen, bedient man sich häufig auch folgender Äquivalenzen:
	\begin{equation*}
		(A\Rightarrow B)\Leftrightarrow
		\begin{cases}
			\lnot B\Rightarrow \lnot A&\text{(Indirekter Schluss)}\\
			\lnot (A\land \lnot B)&\text{(Widerspruchsbeweis)}
		\end{cases}
	\end{equation*}

\paragraph{Beispiel:}
	Für reelle Zahlen $x,y\in\mathbb{R}$ gilt:
	\begin{equation*}
		\left((xy = 0)\Rightarrow (x=0 \lor y=0)\right) \Leftrightarrow \left((xy=0 \land x \neq 0)\Rightarrow (y =0)\right)
	\end{equation*}
	
	bzw. allgemein:
	\begin{equation*}
		(A\Rightarrow (B\lor C))\Leftrightarrow ((A\land\lnot B)\Rightarrow C)
	\end{equation*}

\paragraph{Bemerkung (Mengenlehre):}
	Die Ähnlichkeit mit der Mengensymbolik ist nicht zufällig, z.B. Mengen $X, Y$:
	\begin{gather*}
		(x\in X\cap Y)\Leftrightarrow (x\in X\land x\in Y)\\
		(x\in X\cup Y)\Leftrightarrow (x\in X\lor x\in Y)\\
		(X\subset Y) \Leftrightarrow \{\forall x : (x\in X \Rightarrow x\in Y)\}
	\end{gather*}

\section*{Abbildungen}
\paragraph{Definition:}
	Eine Zuordnung $f: X\to Y$ zwischen zwei Mengen $X$ und $Y$ heißt eine Abbildung, falls $\forall x\in X: \exists ! y\in Y: y=f(x)$.

	X heißt der Definitionsbereich der Abbildung und $f(X):=\{f(x)\mid x\in X \}\subseteq Y$ das Bild.

	Eine Abbildung $f: X\to Y$ heißt
	\begin{itemize}
		\item injektiv, falls $\forall x,x'\in X:f(x) = f(x') \Rightarrow x=x'$
		\item surjektiv, falls $\forall y\in Y:\exists x\in X: y = f(x)$
		\item bijektiv, falls $\forall y\in Y:\exists !x\in X: y = f(x)$
	\end{itemize}

\paragraph{Beispiel:}
	Mit $X=Y=\mathbb{R}$ definiert
	\begin{itemize}
		\item die Relation $x^2 = y$ eine Abbildung $f:X\to Y, x\mapsto f(x)=x^2$
		\item die Relation $x=y^2$ keine Abbildung $f:X\to Y$, denn
		\begin{itemize}
			\item für ein $x$ gibt es zwei $y$-Werte
			\item $x < 0$ ist nicht definiert
		\end{itemize}
	\end{itemize}

\paragraph{Beispiel:}
	Die Identität $id_X :X\to X, x\mapsto id_X(x):= x$ ist eine bijektive Abbildung.
	
\paragraph{Bemerkung:}
	Eine Abbildung ist genau dann bijektiv, wenn sie injektiv und surjektiv ist.
	
\paragraph{Definition:}
	Sind $ f:X\to Y $ und $ g:Y\to Z$ Abbildungen, so ist ihre Komposition/Verkettung die Abbildung $ g\circ f:X\to Z, x\mapsto (g\circ f)(x):= g(f(x)) $.
	
\paragraph{Beispiel:}
	Seien $ X = Y = Z = \mathbb{R} $ und $ f:X\to Y, x\mapsto f(x) :=x^2 $, $ g:Y\to Z, y\mapsto g(x):=y^3 + y $, so ist die Verkettung $ g\circ f: X\to Z, x\mapsto (g\circ f)(x) = (x^2)^3+x^2 = x^6 + x^2 $.

\section*{Inverse}
\paragraph{Lemma:}
	Seien $ f:X\to Y $ und $ g:Y\to X $ Abbildungen. Dann gilt:
	\begin{enumerate}[i)]
		\item ist $ g $ Linksinverse von $ f $, d.h. $ g\circ f = id_X $, so ist f injektiv
		\item ist $ g $ Rechtsinverse von $ f $, d.h. $ f\circ g = id_Y$, so ist f surjektiv
		\item ist $ g $ Links- und Rechtsinverse von $ f $, so heißt $ g =f^{-1}$ Inverse von $ f $
	\end{enumerate}

\paragraph{Beispiel:}
	$ f:\mathbb{N}\to \mathbb{N}, n\mapsto f(n):= n+1 $ hat Linksinverse
	\begin{equation*}
		g:\mathbb{N} \to \mathbb{N}, n\mapsto g(n):=
		\begin{cases}
			15700, & \text{falls } n=0\\
			n-1, & \text{falls } n\neq 0
		\end{cases}
	\end{equation*}

	Tatsächlich ist $ f $ injektiv, da
	\begin{equation*}
		\forall n,n'\in \mathbb{N} : n+1 = f(n) = f(n') = n'+1 \Rightarrow n=n'
	\end{equation*}
	
	jedoch $ f(\mathbb{N}) = \mathbb{N}\setminus \{0\} $, daher kann keine Rechtsinverse existieren.

\paragraph{Beweis (Lemma):}
	Zwei Aussagen sind zu beweisen:
	\begin{enumerate}[i)]
		\item Sei $ g $ Linksinverse von $ f $. Dann gilt für $ x,x'\in X $ mit \\$ f(x) = f(x'): x = g(f(x)) = g(f(x')) = x' $, also ist $ f $ injektiv.
		\item Sei $g $ Rechtsinverse von $ f $ und $ y\in Y $. Setze $ x:= g(y)\in X $, dann gilt $f(x) = f(g(y)) = y$. Damit existiert zu jedem $ y\in Y $ (mindestens) ein $ x = g(y) $, sodass  $ y=f(x) $.
	\end{enumerate}

% % % %Kapitel 1 - Lineare Räume und Abbildungen % % % %
\chapter{Lineare Räume und Abbildungen}
\section{Von Geometrie zu Algebra}
	Euklid führte in den \glqq Elementen\grqq{} (ca. 300 v. Chr.) das bis heute gültige Schema ein:
	\begin{itemize}
		\item Definition
		\item Axiom/Postulat
		\item Lehrsatz
		\item Beweis
	\end{itemize}

\subsection{Parallelenaxiom/-problem (Euklid, Formulierung nach Playfair)}
	Es existiert genau eine Parallele $ g' $ zum Punkt $ P \notin g $ zur Geraden $ g $.

	Kann das Axiom aus den anderen Axiomen hergeleitet/bewiesen werden? Nein, denn es existieren nichteuklidische, hyperbolische Geometrien (18. Jh.) in denen es mehrere derartige Parallelen gibt. Als Beispiel lässt sich eine Geometrie anführen, die nicht auf einer Ebene sondern auf einem Kreis operiert. Dort lassen sich zu einer Sekante mehrere parallele Sekanten betrachten (also Sekanten, die die ursprüngliche nicht schneiden).

	\begin{figure}[H]
		\begin{minipage}{.45\textwidth}
			\begin{tikzpicture}[line cap=round,line join=round,>=triangle 45,x=1.0cm,y=1.0cm]
				\clip(-1.69,-0.64) rectangle (4.14,2.83);
				\draw [domain=-1.69:4.14] plot(\x,{(-1--1*\x)/1});
				\draw [domain=-1.69:4.14] plot(\x,{(-0--1*\x)/1});
				\draw (0.6,1) node[] {P};
				\draw (1.58,0.16) node[] {g};
				\draw (1.52,1.78) node[] {g'};
				\begin{scriptsize}
				\fill [color=blue] (1,1) circle (2pt);
				\end{scriptsize}
			\end{tikzpicture}
		\end{minipage}
		\begin{minipage}{.45\textwidth}
			\begin{tikzpicture}[line cap=round,line join=round,>=triangle 45,x=1.0cm,y=1.0cm]
				\clip(-2.24,-3.38) rectangle (3.15,1.76);
				\draw(0,0) circle (1cm);
				\draw (-0.94,0.35)-- (0.66,0.75);
				\draw (-0.13,0.74) node[] {g};
				\draw (0.32,-0.56) node[] {P};
				\draw (-1,-0.02)-- (0.88,-0.48);
				\draw (-0.35,-0.94)-- (0.91,0.42);
				\begin{scriptsize}
				\fill [color=blue] (0.22,-0.32) circle (1.5pt);
				\end{scriptsize}
			\end{tikzpicture}
		\end{minipage}
	\end{figure}

\paragraph{Was ist eine Geometrie?}
	Eine Geometrie ist durch eine Menge X und eine auf X operierende Transformationsgruppe gegeben.
%%%%%%%%%%%%%%%%% BEGINN VO3-20151013 %%%%%%%%%%%%%%%%%%%%%

\subsection{Definition (Gruppe)}
	\begin{Definition}[Gruppe]
		Ein Paar $(G,\circ)$ bestehend aus einer Menge $G$ und einer Verknüpfung $(\circ : G\times G \to G) : (g,h) \mapsto g \circ h$ heißt Gruppe, falls:

	\begin{enumerate}[(i)]
		\item $\forall f,g,h\in G : f\circ (g\circ h) = (f\circ g)\circ h$ (Assoziativität)
		\item $\exists e\in G\forall g\in G : e\circ g = g$ (Existenz eines neutralen Elements)
		\item $\forall g \in G \exists g^{-1} \in G : g^{-1}\circ g = e$ (Existenz eines inversen Elements)
	\end{enumerate}
	
	Die Gruppe heißt kommutativ oder abelsch, falls zusätzlich gilt:
	\begin{equation*}
		\forall g,h\in G: g\circ h = h\circ g \text{ (Kommutativität)}
	\end{equation*}
	\end{Definition}

\paragraph{Bemerkung}
	Das ist eine axiomatische Definition, d.h. der Begriff \glqq Gruppe\grqq{} wird durch (aus vielen (!) Beispielen abstrahierten) \glqq Rechenregeln\grqq{} definiert.
\paragraph{Beispiel}
	Die rationalen Zahlen $\mathbb{Q}$ bilden mit der Addition eine Gruppe $(\mathbb{Q} ,+)$.
	Die rationalen Zahlen ohne $0$, $\mathbb{Q}^{\times} := \mathbb{Q}\setminus \{0\}$, bilden mit der Multiplikation eine Gruppe $(\mathbb{Q}^\times ,\cdot)$.

\subsection{Definition (Gruppenoperation)}
	\begin{Definition}[Gruppenoperation]
		Sind $(G,\circ )$ eine Gruppe und $X$ eine Menge, so heißt eine Abbildung
		\[ \cdot : G\times X\to X, (g,x)\mapsto g\cdot x \]
	
	eine Gruppenoperation (von $(G,\circ )$ auf $X$), falls

	\begin{enumerate}[(i)]
		\item $\forall g,h\in G :\forall x\in X: g\cdot (h\cdot x) = (g\circ h)\cdot x$ (entspricht nicht der Assoziativität!)
		\item $\forall x\in X: e\cdot x = x$ für das neutrale Element $e$ der Gruppe $(G,\circ )$
	\end{enumerate}
	$(G,\circ )$ heißt dann Transformationsgruppe von X.
	\end{Definition}

\paragraph{Bemerkung}
	Operiert $G$ (kurz für $(G,\circ )$, aus dem Zusammenhang ersichtlich) auf $X$, so ist für jedes $g\in G$ die Abbildung
		\[ g:X\to X, x\mapsto g\cdot x \]
	eine bijektive Abbildung von $X$ auf sich. Wegen der Axiome (i) und (ii) aus der Definition erhält man $g^{-1}: X\to X$ als Inverse der Abbildung.
	
\subsection{Beispiel und Definition (Permutationsgruppe)}
	\begin{Definition}[Permutationsgruppe]
		Die bijektiven Abbildungen einer Menge $X$ auf sich, 
		\[ G:= \{g:X\to X\mid g \text{ bij}\}, \]
	bilden (mit der Komposition $\circ$) eine (Transformations-)Gruppe $(G,\circ )$ (die auf $X$ operiert): die Permutationsgruppe oder symmetrische Gruppe $S_X$ von $X$. Für $X=\{1,2,...,n\}$ schreibt man auch $S_n$ statt $S_{\{1,...,n\}}$.
	\end{Definition}
\paragraph{Bemerkung}
	Im Gegensatz zu allgemeinen Abbildungen stimmen in (Permutations-)Gruppen Links- und Rechtsinverse stets überein.
\subsection{Lemma (Eindeutigkeit des neutralen Elements)}
	\begin{Lemma}[Eindeutigkeit des neutralen Elements]
		Das neutrale Element einer Gruppe $(G,\circ )$ ist eindeutig und $\forall g\in G: g\circ e = g$. Weiters: 
	\begin{equation*}
		\forall g\in G \exists ! g^{-1} \in G: g^{-1}\circ g = g \circ g^{-1} = e
	\end{equation*}
	\end{Lemma}

\paragraph{Beweis}
	Sei $g\in G$ gegeben und (gemäß Gruppenaxiom (iii)):
	\begin{itemize}
		\item $h:= g^{-1}$ (Linksinverse von $g$)
		\item $k:= h^{-1}$ (Linksinverse von $h$)
	\end{itemize}
	
	Damit berechnen wir (multiplikative Schreibweise: $a\circ b = ab$):
	\begin{gather*}
		hg = e = kh = k((hg)h) = k(h(gh)) = (kh)(gh) = gh\\
	\text{und }	ge = g(hg) = (gh)g = eg
	\end{gather*}
	
	Jedes (links-)neutrale Element $e$ ist also auch rechtsneutral:\hfill
	$\forall g\in G: eg = ge = g$
	
	und ist $e'\in G$ auch neutrales Element, dann:\hfill
	$ e' = ee' = e'e = e $

	Weiters ist jedes (Links-)Inverse auch rechtsinvers:\hfill
	$ \forall g \in G: gg^{-1}=g^{-1}g = e $

	und sind $h,h'\in G$ Inverse von $g\in G$, so gilt:\hfill
	$ h' = h'(gh) = (h'g)h = h $

	d.h. Eindeutigkeit des Inversen.

\subsection{Definition (Körper)}
	\begin{Definition}[Körper]
		Ein Tripel $(K,+,\cdot)$, bestehend aus einer Menge $K$ und zwei Verknüpfungen
	\begin{align*}
		+:&K\times K\to K,(x,y)\mapsto x+y\\
		\cdot : &K\times K\to K, (x,y)\mapsto xy
	\end{align*}
	
	heißt Körper, falls:
	\begin{enumerate}[(i)]
		\item $(K,+)$ ist abelsche Gruppe (mit neutralem Element $0$ und inversem Element $-x$ von $x$)
		\item $(K^\times,\cdot)$ ist abelsche Gruppe (mit neutralem Element $1$ und inversem Element $\frac{1}{x} = x^{-1}$ von $x\in K^\times$)
		\item die Distributivgesetze gelten:
			\[ \forall x,y,z\in K :\begin{cases}x\cdot (y+z) = xy+xz\\ (x+y)\cdot z = xz+yz \end{cases} \]
	\end{enumerate}
	\end{Definition}

\paragraph{Bemerkung}
	In einem Körper gilt stets:
	\begin{gather*}
		0\cdot x = x\cdot 0 = 0
		\end{gather*}
		Denn:
		\begin{gather*}
		0\cdot x = (0+0)\cdot x = 0\cdot x + 0\cdot x\\
		\Rightarrow 0 = 0\cdot x + (-(0\cdot x))=0\cdot x+0\cdot x+(-(0\cdot x)) \Rightarrow 0 = 0\cdot x.
	\end{gather*}
	
	Insbesondere folgt damit: $\forall x,y\in K: xy = yx$ (nicht nur für $K^\times$ (Axiom)).
	
\paragraph{Beispiel}
	Die rationalen Zahlen $\mathbb{Q}$, die reellen Zahlen $\mathbb{R}$ und die komplexen Zahlen $\mathbb{C}$ bilden mit den üblichen Verknüpfungen Körper.

\paragraph{Bemerkung und Beispiel}
	Aufgrund der Axiome (i) und (ii) enthält $ K $ mindestens 2 Elemente, also $ \# K \geq 2 $, nämlich:
	\begin{itemize}
		\item $ 0 $, das neutrale Elemente bezüglich $+$
		\item $1 (\neq 0)$, das neutrale Elemente (in $K^\times$ = $K\setminus\{0\}$) bezüglich $\cdot$
	\end{itemize}
	
	Es gibt auch einen Körper mit genau 2 Elementen $(\{0,1\},+,\cdot)$, wobei
	
	\begin{minipage}{0.45\textwidth}
		\begin{equation*}
			\begin{tabular}{c|cc}
				$+$ & 0 & 1\\\hline
				0 & 0 & 1\\
				1 & 1 & 0\\
			\end{tabular}
		\end{equation*}
	\end{minipage}
	\begin{minipage}{0.45\textwidth}
		\begin{equation*}
			\begin{tabular}{c|cc}
				$\cdot$ & 0 & 1\\\hline
				0 & 0 & 1\\
				1 & 1 & 1\\
			\end{tabular}
		\end{equation*}
	\end{minipage}
	
	Dieser Körper wird auch $\mathbb{Z}_2$ bezeichnet.

\subsection{Bemerkung und Definition (Charakteristik)}
	\begin{Definition}[Charakteristik]
		In $\mathbb{Z}_2: 1 + 1 = 0$. Allgemeiner definiert man die Charakteristik eines Körpers $(K,+,\cdot)$ (mit neutralen Elementen 0 und 1 von + bzw. $\cdot$) durch

	\begin{equation*}
		\Char(K):=
		\begin{cases}
			0,\text{falls } \forall n \in \mathbb{N}^\times: \sum_{j = 1}^{n} 1 \neq 0\\
			\min\{n \in \mathbb{N}^\times\mid \sum_{j = 1}^{n} 1 = 1+ ... + 1 = 0\}
		\end{cases}
	\end{equation*}
	\end{Definition}
	
	z.B. $\Char(\mathbb{Z}_2) = 2$, da
	\begin{gather*}
		\{n\in\mathbb{N}^\times\mid 1+...+1=0\}=\\
		=\{n\in\mathbb{N}^\times\mid n=0 \text{ mod } 2\}=\\
		=\{n\in\mathbb{N}^\times\mid n \text{ gerade}\}\\
		\text{und damit: }	\min\{n\in\mathbb{N}^\times\mid 1+...+1=0\}=2
	\end{gather*}
	
	Wir werden mitunter $\Char(K,+,\cdot)\neq 0$ oder (öfter) $\Char(K,+,\cdot)=2$ ausschließen (müssen).

%VO04-2015-10-15
\section{Translationen und Vektoren}

\begin{tikzpicture}[scale=1.5,>=triangle 45]
	\draw[->,color=black] (-0.1,0) -- (10,0);
	\draw[->,color=black] (0,-0.1) -- (0.,4);
	
	\coordinate[label=left:$x$] (x) at (1,2);
	\coordinate[label=below:$y\equal\tau_v(x)$] (y) at (5,1.5);
	\coordinate[label=above:$y'\equal\tau_w(x)$] (y') at (2,3.5);
	\coordinate (z) at (6,3);
	
	\draw [fill] (x) circle (.5pt);
	\draw [fill] (y') circle (.5pt);
	\draw [fill] (y) circle (.5pt);
	\draw [fill] (z) circle (.5pt);
	
	\draw [->] (x) to node[below left]{$ v $} (y);
	\draw [->] (x) --node[above left]{$ w $} (y');
	\draw [->] (y) --node[below right]{$ w $} (z);
	\draw [->] (y') --node[above right]{$ v $} (z);
	\draw [->] (x) --node[above]{$ v+w $} node[below]{$ w+v $} (z);
	
	\draw (z) node[above right] {$z = \tau_w(y)=(\tau_w\circ\tau_v)(x)=\tau_{w+v}(x)$};
	\draw (z) node[below right] {$z' = \tau_v(y')=(\tau_v\circ\tau_w)(x) = \tau_{v+w}(x)$};
	\draw (4,0.5) node[] {Translationen \glqq der\grqq{} Ebene bilden eine abelsche Gruppe};
\end{tikzpicture}

\begin{tikzpicture}[scale=1.5, >=triangle 45]
	\draw[->,color=black] (-0.1,0) -- (9,0);
	\draw[->,color=black] (0,-0.1) -- (0,4);
	
	\coordinate[label=left:$ x $] (x) at (1,3);
	\coordinate[label=right:$ y \equal \tau_v(x) \equal (\tau_{\frac{v}{2}} \circ \tau_{\frac{v}{2}})(x) \equal \tau_{\frac{v}{2} + \frac{v}{2}}(x) $] (y) at (3,1);
	
	\draw [fill] (x) circle (.5pt);
	\draw [fill] (y) circle (.5pt);
	
	\draw [->] (1,2.8) --node[left]{$v$} (3,.8);
	\draw [->] (1,3.2) --node[above] {$\frac{v}{2}$} (2,2.2);
	\draw [->] (2,2.2) --node[above] {$\frac{v}{2}$} (3,1.2);
	
	\draw (5,3) node[text width = 7cm] {Translationen kann man \glqq strecken\grqq{}, sodass gewisse Rechengesetze gelten.};
\end{tikzpicture}

\subsection{Definition (Vektorraum)}
	\begin{Definition}[Vektorraum]
		Sei $K$ ein Körper. Eine Menge $V$ mit zwei Abbildungen
	\begin{align*}
		 +&: V \times V \to V:(v,u)\mapsto v+w,\\
		 \cdot &: K \times V \to V:(x,v)\mapsto vx,
	\end{align*}
	
	heißt \emph{Vektorraum über $K$} ($K$-VR), falls gilt:
	\begin{enumerate}[(i)]
		\item $(V,+)$ ist eine abelsche Gruppe
		\item $\forall v\in V: v\cdot 1=v$ und\\
                      $\forall x,y \in K\ \forall v\in V: (v\cdot x)\cdot y = v\cdot (x\cdot y)$
		\item $\forall x,y \in K\ \forall v\in V: v(x+y) = vx + vy$\\
                      $\forall x\in K\ \forall v,w\in V: (v+w)x = vx + wx$
	\end{enumerate}
	\end{Definition}

\paragraph{Bemerkung}
	Wir notieren die Skalarmultiplikation als Rechtsmultiplikation (Skalar steht rechts):
		\[ \cdot: K \times V \to V : (x,v) \mapsto vx \]

\paragraph{Beispiel}
	Die Translationen eines affinen Raumes bilden einen Vektorraum (vgl. mit der Skizze oben): Dieses Beispiel wird im nächsten Kapitel repräsentiert.
	
\paragraph{Beispiel}
	Jeder Körper $ K $ ist ein $ K $-VR (Vektorraum über sich selbst): das ist ein (trivialer) Spezialfall des folgenden:
	
\subsection{Beispiel und Definition (Standardvektorraum)}
	\begin{Definition}[Standardvektorraum]
		Ist $ I $ eine Menge und $ K $ ein Körper, so bilden die $ K $-wertigen Abbildungen
		\[ v: I \to K: i \mapsto v_i \]
	einen Vektorraum mit der punktweise definierten Addition und Skalarmultiplikation:
	\begin{align*}
		I\ni i \mapsto (v+w)_i &:= v_i+w_i\in K\\
		I\ni i \mapsto (vx)_i &:= v_ix \in K
	\end{align*}

	Dieser Vektorraum wird mit $K^{I}$ bezeichnet und \emph{Standardvektorraum} (über I und K) genannt. Im Falle $ I=\{1,...,n\} $ schreibt man auch $K^{n} := K^{\{1,...,n\}}$
	\end{Definition}

\subsection{Bemerkung und Definition (Familienschreibweise)}
	\begin{Definition}
		Anstelle der normalen Schreibweise
		\[ I\ni i \mapsto v(i) \in K \]
	für die Auswertung einer Abbildung  $v: I \to K$ um einen Punkt $i\in I$ haben wir die \emph{Indexschreibweise} verwendet:	
		\[ I\ni i \mapsto v_i \in K \]
	Wir haben damit eine Abbildung $v: I \to K$ als \emph{Familie} $ (v_i)_{i\in I} $ über der Indexmenge $ I $ aufgefasst -- Familie ist ein \glqq alternativer\grqq{} Begriff für Abbildung.
	\end{Definition}
	
\paragraph{Beispiel}
	Sei $i$ eine \glqq Zahl\grqq{} mit $i^2=-1$ ($i$ entspricht nicht dem Element der Indexmenge aus dem vorherigen Abschnitt). Die Menge der komplexen Zahlen
		\[ \mathbb{C}:=\{{x+iy\mid x,y\in \mathbb{R}}\} \]
	bildet mit der Addition und Multiplikation einen Körper:
	\begin{align*}
		+&:\mathbb{C}\times \mathbb{C} \to \mathbb{C}: ((x+y),(x'+y')) \mapsto (x+iy)+(x'+iy') := (x+x')+i(y+y')\\
		\cdot &:\mathbb{C}\times \mathbb{C} \to \mathbb{C}: ((x+iy),(x'+iy'))\mapsto (x+iy)\cdot (x'+iy') :=(xx'-yy')+i(xy'+x'y)
	\end{align*}

	Der Körper $\mathbb{C}$ bildet einen $\mathbb{R}$-VR mit
		\[ +:\mathbb{C}\times\mathbb{C}\to\mathbb{C} \]
	wie oben und der Skalarmultiplikation
		\[ \cdot:\mathbb{R}\times\mathbb{C}\to\mathbb{C}:(x',(x+iy))\mapsto(x+iy)x':=xx'+iyx' \]
	Diese Skalarmultiplikation ist also gerade die Einschränkung der komplexen Multiplikation auf $\mathbb{R}\times\mathbb{C}$ wobei die Identifikation
		\[ \mathbb{R}\cong \{{x+iy\in\mathbb{C}\mid y=0}\} \]
	verwendet wird.
	
%VO05-2015-10-20
\subsection{Definition (Untervektorraum)}
	\begin{Definition}[Untervektorraum]
		Eine Teilmenge $U\subset V$ eines $K$-VR $V$ heißt \emph{Unter(vektor)raum} (UVR), falls $U$ mit der eingeschränkten Addition und Skalarmultiplikation
	\begin{align*}
		 ^+    & \mid_{U\times U}: U\times U \to V,(v,w) \mapsto v+w \\
		 \cdot & \mid_{K\times U}: K\times U \to V,(x,v) \mapsto vx
	\end{align*}

	selbst ein Vektorraum ist, d.h. wenn insbesondere
	\begin{align*}
		&\forall v,w \in U: v+w\in U \text{ und}\\
		&\forall x\in K\ \forall v\in U: vx\in U.
	\end{align*}
	\end{Definition}

\paragraph{Bemerkung}
	Eine nicht-leere Teilmenge $U\subset V, U\neq\emptyset$, ist genau dann ein UVR, wenn die auf U eingeschränkten Operationen wohldefiniert sind, d.h. wenn $ U $ bzgl. $ + $ und $ \cdot $ abgeschlossen ist.

	Dies kann zum \emph{Unterraumkriterium} zusammengefasst werden:
	\begin{equation*}
		U\subset V \text{ ist UVR }\Leftrightarrow 
 		 \begin{cases}
 		 	U\neq\emptyset\\
 		 	\forall v,w\in U\ \forall x\in K: vx+w\in U
 		 \end{cases}
	\end{equation*}

\paragraph{Beispiel}
	Sei $I=\{1,...,n\}$. Für jedes (feste) $i\in I$ ist
		\[ U_i := \{v:I\to K\mid v_i =0\} \]
	ein UVR von $K^n$, denn
	\begin{enumerate}
		\item $v = 0 \in U_i\text{, also } U_i \neq \emptyset$
		\item Seien $v,w\in U_i$, d.h. $v,w\in K^n$ mit $v_i =w_i =0$, und $x\in K$; dann gilt $(vx+w)_i = v_ix+ w_i = 0\cdot x + 0 = 0$, also $vx+w\in U_i$ und damit ist $U_i$ UVR nach Unterraumkriterium.  
	\end{enumerate}
	
	Kein UVR von $K^n, n\geq 2$, ist jedoch die Menge
		\[ N:=\{v:I\to K\mid v_1\cdot v_2 = 0\}, \]
	denn 
	\begin{enumerate}
		\item $N$ ist zwar nicht-leer, $N\neq \emptyset$, aber
		\item $^+\mid_{N\times N}: N\times N\to N$ nicht wohldefiniert: seien $v,w\in N$, so dass
			\begin{gather*}
				v_1=0, v_2=1\text{ }(v_3 ... v_n \text{ irrelevant})\\
				w_1=1, w_2 = 0\text{ }(w_3 ... w_n \text{ irrelevant})
			\end{gather*}
	\end{enumerate}
	dann gilt:
	\begin{gather*}
		(v+w)_1 = v_1 + w_1 = 0+1=1\\
		(v+w)_2 = v_2 + w_2 = 1+0 = 1
	\end{gather*}
	und damit
		\[ (v+w)_1(v+w)_2 = 1 \Rightarrow v+w\notin N. \]

\paragraph{Bemerkung und Beispiel}
	In analoger Weise definiert man die Begriffe
	\begin{itemize}
		\item einer \emph{Untergruppe} $H\subset G$ einer Gruppe $(G,\cdot)$, bzw.
		\item eines \emph{Unter-} oder \emph{Teilkörpers} $T\subset K$ eines Körpers $(K,+,\cdot )$
	\end{itemize}
	
	z.B. bildet jeder UVR $U\subset V$ eines $K$-VR $V$ (mit der Addition) eine Untergruppe der Gruppe $(V,+)$.\\
        In gleicher Weise ist eine nicht-leere Teilmenge eine \emph{Untergruppe} bzw. einen \emph{Unterkörper}, falls die eingeschränkten Operationen wohldefiniert sind.
    
        z.B. ist $H\subset G$ eine Untergruppe, falls (Untergruppenkriterium):
        \begin{enumerate}
            \item $H\neq \emptyset$
            \item $\forall g,h\in H: g\circ h^{-1} \in H$
        \end{enumerate}
            
	Achtung: Inversenbildung muss im Kriterium explizit formuliert werden, sonst würde z.B.: $\mathbb{N}\subset\mathbb{Z}$ als Teilmenge von $(\mathbb{Z}, +)$ als Gruppe ein Gegenbeispiel liefern.
        
        Weitere Beispiele:
        \begin{itemize}
            \item die Translationen bilden eine Untergruppe der Bewegungsgruppe
            \item $\mathbb{Q}\subset\mathbb{R}$ und $\mathbb{R}\cong \{x+iy\mid y=0\}\subset\mathbb{C}$ bilden Teilkörper von $\mathbb{R}$ bzw. $\mathbb{C}$.
        \end{itemize}

\subsection{Lemma (Schnitt von UVR)}
    \begin{Lemma}[Schnitt von UVR]
    	Ist $(U_i)_{i\in I}$ eine Familie von UVR $U_i\subset V$ eines $K$-VR $V$, so ist ihr Schnitt
        \[ U:= \bigcap_{i\in I}U_i =\{ u\in V\mid \forall i\in I: u\in U_i\} \]
	ein UVR von $V$. (Beweis durch UR-Krit. in Aufgabe 17)
    \end{Lemma}
    
\subsection{Definition (Lineare Hülle)}
	\begin{Definition}
		Die \emph{lineare Hülle} $[S]$ einer Teilmenge $S\subset V$ eines $ K $-VR $ V $ ist der Schnitt aller $S$ enthaltenden UVR $U\subset V$:
		\[ [S] := \bigcap_{S\subset U \text{ UVR}} U \]
	Die lineare Hülle einer Familie $(v_i)_{i\in I}$ von Vektoren $v_i\in V$ in einem $ K $-VR $ V $ ist
		\[ [(v_i)_{i\in I}] := [\{v_i\mid i\in I\}] \]
	\end{Definition}

\paragraph{Bemerkung}
    $[S]$ ist ein UVR (nach Lemma), der \glqq kleinste\grqq{} UVR, der $S$ enthält, d.h. ist $U\subset V$ UVR mit $S\subset U$, so gilt $[S]\subset U$; da aber $[S] = \bigcap_{S\subset \tilde{U}  \text{ UVR}}\tilde{U}\subset U$,
    da $S\subset U$, also $U$ am Schnitt beteiligt ist.

\paragraph{Bemerkung}
	$[\emptyset ] = \{0\}$ und $[V] = V$.

\paragraph{Beispiel}
	Ist $U\subset V$ UVR, so gilt $[U] = U$.

\paragraph{Beispiel}
	$N=\{v:I\to K\mid v_1v_2=0\} \subset K^n,I=\{1,...,n\},n\geq 2$, hat die lineare Hülle $[N]=K^n$.

\paragraph{Beispiel}
	Für $I=\{1,...,n\}$ und $i\in I$ definiere
	$e_i:I\to K , j\mapsto e_i(j):= \delta_{ij}$, wobei 
	\begin{equation*}
		\delta_{ij} :=
		\begin{cases}
			1,& \text{falls }i=j\\
			0,& \text{sonst}
		\end{cases}
	\end{equation*}
	das \emph{Kroneckersymbol} bezeichnet.
	
	Damit ist die lineare Hülle der Familie $(e_i)_{i\in I}$
		\[ [(e_i)_{i\in I}] = K^n. \]
	
	Nämlich: Da $[(e_i)_{i\in I}]\subset K^n$ ist, gilt für beliebige $x_1,...,x_n\in K$
	\begin{gather*}
		\underbrace{e_1x_1\underbrace{+...+\underbrace{e_nx_n + 0}_{\in [(e_i)_{i\in I}]}}_{\in [(e_i)_{i\in I}]}}_{\in [(e_i)_{i\in I}]}\in [(e_i)_{i\in I}]
	\end{gather*}
	da $[(e_i)_{i\in I}] \subset K^n$ UVR ist. 
	Andererseits gilt für beliebiges $v\in K^n$:
		\[ v=\sum^n_{i=1}e_iv(i): I\to K, \]	
	denn
		\[ \forall j\in I: \left(\sum^n_{i=1} e_iv(i)\right)(j) = \sum^n_{i=1}e_i(j)v(i) = v(j) \]
	Damit ist gezeigt, dass die beiden Abbildungen übereinstimmen; da $v\in K^n$ beliebig war, folgt $K^n \subset [(e_i)_{i\in I}]$
	
\subsection{Definition (Linearkombination)}
	\begin{Definition}
		Seien $(v_i)_{i\in I}$ und $(x_i)_{i\in I}$ Familien in einem $ K $-VR bzw. dem Körper $ K $, wobei
	\begin{align*}
		\# &\{i\in I\mid x_i \neq 0\} < \infty\text{ , also}\\
		   &\{i\in I \mid x_i \neq 0\} = \{i_1,...,i_n\}
        \end{align*}
        für ein geeignetes  $n\in \mathbb{N}$;
    	Dann heißt die endliche Summe
            \[\sum_{i\in I} v_ix_i:= \sum^n_{j=1}v_{i_j}x_{i_j}\]
        eine \emph{Linearkombination}.
	\end{Definition}
	
\paragraph{Bemerkung}
	Die Bedingung $\#\{i\in I \mid x_i\neq 0\} <\infty$
	garantiert, dass die Summe wohldefiniert ist $\rightarrow$ vgl. Reihen in der Analysis.

%VO06-2015-10-22
\subsection{Lemma (Lineare Hülle und Linearkombinationen)}
	\begin{Lemma}[Lineare Hülle und Linearkombinationen]
		Ist $(v_i)_{i\in I}$, $I \neq \emptyset$, Familie in einem $K$-VR, so gilt: 
		\[ [(v_i)_{i\in I}] = \left\{\sum_{i\in I} v_ix_i\mid x: I\to K: \# \{i\in I \mid x_i \neq 0\}< \infty\right\}, \]
	d.h. die lineare Hülle der Familie ist die Menge aller Linearkombinationen der Familie.
	\end{Lemma}

\paragraph{Beweis}
	Wir zeigen (wie üblich) zwei Inklusionen:	

	"$\supseteq$":
	
	Sei also $(x_i)_{i\in I}$ eine geeignete Familie in $ K $, dann gilt:
		\[ \sum_{i\in I} v_i x_i = \underbrace{v_{i_1} x_{i_1} + ... + \underbrace{(v_{i_n}x_{i_n}+0)}_{\in [(v_i)_{i\in I}]}}_{\mathrlap{ \in [(v_i)_{i\in I}] \text{ nach UR-Krit. (nach n Schritten)}}} \]

	"$\subseteq$":
	
	Setze $U := \{{\sum_{i\in I} v_ix_i\mid x: I\to K \text{ mit } \#\{{i\in I\mid x_i \neq 0\}} < \infty\}}$, offenbar gilt:
		\[ \forall i\in I: v_i\in U \]
	Wir zeigen, dass $U$ ein Untervektorraum ist. Das heißt:
	\begin{align*}
	^+    & \mid_{U\times U}: U\times U \to U \subset V\\
	\cdot & \mid_{K\times U}: K\times U \to U \subset V,
	\end{align*}
	
	also die Addition und Skalarmultiplikation vererben sich auf $ U $.

Zur Skalarmultiplikation:
        \begin{addmargin}[25pt]{0pt}
	Sind $(x_i)_{i\in I}$ mit $\#\{i\in I \mid x_i \neq 0\}<\infty$ eine Familie in $ K $ und $x\in K$, so gilt für ein geeignetes $n\in \mathbb{N}$
		\[ \{i\in I\mid x_i \neq 0\} = \{i_1, ... , i_n\} \]

	und damit
	\begin{equation*}
		\{i\in I\mid x_ix\neq 0\} =
		\begin{cases}
			\{{i_1,...,i_n\}}& \text{falls }x \neq 0\\
			\emptyset& \text{falls }x = 0.
		\end{cases}
	\end{equation*}

	Also folgt
	\begin{align*}
		(\sum_{i\in I}v_i x_i) x &= (\sum_{j=1}^{n} v_{i_j}x_{i_j})x\\
		&= \sum_{j=1}^{n} v_{i_j}(x_{i_j}x) = \sum_{i\in I} v_i(x_ix) \in U_i,
	\end{align*}

	da $\sum_{i\in I} v_i(x_ix)$ Linearkombination (mit der Familie $(x_ix)_{i\in I}$ in K) ist.
	\end{addmargin}

Zur Addition:
        \begin{addmargin}[25pt]{0pt}
            Ähnlich (Vereinigung zweier Mengen, ist endlich), siehe Aufgabe.
	\end{addmargin}
	
\paragraph{Bemerkung}
	Um triviale Diskussionen zu vermeiden, setzt man $\sum_{i\in \emptyset} ...:=0$.

% % % % Chapter 1 Section 3 % % % %
\section{Basis und Dimensionen}

\paragraph{Definition}
	Eine Teilmenge $S\subset V$ oder eine Familie $(v_i| i\in I)$ in $ V $ heißt:
	\begin{itemize}
		\item Erzeugendensystem von $ V $, falls $[S] = V$ bzw. $[(v_i)_{i\in I}] = V$
		\item linear unabhängig, falls $\forall v\in S: v \notin [S\setminus\{{v\}}]$ bzw. $\forall i\in I: v_i \notin [(v_j)_{j\in I\setminus\{{i\}}}]$
	\end{itemize}

	und sonst linear abhängig. Eine Basis ist ein linear unabhängiges Erzeugendensystem.

\paragraph{Bemerkung}
	Man kann jede (Teil-) Menge $S\subset V$ als Familie in V auffassen mit
	\begin{equation*}
		v: S \to V: v\mapsto id_S(v) = v.
	\end{equation*}
	
	Andererseits gilt für eine Familie $(v_i)_{i\in I} $:
	\begin{equation*}
		(v_i)_{i\in I} \text{ linear unabhängig } \Rightarrow \{v_i| i\in I\} \text{ linear unabhängig.}
	\end{equation*}
	
	Die Umkehrung gilt im Allgemeinen nicht.

	Eine Familie (in $ V $) enthält mehr Information als eine Teilmenge von $ V $.
	
\paragraph{Beispiel und Definition}
	Für $V = K^n$ ist $(e_1, ... , e_n)$,
	\begin{equation*}
		e_i:\{{1, ... ,n\}} =: I\to K: j\mapsto e_i(j)= \delta_{ij}=
		\begin{cases}
			1,& \text{falls } i=j\\
			0,& \text{sonst}
		\end{cases}
	\end{equation*}

	eine Basis -- die Standardbasis des (Standard-)Vektorraumes $K^n$.

\paragraph{Beweis}
	Z.z.: $ (e_i)_{i\in I} $ ist ein linear unabhängiges Erzeugendensystem. Bekannt ist: $ [(e_i)_{i\in I}] = K^n $. Andererseits gilt für jedes $i\in I$ und jede Familie $(x_j| j\in I)$ in $ K $
	\begin{gather*}
		\left(\sum_{j\in I\setminus\{i\}}e_jx_j\right)(i) = \sum_{j\in I\setminus\{i\}}e_j(i)x_j = 0 \neq 1 = e_i(i)\\
		\Rightarrow \sum_{j\in I\setminus\{i\}} e_jx_j \neq e_i,
	\end{gather*}
	
	also gilt:
	\begin{equation*}
		\forall i\in I: e_i \notin [(e_j)_{j\in I\setminus\{i\}}] = \left\{\sum_{j=I\setminus\{i\}} e_jx_j\mid (x_j)_{ j\in I}\right\} \text{ mit } \#\{j\in I| x_j \neq 0\}<\infty
	\end{equation*}
	
\paragraph{Lemma}
	Eine Familie $(v_i)_{i\in I}$ ist linear unabhängig gdw. für jede Linearkombination
	\begin{equation*}
		0 = \sum_{i\in I} v_ix_i \Rightarrow \forall i\in I: x_i = 0.
	\end{equation*}

\paragraph{Beweis}
	Wir zeigen zwei Richtungen der Äquivalenz der Negationen: 
	\begin{equation*}
		(v_i)_{i\in I} \text{ linear abhängig } \Leftrightarrow \exists(x_i)_{i\in I} \neq (0)_{i\in I}: \sum_{i\in I} v_ix_i = 0.
	\end{equation*}

	$\Leftarrow$: Wir nehmen an, es gäbe eine nicht-triviale Linearkombination der Null,
	\begin{equation*}
		0 = \sum_{i\in I} v_ix_i, \text{ wobei } \exists j\in I: x_j \neq 0.
	\end{equation*}

	Für $(y_i)_{i\in I}, y_i := - \frac{x_i}{x_j}$ ist dann
	\begin{equation*}
		0 = v_jx_j + \sum_{i\in I\setminus\{j\}} v_ix_i \Rightarrow v_j = -\left(\sum_{i\in I\setminus\{j\}}v_ix_i\right)x_j^{-1} = \sum_{i\in I\setminus\{j\}} v_iy_i \in [(v_i)_{i\in I\setminus\{j\}}],
	\end{equation*}

	insbesondere ist also $(v_i)_{i\in I}$ linear abhängig.

	$\Rightarrow$: siehe Aufgabe.
	
\paragraph{Korollar}
	Ist $(v_i)_{i\in I}$ Basis von $ V $, so ist jeder Vektor $v\in V$ eindeutig in den $v_i$ darstellbar:
	\begin{equation*}
		\forall v\in V \exists! (x_i)_{i\in I}: v = \sum_{i\in I} v_ix_i
	\end{equation*}

\paragraph{Beweis}
	Sei $v\in V$ beliebig, dann gilt:
	\begin{equation*}
		V = [(v_i)_{i\in I}] \Rightarrow \exists (x_i)_{i\in I}: v = \sum_{i\in I} v_ix_i
	\end{equation*}

	liefern $(x_i)_{i\in I}$ und $(y_i)_{i\in I}$
	\begin{equation*}
		v = \sum_{i\in I} v_ix_i = \sum_{i\in I}v_iy_i \Rightarrow 0 = \sum_{i\in I} v_i(x_i-y_i)
		\begin{array}{l}
			\Rightarrow \forall i\in I: x_i = y_i\\
			\Rightarrow (x_i)_{i\in I} = (y_i)_{i\in I}
		\end{array}
	\end{equation*}

	Damit ist die Basisdarstellung $v = \sum_{i\in I} v_ix_i$ von $ v $ auch eindeutig.


\paragraph{Basislemma}
    Sei $S\subset V$ lin. unabh. und $E\subset V$ ein Erzeugendensystem mit $S\subset E$. Dann existiert eine Basis $B$ von $V$ mit $S\subset B\subset E$.

\paragraph{Beweis}
    Wir gehen für den Beweis davon aus, dass $\#E<\infty$. Betrachte alle Teilmengen $X\subset V$ mit $S\subset X\subset E$ und $X$ lin. unabh. Sei $B$ eine solche Menge, die maximal ist, d.h.
    \begin{equation*}
        \forall X\subset E: ((B\subset X\land X\text{ lin. unabh.}) \Rightarrow X= B)
    \end{equation*}
    
    Nach Konstruktion ist $B=\{b_1,...,b_n\}$ lin. unabh. Zu zeigen: $V=[B]$.\\
    Ist $B=E$, so folgt $[B]=[E]=V$.\\
    Ist $B\neq E$, so ist $B\cup \{v\} $ für (jedes) $v\in E\setminus B$ lin. abh., da $B$ maximal (Existenz einer maximalen Menge ist problematisch!) und lin. unabh. ist; also existiert eine nicht-triviale Linearkombination des Nullvektors.
    \begin{equation*}
    \exists x,x_1,...,x_n \in K: o=vx+\sum^n_{i=1}b_ix_i
    \end{equation*}

    Wäre $x=0$, so würde folgen $x_1=...=x_n=0$, da $B$ lin. unabh. ist. 
    Also ist $x\neq 0$ und 
    \begin{equation*}
    	v=-\sum^n_{i=1} b_i\frac{x_i}{x} \in [B].
    \end{equation*}
    
    Da dies für beliebiges $v\in E\setminus B$ gilt, folgt
    \begin{equation*}
    	E\subset [B] \Rightarrow V=[E]\subset [[B]] = [B],
    \end{equation*}
    
    d.h., $ B $ ist Erzeugendensystem und damit eine Basis mit $S\subset B\subset E$.

\paragraph{Bemerkung}
    Ist $\#E = \infty$, so kann man einen analogen Beweis führen, falls man an die Existenz einer maximalen Menge glaubt: Dies garantiert das Zornsche Lemma bzw. Auswahlaxiom.
    Wir werden das Lemma auch im Falle $\#E = \infty$ benutzen!

\paragraph{Beispiel}
    Für $V=K^3=K^I$ mit $I=\{1,2,3\}$ betrachte die Standardbasisvektoren 
    \begin{align*}
        e_i &:I\to K, j\mapsto e_i(j) = \delta_{ij}\text{, und}\\
        f_i &: I\to K, j\mapsto f_i(j):= 1-\delta_{ij};
    \end{align*}

    dann sind $S:= \{e_1,f_1\}$ und $E:= \{e_i,f_i\mid i\in I\}$ lin. unabh. bzw. Erzeugendensystem von $K^3$. Ergänzung von $S$ durch einen Vektor $e_i$ oder $f_i, i = 2,3$ liefert eine Basis $B$ mit $S\subset B\subset E$.
    
    Zum Beispiel: $B=\{e_1,f_1,f_2\}$ eine Basis, da sich jede Funktion $v\in K^3$ aus den Funktionen $e_1,f_1$ und $f_2$ linear kombinieren lässt.
    \begin{gather*}
        v=e_1x_1+f_1y_1 + f_2y_2\Leftrightarrow \left\{
            \begin{array}{l}
                v(2)=y_1\\
                v(3) - v(2) = y_1 + y_2 - y_1 = y_2\\
                v(1) + v(2) - v(3) = x_1 + y_2 - y_2 = x_1
            \end{array}
    	\right.
    \end{gather*}

    Dass $B$ lin. unabh. folgt dann; Wäre $B$ lin. abh., so würde folgen $f_2\in [\{e_1,f_1\}]\Rightarrow [B] \subset [\{e_1,f_1\}] \neq K^3$, was nicht der Fall ist.

\paragraph{Basisergänzungssatz}
    Jede lin. unabh. Menge $S\subset V$ kann zu einer Basis $B$ von $V$ ergänzt werden: Es existiert eine Basis $B$ von $V$ mit $S\subset B$.

\paragraph{Beweis}
    Sei $E\subset V$ ein Erzeugendensystem von $V$ (z.B. $E=V$). Dann ist $S\cup E$ ein Erzeugendensystem von $V$ mit $S\subset S\cup E$, das Basislemma liefert dann die gesuchte Basis.

\paragraph{Bemerkung und Definition}
    Strikt genommen haben wir den Basisergänzungssatz (BES) nur unter der Annahme bewiesen, dass $V$ endlich erzeugt sei, d.h. $V$ ein endliches Erz. Syst. $E$ besitzt, $V=[E]$ und $\#E<\infty$.
    
\paragraph{Bemerkung}
    Wir haben für den BES die (in diesem Falle einfachere) Mengenschreibweise (anstelle der Familienschreibweise) verwendet.
    
\paragraph{Bemerkung}
    Ähnlich kann man einen Verkürzungssatz beweisen: Jedes Erzeugendensystem eines Vektorraums $V$ kann zu einer Basis verkürzt werden.

\paragraph{Austauschlemma}
    Seien $B,B' \subset V$ Basen von $V$. Dann gilt:
    \begin{equation*}
        \forall b\in B \exists b' \in B': (B\setminus\{b\})\cup\{b'\} \text{ ist Basis}
    \end{equation*}
    
\paragraph{Beweis}
    Sei $b\in B$ beliebig gewählt und $S:= B\setminus \{b\}$. Da $B$ lin. unabh. ist, gilt $b\notin [S] \Rightarrow \emptyset \neq V\setminus [S] = [B']\setminus [S] \Rightarrow B' \not\subset [S]$, d.h. es existiert $b' \in B'$ mit $b' \notin [S]$. Wir zeigen, dass $B'' := S\cup \{b'\} = (B\setminus\{b\})\cup \{b'\}$ Basis ist. $B''$ ist Erzeugendensystem: Da $b'\in [B]$ existiert $(x_j)_{j\in B}$ mit $$b' = \sum_{j\in B} jx_j $$ mit $x_b \neq 0$, da $b' \notin [S]$.

    Damit ist $b=(b'-\sum_{j\in S} jx_j)\frac{1}{x_b} \in [B'] \Rightarrow V = [B] \subset [B'' \cup \{b\}] \subset [B'']$.
    
    $B''$ ist linear unabhängig. $B''$ ist Erz. Syst. und $S\subset B' = S \cup \{b'\}$ lin unabh., kann also (nach Basislemma) erg"anzt werden zu einer Basis $\tilde{B}$ mit $S\subset \tilde{B}\subset B''$.
    Da $[S] \neq V$ gilt $\tilde{B} \neq S$ und damit $\tilde{B} = B''$ Basis, insbesondere linear unabhängig.
    
\paragraph{Bemerkung}
    Hier haben wir die Familienschreibweise (mit $B$ bzw. $S$ als Indexmenge) verwendet, um Linearkombinationen darzustellen.
    
\paragraph{Basissatz}
    Sei $V$ ein endlich erzeugter $K$-VR, $V=[E]$ mit $\#E < \infty$. Dann gilt:
    \begin{enumerate}[(i)]
   		\item $V$ besitzt eine endliche Basis $B$ mit $n:= \#B \leq \#E$.
    	\item Ist $B'\subset V$ eine Basis von $V$, so ist $\#B' = \#B = n$.
    \end{enumerate}
    
\paragraph{Beweis}
    \begin{enumerate}[(i)]
        \item  Dies folgt direkt aus dem Basislemma (mit $S=\emptyset$).
        \item Seien $B,B'$ Basen von V, $B = (b_1,...,b_n)$.\\
        Annahme: $\#B' < n, B' = (b'_1,...,b'_k)$ mit $k < n$. Wiederholte Anwendung des Austauschlemmas auf die Basen $B$ und $B'$ liefert nach (spätestens) $k+1\leq n$ Schritten einen Widerspruch zur linearen Unabhängigkeit der neuen Basis $B''$, da Vektoren $b'_i$ doppelt vorkommen müssen.\\
        Annahme: $\#B' > n, B' = (b'_1,...,b'_n,b'_{n+1})$: Das gleiche Argument mit vertauschten Rollen der Basen führt wieder zum Widerspruch.
     \end{enumerate}

\paragraph{Definition}
    Sei $V$ ein $K$-VR, die Dimension von $ V $ ist dann:
    \begin{itemize}
        \item $\dim V:= \#B$, falls $ V $ endlich erzeugt und $B$ eine Basis von $V$ ist;
        \item $\dim V:= \infty$, falls $V$ nicht endlich erzeugt ist.
    \end{itemize}
    
\paragraph{Bemerkung: }
    Nach dem Basissatz hängt $\dim V = \#B$ (falls $V$ endlich erz.) nicht von der Basis $B$ ab, d.h. $\dim V$ ist wohldefiniert.
    
\paragraph{Beispiel: }
    $\dim K^n = \#\{e_1,...,e_n\} = n$ (Standardbasis).

\paragraph{Korollar: } Sei $ V $ ein $ K $-VR mit $\dim V =: n\in \mathbb{N}$. Dann gilt:
    \begin{enumerate}[(i)]
    	\item Ist $S \subset V$ linear unabhängig, so ist $\# S \leq n$ und $\# S = n$ genau dann, wenn $ S $ Basis ist.
    	\item Ist $E \subset V$ Erzeugendensystem, so ist $\#E \geq n$, bzw. $\#E = n$ genau dann, wenn $ E $ eine Basis ist.
    \end{enumerate}
    
\paragraph{Bemerkung:}
	Insbesondere: Ist $U\subset V$ UVR mit $\dim U=\dim V < \infty$, so gilt $ U=V $.
   
\paragraph{Beweis (Korollar):}
    \begin{enumerate}[(i)]
    	\item Ist $ S $ linear unabhängig, so existiert (nach BES) eine Basis $ B $ von $ V $ mit 
			 \begin{gather*}
			    S\subset B\Leftrightarrow \left\{
				    \begin{array}{l}
					    \#S \leq \#B\\
						\#S = \#B \Leftrightarrow S = B
					\end{array}
			    \right.
		    \end{gather*}
		    \item Analog (mit Basislemma), siehe Aufgabe 23.
	 \end{enumerate} 

%VO08-2015-10-29
\section{Homomorphismen}
 \subsection{Definition}
 	\begin{Definition}[Homomorphismus]
 		Sind $ V $ und $ W $ $ K $-VR, so heißt eine Abbildung $f: V \rightarrow W$ \emph{($K$-)linear} oder ein \emph{(Vektorraum-)""Homomorphismus} $f\in \hom(V,W)$, falls gilt:

 		\begin{enumerate}[(i)]
 			\item $\forall v,w \in V: f(v+w) = f(v) + f(w)$,
 			\item $\forall v\in V\ \forall x\in K: f(vx) = f(v)x$,
 		\end{enumerate}

 		das heißt, $ f $ ist verträglich mit den Vektorraumoperationen in $ V $ und $ W $.
 	\end{Definition}

 	\paragraph{Bemerkung}
 		Damit die Verträglichkeit mit der Skalarmultiplikation sinnvoll ist, müssen $ V $ und $ W $ Vektorräume über demselben Körper $ K $ sein.

 	\paragraph{Bemerkung}
 		Für $f\in \hom(V,W)$ gilt stets $f(0_V) = f(0_V\cdot0_K) = f(0_V)\cdot0_K = 0_W$.

 \subsection{Bemerkung \& Definition}
 	Ebenso erklärt man zum Beispiel \emph{Gruppenhomomorphismen} oder \emph{Körperhomomorphismen}. Sind etwa $(G,\circ)$ und $(H,\cdot)$ Gruppen, so ist eine Abbildung $f: G \to H$ ein Gruppenhomomorphismus, falls
 	\begin{equation*}
 		\forall g,h \in G: f(g\circ h) = f(g) \cdot f(h)
 	\end{equation*}

 	\paragraph{Beispiel}
 		Ist $f\in \hom(V,W)$ ein Vektorraumhomomorphismus so ist $ f $ nach (i) Gruppenhomomorphismus von $ (V,+) $ in $ (W,+) $.

 	\paragraph{Beispiel}
 		Sei $ V $ ein $ K $-VR und $y\in K$ fest, dann ist die Streckung um $y$
 		\[
 			\eta_y:V\to V,\ v\mapsto \eta_y(v) := vy
 		\]
 		ein Homomorphismus von $ V $ in sich, $\eta_y\in \hom(V,V)$. Eine Streckung nennt man auch \emph{Homothetie}.

 	\paragraph{Beispiel}
 		Sei $V := \mathbb{C} = \{z = x+iy\mid x,y\in \mathbb{R}\}$, dann ist die komplexe Konjugation
 		\[
 			\mathbb{C}\ni z = x+iy \mapsto x-iy =: \bar{z} \in \mathbb{C}
 		\]
 		kein Homomorphismus von $\mathbb{C}$ in sich, wenn man $\mathbb{C}$ als $\mathbb{C}$-VR auffasst. Hingegen ist sie ein Homomorphismus von $\mathbb{C}$ in sich, wenn man $\mathbb{C}$ als $ \mathbb{R} $-VR auffasst.

 \subsection{Lemma (Linearkombinationen und Homomorphismen)}
 	\begin{Lemma}[Linearkombinationen und Homomorphismen]
 		$f:V\to W$ ist genau dann ein Homomorphismus, wenn für jede beliebige Linearkombination gilt:
 		\begin{equation*}
 			f\Big(\sum_{i\in I}v_ix_i\Big) = \sum_{i\in I}f(v_i)x_i
 		\end{equation*}
 	\end{Lemma}

 	\paragraph{Beweis}
 		Eine Richtung ist trivial, die andere ist mit vollständiger Induktion zu zeigen.

 \subsection{Fortsetzungssatz}
 	\begin{Satz}[Fortsetzungssatz]
 		Seien $ V $ und $ W $ $K$-VR, $(b_i)_{i\in I}$ eine Basis von $ V $ und $(c_i)_{i\in I}$ eine Familie in $ W $.
 		Dann gilt:
 		\begin{equation*}
 			\exists!f\in \hom(V,W) \forall i\in I: f(b_i) = c_i
 		\end{equation*}
 	\end{Satz}

 	\paragraph{Bemerkung}
 		Anders ausgedrückt: Ist $B\subset V$ eine Basis von $ V $, so kann jede Abbildung $f: B\to C\subset W$ eindeutig zu einem Homomorphismus $f: V\to W$ fortgesetzt werden.

 	\paragraph{Beweis}
 		Wir beweisen Existenz und Eindeutigkeit getrennt.
 		\begin{enumerate}
 			\item Eindeutigkeit: Sei $f\in \hom(V,W)$ so, dass $\forall i\in I: f(b_i)=c_i$. Sei $v\in V$ beliebig. Da $ B $ Erzeugendensystem ist, lässt sich $ v $ als Linearkombination in $(b_i)_{i\in I}$ mit geeigneten Koeffizienten $(x_i)_{i\in I}$ in $ K $ darstellen.
 			      \begin{gather*}
 			      	v=\sum_{i\in I}b_ix_i \Rightarrow f(v) = \sum_{i\in I} f(b_i)x_i = \sum_{i\in I}c_ix_i
 			      \end{gather*}

 			      Damit ist $ f(v) $ eindeutig durch $ v $ und die $c_i = f(b_i)x_i$ bestimmt.

 			\item Existenz: Da $(b_i)_{i\in I}$ auch linear unabhängig ist, ist jedes $v\in V$ eindeutig als Linearkombination in $(b_i)_{i\in I}$ dargestellt, damit ist durch
 			      \begin{equation*}
 			      	f:V\to W,\ v=\sum_{i\in I}b_ix_i \mapsto f(v):=\sum_{i\in I}c_iv_i
 			      \end{equation*}
 			      eine Abbildung wohldefiniert.

 			      Weiters ist $f\in\hom(V,W)$ wegen
 			      \begin{align*}
 			      	  & f(v+w) =\sum_{i\in I}c_i(x_i+y_i)= f(v) + f(w) \text{ für alle } v=\sum_{i\in I}b_ix_i \in V, w=\sum_{i\in I}b_iy_i \in V
 			      	\intertext{und}
 			      	  & f(vx) =\sum_{i\in I}c_i(x_ix)=\sum_{i\in I}(c_ix_i)x = \Big(\sum_{i\in I}c_ix_i\Big)x= f(v)x \text{ für }  x\in K\text{ und }v= \sum_{i\in I}b_ix_i \in V.
 			      \end{align*}
 			      Damit ist die Linearität von $ f $ gezeigt.
 		\end{enumerate}

 \subsection{Beispiel und Definition (Dualraum)}
 	\begin{Definition}[Dualraum]
 		Der \emph{Dualraum} $V^\ast := \hom(V,K)$ eines $K$-VRs $V$ ist ein $ K $-VR $(\subset K^V)$. Ist $\dim V=:n<\infty$ so ist $\dim V^\ast=n$.

 		Nämlich: Ist $B=(b_1, ... ,b_n)$ eine Basis von $V$  $(\dim V < \infty)$, so definieren wir für $ i = \{1, ... ,n\} $ (nach Fortsetzungssatz):
 		\[
 			\forall j\in \{1,...,n\}: V^* \ni b_i^*(b_j)=\delta_{ij}
 		\]
 		eine Basis $B^*=(b^*_1,\dots,b^*_n)$ von $V^*$: die zu $B$ \emph{duale Basis} von $V^\ast$.
 	\end{Definition}

%VO09-2015-11-03
 	\paragraph{Beweis} z.z: $ V^* $ ist $ K $-VR. Wir zeigen $ V^*\subset K^V $ ist UVR.
 		\begin{itemize}
 			\item $ 0: V\to K $ ist linear, d.h. $ 0 \in V^* \Rightarrow V^* \neq \emptyset $
 			\item Seien $ f,g \in V^* $ und $ x\in K $; dann gilt
 			      \begin{align*}
 			      	\forall v,w\in V: (fx+g)(v+w)      & = f(v+w)x+g(v+w)            \\
 			      	                                   & = (f(v)+ f(w))x+(g(v)+g(w)) \\
 			      	                                   & = (f(v)x+g(v))+(f(w)x+g(w)) \\
 			      	                                   & = (fx+g)(v)+(fx+g)(w)
 			      	\intertext{genauso:}
 			      	\forall v\in V, y\in K: (fx+g)(vy) & = f(vy)x+g(vy)              \\
 			      	                                   & = f(v)yx + g(v)y            \\
 			      	                                   & = (f(v)x +g(v))y            \\
 			      	                                   & = (fx+g)(v)y
 			      \end{align*}
 			      Damit gilt: $ fx+g\in \hom (V,K) = V^* $
 		\end{itemize}

 		Da $ f,g\in V^* $ und $ x\in K $ beliebig waren, zeigt das UR-Kriterium, dass $ V^*\subset K^V $ ein UVR ist und damit selbst $ K $-VR ist.

 	\paragraph{Beweis} z.z: $B^*$ ist Basis. Wir zeigen $B^*$ ist linear unabhängig und Erzeugendensystem.
 		\begin{itemize}
 			\item $ B^* $ ist linear unabhängig: Seien $ x_1,...,x_n $ so, dass
 			      \begin{align*}
 			      	0 =                                      & \sum_{i=1}^{n}b_i^*x_i                                                                           \\
 			      	\Rightarrow \forall j=1,...,n: 0 = \Big( & \sum_{i=1}^{n}b_i^*x_i\Big)(b_j) = \sum_{i=1}^{n}\underbrace{b_i^*(b_j)}_{\delta_{ij}}x_i = x_j.
 			      \end{align*}
 			      Also $ x_1 = ... = x_n = 0 $ und damit ist $ B^* $ linear unabhängig.
 			\item $ B^* $ ist Erzeugendensystem: Sei $ f\in V^* $ beliebig, dann gilt:
 			      \begin{align*}
 			      	\forall j = 1,...,n:f(b_j)                             & = \sum_{i=1}^{n}\underbrace{b_i^*(b_j)}_{\delta_{ij}}f(b_i) = \Big(\sum_{i=1}^{n}b_i^*f(b_i)\Big)(b_j) \\
 			      	\underbrace{\Rightarrow}_{\text{(Fortsetzungssatz)}} f & = \sum_{i=1}^{n}b_i^*f(b_i)\in [B^*]
 			      \end{align*}
 			      Da $ f\in V^* $ beliebig war, ist also $ V^* = [B^*]$.
 		\end{itemize}
 		Damit ist $ B^* = \{b_1^*,...,b_n^*\}$ eine Basis von $ V^* $ -- insbesondere also $ \dim V^* = n = \dim V = \underbrace{\dim K}_{=1}\cdot \dim V $.

 	\paragraph{Bemerkung}
 		Ist $\dim V = \infty$ und $B=(b_i)_{i\in I}$ eine Basis von $V$, so liefert $B^\ast=(b_i^\ast)_{i\in I}$ mit
 		\[
 			\forall j\in I:b_i^\ast(b_j)=\delta_{ij}
 		\]
 		eine linear unabhängige Familie, die jedoch kein Erzeugendensystem von $V^\ast$ ist:
 		\[
 			f\in\hom(V,K)=V^\ast \text{ mit } \forall j\in I:f(b_j)=1
 		\]
 		lässt sich nicht in $B^\ast$ linear kombinieren. Wäre $f=\sum_{i\in I}b_i^\ast x_i$, so gälte
 		\[
 			\forall j\in I: x_j =\sum_{i\in I}\underbrace{b_i^\ast(b_j)}_{\delta_{ij}}x_j = f(b_j) = 1
 		\]
 		Das heißt, $(x_i)_{i\in I}$ wäre eine Familie in $ K $ mit $\#\{i\in I\mid x_i\neq 0\}=\infty$, also ist $\sum_{i\in I}b_i^\ast x_i$ keine Linearkombination.

 \subsection{Satz (Homomorphismen als VR)}
 	\begin{Satz}[Homomorphismen als VR]
 		$ \hom (V,W) $ ist ein VR, mit $\dim\hom (V,W) = m\cdot n$, falls $m:=\dim W<\infty, n:=\dim V< \infty$.
 	\end{Satz}
 	\paragraph{Beweis}
 		Addition und Skalarmultiplikation in $\hom (V,W)$ werden (wie für $K$-wertige Abbildungen oder in $V^*$) punktweise definiert:
 		\begin{itemize}
 			\item für $f,g \in \hom (V,W)$ setzt man $(f+g)(v) := f(v) + g(v)$ für alle $v\in V$,
 			\item für $f\in \hom (V,W)$ und $x\in K$ setzt man $(fx)(v) := f(v)x$ für alle $v\in V$.
 		\end{itemize}
 		Die so definierten Abbildungen $f+g,fx: V\to W$ sind linear, $f+g, fx\in \hom (V,W)$, aufgrund der VR-Eigenschaften von $V$. Damit zeigt man: $\hom (V,W)$ ist $K$-VR (siehe Aufgabe 27).

 		Seien nun $\dim V = n < \infty$ und $\dim W = m < \infty$.
 		Wir wählen (nach BES) Basen $B = (b_1,...,b_n)$ von $V$ und $C=(c_1,...,c_m)$ von $W$ und definieren
 		\begin{equation*}
 			\hom (V,W) \ni f_{ij}:= c_i\cdot b_j^* \text{ für }
 			\begin{cases}
 				i\in I := \{1,...,m\} \\
 				j\in J := \{1,...,n\}
 			\end{cases}
 		\end{equation*}
 		Behauptung: $F=(f_{ij})_{\substack{i\in I\\j \in J}}$ ist Basis von $\hom (V,W)$.

 		Da $(c_i)_{i\in I}$ linear unabhängig in $W$ ist, gilt für jede Familie $(x_{ij})_{\substack{i\in I\\j \in J}}$ in $K$:
 		\begin{align*}
 			0 = \sum_{\substack{i\in I \\j \in J}} f_{ij}x_{ij} &\Rightarrow \forall k \in J: 0 = \sum_{\substack{i\in I\\j \in J}} (f_{ij}x_{ij})(b_k) = \sum_{\substack{i\in I\\j \in J}} c_i b_j^* (b_k) x_{ij} = \sum_{i\in I} c_ix_{ik} \\
                                                                            & \Rightarrow \forall k\in J\ \forall i\in I:x_{ik} = 0
 		\end{align*}
 		Also ist $F$ linear unabhängig.

 		Da $(c_i)_{i\in I}$ Erzeugendensystem von $W$ ist, existiert zu jedem (fest gegebenen) $f\in\hom (V,W)$ eine Familie $(x_{ij})_{\substack{i\in I\\j \in J}}$ in $K$, sodass
 		\begin{align*}
 			\forall k\in J: f(b_k) & = \sum_{i\in I} c_i x_{ik} = \sum_{\substack{i\in I \\j \in J}}c_ib_j^*(b_k)x_{ij} = \Big(\sum_{\substack{i\in I\\j \in J}} f_{ij}x_{ij}\Big)(b_k)
 			\intertext{also (nach Fortsetzungssatz):}
 			f                      & =\sum_{\substack{i\in I\\j \in J}}f_{ij}x_{ij} \in [F]
 		\end{align*}
 		Da $f\in\hom (V,W)$ beliebig war, gilt also $\hom (V,W) = [F]$. Damit ist $F$ Basis von $\hom (V,W)$ und $\dim\hom (V,W) = \# F = m\cdot n$


 \subsection{Lemma und Definition (Bild, Kern, Rang \& Defekt)}
 	\begin{Definition}[Bild, Kern, Rang \& Defekt]
 		Sei $f\in \hom (V,W)$. Dann sind \emph{Bild} und \emph{Kern} von $f$,
 		\begin{align*}
 			f(V)       & = \{f(v)\in W\mid v\in V \}\subset W \\
 			\ker (f) : & = \{v\in V\mid f(v) = 0 \} \subset V
 		\end{align*}
 		UVR von $W$ bzw. $V$. Ihre Dimensionen heißen \emph{Rang} und \emph{Defekt} von $f$:
 		\begin{align*}
 			\rg f   & := \dim f(V)   \\
 			\dfkt f & := \dim \ker f
 		\end{align*}
 	\end{Definition}

 	\paragraph{Bemerkung}
 		Da $f(0)=0$ für  $f\in \hom (V,W)$ gilt: $\{0_V \}\in \ker f$ und $\{0_W \}\in f(V)$.

 	\paragraph{Beweis}
 		Zu zeigen: Das Bild $f(V)\subset W$ und $\ker f\subset V$ sind UVR.
 		Nach Bemerkung gilt $f(V)\neq \emptyset$ und $\ker f \neq \emptyset$ -- wir verwenden dann das UR-Kriterium.

 		Das Bild $f(V)$ ist UVR:
 		\begin{addmargin}[25pt]{0pt}
 			$f(V) \neq \emptyset$. Es bleibt zu zeigen:
 			\[
 				\forall w_1,w_2\in f(V), \forall x\in K: w_1x+w_2 \in f(V)
 			\]
 			Seien also $w_1 = f(v_1), w_2 = f(v_2) \in f(V)$ und $x\in K$; dann gilt:
 			\[
 				w_1x+w_2 = f(v_1)x+f(v_2) = f(v_1x+v_2)\in f(V)
 			\]
 		\end{addmargin}

 		Der Kern $\ker f$ ist UVR:
 		\begin{addmargin}[25pt]{0pt}
 			$\ker f\neq \emptyset$. Seien $v_1,v_2\in \ker f$ und $x\in K$, dann gilt:
 			\[
 				f(v_1x+v_2) = f(v_1)x+f(v_2) = 0\cdot x + 0 = 0 \Rightarrow v_1x+v_2\in \ker f
 			\]
 		\end{addmargin}

 	\paragraph{Bemerkung}
 		Allgemeiner kann man für $f\in \hom (V,W)$ zeigen:
 		\begin{enumerate}[(i)]
 			\item Ist $U\subset V$ UVR, so ist $f(U)\subset W$ UVR.
 			\item Ist $U\subset W$ UVR, so ist $f^{-1}(U) = \{v\in V\mid f(v) \in U \}\subset V$ ein UVR.
 		\end{enumerate}

 	\paragraph{Bemerkung}
 		Die Funktion $f\in \hom (V,W)$ ist genau dann injektiv, wenn $\ker f = \{0\}$. Nämlich:
 		\begin{itemize}
 			\item ist $f$ injektiv und $v\in \ker f$, so gilt $f(v) = 0 = f(0) \Rightarrow v=0$
 			\item ist $\ker f = \{ 0 \}$ und sind $v,w \in V$ mit $f(v) = f(w)$, so folgt $0=f(v)-f(w) = f(v-w) \Rightarrow v-w\in \ker f = \{0\} \Rightarrow v = w$
 		\end{itemize}

 	\paragraph{Bemerkung}
 		Eine lineare Abbildung $ f\in \hom (V,W) $ ist genau dann
 		\begin{enumerate}[(i)]
 			\item injektiv, wenn $ \forall S\subset V: S$ lin. unabh. $ \Rightarrow f(S) $ lin. unabh.
 			\item surjektiv, wenn $ \forall E \subset V:E $ Erz. Syst. $ \Rightarrow f(E)$ Erz. Syst.
 			\item bijektiv, wenn $ \forall B\subset V: B$ Basis $ \Rightarrow f(B)$ Basis
 		\end{enumerate}

 		Ist $ f\in \hom (V,W) $ bijektiv, so ist $ f^{-1}\in \hom (W,V) $.

 \subsection{Rangsatz}
 	\begin{Satz}[Rangsatz]
 		Sei $ f\in \hom (V,W) $. Ist $ \dim V = n < \infty $,  so gilt
 		\[
 			\rg f + \dfkt f = \dim V
 		\]
 		Ist $ \dim V = \infty $, so gilt $ \rg f = \infty $ oder $ \dfkt f = \infty $.
 	\end{Satz}

%VO09-2015-11-05
 	\paragraph{Beweis}
 		Wir nehmen an, dass $ \dfkt f = k < \infty $.
 		Sei $ (b_1,...,b_k) $ eine Basis von $ \ker f $;
 		nach BES ergänzen wir zu einer Basis $ (b_j)_{j\in J} $ von $ V $ (bemerke: $ \{1,...,k\}\subset J $).
 		Wir setzen
 		\[
 			c_i:=f(b_i)\ \text{ für } i\in I:=J\setminus \{1,...,k\}
 		\]
 		Behauptung: $(c_i)_{i\in I}$ ist eine Basis von $f(V)$.

 		Lineare Unabhängigkeit:
 		\begin{addmargin}[25pt]{0pt}
 			Gilt für eine Linearkombination in $(c_i)_{i\in I}$:
 			\[
 				0=\sum_{i\in I}c_ix_i = \sum_{i\in I}f(b_i)x_i = f\Big(\sum_{i\in I}b_ix_i\Big)
 			\]
 			so folgt
 			\begin{align*}
 				                                       & \sum_{i\in I}b_ix_i \in \ker f             \\
 				\Rightarrow \exists y_1,...,y_k\in K:&\sum_{i\in I}b_ix_i=\sum_{j=1}^{k}b_jy_j &                                            \\
 				\Rightarrow 0 =                        & \sum_{i\in I}b_ix_i - \sum_{j=1}^{k}b_jy_j \\
 				\Rightarrow                            &
 				\begin{cases}
 				\forall j = 1, ... ,k:y_j=0\\
 				\forall i\in I: x_i = 0,
 				\end{cases}
 				\text{da $(b_j)_{j\in J}$ lin. unabh.}
 			\end{align*}

 			Insbesondere gilt also $\forall i\in I: x_i = 0$; damit folgt die lineare Unabhängigkeit nach Lemma.
 		\end{addmargin}

 		Erzeugendensystem:
 		\begin{addmargin}[25pt]{0pt}
 			Sei $w\in f(V)$, also existiert $v\in V$ mit $w = f(v)$. Da $(b_j)_{j\in J}$ Basis von $V$ ist, existiert eine Familie $(x_j)_{j\in J}$ in $K$ so, dass
 			\begin{equation*}
 				v = \sum_{j\in J} b_jx_j
 			\end{equation*}

 			Dann gilt
 			\begin{align*}
 				w & = f(v) = f\Big(\sum_{j\in J} b_jx_j\Big) = \sum_{j\in J}f(b_jx_j)                                                                  \\
 				  & = \sum_{j=1}^{k}\underbrace{f(b_j)}_{=0}x_j + \sum_{i\in I}\underbrace{f(b_i)}_{=c_i}x_i = \sum_{i\in I}c_ix_i\in[(c_i)_{i\in I}].
 			\end{align*}
 		\end{addmargin}

 		Da $(c_i)_{i\in I}$ also Basis von $f(V)$ ist folgt:
 		\begin{enumerate}[1.{ Fall:}]
 			\item $(\dim V = n<\infty)$ dann ist $\# J = n$ und $\# I = \# J-k$, also $\rg f = n-k = \dim V - \dfkt f.$
 			\item $(\dim V = \infty)$, dann ist $\# J = \infty $ und damit auch $\#I =\#(J\setminus \{{1,...,k\}})=\infty $, also $\rg f= \infty$.
 		\end{enumerate}

 	\paragraph{Bemerkung}
 		Die Annahme $\dfkt f = k<\infty$, im Beweis ist keine Einschränkung:
 		\begin{enumerate}
 			\item ist $\dim V < \infty$, so folgt $\dfkt f<\infty$, da $\ker f\subset V$ Untervektorraum ist;
 			\item ist $\dim V = \infty$, so ist man mit dem Beweis fertig, falls $\dfkt f = \infty$.
 		\end{enumerate}

 	\paragraph{Korollar (Homomorphismen zwischen gleichdimensionalen VR)}
 		\begin{Korollar}[Homomorphismen zwischen gleichdimensionalen VR]
 			Sei $f\in \hom(V,W)$ und $\dim W = \dim V = n<\infty$.
 			Dann gilt: Ist $f$ injektiv oder surjektiv, so ist $f$ bijektiv.
 		\end{Korollar}

 	\paragraph{Beweis}
 		Der Rangsatz liefert:
 		\begin{align*}
 			f \text{ injektiv} \Leftrightarrow \ker f = \{{0\}} & \Rightarrow \dfkt f = 0                                                            \\
 			                                                    & \Rightarrow \rg f = \dim V-0 = \dim V = \dim W                                     \\
 			                                                    & \stackrel{\footnotemark}{\Rightarrow} f(V) = W \Leftrightarrow f \text{ surjektiv}
 		\end{align*}
 		\footnotetext{$f(V)\subset W$ UVR und $\dim f(V) = \dim W \Rightarrow f(V)=W$}
 		\begin{align*}
 			f(V) = W & \Rightarrow \rg f = \dim W = \dim V   \\
 			         & \Rightarrow \dfkt f= \dim V - \rg f=0 \\
 			         & \Rightarrow \ker f = \{{0}\}
 		\end{align*}

 	\paragraph{Beispiel}
 		Der \emph{Shiftoperator} für Folgen $(x_i)_{i\in \mathbb{N}}$ in $K$, $s: K^{\mathbb{N}} \to K^{\mathbb{N}}: (x_i)_{i\in \mathbb{N}} \mapsto (y_i)_{i\in \mathbb{N}}$ wobei
 		\begin{equation*}
 			y_i :=
 			\begin{cases}
 				0       & \text{ für } i = 0    \\
 				x_{i-1} & \text{ für } i \neq 0
 			\end{cases}
 		\end{equation*}
 		ist ein injektiver Homomorphismus, $s\in \hom(K^\mathbb{N},K^\mathbb{N})$ von $K^\mathbb{N}$ in sich (damit gilt $\dim $ Definitionsbereich $= \dim K^\mathbb{N}= \dim$ Wertebereich). Aber $s$ ist nicht surjektiv, also auch nicht bijektiv.\\
 		Damit folgt $\dim K^\mathbb{N} =\infty$ (sonst hätte man einen Widerspruch zum Korollar).

 \subsection{Definition (Spezielle Homomorphismen)}
 	\begin{Definition}[Spezielle Homomorphismen]
 		Sei $f\in \hom(V,W)$ ein Homomorphismus, dann heißt $f$
 		\begin{itemize}
 			\item \emph{Endomorphismus}, $f\in \End (V)$, falls $W = V$;
 			\item \emph{Isomorphismus}, $f\in \Iso(V,W)$, falls $f$ bijektiv ist;
 			\item \emph{Automorphismus}, $f\in \Aut(V)$, falls $W=V$ und $f$ bijektiv ist.
 		\end{itemize}

 		Zwei $K$-VR $V$ und $W$ heißen \emph{isomorph} $W \cong V$, falls $\Iso(V,W) \neq \emptyset$.
 	\end{Definition}

 	\paragraph{Bemerkung}
 		Ein Isomorphismus $f\in \Iso(V,W)$ bildet jede Basis $B$ von $V$ auf eine Basis $C = f(B)$ von $W$ ab.

 		Andererseits: Bildet eine lineare Abbildung $f\in \hom(V,W)$, eine Basis $B$ von $V$ auf eine Basis $C = f(B)$ von $W$ ab, so ist $f$ ein Isomorphismus.

 		Nämlich: Ist $B$ Basis von $V$ und $ C = f(B)$ Erzeugendensystem, so ist $f$ surjektiv, da
 		\[
 			f(V) = f ([B]) = [f(B)] = [C] = W;
 		\]
 		Ist $C =f(B)$ linear unabhängig, so ist $f$ injektiv, denn für
 		\begin{align*}
 			  & v = \sum_{b\in B} bx_b \in \ker f                                            \\
 			  & \Rightarrow 0 = f(v) = f\Big(\sum_{b\in B}bx_b\Big) = \sum_{b\in B}f(b)x_b   \\
 			  & \Rightarrow \forall b \in B: x_b = 0                                         \\
 			  & \Rightarrow v = 0, \text{ d.h., } \ker f=\{{0}\} \Rightarrow f \text{ inj.}.
 		\end{align*}

 \subsection{Isomorphielemma}
 	\begin{Lemma}[Isomorphielemma]
 		Seien $V$ und $W$ $ K $-VR mit $\dim V, \dim W < \infty$.
 		Dann gilt:
 		\[
 			V \cong W \Leftrightarrow \dim V = \dim W
 		\]
 	\end{Lemma}

 	\paragraph{Beweis}
 		Folgt aus obiger Bemerkung:

 		"$\Rightarrow$":

 		Annahme: $V \cong W$; sei $f\in \Iso(V,W)\neq \emptyset$.
 		Wähle eine Basis $B = (b_1, \dots, b_n)$ von $V$ (BES); da $f$ bijektiv, ist dann
 		\[
 			C = f(B) = (f(b_i), \dots , f(b_n))
 		\]
 		eine Basis von W, womit $\dim W = n = \dim V$.

 		"$\Leftarrow$":

 		Sei $\dim W = \dim V = n$.
 		Wähle Basen $B = (b_1, \dots ,b_n)$ von $V$ und $C = (c_1, \dots ,c_n)$ von $W$ (BES und Basissatz) und definiere $f\in \hom(V,W)$ durch (Fortsetzungssatz):
 		\[
 			\forall i = 1, ... ,n : f(b_i) = c_i
 		\]
 		Da $f$ eine Basis auf eine Basis abbildet ist $f\in \Iso(V,W)$.
 		Damit folgt also $\Iso(V,W) \neq \emptyset \Rightarrow V \cong W$.

 	\paragraph{Beispiel: }
 		Ist $V$ $K$-VR mit $\dim V < \infty$, so ist $V^\ast \cong V$. (Achtung: Es gibt aber viele Isomorphismen, keiner ist besonders d.h., \emph{kanonisch}.)

 	\paragraph{Bemerkung: }
 		Ist $f\in \Iso(V,W)$, so ist $f^{-1}\in \Iso(W,V)$, denn
 		\begin{gather*}
 			(f\circ f^{-1})\Big(\sum_{i\in I}v_ix_i\Big) = f\Big(\sum_{i\in I}f^{-1}(v_i)x_i\Big) = \sum_{i\in I}(f\circ f^{-1})(v_i)x_i\\
 			\Rightarrow f^{-1}\Big(\sum_{i\in I}v_ix_i\Big) = \sum_{i\in I}f^{-1}(v_i)x_i.
 		\end{gather*}

%VO_10.11.15
\section{Summen, Produkte und Quotienten}
\paragraph{Definition: }
	Die Summe einer Familie $ (U_i)_{i\in I} $ von UVR $ U\subset V $ eines $ K $-VR ist die Menge
		\begin{equation*}
		\sum_{i\in I} U_i := \{\sum_{i \in I}u_i\mid \forall i\in I: u_i\in U_i \land \# \{i\in I\mid u_i \neq 0\}<\infty\}.
		\end{equation*}
		
\paragraph{Bemerkung: }
	Offenbar ist $ \sum_{i\in I} U_i\subset V $ UVR mit
	\[  \bigcup_{i\in I}U_i \subset \sum_{i\in I} U_i \Rightarrow [\bigcup_{i\in I}U_i]\subset \sum_{i\in I} U_i; \]
	andererseits gilt:
	\[ \sum_{i\in I}U_i \subset \{\sum_{j\in J}v_jx_j\mid \forall j\in Jv_j\in \bigcup_{i\in I}U_i \land \#\{j\in J\mid x_j\neq 0\}<\infty\}\subset [\bigcup_{i\in I}U_i]. \]
	
	Damit ist die Summe einer Familie $ (U_i)_{i\in I} $ gerade die lineare Hülle ihrer Vereinigung $ \bigcup_{i\in I}U_i $,
	\[ \sum_{i\in I}U_i= [\bigcup_{i\in I}U_i]. \]
		
\paragraph{Beispiel: }
	Sei $ V=\mathbb{R}^\mathbb{N} $ der Raum der reellen Folgen. Für $ n\in \mathbb{N} $ setze
		\begin{equation*}
		U_n := \{v\in \mathbb{R}^\mathbb{N}\mid \forall j\in \mathbb{N}: j>n\Rightarrow v_j = 0 \} \subset \mathbb{R}^\mathbb{N};
		\end{equation*}
		
	dann gilt $ \forall n\in \mathbb{N}: U_n\subset U_{n+1} $, und damit auch
		\begin{gather*}
		\sum_{i\leq n} U_i = U_n = \bigcup_{i\leq n}U_i \text{, aber}\\
		\sum_{i\in \mathbb{N}}U_i = \bigcup_{i\in \mathbb{N}}U_i \neq V.
		\end{gather*}
		
	Nun setze für $ i\in \{0,1\} $
		\begin{equation*}
		\tilde{U}_i := \{v\in \mathbb{R}^\mathbb{N}\mid \forall j\in \mathbb{N}: j=i\operatorname{mod} 2\Rightarrow v_j = 0\}
		\end{equation*}
		
	dann ist 
		\begin{equation*}
		\bigcup_{i\in \{0,1\}}\tilde{U}_i \neq \sum_{i\in \{0,1\}}\tilde{U}_i = V.
		\end{equation*}
\paragraph{Dimensionssatz: }
	Sind $ U_i \subset V $ UVR mit $ \dim U_i < \infty $ für $ i\in \{1,2\} $, so ist
		\begin{equation*}
		\dim (U_1+U_2) + \dim (U_1\cap U_2) = \dim U_1 + \dim U_2.
		\end{equation*}
		
	Ist $ \dim U_1 = \infty$ oder $ \dim U_2=\infty $, so ist auch $ \dim (U_1+U_2)=\infty $.
\paragraph{Beweis: }
	Seien
		\begin{itemize}
		\item $ B_0 \subset U_1\cap U_2 $ eine Basis von $ U_0 := U_1\cap U_2 $;
		\item $ S_i \subset U_i $ lin. unabh., sodass $ B_i = B_0 \cup S_i $ Basen von $ U_i $ sind ($ i = 1,2; $ BES).
		\end{itemize}
	
	Offenbar gilt dann, da $ B_i = B_0\cup S_i $ lin. unabh. sind,
		\begin{equation*}
		B_0\cap S_1 = \emptyset \text{ und } B_0\cap S_2 = \emptyset
		\end{equation*}
	und 
		\begin{equation*}
		S_1\cap S_2 \subset U_1\cap U_2 = [B_0] \Rightarrow S_1\cap S_2 = \emptyset.
		\end{equation*}
		
	Wir zeigen, dass $ B:= B_0\cup S_1\cup S_2 $ Basis von $ U_1 + U_2 =: U $ ist.
	
	$ B\subset U $ ist Erz. Syst. nach Konstruktion:
		\begin{gather*}
		\forall i\in \{1,2\} : U_i=[B_i]\subset [B]\\
		\Rightarrow U_1+U_2 = [U_1\cup U_2]\subset [B]
		\end{gather*}
	
	$ B $ ist linear unabhängig: Gegeben sei eine Linearkombination von $ 0\in U $,
		\begin{gather*}
		0 = \sum_{b\in B}bx_b = \sum_{b\in B_0}bx_b + \sum_{b\in S_1}bx_b + \sum_{b\in S_2}bx_b =: b_0 + s_1+ s_2\\
		\text{mit } b_0\in [B_0] = U_0 \text{ und } s_i\in [S_i] \text{ für } i= 1,2;
		\end{gather*}
		
	dann gilt etwa, $ B_1 = B_0 \cup S_1 $ lin. unabh.,
		\begin{equation*}
		b_0+s_1 = -s_2 \in U_1\cap [S_2]\subset U_0 \Rightarrow s_1 = 0
		\end{equation*}
		
	und damit, da $ B_2 = B_0 \cup S_2 $ lin. unabh. ist,
		\begin{equation*}
		0 = b_0 + s_1 + s_2 \Rightarrow b_0=s_2 = 0.
		\end{equation*}
	
	Mit der linearen Unabhängigkeit von $ B_0, S_1 $ und $ S_2 $ folgt dann
		\begin{equation*}
		0 = \sum_{b\in B_0}bx_b = \sum_{b\in S_1}bx_b = \sum_{b\in S_2}bx_b \Rightarrow \forall b\in B: x_b = 0.
		\end{equation*}
	
	Mit
		\begin{gather*}
		\#B + \#B_0 = (\#B_0 + \#S_1 + \#S_2) + \#B_0\\
		= (\#B_0+\#S_1)+(\#B_0 + \#S_2) = \#B_1+\#B_2
		\end{gather*}
	
	folgt dann die Behauptung.

\paragraph{Bemerkung: }
	Im Beweis haben wir benutzt: 
	
	Ist z.B. $ B_1 = B_0\cup S_1 $ lin. unabh., und $ b_0\in [B_0] $ und $ s_1\in [S_1] $ mit $ b_0 + s_1 = 0 $, so folgt $ b_0 = s_1 = 0 $:
	sind nämlich $ b_0 = \sum_{b\in B_0} bx_b $ und $ s_1 = \sum_{b\in S_1}bx_b $, so gilt 
		\begin{equation*}
		0 = b_0+s_1 = \sum_{b\in B_0} bx_b+\sum_{b\in S_1} bx_b = \sum_{b\in B_1} bx_b \Rightarrow \forall b\in B_1: x_b = 0\Rightarrow b_0 = s_1 = 0.
		\end{equation*}

\paragraph{Bemerkung: }
	Ist $ U_1\cap U_2 = \{o\} $ bzw. $ \dim (U_1\cap U_2)  = 0 $, so zeigt der Beweis auch:
		\begin{equation*}
		\forall v\in U_1+U_2\exists ! u_1 \in U_1\exists ! u_2\in U_2: v= u_1+u_2
		\end{equation*}

\paragraph{Definition: }
	Zwei UVR $ U_1,U_2 \subset V $ heißen komplementär in $ V $, falls
		\begin{equation*}
		U_1+U_2=V\text{ und } U_1\cap U_2 = \{o\}.
		\end{equation*}
		
\paragraph{Lemma: }
	Zu jedem UVR $ U\subset V $ existiert ein (in $ V $) komplementärer UVR.
	
\paragraph{Beweis: }
	Sei $ U\subset V $ UVR eines $ K $-VR $ V $.
	Seien 
		\begin{itemize}
		\item $ B\subset U $ eine Basis von $ U $;
		\item $ S\subset V $ lin. unahb., sodass $ C=B\cup S $ Basis von $ V $ ist (BES).
		\end{itemize}
	
	Definiere $ U':= [S] $. Dann ist $ U'\subset V $ UVR mit
		\begin{enumerate}[(i)]
		% TODO: wir zeigen, dass U+U' übermenge ist, wollen aber Gleichheit zeigen ... wo ist die andere inklusion?
		\item $ U+U' \supset [C] = V $, da $ C\subset U\cup U' $ Erz. Syst. von $ V $ ist;
		\item $ U\cap U' = [B]\cap [S] = \{o\}$, da $ C=B\cup S $ linear unabhängig ist.
		\end{enumerate}
\paragraph{Bemerkung: }
	Zu einem UVR $ U\subset V $ gibt es normalerweise viele komplementäre UVR $ U'\subset V $.
	Z.B.: Zu
		\begin{equation*}
		U:= \{v\in K^2\mid v_2 = 0\}
		\end{equation*}
	
	ist jeder UVR $ U' = [u']$ mit $u'_2\neq 0 $ komplementär in $ K^2 $.
	
\paragraph{Lemma \& Definition: }
	Sei $ U= \sum_{i\in I}U_i\subset V $ Summe einer Familie von UVR $ U_i\in V $; dann besitzt jeder Vektor $ u\in U $ eine eindeutige Zerlegung als Summe von $ u_i $, genau dann, wenn
		\begin{equation*}
		\forall i\in I: U_i\cap \sum_{j\in I\setminus \{i\}}U_j = \{o\}.
		\end{equation*}
		
	In diesem Falle heißt die Summe "`direkt"' und man schreibt
	\[ U = \bigoplus_{i\in I} U_i. \]
		
\paragraph{Bemerkung: }
	Eine Summe $ V = \sum_{i\in I} U_i $ ist genau dann direkt, wenn
		\begin{equation*}
		\forall i\in I: U_i, \sum_{j\in I\setminus\{i\}}U_j \subset V
		\end{equation*}
		
	komplementäre UVR in $ V $ sind.

\paragraph{Beweis: }
	Zu zeigen ist die Eindeutigkeitsaussage. Sei also $ u \in \bigoplus_{i\in I}U_i $,
		\begin{gather*}
		u = \sum_{i\in I} u_i = \sum_{i\in I} u_i' \text{ mit } \forall i\in I: n_i,n'_i\in U_i;
		\end{gather*}
	dann gilt für jedes $ i\in I$:
		\begin{equation*}
		u_i-u'_i = \sum _{j\neq i}u_j-\sum_{j\neq i} u'_j = \sum_{j\neq i}u_j-u'_j \in U_i\cap \sum_{j\neq i} U_j = \{o\},
		\end{equation*}
	da die Summe als direkt angenommen wurde; damit folgt $ \forall i \in I: u_i = u'_i $, d.h. die Zerlegung ist eindeutig.
	
	Die Umkehrung ist trivial:
		\begin{gather*}
		\exists i\in I:U_i\cap \sum_{j\neq i} U_j \neq \{o\} \Rightarrow \exists i\in I\exists u_i\in U_i\setminus\{o\}\exists (u_j)_{j\in I\setminus\{i\}}:\\
		(\forall j\in I\setminus\{i\}:u_j \in U_j)\land u_i = \sum_{j\neq i} u_j,
		\end{gather*}
		
	d.h., die Zerlegung von $ u_i\in \sum_{i\in I}U_i $ ist nicht eindeutig.

\paragraph{Bemerkung: }
	Sind $ \dim V <\infty $ und $ \# I < \infty $ so gilt
		\begin{equation*}
		\forall i\in I: \dim U_i < \infty
		\end{equation*}
		
	und es gilt die Dimensionsformel für direkte Summen (Beweis in Aufgabe 35):
		\begin{equation*}
		\dim \bigoplus_{i\in I}U_i = \sum_{i\in I} \dim U_i.
		\end{equation*}
	
	Ist insbesondere $ B=(b_1,...,b_n) $ eine Basis von $ V $, so gilt
		\begin{equation*}
		\dim V = \dim \bigoplus_{i=1}^n [b_i]=\sum_{i=1}^{n}1 = n.
		\end{equation*}
\paragraph{Bemerkung: }
	Seien $ U,U'\subset V $ komplementäre UVR, also $ V = U \oplus U' $, dann werden durch
		\begin{equation*}
		v = u+u' \mapsto
			\begin{cases}
			p(v):=u\\
			p':= u'
			\end{cases}
		\end{equation*}
	
	Endomorphismen $ p,p'\in \operatorname{End}(V) $ (wohl-)definiert, da $ n,n' $ durch $ v $ eindeutig bestimmt sind (Linearität von $ p,p' $ ist klar).
	Offenbar ist 
		\[ p(V) = U \text{ und } \ker p = U'\]
	und es gilt
		\[ p^2 := p\circ p = p \]
	und analog für $ p' $; außerdem gilt ($ \circ $ ausgelassen)
		\[ p+p' = id_V \text{ und } p'p = 0 = pp'.\]
		
\paragraph{Definition: }
	$ p\in \operatorname{End}(V) $ heißt Projektion, falls $ p^2 = p $ (d.h. falls $ p $ idempotent ist).
	
\paragraph{Satz: }
	Sei $ p\in \operatorname{End}(V) $ Projektion, dann ist $ p'= id_V-p $ Projektion mit $ pp' = p'p = 0 $. Gilt $ p+p' = id_V $ und $ pp' = 0 $ für $ p,p' \in \operatorname{End}(V) $, so sind $ p,p' $ Projektionen mit
		\[ V = p(V)\oplus p'(V) = \ker p \oplus \ker p'. \]
		
\paragraph{Beweis: }
	Seien $p\in \operatorname{End}(V)$ Projektion und $p' := id_V -p$; dann gilt:
		\[p\circ p' = p(id_V-p)=p-p^{2} = 0; (p^{2} = p\circ p)\]
		\[p\circ p = (id_V-p)\circ p = p - p^{2} = 0\]
	und
		\[p' \circ p' = p^{1,2}=p' \circ(id_V-p)=p'-p'\circ p = p^{\prime\prime},\]
	d.h., $p'\in\operatorname{End}(V)$ ist Projektion.
	Anderseits: Seien $p,p' = id_V \text{ und } p' p = 0$.
		
	Dann gilt:
		\[p-p^2 = p(id_V-p) = pp' = 0\]
	d.h., $p\in\operatorname{End}(V)$ ist Projektion, damit ist auch $p'$ Projektion (erster Teil)

	und $p' p = 0 $. Weiters liefert
		\[\forall v\in V: v=id_v(v) = p(v) + p'(v) \Rightarrow V = p(V)+p'(V),\]

	und ist $w = p(V)= p'(v')$ für geeignete $v,v'\in V$ (d.h., $w  = p(V)\cap p'(V))$ so gilt
		\[w = p(v) = p^2(v) = p(p(v)) = p(w) = p(p'(v')) = pp'(v') = 0 (pp' = 0),\]
		
	also $p(V)\cap p'(V) = {0}$ und damit $V = p(V)\oplus p'(V)$. Weiters gilt
		\[0 = p \circ p' \rightarrow p'(V)\subset \ker p\]
	und ist $v\in \ker p$, so folgt
		\[v = p(v) + p'(v) = 0 + p'(v)\in p'(V) \Rightarrow \ker p \subset p'(V).\]
		
	Für $p'$ gilt das Gleiche und wir haben $\ker p = p'(V)$ und $\ker p' = p(V)$. Damit folgt die letzte Behauptung 
		\[V = \ker p \oplus\ker p'.\]
	
\paragraph{Bemerkung: }
		Im Beweis haben wir etwas mehr bewiesen als behauptet - nämlich:
			\[ ker p = p'(V)\text{ und }\ker p' = p(V) \]
		
\paragraph{Beispiel und Definition: }	
		Sei $s\in \operatorname{End}(V)$ eine Involution d.h., 
			\[s^2 = id_v\] 
		und 
			\[p\pm := 1/2(id_v\pm s).\]
		Offenbar gilt dann
			\[p_{+} + p_{-} = id_V\] 
		und 
			\[p + p_{-} = 1/4(id_V +s)(id_V -s) = 1/4(id_v^2-s^2)=0\]
		also (Satz) sind $p_\pm\in\operatorname{End}(V)$ Projektionen mit komplementären Bildern bzw. Kernen.
		
\paragraph{Lemma und Definition: }
		Ist $(V_i)_{i\in I}$ eine Familie von K-VR $V_i$, so wird das (mengenthoretische) Produkt:
			\[V:= \prod_{i\in I}V_i=\{(v_i)_{i\in I}\mid\forall i\in I:v_i\in V_i\}\]
		mit den komponentenweise definierten VR-Operationen zu einem $ K $-VR. Dies ist der Produktraum der Familie	$(V_i)_{i\in I}$.
		
\paragraph{Beweis: } Aufgabe!

\paragraph{Bemerkung: } 
		Ist $V = \prod_{i\in I} V_i$ ein Produktraum, so erhält man einen kanonischen UVR
			\[U_i:=\{v=(v_i)_{i\in I}\in V\mid\forall j = i:v_j = 0\}\subset V,\]
		die isomorph zu $\dim(V_i)$ sind der Faktorprojektionen,
			\[\prod_i:V\to V_i:(v_j)_{j\in I} \mapsto v_i,\]
		mittels Faktor-Injektionen
			\[\iota_i:V_i\to V: v_i \mapsto(v_j)_{j\in I},\]
		wobei
			\[v_j :=
				\begin{cases}
					v_i \text{ falls } j=i\\
					0 \text{ sonst}
				\end{cases}
			\]
		Ist dann $\# I < \infty$, so erhält man
			\[\prod_{i\in I} V_i\cong \oplus_{i\in I}U_i (=: \oplus_{i\in I}V_i);\]
		ist $\#I=\infty$, so ist diese Identifikation im Allgemeinen falsch!
	
\paragraph{Beispiel: }
	Für einen Körper K liefert das n-fache Produkt den Standardraum
		\[\prod_{i=1}^{n}K = \{(x_i)_{i = 1,...,n}\mid\forall i \in \{1,...,n\}: x_i \in K\} = K^n \cong \oplus_{i=1}^n\{(x_i)_{i\in {1,...,n}}\mid\forall j\neq i: x_j = 0\};\]
	für den Raum der K-wertigen Folgen ist jedoch
		\[\prod_{i\in \mathbb{N}}K=K^{\mathbb{N}}\neq\oplus_{i\in \mathbb{N}}\{(x_i)_{i\in \mathbb{N}}\mid\forall j\neq i: x_j=0\}.\]
			
\paragraph{Lemma und Definition: }
	Sei $U\subset V$ UVR. Die Menge der Nebenklassen 
		\[V/U := \{v+U\mid v\in V\},\]
	wobei
		\[v+U:=\{v+u\mid u\in U\}\]
	die Nebenklasse zu $v\in V$ bezeichnet, wird mit den durch
		\[(v+U)+(w*U):=(v+w)+U\]
	und
		\[(V+U)x := Vx + U\]
	definierten Operationen ein Vektorraum: der Quotientenraum $V/u$.
			
\paragraph{Beweisen: }
	Zu zeigen: Wohldefiniertheit der Operationen und VR-Axiome. Wohldefiniertheit der Skalarmultiplikation: Ist $x\in K$ und sind $(v+U),(v'+U)\in V/U$ gleich, $v+U = v' + U$, so gilt
		\[v+U = v'+U \Leftrightarrow v = v'\in U\] \[\Rightarrow (v-v')x \in U\]
		\[\Rightarrow vx+U=v' x+U\]
	Das Resultat der Skalarmultiplikation bringt also nicht von dem "Repräsentanten" $v$ einer Nebenklasse $v+U$ ab, sondern nur von der Nebenklasse.
	
	Die Wohldefiniertheit der Addition ist analog.
	VR-Axiome = langweilig, nachrechnen.

% % 2015-11-19 % %
\chapter{Affine Geometrie}
\begin{tikzpicture}[scale=1.5,>=triangle 45]
	\draw[->,color=black] (-0.1,0) -- (10,0);
	\draw[->,color=black] (0,-0.1) -- (0.,4);
	
	\coordinate[label=left:$x$] (x) at (1,1);
	\coordinate[label=below:$\tau_v(x)$] (y) at (5,1.5);
	\coordinate[label=above:$\tau_w(x)$] (y') at (2,2.5);
	\coordinate (z) at (6,3);
	
	\draw [fill] (x) circle (.5pt);
	\draw [fill] (y') circle (.5pt);
	\draw [fill] (y) circle (.5pt);
	\draw [fill] (z) circle (.5pt);
	
	\draw [->] (x) to node[below left]{$ v $} (y);
	\draw [->] (x) --node[above left]{$ w $} (y');
	\draw [->] (y) --node[below right]{$ w $} (z);
	\draw [->] (y') --node[above right]{$ v $} (z);
	
	\draw (z) node[above right] {$\tau_w(\tau_v(x))=(\tau_w\circ\tau_v)(x)=\tau_{w+v}(x)$};
	\draw (z) node[below right] {$\tau_v(\tau_w(x))=(\tau_v\circ\tau_w)(x) = \tau_{v+w}(x)$};
\end{tikzpicture}
\paragraph{Definition (nach Klein): }
	Eine Geometrie besteht aus einer Menge $ A $ (z.B. Punktmenge) und einer darauf operierenden Gruppe $ (G,*) $, d.h.,
	es gibt eine Gruppenoperation
		\[ \rho: G\times A\to A,(g,a)\mapsto \rho_g(a)  \]
	wobei gilt
		\begin{enumerate}[(i)]
			\item $ \forall a\in A\forall g,h,\in G:(\rho_g\circ \rho_h)(a) = \rho_{g*h}(a) $
			\item $ \forall a\in A:\rho_e(a) = a $ für das neutrale Element $ e \in G $
		\end{enumerate}

\paragraph{Definition: }
	Sei $ K $ ein Körper. Ein affiner Raum (AR) $ (A,V,\tau) $ über $ K $ besteht aus einer Menge $ A $, einem $ K $-Vektorraum $ V $ und einer Gruppenoperation
		\[ \tau:V\times A\to A,(v,a)\mapsto \tau_v(a) \]
	von $ V $ (als additive Gruppe $ (V,+) $) auf $ A $, die einfach transitiv ist, d.h.
		\[ \forall a,b\in A\exists!v\in V:b=\tau_v(a) \]
\begin{figure}\centering
\begin{tikzpicture}[scale=1.5,>=triangle 45]
	\draw[->,color=black] (-0.1,0) -- (5,0);
	\draw[->,color=black] (0,-0.1) -- (0.,2);
	
	\coordinate[label=left:$a$] (x) at (1,0.5);
	\coordinate[label=right:${b=\tau_v(a)}$] (y) at (3,1.5);

	\draw [fill] (x) circle (.5pt);
	\draw [fill] (y) circle (.5pt);
	
	\draw [->] (x) to node[below]{$ v $} (y);
	\draw (5,0.5) node[] {Der Verbindungsvektor ist eindeutig.};
	
\end{tikzpicture}
\end{figure}

	Weiters nennen wir
		\begin{itemize}
			\item Elemente von $ A $ Punkte,
			\item $ V $ den Richtungsvektorraum oder Tangentialraum von $ A $,
			\item $ v $ mit $ \tau_v(a)=b $ den Verbindungsvektor von $ a $ nach $ b $,
			\item $ \tau_v:A\to A, a\mapsto \tau_v(a) $ die Translation von $ v $
			\item und $ \dim V $ die Dimension des affinen Raums $ A $
		\end{itemize}
		
\subparagraph{Bemerkung: }
	Die Translationen eines AR $ A $ bilden eine abelsche Gruppe.
	
	Alternative Notation:
		\[ a+v:=\tau_v(a) \text{ und } b-a:= v\text{, falls } b=\tau_v(a) \]
	Mit dieser alternativen Schreibweise für die Operation von $ (V,+) $ auf $ A $, erscheinen die Bedingungen, dass $ V=(V,+) $ einfach transitiv auf $ A $ operiert, "`offensichtlich"'.
	
	Gruppenoperation:
		\begin{enumerate}[(i)]
			\item $ \forall a\in A\forall v,w,\in V: (a+v)+w = a+(v+w) $ ist kurz für $ \tau_w(\tau_v(a)) = \tau_{v+w}(a) $, entspricht also nicht der Assoziativität.
			\item $ \forall a\in A:a+0=a $ entspricht $ \tau_0(a) = a $
		\end{enumerate}
	Transitivität:
		\[ \forall a,b\in A\exists v\in V: b=a+v \]
	Nämlich: sind $ a,b\in A $ gegeben, so liefert $ v:=b-a $ (weil $ V $ einfach transitiv operiert) eindeutig den gesuchten Vektor.

\paragraph{Beispiel \& Definition: }
	Jeder $ K $-VR liefert einen affinen Raum $ (V,V,\tau) $ mit der Operation
		\[ \tau: V\times V\to V, (v,a)\mapsto \tau_v(a):= a+v \]
	von $ V $ auf sich selbst -- die Unterscheidung zwischen Punkten und Vektoren wird dann etwas undurchsichtig.
	
	Der affine Standardraum $ (K^n,K^n,\tau) $ wird mit $ A^n $ bezeichnet.
\paragraph{Beispiel \& Definition}
	Sei $ (A,V,\tau) $ AR, für jede Wahl eines Ursprungs $ o\in A $ ist
		\[ \tau(o) :V\to A,v\mapsto \tau_v(o) \]
	eine Bijektion -- ein VR ist also ein "`AR mit Ursprung"'.
	
\subparagraph{Beispiel: }
	Auf einem Zylinder
	\[ Z^2 := S^1\times \mathbb{R}:= (\mathbb{R}/2\pi\mathbb{Z})\times \mathbb{R} \]
	liefert die Operation
		\[ \tau:\mathbb{R}^2\times Z^2\to Z^2,(v,a)\mapsto a+v \]
	keinen affinen Raum, da diese Operation nicht einfach transitiv ist: zu je zwei Punkten gibt es unendlich viele "`Verbindungsvektoren"'.
\paragraph{Beispiel \& Definition: }
	Ist $ U\subset V $ UVR eines $ K $-VR $ V $, so liefert jedes $ v\in V $ die Nebenklasse
		\[ A = v+U \]
	einen affinen Raum $ (A,U,\tau) $ mit 
		\[ \tau:U\times A\to A,(u,a)\mapsto \tau_u(a):= a+u; \]
	offensichtlich ist die Operation wohldefiniert (operiert auf der Nebenklasse) und einfach transitiv.
	
	Eine Nebenklasse $ A= v+U\subset V $ nennt man daher auch einen affinen Unterraum des VR $ V $.
\paragraph{Definition: }
	$ A'\subset A $ ist affiner Unterraum (AUR) des affinen Raumes $ (A,V,\tau) $, falls
		\[ \exists a\in A\exists U\subset V \text{UVR}:A' = a+U = \{\tau_u(a)\mid u\in U\}.\]
	Ist $ \dim A' =1 $ oder $ \dim A' = 2 $, so heißt $ A' $ (affine) Gerade bzw. Ebene; ist $ \dim A' < \infty $ und $ \dim A' = \dim A-1 $, so heißt $ A' $ (affine) Hyperebene.
\subparagraph{Bemerkung: }
	Jeder AUR ist selbst AR mit der "`geerbten"' (eingeschränkten) Operation.
\subparagraph{Beispiel: }
	Ist $ f\in \hom(V,W) $ und $ w\in f(V) $, so erhält man einen affinen Raum
		\[ (f^{-1}(\{w\}),\ker f,\tau) \text{ mit }\tau_u(a):= a+u.\]
	Ist $ f\in V^*\setminus \{o\} $ (und $ \dim V<\infty $), so wird $ f^{-1}(\{x\})\subset V $ für jedes $ x\in K (=f(V)) $ eine affine Hyperebene in $ (V,V,\tau) $ -- nach Rangsatz.
% % % 2015-11-26 Teil 2 % % %

\section{Affine Abbildungen \& Transformationen}
\subsection{Definition}
	\begin{Definition}[Affine Abbildung/Affinität]
		Eine Abbildung $ \alpha:A\to A' $ zwischen affinen Räumen $ A $ und $ A' $ (über dem gleichen Körper $ K $) heißt affin, falls sie
			\begin{enumerate}[(i)]
				\item \emph{geradentreu} ist, d.h. die Bilder kollinearer Punkte sind kollinear;
				\item \emph{teilverhältnistreu} ist, d.h. das Teilverhältnis kollinearer Punkte wird erhalten (solange die Punkte nicht alle zusammenfallen).
			\end{enumerate}
		Eine bijektive affine Abbildung $ \alpha:A\to A $ heißt Affinität oder affine Transformation.
	\end{Definition}
	
\paragraph{Bemerkung}
	Sei $ \alpha:A\to A' $ und $ a,b\in A $ sodass $ \alpha(a)\neq \alpha(b) $; insbesondere ist dann auch $ a\neq b $. Ist $ \alpha $ geradentreu, so gilt für jeden Punkt
		\[ c_s = a(1-s)+bs;\ s=(ca:ba), \]
	dass $ c_s\in [\{a,b\}] $, d.h.
		\[ \forall s\in K\exists t\in K:\alpha(c_s) = \alpha(a(1-s)+bs) = \alpha(a)(1-t)+\alpha(b)t \in [\{\alpha(a),\alpha(b)\}] \]
	Ist $ \alpha $ dann auch teilverhältnistreu, so folgt
		\[ \frac{-t}{1-t} = (\alpha(a)\alpha(c_s):\alpha(b)\alpha(c_s)) = (ac_s:bc_s) = \frac{-s}{1-s} \Rightarrow t = s. \]
	Insbesondere bildet $ \alpha $ die Gerade $ [ab] $ dann bijektiv auf die Gerade $ [\alpha(a),\alpha(b)] $ durch die Bildpunkte von $ a $ und $ b $ ab, und 
		\[ \forall s\in K:\alpha(a(1-s)+bs)=\alpha(a)(1-s)+\alpha(b)s. \]
	Enthält die Gerade durch $ a $ und $ b $, $ a\neq b $ keine Punkte, deren Bilder verschieden sind, so wird die Gerade auf einen einzigen Punkt abgebildet -- und die vorherige Gleichung gilt ebenfalls.
	
\paragraph{Beispiel}
	Die Translationen eines affinen Raumes sind Affinitäten, denn für
		\[ c_s = a(1-s)+bs = a + ws, \text{ mit } w:=b-a \]
	gilt, mit Translationsvektor $ v\in V $,
		\[ \tau_v(c_s) = \tau_v(a+ws) = \tau_v(\tau_{ws}(a)) = \tau_{v+ws}(a) = \tau_{ws+v}(a) = \tau_{ws}(\tau_v(a)) =  \tau_v(a) + ws, \]
	insbesondere gilt also
		\[ \tau_v(b) = \tau_v(a)+w \]
	und damit
		\[ \tau_v(c_s) = \tau_v(a)+ws = \tau_v(a)+(\tau_v(b)-\tau_v(a))s = \tau_v(a)(1-s)+\tau_v(b)s.\]
	Also sind $ \tau_v(a),\tau_v(b) $ und $ \tau_v(c_s) $ kollinear und erhalten das Teilverhältnis
		\[ (\tau_v(a)\tau_v(c_s):\tau_v(b)\tau_v(c_s)) = (ac_s:bc_s). \]
		
\subsection{Lemma}
	\begin{Lemma}[]
		$ \alpha:A\to A' $ ist genau dann affin, wenn für jede Affinkombination in $ A $ gilt:
			\[ \alpha(\sum_{i\in I}a_ix_i) = \sum_{i\in I} \alpha(a_i)x_i. \]
	\end{Lemma}
	
\paragraph{Beweis}
	Wir haben schon gesehen: $ \alpha:A\to A' $ ist affin genau dann, wenn
		\[ \forall a,b,\in A\forall x\in K:\alpha(a(1-s)+bs) = \alpha(a)(1-s)+\alpha(b)s \]
	Offenbar ist die vorherige Bemerkung ein Spezialfall des Lemmas. Es bleibt die andere Richtung zu zeigen. Wir benutzen vollständige Induktion über $k = \#\{{i\in I}\mid x_i\neq 0\}<\infty $.
	
\subparagraph{Induktionsanfang}
	Für $ k=1 $ trivial.

\subparagraph{Induktionsannahme}
	Für $ a_1,...,a_k\in A $ und $ x_1,...,x_k \in K^\times$ mit $ \sum_{i=1}^{k}x_i=1 $ gelte
		\[ \alpha(\sum_{i=1}^{k}a_ix_i) = \sum_{i=1}^{k}\alpha(a_i)x_i. \]
	
\subparagraph{Induktionsschluss}
	Seien $ a_1,...,a_{k+1} \in A$ und $ x_1,...,x_{k+1} \in K^\times$ Gewichte, sodass $ \sum_{i=1}^{k+1}x_i = 1 $, o.B.d.A. $ x_{k+1}\neq 1 $; dann gilt
		\[ \alpha(\sum_{i=1}^{k+1}a_ix_i) = \alpha((\sum_{i=1}^{k}a_i\frac{x_i}{1-x_{k+1}})(1-x_{k+1})+a_{k+1}x_{k+1}) \]
		\[ = \alpha(\sum_{i=1}^{k}a_i\frac{x_i}{1-x_{k+1}})(1-x_{k+1})+\alpha(a_k+1)x_{k+1} \]
		\[ = \sum_{i=1}^{k}\alpha(a_i)\frac{x_i}{1-x_{k+1}}(1-x_{k+1})+\alpha(a_{k+1})x_{k+1} \]
		\[ = \sum_{i=1}^{k+1}\alpha(a_i)x_i. \]
	Damit ist die Behauptung für affine Abbildungen $ \alpha $ bewiesen.

\paragraph{Bemerkung}
	Im Beweis wurde benutzt: für Affinkombinationen ist (falls $ x_j \neq 1$)
		\[ \sum_{i\in I}a_ix_i = \left(\sum_{i\neq j}a_i\frac{x_i}{1-x_j}\right)(1-x_j)+a_jx_j \]
\paragraph{Bemerkung}
	Mit der Verträglichkeit affiner Abbildungen mit Affinkombinationen folgt, dass die Inverse $ \alpha^{-1}:A'\to A $ einer bijektiven affinen Abbildung $ \alpha:A\to A' $ ebenfalls affin ist:
		\begin{gather*}
		\alpha\left(\sum_{i\in I} \alpha^{-1}(a_i')x_i\right)=\sum_{i\in I}(\alpha\circ\alpha')(a_i')x_i = \sum_{i\in I}a_i'x_i = \alpha\left(\alpha^{-1}(\sum_{i\in I}a_i'x_i)\right) \\
		\Rightarrow \sum_{i\in I} \alpha^{-1}(a_i')x_i =\alpha^{-1}(\sum_{i\in I}a_i'x_i),
		\end{gather*}
	da die Affinkombination $ \sum_{i\in I}a_i'x_i\in A' $ beliebig war, folgt damit die Behauptung. Insbesondere sind damit auch die Inversen von Affinitäten Affinitäten.
\paragraph{Bemerkung}
	Sind $ \alpha:A \to A' $ und $ \beta:A'\to A'' $ geraden- und teilverhältnistreu, so ist auch
		\[ \beta\circ\alpha:A\to A'' \]
	geraden- und teilverhältnistreu, d.h. die Komposition affiner Abbildungen ist affin. Insbesondere ist damit die Menge $ G $ aller affinen Transformationen eines affinen Raumes $ A $ abgeschlossen unter der Komposition
		\[ \circ: G\times G\to G; \]
	außerdem ist $ G $ abgeschlossen unter Inversenbildung. Damit folgt: $ G $ ist Untergruppe der Permutationsgruppe (der symmetrischen Gruppe) des affinen Raumes $ A $: Diese Gruppe bezeichnet man als \emph{affine Gruppe}.
\subsection{Definition}
	\begin{Definition}[Affine Geometrie]
	Die auf einem affinen Raum $ A $ operierende Gruppe $ G $ der Affinitäten von $ A $ bestimmt eine \emph{affine Geometrie}.
	\end{Definition}
	
\paragraph{Bemerkung}
	Die Verträglichkeit einer affinen Abbildung $ \alpha:A\to A' $ mit Affinkombinationen lässt sich auch mithilfe von Vektoren formulieren (unabhängig von der Wahl des Ursprungs $ o \in A$):
		\begin{gather*}
		v_i = a_i-o \Rightarrow \alpha\left(\sum_{i\in I}a_ix_i\right)=\alpha\left(o+\sum_{i\in I}v_ix_i\right)\\
		\sum_{i\in I}\alpha(a_i)x_i = \sum_{i\in I}\alpha(o+v_i)x_i\\
		\Rightarrow \alpha\left(o+\sum_{i\in I}v_ix_i\right)-\alpha(o) = \sum_{i\in I}\alpha(o+v_i)x_i-\alpha(o)\sum_{i\in I}x_i = \sum_{i\in I}(\alpha(o-v_i)-\alpha(o))x_i,
		\end{gather*}
	setzt man also
		\[ \lambda:V\to V',v\mapsto \lambda(v):= \alpha(o+v)-\alpha(o),  \]
	wobei $ V $ und $ V' $ die zu $ A $ bzw. $ A' $ gehörenden Richtungsvektorräume sind, so erhält man einen Homomorphismus $ \lambda\in \hom(V,V') $, da sie mit Linearkombinationen verträglich ist:
		\[ \lambda\left(\sum_{i\in I}v_ix_i\right)=\alpha\left(o+\sum_{i\in I}v_ix_i\right)-\alpha(o)= \sum_{i\in I}\left(\alpha(o+v_i)-\alpha(o)\right)x_i = \sum_{i\in I}\lambda(v_i)x_i\]
		
\subsection{Lemma \& Definition}
	\begin{Lemma}
	Seien $ A $ und $ A' $ AR mit RVR $ V $ bzw. $ V' $; dann ist eine Abbildung $ \alpha:A\to A' $ genau dann affin, wenn es $ \lambda\in\hom(V,V') $ gibt, sodass 
		\[ \forall a\in A\forall v\in V:\alpha(a+v)=\alpha(a)+\lambda(v). \]
	\end{Lemma}
	\begin{Definition}
	Wir nennen $ \lambda $ den \emph{linearen Anteil} einer affinen Abbildung $ \alpha $.
	\end{Definition}
\paragraph{Beweis}
	Es sind zwei Richtungen zu zeigen:
	
	$ \Rightarrow: $ Sei $ \alpha:A\to A' $ affin. Zu zeigen ist nun die Existenz eines geeigneten $ \lambda \in\hom(V,V') $. Nämlich: Wähle $ o\in A $ und definiere
		\[ \lambda:V\to V',v\mapsto \lambda(v):=\alpha(o+v)-\alpha(o). \]
	Wegen der Verträglichkeit von $ \alpha $ mit Affinkombinationen ist $ \lambda $ linear. Für $ a\in A,v\in V $ gilt dann mit $ w:=a-o $:
		\begin{gather*}
		\alpha(a+v) = \alpha(o+w+v) = \alpha(o)+\lambda(w+v) =\\ \alpha(o)+\lambda(w)+\lambda(v) = \alpha(o+w)+\lambda(v) = \alpha(a)+\lambda(v)
		\end{gather*}
	Insbesondere ist der lineare Anteil $ \lambda \in\hom(V,V')$ von $ \alpha $ wohldefiniert, d.h. unabhängig von der Wahl des Ursprungs.
	
	$ \Leftarrow: $ Für $ \alpha:A\to A' $ gilt mit einem $ \lambda\in\hom(V,V') $
		\[ \forall a\in A\forall v\in V:\alpha(a+v)=\alpha(a)+\lambda(v) \]
	Wegen der Verträglichkeit von $ \lambda $ mit Linearkombinationen ist $ \alpha $ verträglich mit Affinkombinationen (siehe oben) und damit affin.
\paragraph{Bemerkung}
	Jede affine Transformation setzt sich also zusammen aus einer Translation und und einem Automorphismus $ \lambda \in \Aut(V) $. Insbesondere: Ist $ \tau_w:A\to A' $ Translation eines affinen Raumes $ A $ über $ V $, so ist für $ a\in A $ und $ v\in V $
		\[ \tau_w(a+v) = (a+v)+w = a+(v+w) = a+(w+v) = (a+w)+v = \tau_w(a)+v = \tau_w(a)+\id_V(v), \]
		d.h. der lineare Anteil einer Translation ist trivial -- also die Identität auf $ V $.
\subsection{Definition}
	\begin{Definition}[Allgemeine lineare Gruppe]
	Die Automorphismen eines VR $ V $ bilden seine \emph{allgemeine lineare Gruppe}
		\[ Gl(V):= \{\lambda\in \End(V)\mid \lambda \text{ invertierbar}\}. \]
	\end{Definition}
\subsection{Bemerkung \& Definition}
	\begin{Definition}[Parallele Geraden]
	Sind $ g_i = [a_ib_i]=a_i + [v] $ mit $ b_i = a_i + v $ für $ i = 1,2 $ zwei Geraden mit dem gleichen RVR $ [v] $, d.h. \emph{parallel}, so sind auch ihre Bilder unter einer affinen Transformation $ \alpha $ parallele Geraden,
		\[ \alpha(g_i) = \alpha(a_i) + [\lambda(v)] \text{ mit } \lambda\in Gl(V). \]
	\end{Definition}
\subsection{Beispiel \& Definition}
	\begin{Definition}[Streckung]
	Sei $ (A,V,\tau) $ ein AR über $ K $ und $ \lambda\in\End(V) $ eine \emph{Homothetie},\hfill
		$ \lambda= \id_V\cdot c \text{ für ein }c\in K. $
		
	Ist die zugehörige affine Abbildung\hfill
		$ \alpha:A\to A,o+v\mapsto \alpha(o+v):= o+v\cdot c $
		
	eine affine Transformation, d.h.,\hfill
		$ \lambda \in Gl(V)\Leftrightarrow c\in K^\times, $
	
	so nennt man $ \alpha $ eine \emph{Streckung} mit \emph{Zentrum} $ o\in A $. Ist $ c\neq 1 $, d.h. $ \alpha \neq \id_A $, so gilt 
		\[ \alpha(a) = a\Leftrightarrow a = o. \]
	also hat die Abbildung $ \alpha $ genau einen \emph{Fixpunkt} $ a = o $.
	\end{Definition}
\subsection{Beispiel \& Definition}
	\begin{Definition}[Parallelprojektion]
	Sind $ p\in \End(V) $ eine Projektion ($ p^2 = p $) und $ o\in A $, so liefert
		\[ \pi:A\to A,o+v\mapsto \pi(o+v):= o+p(v) \]
	eine \emph{Parallelprojektion} von $ A $ auf dem affinen Unterraum $ o+p(V) $. Ist $ p\neq \id_V $, so ist $ p\notin Gl(V) $ und also $ \pi $ keine affine Transformation (sondern eine nicht bijektive affine Abbildung), so hat $ \pi $ nicht-triviale \emph{Fasern} 
		\[ \pi^{-1}(\{a'\})\subset A \text{ für } a'\in \pi(A), \]
	wobei $ \dim\pi^{-1}(\{a'\}) = \dfkt p \geq 1 $.
	\end{Definition}
\subsection{Beispiel \& Definition}
	\begin{Definition}[Scherung]
	Seien $ \omega\in V^* $ und $ w\in \ker \omega $, sei $ o\in A $; die \emph{Scherung}
		\[ \sigma:A\to A, o+v\mapsto \sigma(o+v):= o+v+w\omega(v) \]
	ist dann eine affine Transformation, denn\hfill
		$ \lambda = \id_V + w\cdot \omega \in Gl(V) $
		
	mit\hfill
		$ \lambda^{-1} = \id_v - w\cdot \omega. $
	
	Ist $ w\cdot\omega\in\End(V)\setminus\{o\} $, so hat $ \sigma $ Fixpunktmenge $ \text{Fix}_\sigma = o+\ker\omega $ und jeder Punkt und sein Bild liegen auf einer zu $ o+[w] $ parallelen Geraden:
		\[ \forall a\in A\setminus \text{Fix}_\sigma : [a,\sigma(a)] \parallel o+[w] \]
	\end{Definition}

%VO_03.12.15
\subsection{Korollar (Fortsetzungssatz für affine Abbildungen)}
	\begin{Korollar}[Fortsetzungssatz für affine Abbildungen]
		Eine affine Abbildung $ \alpha:A\to A' $ ist durch die (beliebige) Angabe der Bilder $ a_i' = \alpha(a_i) $ der Punkte eines baryzentrischen Bezugssystems $ (a_i)_{i\in I} $ von $ A $ eindeutig bestimmt.
	\end{Korollar}
	Beweis ist analog dem des Fortsetzungssatzes für lineare Abbildungen: Mit der Verträglichkeit der gesuchten affinen Abbildung mit Affinkombinationen muss gelten:
		\[ \alpha\left(\sum_{i\in I}a_ix_i\right)=\sum_{i\in I}\alpha(a_i)x_i \text{ für } a =\sum_{i\in I}a_ix_i \text{ mit } \sum_{i\in I}x_i = 1 \]
	Eindeutigkeit folgt, da jeder Punkt $ a\in A $ eine Affindarstellung $ a = \sum_{i\in I}a_ix_i $ besitzt. Existenz von $ \alpha $ folgt aus der Eindeutigkeit der Affindarstellung jedes Punktes $ a\in A $ im baryzentrischen Bezugssystem $ (a_i)_{i\in I} $.
	
	\paragraph{Beispiel}
		% % % % Baryzentrisches Bezugssystem mit affiner Abbildung
		\begin{figure}[H]\centering
			\begin{tikzpicture}[scale=1.5,>=triangle 45]
				\draw[->,color=black] (-0.1,0) -- (5,0);
				\draw[->,color=black] (0,-0.1) -- (0.,2);
					
				\coordinate[label=above:$ a $] (a) at (2.5,1.5);
				\coordinate[label=left:$b $] (b) at (1.5,0.5);
				\coordinate[label=above:$c$] (c) at (3,1);
				
				\coordinate[label=right:$ a' $] (a') at (6,1);
				\coordinate[label=right:$b' $] (b') at (3.5,0.5);
				\coordinate[label=right:$c'$] (c') at (4,1.5);
				
				\draw [->] (a) to node[above right]{$ $} (a');
				\draw [->] (b) to node[above left]{${}$} (b');
				\draw [->] (c) to node[below right]{${ }$} (c');
	
				\draw[fill] (a) circle (0.5pt);
				\draw[fill] (b) circle (0.5pt);
				\draw[fill] (c) circle (0.5pt);
				
				\draw[fill] (a') circle (0.5pt);
				\draw[fill] (b') circle (0.5pt);
				\draw[fill] (c') circle (0.5pt);
			\end{tikzpicture}
		\end{figure}
		
		Gegeben sind die Ecken eines nicht-degenerierten Dreiecks $ a,b,c\in A^2 $ und drei Punkte $ a',b',c'\in A^2 $; es existiert genau eine affine Abbildung $ \alpha:A^2\to A^2 $ mit $ (a,b,c)\mapsto (a',b',c') $. Dieses $ \alpha $ ist genau dann eine affine Transformation von $ A^2 $, wenn das Bilddreieck $ \{a',b',c'\} $ nicht-degeneriert ist, d.h. $ (a',b',c') $ ein baryzentrisches Bezugssystem ist (dann bekommt man die Inverse mittels Fortsetzungssatz durch $ (a',b',c') \overset{\alpha^{-1}}{\mapsto} (a,b,c) $).

%VO_03.12.15
\section{Dreiecke in der Affinen Geometrie}
\subsection{Beispiel \& Definition}
	\begin{Definition}[Mittelpunkt]
	Der (geometrische) Schwerpunkt zweier Punkte $ a,b\in A $ eines affinen Raumes über dem Körper $ K $ ist ihr Mittelpunkt
		\[ s_{a,b} = a\cdot \frac{1}{2}+b\cdot \frac{1}{2}. \]
	\end{Definition}
	
	Dies ist sinnlos, falls $ \Char K = 2 $ ist, was wir also ausschließen müssen.
	Ist etwa $ A $ AR über $ K=\mathbb{Z}_2 $, so enthält jede Gerade genau zwei Punkte,
		\[ \forall a,b\in A: [ab] = 
			\begin{cases}
				\{a,b\},& \text{falls }a\neq b\\
				\{a\},& \text{falls } a=b
			\end{cases} \]
	Für den Rest des Kapitels wird $ \Char K \neq 0 $ ausgeschlossen.
\paragraph{Bemerkung}
	Ist $ K $ ein geordneter Körper, e.g. $ K=\mathbb{R} $, so kann man die \emph{Strecke}
		\[ \overline{ab}:= \{a(1-s)+bs\mid 0\leq s\leq 1\} \]
	zwischen zwei Punkten $ a,b\in A $ definieren. Jeder Punkt $ c\in \overline{ab} $ auf der Strecke liegt \emph{zwischen} ihren \emph{Endpunkten} $ a $ und $ b $; $ s_{ab} $ ist dann auch Mittelpunkt der Strecke $ \overline{ab} $.
	
	Offenbar ist das sinnlos, wenn der Körper $ K $ nicht angeordnet ist.
	
	
\subsection{Schwerpunktsatz}
	\begin{Satz}[Schwerpunktsatz]
	Der Schwerpunkt eines nicht-degenerierten Dreieck $ \{a,b,c\}\subset A $ ist der Schnittpunkt der Seitenhalbierenden, die er im Verhältnis $ -\frac{1}{2} $ teilt.
	\end{Satz}
% % % Grafik Schwerpunktsatz
	\begin{figure}[H]\centering
	\definecolor{uququq}{rgb}{0.25,0.25,0.25}
	\definecolor{zzttqq}{rgb}{0.6,0.2,0}
	\definecolor{qqqqff}{rgb}{0,0,1}
		\begin{tikzpicture}[line cap=round,line join=round,>=triangle 45,scale=1.5] %,x=1.0cm,y=1.0cm]
		\clip(0.71,0.52) rectangle (5.68,4.35);
		\coordinate (a) at (1.36,2.12);
		\coordinate (b) at (3,4);
		\coordinate (c) at (4.56,0.82);
		\fill[color=zzttqq,fill=zzttqq,fill opacity=0.1] (a) -- (b) -- (c) -- cycle;
		\draw [color=zzttqq] (a)-- (b);
		\draw [color=zzttqq] (b)-- (c);
		\draw [color=zzttqq] (c)-- (a);
		
		% Halbierungspunkte:
		\coordinate (Sab) at (2.18,3.06);
		\coordinate (Sbc) at (3.78,2.41);
		\coordinate (Sac) at (2.96,1.47);
		\coordinate (S) at (2.97,2.31);
		
		\fill [color=qqqqff] (a) circle (1.5pt);
		\draw[color=qqqqff] (a) node[left] {$a$};
		\fill [color=qqqqff] (b) circle (1.5pt);
		\draw[color=qqqqff] (b) node[above] {$b$};
		\fill [color=qqqqff] (c) circle (1.5pt);
		\draw[color=qqqqff] (c) node[right] {$c$};
		\fill [color=uququq] (Sab) circle (1.5pt);
		\draw[color=uququq] (Sab) node[left] {$S_{ab}$};
		\fill [color=uququq] (Sbc) circle (1.5pt);
		\draw[color=uququq] (Sbc) node[right] {$S_{bc}$};
		\fill [color=uququq] (Sac) circle (1.5pt);
		\draw[color=uququq] (Sac) node[below] {$S_{ac}$};
		\fill [color=uququq] (S) circle (1.5pt);
		\draw[color=uququq] (S) node[below left] {$S$};
		\draw [dash pattern=on 2pt off 2pt] (Sbc)-- (a);
		\draw [dash pattern=on 2pt off 2pt] (Sac)-- (b);
		\draw [dash pattern=on 2pt off 2pt] (Sab)-- (c);
		\end{tikzpicture}
	\end{figure}
\paragraph{Beweis}
	Der (geometrische) Schwerpunkt des Dreiecks $ \{a,b,c\}\subset A $ ist
		\[ s = a\cdot \frac{1}{3}+ b\cdot \frac{1}{3}+ c\cdot \frac{1}{3} = (a\cdot \frac{1}{2}+b\frac{1}{2})\frac{2}{3}+c\cdot \frac{1}{3} = s_{ab}\cdot\frac{2}{3}+c\cdot \frac{1}{3} \in [s_{ab}c];\]
	weiters gilt
			\[ (s_{ab}s:cs)= -\frac{\frac{1}{3}}{1-\frac{1}{3}} = -\frac{1}{2}, \]
	$ s $ teilt die Strecke $ \overline{s_{ab}c} $ im Verhältnis $ -\frac{1}{2} $. Aus Symmetriegründen gelten diese Resultate genauso für die anderen Seitenhalbierenden.
\paragraph{Bemerkung}
	Andere bekannte Schnittsätze im Dreieck machen in der affinen Geometrie keinen Sinn. Sätze wie der Höhensatz oder über den Umkreismittelpunkt können gar nicht erst formuliert werden: in der affinen Geometrie kennt man weder Längen- noch Winkelmessung.
	
	Dem gegenüber sind die Sätze von Menelaos und Ceva "`affine Sätze"', d.h. sie können rein affin formuliert werden und beschreiben unter affinen Transformationen \emph{invariante} Sachverhalte.
\subsection{Bemerkung \& Definition}
	Sind $ \alpha:A\to A' $ und $ \beta:A'\to A'' $ affine Abbildungen und bezeichnen $ \lambda:V\to V' $ bzw. $ \mu:V'\to V'' $ ihre linearen Anteile,
		\[ \forall a\in A\forall v\in V:\alpha(a+v) = \alpha(a)+\lambda(v)\text{ und }\forall a'\in A'\forall v'\in V':\beta(a'+v') = \beta(a')+\mu(v'), \]
	so gilt für ihre Komposition
		\[ (\beta\circ\alpha)(a+v) = \beta(\alpha(a)+\lambda(v)) = \beta(\alpha(a))+\mu(\lambda(v)) = (\beta\circ\alpha)(a)+(\mu\circ\lambda)(v), \]
	d.h. der lineare Anteil einer Komposition von affinen Abbildungen ist die Komposition der linearen Anteile.
	
	\begin{Definition}[Dilatationsgruppe]
	Da eine affine Transformation, deren linearer Anteil Vielfaches der Identität ist, eine Translation oder eine Streckung ist, bilden die Translationen und Streckungen eines affinen Raumes eine Gruppe, die \emph{Dilatationsgruppe}.
	\end{Definition}
	
\subsection{Satz von Menelaos}
	\begin{Satz}[Satz von Menelaos]
		Seien $ \{a,b,c\}\subset A $ ein nicht-degeneriertes Dreieck und $ g\subset A $ eine Gerade, die die drei Seiten des Dreiecks außerhalb der Ecken des Dreiecks schneidet;
			\[ a'\in g\cap [bc],b'\in g\cap [ca] \text{ und }c'\in g\cap [ab] \]
		bezeichne die Schnittpunkte. Dann gilt:
			\[ (ac':bc')(ba':ca')(cb':ab') = 1 \]
		Umgekehrt garantiert die TV-Bedingung, dass drei Punkte $ a'\in [bc],b'\in [ca] $ und $ c'\in [ab] $ auf den Seiten des Dreiecks kollinear sind.
	\end{Satz}
	
	\begin{figure}[H]\centering
	\definecolor{zzttqq}{rgb}{0.6,0.2,0}
	\definecolor{qqqqff}{rgb}{0,0,1}
	\begin{tikzpicture}[line cap=round,line join=round,>=triangle 45,x=1.0cm,y=1.0cm]
	\clip(1.05,-0.85) rectangle (9.76,4.01);
	\fill[color=zzttqq,fill=zzttqq,fill opacity=0.1] (2.1,0.26) -- (6.42,0.81) -- (4.5,3) -- cycle;
	\draw [color=zzttqq] (2.1,0.26)-- (6.42,0.81);
	\draw [color=zzttqq] (6.42,0.81)-- (4.5,3);
	\draw [color=zzttqq] (4.5,3)-- (2.1,0.26);
	\draw [domain=1.05:9.76] plot(\x,{(-0.02--0.55*\x)/4.32});
	\draw [domain=1.05:9.76] plot(\x,{(-13.58--1.45*\x)/-3.33});

	\fill [color=qqqqff] (2.1,0.26) circle (1.5pt);
	\draw[color=qqqqff] (2.14,0.33) node {$A$};
	\fill [color=qqqqff] (6.42,0.81) circle (1.5pt);
	\draw[color=qqqqff] (6.46,0.88) node {$B$};
	\fill [color=qqqqff] (4.5,3) circle (1.5pt);
	\draw[color=qqqqff] (4.54,3.07) node {$C$};
	\fill (3.94,2.36) circle (1.5pt);
	\draw (3.99,2.43) node {$b'$};
	\fill (7.28,0.92) circle (1.5pt);
	\draw (7.33,0.99) node {$c'$};
	\draw[color=black] (1.33,3.19) node {$g$};
	\fill (5.74,1.59) circle (1.5pt);
	\draw (5.78,1.66) node {$a'$};
	\end{tikzpicture}
	\end{figure}
	
\paragraph{Beweis}
	Betrachte Streckung $ \gamma $ mit Zentrum $ c' $ und Faktor $ s_{ab}\in K^\times $,
		\[ \gamma:A\to A, c'+ v\mapsto \gamma(c'+v) := c'+vs_{ab}; \]
	insbesondere ist für $ s_{ab}=\frac{1}{(ac':bc')} $
		\[ \gamma(a)=c'+(a-c')\frac{1}{(ac':bc')}=c'+(b-c') = b. \]
	Definiert man Streckungen $ \alpha $ und $ \beta $ entsprechend, mit Zentren $ a' $ bzw. $ b' $ und Faktoren $ s_{bc} = \frac{1}{(ba':ca')} $ bzw. $ s_{ca}=\frac{1}{cb':ab'} $, so liefert die Komposition eine affine Transformation
		\[ \delta:= \beta\circ\alpha\circ\gamma:A\to A:,a+v \mapsto \delta(a+v) := a+vs_{ab}s_{bc}s_{ca}, \]
	da
		\[ a\overset{\gamma}{\mapsto}b\overset{\alpha}{\mapsto}c\overset{\beta}{\mapsto}a. \]
	Damit gilt
		\[ (ac':bc')(ba':ca')(cb':ab') = 1 \Leftrightarrow \delta = \id_A \]
	Wegen $ a\notin [c'a'] $ ist andererseits
		\[ \delta = \id_A \Leftrightarrow [c'a'] = \delta([c'a']) = \beta([c'a']), \]
	da $ \gamma([c'a']) = [c'a'] $ und $ \alpha([c'a']) = [c'a'], $ was die letzte Gleichung liefert, damit ist
		\[ \delta = \id_A \Leftrightarrow [c'a']=\beta([c'a'])\Leftrightarrow b'\in [c'a'], \]
	da $ \beta $ Streckung mit Zentrum $ b' $ ist. Damit ist die Behauptung bewiesen.
\subsection{Satz von Ceva}
	\begin{Satz}[Satz von Ceva]
		Seien $ \{a,b,c\}\subset A $ ein nicht-degeneriertes Dreieck und
			\[ a'\in [bc]\setminus \{b,c\},b'\in [ac]\setminus \{a,c\},c'\in [ab]\setminus \{a,b\}. \]
		Schneiden sich die drei Transversalen $ [aa'],[bb'] $ und $ [cc'] $ in einem Punkt, so gilt
			\[ (ac':bc')(ba':ca')(cb':ab')=-1. \]
	\end{Satz}
	\begin{figure}[H]\centering
	\definecolor{zzttqq}{rgb}{0.6,0.2,0}
	\definecolor{qqqqff}{rgb}{0,0,1}
	\begin{tikzpicture}[line cap=round,line join=round,>=triangle 45,x=1.0cm,y=1.0cm]
	\clip(0.25,0.33) rectangle (9.57,6.25);
	\fill[color=zzttqq,fill=zzttqq,fill opacity=0.1] (1.5,1.52) -- (6.8,1.78) -- (4.08,5.58) -- cycle;
	\draw [color=zzttqq] (1.5,1.52)-- (6.8,1.78);
	\draw [color=zzttqq] (6.8,1.78)-- (4.08,5.58);
	\draw [color=zzttqq] (4.08,5.58)-- (1.5,1.52);
	\draw [domain=0.25:9.57] plot(\x,{(--15.39-1.1*\x)/4.43});
	\draw [domain=0.25:9.57] plot(\x,{(-13.46--3.96*\x)/0.48});
	\draw [domain=0.25:9.57] plot(\x,{(--1.82--1.03*\x)/2.21});
	\fill [color=qqqqff] (1.5,1.52) circle (1.5pt);
	\draw[color=qqqqff] (1.58,1.64) node {$a$};
	\fill [color=qqqqff] (6.8,1.78) circle (1.5pt);
	\draw[color=qqqqff] (6.88,1.9) node {$b$};
	\fill [color=qqqqff] (4.08,5.58) circle (1.5pt);
	\draw[color=qqqqff] (4.14,5.7) node {$c$};
	\fill (2.37,2.88) circle (1.5pt);
	\draw (2.45,3.01) node {$b'$};
	\fill (3.6,1.62) circle (1.5pt);
	\draw (3.68,1.75) node {$c'$};
	\fill (5.62,3.43) circle (1.5pt);
	\draw (5.7,3.56) node {$a'$};
	\end{tikzpicture}
	\end{figure}
	Beweis in Aufgabe 59.
\paragraph{Bemerkung}
	Für die Seitenmitten gilt der Satz (Schwerpunktsatz).

%VO18-2015-12-10
\chapter{Buchhaltung}
Dieses Kapitel zeigt eine Art "`Tabellenkalkül"' -- eine effiziente Rechenmethode in der linearen Algebra.

Vorteil: Selbst durch einen Trottel (e.g. einen Computer) ausführbar.

Nachteil: Selbst durch einen Trottel ausführbar.

\paragraph{Generalvoraussetzung} Alle VR haben in diesem Kapitel endliche Dimension.
\section{Matrizen}
 \paragraph{Idee}
 	Ein Homomorphismus $ f\in \hom(V,W) $ wird (nach Fortsetzungssatz) durch die Bilder $ f(b_j) $ der Vektoren einer Basis $ (b_j)_{j\in J} $ eindeutig festgelegt; ist $ (c_i)_{i\in I} $ eine Basis von $ W $, so hat jedes dieser $ f(b_j) $ eine eindeutige Basisdarstellung.
 	\[
 		\forall {j\in J}\exists! (x_i)_{i\in I}:f(b_j) = \sum_{i\in I}c_ix_{ij}
 	\]
 	Sind $ n=\dim V $ und $ m=\dim W $ endlich, so kann man also $ f $ mithilfe der Basen $ (b_j)_{j\in J} $ von $ V $ und $ (c_i)_{i\in I} $ von $ W $ komplett durch die Tabelle der Koeffizienten beschreiben:

 	\begin{figure}[H]\centering
 		$
 		\begin{array}{c|cccccc}
 			f      & f(b_1) & f(b_2) & \dots & f(b_j) & \dots & f(b_n) \\\hline
 			c_1    & x_{11} & x_{12} & \dots & x_{1j} & \dots & x_{1n} \\
 			c_2    & x_{21} & x_{22} & \dots & x_{2j} & \dots & x_{2n} \\
 			\vdots & \vdots & \vdots &       & \vdots &       & \vdots \\
 			c_i    & x_{i1} & x_{i2} & \dots & x_{ij} & \dots & x_{in} \\
 			\vdots & \vdots & \vdots &       & \vdots &       & \vdots \\
 			c_m    & x_{m1} & x_{m2} & \dots & x_{mj} & \dots & x_{mn}
 		\end{array}
 		$
 	\end{figure}

 	Dabei spielt es prinzipiell keine Rolle, ob die Bilder $ f(b_j) $ der Basisvektoren in den Spalten stehen (wie oben) oder in den Zeilen der Tabelle -- es ist aber wichtig, dass dies konsistent gemacht wird.

 	In dieser LVA: Bilder $ f(b_j) $ der Basisvektoren werden durch Spalten beschrieben.

\subsection{Definition}
	\begin{Definition}[Matrix]
		Eine \emph{Matrix} $ X\in K^{m\times n} $ ist eine Tabelle von Elementen $ x_{ij}\in K $ mit $ m $ Zeilen und $ n $ Spalten:
		\[
			X =
			\begin{pmatrix}
				x_{11} & \dots & x_{1n} \\
				\vdots &       & \vdots \\
				x_{m1} & \dots & x_{mn}
			\end{pmatrix}
		\]
		Die \emph{(Darstellungs-)Matrix} eines $ f\in \hom(V,W) $ bzgl. Basen $ B= (b_1,\dots,b_n) $ und $ C=(c_1,\dots,c_m) $ von $ V $ bzw. $ W $, ist die Matrix
		\[
			X = \xi^C_B(f)\in K^{m\times n}\text{ mit }\forall j=1,\dots,n:f(b_j) = \sum_{i=1}^{m}c_ix_{ij}.
		\]
	\end{Definition}

	\paragraph{Bemerkung}
		Mit $ I:= \{1,\dots,m\} $ und $ J:= \{1,\dots,n\} $ kann eine Matrix auch als Abbildung aufgefasst werden
		\[
			X = (x_{ij})_{i\in I,j\in J} \quad\text{ bzw. }\quad X:I\times J\to K,\ (i,j)\mapsto x_{ij}.
		\]
		Ist $ f\in \hom(V,W) $ und sind $ B=(b_1,\dots,b_n) $ und $ C=(c_1,\dots,c_m) $ Basen von $ V $ bzw. $ W $, so sind
		\[
			x_{ij} = c_i^*(f(b_j))
		\]
		die Komponenten der Darstellungsmatrix $ \xi_B^C(f) $ von $ f $ bzgl. der Basen $ B $ und $ C $ mit der zu $ C $ dualen Basis $ C^*=(c_1^*,\dots c_m^*) $ von $ W^* $.\\
		Mit der zu $ B $ dualen Basis $ B^*= (b_1^*,\dots,b_n^*) $ von $ V^* $ ist dann auch
		\[
			f=\sum_{i=1}^{m}\sum_{j=1}^{n} c_ix_{ij}b_j^*.
		\]
\subsection{Lemma}
	\begin{Lemma}[Matrizen als VR]
		Mit der komponentenweisen Addition und Skalarmultiplikation auf $ K^{m\times n} $,
		\[
			(x_{ij})+(y_{ij}) := (x_{ij}+y_{ij}) \text{ und } (x_{ij})\cdot z := (x_{ij}\cdot z),
		\]
		wird $ K^{m\times n} $ ein Vektorraum und man erhält einen Isomorphismus zu Basen $B$ und $C$ von $V$ bzw. $W$.
		\[
			\xi_B^C:\hom(V,W)\to K^{m\times n},\ f\mapsto \xi_B^C(f).
		\]
	\end{Lemma}
	\paragraph{Bemerkung}
		Die komponentenweise Addition und Skalarmultiplikation sind gerade die Addition und Skalarmultiplikation von Matrizen als Abbildungen.

	\paragraph{Beweis}
		Dass $ \hom(V,W) $ und $ K^{m\times n} \ K $-VR sind, ist bekannt (vgl. Kap. 1.4 bzw. Kap. 1.1). Die Linearität von $ \xi_B^C $ folgt direkt, da mit der zu $ C $ dualen Basis $ C^* $ von $ W^* $
		\[
			\forall_{i=1,\dots,m} \forall_{j=1,\dots,n}: x_{ij}= c_i^*(f(b_j)).
		\]
		Nämlich: für $ f,g\in \hom(V,W) $ und $ x,y\in K $ ist dann
		\begin{align*}
			\forall_{i= 1,\dots,m} \forall_{j=1,\dots, n} : c_i^*((fx+gy)(b_j)) & = c_i^*(f(b_j)x+g(b_j)y)         \\
			                                                                    & = c_i^*(f(b_j))x+c_i^*(g(b_j))y.
		\end{align*}
		Die Abbildung
		\[
			K^{m\times n}\ni X=(x_{ij})\mapsto \sum_{i=1}^{m}\sum_{j=1}^{n}c_ix_{ij}b_j^* = f\in \hom(V,W)
		\]
		liefert die Inverse von $ f\mapsto\xi_B^C(f) $, also ist $ \xi_B^C $ ein Isomorphismus.
	\paragraph{Bemerkung}
		Damit folgt (vgl. Kap. 1.4): $ \dim \hom(V,W) = \dim K^{m\times n} = m\cdot n $.
\subsection{Lemma \& Definition}
	\begin{Lemma}[Darstellungsmatrix einer Komposition]
		Sind $ U,V,W \ K$-VR mit Basen $ A=(a_1,\dots,a_p),\ B=(b_1,\dots,b_n),\ C=(c_1,\dots,c_m) $, so gilt für $ g\in \hom(U,V) $ und $ f\in \hom(V,W) $
		\[
			\xi_A^C(f\circ g) = \xi_B^C(f)\cdot \xi_A^B(g),
		\]
	\end{Lemma}
	\begin{Definition}
		wobei die \emph{Matrixmultiplikation}
		\[
			\cdot:K^{m\times n}\times K^{n\times p} \to K^{m\times p},\ (X,Y)\mapsto X\cdot Y = Z
		\]
		definiert ist durch
		\[
			z_{ik} := \sum_{j=1}^{n}x_{ij}y_{jk}.
		\]
	\end{Definition}
	\paragraph{Bemerkung}
		Das Element $ z_{ik} $ in der $ i $-ten Zeile und $ k $-ten Spalte von $ Z = XY $ wird also aus der $ i $-ten Zeile von $ X $ und der $k$-ten Spalte von $ Y $ berechnet.
		\begin{align*}
			\left(
			\begin{array}{ccc}
			       &        &        \\
			x_{i1} & \dots  & x_{in} \\
			       &        &        
			\end{array}
			\right)
			\cdot
			\left(
			\begin{array}{ccc}
			       & y_{1k} &        \\
			       & \vdots &        \\
			       & y_{nk} &        
			\end{array}
			\right)
			=
			\left(
			\begin{array}{ccc}
			       &        &	     \\
			       & z_{ik} &        \\
			       &        &        
			\end{array}
			\right)
		\end{align*}
	\paragraph{Beweis}
		Wir verwenden die Darstellungsmatrizen
		\[
			\begin{cases}
				X = \xi^C_B(f)\in K^{m\times n} & \text{ von } f\in \hom(V,W) \\
				Y = \xi_A^B(g)\in K^{n\times p} & \text{ von } g\in \hom(U,V)
			\end{cases}
		\]
		bezüglich $ B $ und $ C $ bzw. $ A $ und $ B $, dann gilt für $ k=1,\dots,p $
		\[
			(f\circ g)(a_k)=f\Big(\sum_{j=1}^{n}b_jy_{jk}\Big) = \sum_{j=1}^{n}f(b_j)y_{jk} = \sum_{j=1}^{n}\sum_{i=1}^{m}c_ix_{ij}y_{jk} = \sum_{i=1}^{m}c_i\Big(\sum_{j=1}^{n}x_{ij}y_{jk}\Big),
		\]
		d.h. durch $ I=\{1,\dots,m\},\ J=\{1,\dots,n\},\ K=\{1,\dots,p\} $ und
		\[
			\xi_A^C(f\circ g) = Z = (z_{ik})_{i\in I,k\in K} \quad\text{mit}\quad \forall i\in I\ \forall k\in K: z_{ik}= \sum_{j=1}^{n}x_{ij}y_{jk}
		\]
		erhält man die Darstellungsmatrix
		\[
			\xi_A^C(f\circ g) = \xi_B^C(f)\xi_A^B(g)
		\]
		der Komposition als Produkt der Darstellungsmatrizen von $ f $ und $ g $.
\subsection{Notation \& Definition}
	\begin{Definition}[Kurzform der def. Gleichung einer Darst.-Matrix]
		Wir notieren die definierende Gleichung einer Darstellungsmatrix $ X=\xi_B^C(f) $ von $ f\in \hom(V,W) $ auch in Kurzform
		\[
			CX=(c_1,\dots,c_m)X = (f(b_1),\dots,f(b_n)) = f(B).
		\]
		Für die \emph{Koordinatenspalte eines Vektors}
		\[
			Y\in K^{n\times 1} \text{ mit } v=\sum_{j=1}^{n}b_jy_{j1}
		\]
		ist dann
		\[
			f(v) = (f(b_1),\dots,f(b_n))Y = (c_1,\dots,c_m)XY.
		\]
	\end{Definition}
	Die Familien $ (c_1,\dots,c_m) $ und $ (f(b_1),\dots,f(b_n)) $ sind keine Matrizen, denn die Elemente sind Vektoren!

%VO19-2015-12-15
	\paragraph{Bemerkung}
		Wir schreiben die Skalarmultiplikation als Rechts-Multiplikation.
	\paragraph{Beispiel}
		Die neue Notation liefert einen alternativen "`Beweis"' für $ \xi_A^C(f\circ g) = \xi_B^C(f)\xi_A^B(g) $:

		Gilt für jeden Vektor $ a_k,\ k=1,\dots,p $
		\[
			g(a_k) = \sum_{j=1}^{n}b_jy_{jk}, \text{ wobei } Y = \xi_A^B(g)
		\]
		so erhalten wir
		\[
			(f(g(a_1)),\dots, f(g(a_p))) = (f(b_1),\dots , f(b_n))Y = (c_1,\dots,c_m)\xi_B^C(f)\cdot Y = C\cdot XY
		\]
		womit nun
		\[
			(f\circ g)(A) = C\cdot XY,
		\]
		also
		\[
			\xi_A^C(f\circ g) = X\cdot Y = \xi_B^C(f)\cdot \xi_A^B(g).
		\]

		Einfacher (aber weniger überzeugend) ist die folgende, die Linearität von $ f $ benutzende Version:
		\[
			( f\circ g )(A) = f(g(A)) = f(BY) = f(B)\cdot Y = C\cdot XY
		\]
	\paragraph{Bemerkung}
		Sei $ f\in \hom(V,W) $ mit $ r:= \rg f $, dann existieren Basen $ B $ und $ C $ von $ V $ bzw. $ W $, sodass
		\[
			\xi_B^C(f) = X \text{ mit } x_{ij} =
			\begin{cases}
				1, & \text{falls }i=j\leq r \\
				0, & \text{sonst}
			\end{cases}
		\]
		d.h.
		\[
			X = \left(
			\begin{array}{ccc|ccc}
				1      & \dots  & 0      & 0      & \dots  & 0      \\
				\vdots & \ddots & \vdots & \vdots &        & \vdots \\
				0      & \dots  & 1_{rr} & 0      & \dots  & 0      \\\hline
				0      & \dots  & 0      & 0      & \dots  & 0      \\
				\vdots &        & \vdots & \vdots &        & \vdots \\
				0      & \dots  & 0      & 0      & \dots  & 0      \\
			\end{array}
			\right)
		\]

		Nämlich -- wie im Beweis des Rangsatzes: Die Basen $ B $ und $ C $ werden so gewählt, dass
		\begin{enumerate}[(i)]
			\item $ (b_{r+1},\dots,b_n) $ Basis von $ \ker f $ ist, und dann
			\item $ c_i := f(b_i) $ für $ i= 1,\dots,r $ eine Basis von $ f(V) $ liefert.
		\end{enumerate}
		Offenbar hat $ \xi_B^C(f)$ dann die gewünschte Form:
		\[
			f(b_1)=c_1,\dots,f(b_r)=c_r,f(b_{r+1})=0,\dots,f(b_n)=0,
		\]
		d.h.
		\[
			f(B) = (f(b_1),\dots,f(b_n))=(c_1,\dots,c_m)\left(
			\begin{array}{ccc|ccc}
				1      & \dots  & 0      & 0      & \dots  & 0      \\
				\vdots & \ddots & \vdots & \vdots &        & \vdots \\
				0      & \dots  & 1_{rr} & 0      & \dots  & 0      \\\hline
				0      & \dots  & 0      & 0      & \dots  & 0      \\
				\vdots &        & \vdots & \vdots &        & \vdots \\
				0      & \dots  & 0      & 0      & \dots  & 0      \\
			\end{array}
			\right) = CX.
		\]
		Umgekehrt: gibt es eine Darstellungmatrix von $ f $ dieser Form, so ist $ \rg f = r $.
\subsection{Beispiel \& Definition}
	\begin{Definition}[Einheitsmatrix]
		Ist $ B=(b_1,\dots,b_n) $ Basis von $ V $, so hat der Isomorphismus
		\[
			\phi:V\to K^n \text{ mit } \forall j=1,\dots,n:\phi(b_j)=e_j
		\]
		bezüglich $ B $ und der Standardbasis $ E = (e_1,\dots,e_n) $ von $ K^n $ die \emph{$ n $-reihige Einheitsmatrix} als Darstellungsmatrix:
		\[
			\xi_B^E(\phi) = E_n := (\delta_{ij})_{i,j = 1,\dots,n}
		\]
	\end{Definition}
	\paragraph{Beispiel}
		Sind $ B=(b_1,\dots,b_n) $ und $ B'=(b'_1,\dots,b'_n) $ Basen von $ V $, wobei
		\[
			\forall j=1,\dots,n:b_j = \sum_{i=1}^{n}b'_ix_{ij},
		\]
		so hat die Identität $ \id_V $ die Darstellungsmatrix
		\[
			\xi_B^{B'}(\id_V)= X = (x_{ij})_{i,j=1\dots,n}.
		\]
		Sind dann $ B $ und $ B' $ Basen von $ V $ und $ C $ und $ C' $ Basen von $ W $, so erhält man für $ f\in\hom(V,W) $ die Transformationsformel
		\[
			\xi_{B'}^{C'}(f) = \xi_{B'}^{C'}(\id_W\circ f\circ \id_V) = \xi_C^{C'}(\id_W)\cdot \xi_B^C(f)\cdot \xi_{B'}^{B}(\id_V)
		\]
\subsection{Beispiel \& Definition}
	Ist $ f\in \Iso(V,W) $ mit Basen $ B $ und $ C $ von $ V $ bzw. $ W $, so gilt (mit $ n=\dim V = \dim W $)
	\[
		\xi_C^B(f^{-1})\cdot \xi_B^C(f) = \xi_B^B(f^{-1}\circ f) = \xi_B^B(\id_V) = E_n
	\]
	und
	\[
		\xi_B^C(f)\cdot \xi_C^B(f^{-1})=\xi_C^C(f\circ f^{-1}) = \xi_C^C(\id_W)=E_n.
	\]

	\begin{Definition}[Invertierbare Matrix]
		Eine Matrix $ X\in K^{n\times n} $ nennt man invertierbar mit Inverser $ X^{-1} $, falls
		\[
			\exists X^{-1}\in K^{n\times n}:X^{-1}X = E_n
		\]
		Damit ist die Darstellungsmatrix der Inversen die Inverse der Darstellungsmatrix:
		\[
			\xi_C^B(f^{-1}) = (\xi_B^C(f))^{-1}
		\]
	\end{Definition}
\subsection{Bemerkung \& Definition}
	Jedes $ X\in K^{m\times n} $ liefert (eindeutig) $ f_X\in \hom(K^n,K^m) $ nach Fortsetzungssatz via
	\[
		f_X:K^n\to K^m,\ f_X(e_j) = \sum_{i=1}^{m} e'_ix_{ij} \text{ für } j=1,\dots,n.
	\]
	Bezüglich der Standardbasen $ E = (e_1,\dots,e_n) $ von $ K^n $ und $ E'=(e'_1,\dots,e'_m) $ von $ K^m $ ist dann
	\[
		\xi_E^{E'}(f_X) = X.
	\]

	\begin{Definition}[Rang einer Matrix]
		Damit definiert man den Rang einer Matrix $ X\in K^{m\times n} $ als
		\[
			\rg X:=\rg f_X
		\]
		Eine Matrix $ X\in K^{n\times n} $ ist genau dann invertierbar, wenn $ \rg X = n $. Man setzt
		\[
			\mathrm{Gl}(n):= \{X\in K^{n\times n}\mid \rg X =n\}.
		\]
	\end{Definition}
\subsection{Bemerkung \& Definition}
	Nach der Transformationsformel für Darstellungsmatrizen gilt bei Basiswechseln in $ V $ und $ W $ für $ f\in \hom(V,W) $
	\[
		\xi_{B'}^{C'}(f) = \xi_C^{C'}(\id_W)\cdot \xi_B^C(f)\cdot \xi_{B'}^B(\id_V).
	\]
	Dabei sind $ \xi_{B'}^B(\id_V)\in \mathrm{Gl}(n) $ und $ \xi_C^{C'}(\id_W)\in \mathrm{Gl}(m) $ invertierbar, da etwa
	\[
		\xi^B_{B'}(\id_V)\cdot\xi_B^{B'}(\id_V) = \xi_B^B(\id_V)=E_n;
	\]
	Sind andererseits die Basis $ B $ und $ P\in \mathrm{Gl}(n) $ gegeben, so ist
	\[
		\xi_B^{B'}(\id_V)=P^{-1} \text{ für } B':= BP,
	\]
	d.h. jedes $ P\in \mathrm{Gl}(n) $ realisiert einen Basiswechsel in $ V $, kommt also in der Transformationsformel vor.

	\begin{Definition}[Äquivalente Matrizen]
		Daher definiert man auch Matrizen $ X,X'\in K^{m\times n} $ als \emph{äquivalent},
		\[
			X\sim X',\quad \text{falls }\exists P\in \mathrm{Gl}(n)\exists Q\in \mathrm{Gl}(m):X' = QXP^{-1}.
		\]
	\end{Definition}

%VO19-2015-12-15
\section{Lineare Gleichungssysteme}
 Mission: Viele Probleme in Anwendungen oder Naturwissenschaften werden zu "`linearen Problemen"' reduziert, d.h. auf lineare Gleichungssysteme unterschiedlicher Komplexität.
 Diese Reduktion ist etwa eine wichtige Aufgabe der Analysis; Aufgabe der linearen Algebra ist dann die Lösung bzw. Strukturanalyse der linearen Gleichungssysteme.
 \subsection{Definition}
 	\begin{Definition}[Lineares Gleichungssystem]
 		Ein \emph{lineares Gleichungssystem} (LGS) ist ein System von $ m $ Gleichungen
 		\[
 			\begin{array}{cccc}\tag{$\star\star$}
 				a_{11}x_1+  & \dots & + a_{1n}x_n & =y_1   \\
 				\vdots      &       & \vdots      & \vdots \\
 				a_{m1}x_1 + & \dots & +a_{mn}x_n  & = y_m
 			\end{array}
 		\]
 		für $ n $ Unbekannte $ x_1,\dots,x_n\in K $, wobei die Parameter $ a_{ij},y_i\in K $ gegeben sind. Ist $ y_1 = \dots = y_m = 0 $, so heißt das System \emph{homogen}, anderenfalls \emph{inhomogen}.
 	\end{Definition}
 	\paragraph{Bemerkung}
 		Mit Matrizen $ A\in K^{m\times n},X\in K^{n\times 1} $ und $ Y\in K^{m\times 1} $ lässt sich ein lineares Gleichungssystem kompakter schreiben als
 		\[
 			AX = Y \tag{$\star$}
 		\]
 		Die Standardbasen $ E $ und $ E' $ von $ K^n $ bzw. $ K^m $ liefern den Isomorphismus
 		\[
 			K^{m\times n}\ni A\mapsto f_A\in \hom(K^n,K^m)\text{, wobei }f_A(E) = E'A,
 		\]
 		damit lässt sich ($\star$) umformulieren als Gleichung eines affinen Unterraumes von $ K^n: $
 		\[
 			f_A(x) = y \text{ mit } x=EX \text{ und } y=E'Y.
 		\]
 		Nämlich: Existiert eine Lösung $ x\in f_A^{-1}(\{y\})\neq \emptyset $, so ist der Lösungsraum
 		\[
 			f_A^{-1}(\{y\}) = x+\ker f_A\subset K^n
 		\]
 		ein affiner Unterraum.

 		Das nächste Lemma folgt dann mit dem Basisisomorphismus:
 		\[
 			K^{n\times 1} \ni X \mapsto EX =: x\in K^n
 		\]
 \subsection{Definition \& Lemma}
 	\begin{Lemma}[Lösungsraum]
 		Der \emph{Lösungsraum} $ L_{A,Y} $ eines linearen Gleichungssystems,
 		\[
 			L_{A,Y}:=\{X\in K^{n\times 1}\mid AX=Y\}\subset K^{n\times 1}
 		\]
 		ist leer oder ein affiner Unterraum der Dimension $ k = n-\rg A $.

 		Ist $ Y = 0 $, so gilt $ 0\in L_{A,Y} $ und $ L_{A,Y}\subset K^{n\times 1} $ ist ein linearer Unterraum (UVR).
 	\end{Lemma}

%VO20-2015-12-17
 	\paragraph{Bemerkung}
 		Jede Lösung $ X_l $ eines (inhomogenen) LGS $AX=Y$ lässt sich schreiben als Summe einer \emph{Partikulärlösung} $ X_0 \in K^{n\times 1},AX_0 = Y $, und einer Lösung $V$ des homogenen LGS $ AX = 0 $:
 		\[
 			\forall X_l\in L_{A,Y}\exists V\in L_{A,0}:X_l=X_0+V=\tau_V(X_0)
 		\]
 	\paragraph{Bemerkung}
 		Der Lösungsraum eines "`unendlichen linearen Gleichungssystems"' hat die gleiche Struktur eines affinen Unterraums wie im endlichen Fall, z.B.:
 		\[
 			\{x\in C^\infty(\mathbb{R})\mid \forall t\in \mathbb{R}:x''(t) = t^2 \}
 		\]
 		ist ein (2-dim) AUR, des unendlich-dim. R-VR $C^\infty$, wobei $ C^\infty(\mathbb{R}) $ den (Vektor-)Raum der beliebig oft differenzierbaren Funktionen auf $ \mathbb{R} $ notiert.
 \subsection{Bemerkung \& Definition}
 	\begin{Definition}[Erweiterte Koeffizientenmatrix]
 		Ist $ AX=Y \neq 0$ ein inhomogenes LGS, so gilt
 		\[
 			L_{A,Y} = \emptyset \Leftrightarrow y\notin f_A(K^n)
 		\]
 		mit der \emph{erweiterten Koeffizientenmatrix}
 		\[
 			(A\mid Y) \in K^{m\times (n+1)}
 		\]
 		lässt sich dies formulieren als
 		\[
 			f_{(A\mid Y)}(K^{n+1})\neq f_A(K^n) \Leftrightarrow \rg f_{(A\mid Y)}\neq \rg f_A \Leftrightarrow \rg (A\mid Y) \neq \rg A.
 		\]
 		Folglich ist
 		\[
 			L_{A,Y} \neq \emptyset \Leftrightarrow \rg (A\mid Y) = \rg A
 		\]
 	\end{Definition}
 \subsection{Bemerkung \& Definition, Gaußsches Eliminationsverfahren}
 	\begin{Definition}[Äquivalente LGS]
 		Eine Idee zur Lösung eines LGS ist, das Gleichungssystem zu "`vereinfachen"', ohne dabei den Lösungsraum zu verändern: Man nennt zwei LGS $ AX=Y$ und $A'X=Y' $ \emph{äquivalent}, wenn sie den gleichen Lösungsraum haben,
 		\[
 			(AX=Y)\sim (A'X=Y'):\Leftrightarrow L_{A,Y} = L_{A',Y'}
 		\]
 	\end{Definition}
 	\emph{Links}multiplikation der erweiterten Koeffizientenmatrix $ (A\mid Y) $ mit den folgenden Matrizen (mit $ i\neq j $) liefert z.B. äquivalente Systeme:

 	$ D_i = (d_{kl})\in \mathrm{Gl}(m), \qquad
 	d_{kl} := \delta_{kl}+(d-1)\delta_{ik}\delta_{il}\quad (d\in K^x) $
 	\[
 		\bordermatrix{
 			&   &   & i & &\cr
 			& 1 & 0 & \dots & \dots & 0\cr
 			& 0 & \ddots & \ddots &   & \vdots \cr
 			i & \vdots & \ddots & d & \ddots & \vdots \cr
 			& \vdots &   &  \ddots & \ddots & 0 \cr
 			& 0 &  \dots & \dots  & 0 & 1 \cr
 		}
 	\]
 	$ T_{ij} = (t_{kl})\in \mathrm{Gl}(m), \qquad
 	t_{kl} := \delta_{kl}-(\delta_{ik}-\delta_{jk})(\delta_{il}-\delta_{jl})$
 	\[
 		\bordermatrix{
 			&        & i      & \dots  & j     &         \cr
 			& 1      &        &        &       &         \cr
 			i     &        & 0      &        & 1     &         \cr
 			\vdots&        &        & 1      &       &         \cr
 			j     &        & 1      &        & 0     &         \cr
 			&        &        &        &       & 1       \cr
 		}
 	\]

 	$ S_{ij}=(s_{kl})\in \mathrm{Gl}(m), \qquad
 	s_{kl} := \delta_{kl}+s\delta_{ik}\delta_{jl} \quad (s\in K)$
 	\[
 		\bordermatrix{
 			&   &   &  & j &\cr
 			& 1 & 0 & \dots & \dots & 0\cr
 			i & 0 & \ddots & \ddots & s & \vdots \cr
 			& \vdots & \ddots & 1 & \ddots & \vdots \cr
 			& \vdots &   &  \ddots & \ddots & 0 \cr
 			& 0 &  \dots & \dots  & 0 & 1 \cr
 		}
 	\]

 	Die entsprechenden Operationen auf dem LGS werden als \emph{elementare Zeilenoperationen/-umformungen} bezeichnet (elZumf) bezeichnet:
 	\begin{itemize}
 		\item $ (A\mid Y) \to D_i (A\mid Y) $, Multiplikation der $ i $-ten Gleichung mit $ d\neq 0 $;
 		\item $ (A\mid Y) \to T_{ij} (A\mid Y) $, Vertauschung der $ i $-ten und $ j $-ten Gleichung;
 		\item $ (A\mid Y) \to S_{ij} (A\mid Y)$, Addition des $ s $-fachen der $ j $-ten Gleichung zur $ i $-ten Gleichung.
 	\end{itemize}
 	Da $ D_i,T_{ij},S_{ij}\in \mathrm{Gl}(m) $, sind die elementaren Zeilenumformungen reversibel, verändern daher den Lösungsraum nicht: für $ D_i $ und $ T_{ij} $ ist das klar; $ S_{ij} = S_{ij}(s) $ ist invertierbar mit
 	\[
 		(S_{ij})^{-1} = (S_{ij}(s))^{-1} = S_{ij}(-s).
 	\]
 	Geometrisch ist $ S_{ij} $ Darstellungsmatrix einer Scherung.

 	\begin{Definition}[Zeilenstufenform]
 		Mit Hilfe der elementaren Zeilenumformungen kann man das LGS auf Zeilenstufenform bringen:
 		\[
 			\begin{pmatrix}
 				a_{11} & \dots & a_{1n} & y_1    \\
 				a_{21} & \dots & a_{2n} & y_2    \\
 				a_{31} & \dots & a_{3n} & y_3    \\
 				\vdots &       & \vdots & \vdots \\
 				a_{m1} & \dots & a_{mn} & y_m
 			\end{pmatrix}
 			\xrightarrow{\text{elZUmf}}
 			\begin{pmatrix*}[l]
 				1 & \dots & \dots &\dots & \dots & y'_1\\
 				0 & 1 & \dots &\dots & \dots & y'_2\\
 				\vdots &\ddots & \ddots & &  &\vdots \\
 				0  & \dots  &  0 & 1 & \dots &y'_r\\
 				0  & \dots  & \dots    & 0 & 0 & y'_{r+1}\\
 				\vdots & & & & \vdots & \vdots \\
 				0 & \dots & \dots   & \dots &  0 & y'_{m}\\
 			\end{pmatrix*}
 		\]
 		Ein System in Zeilenstufenform kann dann einfach gelöst werden -- oder auch nicht, falls eine Gleichung $ 0 = y' \neq 0 $ auftaucht.
 	\end{Definition}
 	\paragraph{Beispiel}
 		Wir betrachten das LGS $ AX=Y $ mit
 		\[
 			A =
 			\begin{pmatrix}
 				0 & 3 & 6 \\
 				1 & 4 & 7 \\
 				2 & 5 & 8
 			\end{pmatrix}
 			\text{ und }
 			Y =
 			\begin{pmatrix}
 				y_1 \\ y_2 \\ y_3
 			\end{pmatrix}
 			.
 		\]
 		Elementare Zeilenumformungen liefern dann:
 		\begin{align*}
 			\begin{pmatrix}
 			0 & 3  & 6  & y_1             \\
 			1 & 4  & 7  & y_2             \\
 			2 & 5  & 8  & y_3
 			\end{pmatrix}
 			\quad
 			\overset{T_{12}}{\to}\quad
 			&
 			\begin{pmatrix}
 			1 & 4  & 7  & y_2             \\
 			0 & 3  & 6  & y_1             \\
 			2 & 5  & 8  & y_3
 			\end{pmatrix}
 			\\
 			\overset{S_{31}(-2)}{\to}\quad
 			&
 			\begin{pmatrix}
 			1 & 4  & 7  & y_2             \\
 			0 & 3  & 6  & y_1             \\
 			0 & -3 & -6 & y_3-2y_2
 			\end{pmatrix}
 			\\
 			\overset{S_{32}(1)}{\to}\quad
 			&
 			\begin{pmatrix}
 			1 & 4  & 7  & y_2             \\
 			0 & 3  & 6  & y_1             \\
 			0 & 0  & 0  & y_3-2y_2+y_1
 			\end{pmatrix}
 			\\
 			\overset{D_2(\frac{1}{3})}{\to}\quad
 			&
 			\begin{pmatrix}
 			1 & 4  & 7  & y_2             \\
 			0 & 1  & 2  & y_1 \frac{1}{3} \\
 			0 & 0  & 0  & y_3-2y_2
 			\end{pmatrix}
 		\end{align*}
 		d.h. ein äquivalentes LGS $ A'X=Y' $ ist gefunden mit
 		\[
 			A' =
 			\begin{pmatrix}
 				1 & 4 & 7 \\
 				0 & 1 & 2 \\
 				0 & 0 & 0
 			\end{pmatrix}
 			\text{ und }
 			Y =
 			\begin{pmatrix}
 				y_2 \\ y_1\frac{1}{3} \\ y_1-2y_2+y_3
 			\end{pmatrix}
 			.
 		\]
 		Das LGS $ AX=Y $ ist also genau dann lösbar, wenn $ y_1-2y_2+y_3 = 0 $; in diesem Falle ist dann
 		\[
 			L_{A,Y} = \{X =
 			\begin{pmatrix} y_2-7t-4(-2t+\frac{1}{3}y_1)\\-2t+\frac{1}{3}y_1\\t\end{pmatrix}
 			, t\in \mathbb{R}\}
 			= \{X =
 			\begin{pmatrix} t-\frac{4}{3}y_1+y_2\\-2t+\frac{1}{3}y_1\\t\end{pmatrix}
 			, t\in \mathbb{R}\}
 		\]
 	\paragraph{Historische Bemerkung}
 		Das Gaußsche Eliminationsverfahren ist schon seit ca. 2000 Jahren bekannt, also schon lange vor Gauß (1777-1855) entwickelt worden.

 	\paragraph{Nutzen der Methode}
 		\begin{itemize}
 			\item lässt sich einfach programmieren (leider ggf. numerisch instabil)
 			\item nützlich für mittelgroße Systeme (tausende Gleichungen)
 			\item nicht effizient für große Systeme (Millionen von Gleichungen)
 		\end{itemize}
 	\paragraph{Bemerkung}
 		Mehrere LGS $ AX = Y_1, AX = Y_2, \dots AX = Y_k $ mit derselben Koeffizientenmatrix $ A $ können simultan gelöst werden, indem man elementare Zeilenumformungen auf die um alle $ Y $ erweiterte Koeffizientenmatrix $ (A\mid Y_1\mid Y_2\mid \dots \mid Y_k) $ anwendet.
 	\paragraph{Bemerkung}
 		Das Gaußsche Eliminationsverfahren kann zur Bestimmung der Inversen einer Matrix $ A\in \mathrm{Gl}(n) $ verwendet werden. Insbesondere ist eine \emph{untere Dreiecksmatrix} $ A\in K^{n\times n} $, d.h $ a_{ij} = 0 $ für $ i<j $ genau dann invertierbar, wenn $ a_{ii}\neq 0 $ für alle $ i=1,\dots,n $.

\chapter{Volumenmessung}
	Grundlegende Idee: Wir definieren ein Spat- oder Parallelotop-Volumen.

	Algebraisch: Dieses Volumen kann dann benutzt werden, um zu testen, wann ein Spat/Parallelotop "`zusammenklappt"'.
\section{Determinantenformen}
	Idee: Für den Flächeninhalt $ F(v,w) $ eines von zwei Vektoren $ v,w\in V $ aufgespannten Parallelogramms gilt
	\begin{gather*}
		F{(vx,w)} = F{(v,w)}x\\
		F(v+v',w) = F(v,w)+F(v',w)
	\end{gather*}
	und entsprechend für das zweite Argument; außerdem verschwindet der Flächeninhalt, wenn das Parallelogramm "`zusammenklappt"', also insbesondere gilt
		\[ w=v\Rightarrow F(v,w)=0 \]
	Die folgende Definition verallgemeinert diese Eigenschaften:

\definecolor{ttttff}{rgb}{0.2,0.2,1}
\definecolor{ttfftt}{rgb}{0.2,1,0.2}
\definecolor{uququq}{rgb}{0.25,0.25,0.25}
\definecolor{qqqqff}{rgb}{0,0,1}
\begin{tikzpicture}[line cap=round,line join=round,>=triangle 45,scale=1.5]
\draw[->,color=black] (-1.1,0) -- (7.03,0);
\draw[->,color=black] (0,-0.34) -- (0,4.89);

\clip(-1.1,-0.34) rectangle (7.03,4.89);
\fill[color=ttfftt,fill=ttfftt,fill opacity=0.5] (2,2) -- (1.33,3.75) -- (2.83,3.21) -- (3.51,1.46) -- cycle;
\fill[color=ttttff,fill=ttttff,fill opacity=0.1] (2,2) -- (1.33,3.75) -- (3.59,2.94) -- (4.26,1.19) -- cycle;
\draw [->] (2,2) -- (3.51,1.46);
\draw [->] (2,2) -- (1.33,3.75);
\draw [->] (3.51,1.46) -- (2.83,3.21);
\draw [->] (1.33,3.75) -- (2.83,3.21);
\draw [->] (2,2) -- (4.26,1.19);
\draw [->] (1.33,3.75) -- (3.59,2.94);
\draw [->] (4.26,1.19) -- (3.59,2.94);
\draw [color=ttfftt] (2,2)-- (1.33,3.75);
\draw [color=ttfftt] (1.33,3.75)-- (2.83,3.21);
\draw [color=ttfftt] (2.83,3.21)-- (3.51,1.46);
\draw [color=ttfftt] (3.51,1.46)-- (2,2);
\draw [color=ttttff] (2,2)-- (1.33,3.75);
\draw [color=ttttff] (1.33,3.75)-- (3.59,2.94);
\draw [color=ttttff] (3.59,2.94)-- (4.26,1.19);
\draw [color=ttttff] (4.26,1.19)-- (2,2);
\begin{scriptsize}
\fill [color=qqqqff] (2,2) circle (1.5pt);
\fill [color=qqqqff] (1.33,3.75) circle (1.5pt);
\fill [color=qqqqff] (3.51,1.46) circle (1.5pt);
\draw[color=black] (2.79,1.84) node {$v$};
\draw[color=black] (1.8,2.91) node {$w$};
\fill [color=uququq] (2.83,3.21) circle (1.5pt);
\fill [color=uququq] (4.26,1.19) circle (1.5pt);
\fill [color=uququq] (3.59,2.94) circle (1.5pt);
\draw[color=ttttff] (3.65,1.31) node {$vx$};
\end{scriptsize}
\end{tikzpicture}

%-------------------Begin Addition mit einem Vektor ----------------  
\begin{figure}[H]\centering
\tdplotsetmaincoords{0}{0} %-27
\begin{tikzpicture}[yscale=1,tdplot_main_coords]

\def\xstart{0} %x Koordinate der Startposition der Grafik
\def\ystart{0} %y Koordinate der Startposition der Grafik
\def\myscale{0.9} %ändert die Größe der Grafik (Skalierung der Grafik) 

\def\xstartdraw{(\xstart + 1.5)} %xKoordinate des Referenzstartpunktes (in dieser Zeichnung: a)
\def\ystartdraw{(\ystart + 3.5)}%yKoordinate des Referenzstartpunktes (in dieser Zeichnung: a)

\def\balkenhoehe{(5.3)}% Länge des vertikalen blauen Balkens
\def\balkenlaenge{(10)}% Länge des horizontalen blauen Balkens
\def\balkenbreite{0.4} %Balkenbreite

%---------Begin Balken----------
\def\drehwinkel{0}
\node (VekV) at ({\xstart+0.7*cos(\drehwinkel)-\balkenbreite*sin(\drehwinkel)},{\ystart+0.5*sin(\drehwinkel)+\balkenbreite*cos(\drehwinkel)})[right, xshift=1,color=blue] {$V=\mathbb{R}^2$};
\node (AffA) at ({\xstart+(\balkenlaenge-1)*cos(\drehwinkel)},{\ystart+(\balkenlaenge-1)*sin(\drehwinkel)+\balkenbreite*cos(\drehwinkel)})[color=red] {$A$};

\path[ shade, top color=white, bottom color=blue, opacity=.6] 
    ({\xstart},{\ystart},0)  -- ({\xstart - \balkenbreite * cos(\drehwinkel)- (-\balkenbreite+0)*sin(\drehwinkel)},{\ystart - \balkenbreite * sin(\drehwinkel)+ (-\balkenbreite+0)*cos(\drehwinkel)},0)  -- ({\xstart - \balkenbreite * cos(\drehwinkel)- (\balkenhoehe+0.5)*sin(\drehwinkel)},{\ystart - \balkenbreite * sin(\drehwinkel)+ (\balkenhoehe+0.5)*cos(\drehwinkel)},0) -- ({\xstart - 0 * cos(\drehwinkel)- (\balkenhoehe+0)*sin(\drehwinkel)},{\ystart - 0 * sin(\drehwinkel)+ (\balkenhoehe+0)*cos(\drehwinkel)},0) -- cycle;
        
\path[ shade, right color=white, left color=blue, opacity=.6] 
	({\xstart},{\ystart},0)  -- ({\xstart - \balkenbreite * cos(\drehwinkel)- (-\balkenbreite+0)*sin(\drehwinkel)},{\ystart - \balkenbreite * sin(\drehwinkel)+ (-\balkenbreite+0)*cos(\drehwinkel)},0) --
    ({\xstart + (\balkenlaenge+0.5) * cos(\drehwinkel)- (-\balkenbreite+0)*sin(\drehwinkel)},{\ystart + (\balkenlaenge+0.5) * sin(\drehwinkel)+ (-\balkenbreite+0)*cos(\drehwinkel)},0) --   
    ({\xstart + \balkenlaenge * cos(\drehwinkel)},{\ystart + \balkenlaenge * sin(\drehwinkel)},0)--
    cycle;       
%---------End Balken----------
\def\lightoffset{0.2*\myscale} %offeset der Vektoren

%Punkte Definition
\node (pointa1) at ({\xstartdraw},{\ystartdraw}) {};
\node (pointa2) at ({\xstartdraw+(1 *\myscale)},{\ystartdraw-(2.0*\myscale)}) {};
\node (pointb1) at ($(pointa1) + (3.0*\myscale,-1.0*\myscale) $) {};
\node (pointb2) at ($(pointb1) + (1.0*\myscale,-2.0*\myscale) $) {};

\node (pointc1) at ($(pointa1) + (6.5*\myscale,1.3*\myscale) $) {};
\node (pointc2) at ($(pointa2) + (6.5*\myscale,1.3*\myscale) $) {};

\node (pointFvwi) at ($(pointb2) + (-0.9*\myscale,0.9*\myscale) $) {};
\node (pointFvwa) at ($(pointb2) + (-1.5*\myscale,-0.3*\myscale) $) {};

\node (pointFvswi) at ($(pointc2) + (-2.9*\myscale,-1.2*\myscale) $) {};
\node (pointFvswa) at ($(pointc2) + (-1.3*\myscale,-1.9*\myscale) $) {};

\node (pointfgi) at ($(pointa1) + (2.2*\myscale,-0.2*\myscale) $) {};
\node (pointfga) at ($(pointfgi) + (-1.4*\myscale,1.8*\myscale) $) {};



%Flächen füllen
%blaue Flaeche
\fill[color=ttttff,fill=ttttff,fill opacity=0.15] (pointa1.center) -- (pointb1.center) -- (pointc1.center) -- (pointc2.center) -- (pointb2.center)-- (pointa2.center)-- cycle;
%gruene Flaeche
\fill[color=ttfftt,fill=ttfftt,fill opacity=0.5] (pointa1.center) -- (pointc1.center) -- (pointc2.center) -- (pointa2.center)-- cycle;

%Vektoren blau
\draw[-{>[scale=1,length=10,width=6]},shorten >=2pt, shorten <=2pt,line width=0.2pt,color=blue] (pointa1) -- (pointb1);
\draw[-{>[scale=1,length=10,width=6]},shorten >=2pt, shorten <=2pt,line width=0.2pt,color=blue] (pointa2) -- (pointb2);
\node [color=blue] (pointlabelg1) at ($(pointa1)!0.5!(pointb1)$) [above, xshift=0, yshift=0] {$v$} ;
\node [color=blue] (pointlabelg2) at ($(pointa2)!0.5!(pointb2)$) [above, xshift=0, yshift=0] {$v$} ;

\draw[-{>[scale=1,length=10,width=6]},shorten >=2pt, shorten <=2pt,line width=0.2pt,color=blue] (pointb1) -- (pointc1);
\draw[-{>[scale=1,length=10,width=6]},shorten >=2pt, shorten <=2pt,line width=0.2pt,color=blue] (pointb2) -- (pointc2);
\node [color=blue] (pointlabelg3) at ($(pointb1)!0.5!(pointc1)$) [above, xshift=0, yshift=0] {$v'$} ;
\node [color=blue] (pointlabelg4) at ($(pointb2)!0.5!(pointc2)$) [above, xshift=0, yshift=0] {$v'$} ;

\draw[-{>[scale=1,length=10,width=6]},shorten >=2pt, shorten <=2pt,line width=0.2pt,color=blue] (pointa2) -- (pointa1);
\draw[-{>[scale=1,length=10,width=6]},shorten >=2pt, shorten <=2pt,line width=0.2pt,color=blue] (pointb2) -- (pointb1);
\draw[-{>[scale=1,length=10,width=6]},shorten >=2pt, shorten <=2pt,line width=0.2pt,color=blue] (pointc2) -- (pointc1);

\node [color=blue] (pointlabelga2a1) at ($(pointa2)!0.5!(pointa1)$) [left, xshift=0, yshift=0] {$w$} ;
\node [color=blue] (pointlabelgb2b1) at ($(pointb2)!0.5!(pointb1)$) [right, xshift=0, yshift=0] {$w$} ;
\node [color=blue] (pointlabelgc2c1) at ($(pointc2)!0.5!(pointc1)$) [right, xshift=0, yshift=0] {$w$} ;

%Vektoren gruen
\draw[-{>[scale=1,length=10,width=6]},shorten >=4pt, shorten <=4pt,line width=0.2pt,color=green] (pointa1) -- (pointc1);
\draw[-{>[scale=1,length=10,width=6]},shorten >=4pt, shorten <=4pt,line width=0.2pt,color=green] (pointa2) -- (pointc2);
\node [color=green] (pointlabelac1) at ($(pointa1)!0.5!(pointc1)$) [above, xshift=0, yshift=0] {$v+v'$} ;
\node [color=green] (pointlabelac2) at ($(pointa2)!0.5!(pointc2)$) [above, xshift=0.5, yshift=0] {$v+v'$} ;

%Punkte malen
\draw[fill,color=red] (pointa1) circle [x=1cm,y=1cm,radius=0.08]node[above, xshift=0, yshift=0]{};
\draw[fill,color=red] (pointb1) circle [x=1cm,y=1cm,radius=0.08]node[above, xshift=0, yshift=0]{};
\draw[fill,color=red] (pointa2) circle [x=1cm,y=1cm,radius=0.08]node[below, xshift=5, yshift=0]{};
\draw[fill,color=red] (pointb2) circle [x=1cm,y=1cm,radius=0.08]node[below, xshift=5, yshift=0]{};
\draw[fill,color=red] (pointc1) circle [x=1cm,y=1cm,radius=0.08]node[below, xshift=5, yshift=0]{};
\draw[fill,color=red] (pointc2) circle [x=1cm,y=1cm,radius=0.08]node[below, xshift=5, yshift=0]{};

\draw[->,shorten >=2pt, shorten <=2pt,line width=0.2pt,color=blue] (pointFvwa) -- (pointFvwi);
\draw[->,shorten >=2pt, shorten <=2pt,line width=0.2pt,color=blue] (pointFvswa) -- (pointFvswi);
\draw[->,shorten >=2pt, shorten <=2pt,line width=0.2pt,color=green] (pointfga) -- (pointfgi);

\node [color=blue] (pointlabelFvwl) at (pointFvwa) [xshift=0.5, yshift=-0.5] {$F(v,w)$} ;
\node [color=blue] (pointlabelFvswl) at (pointFvswa) [xshift=-0.5,  yshift=-5] {$F(v',w)$} ;
\node [color=green] (pointlabelFvswl) at (pointfga) [xshift=35 ] {$F(v+v',w)=\textcolor{blue}{F(v,w)+F(v',w)}$} ;

\end{tikzpicture}
\end{figure}
%-------------------End Addition mit einem Vektor ----------------

\subsection{Definition}
	\begin{Definition}[Linearform/Determinantenform]
	Sei $ V $ ein $ K $-VR. Eine Abbildung $ \omega:V^m\to K $ heißt
		\begin{itemize}
		\item \emph{$ m $-linear}, bzw. eine \emph{$ m $-(Linear-)Form}, falls $ \omega $ in jedem Argument linear ist, d.h.
			\[ \forall i=1,\dots, m: V\ni v_i\mapsto \omega(v_1,\dots,v_{i-1},v_i,v_{i+1},\dots,v_m)\in K \]
			ist linear;
		\item \emph{alternierend}, falls $ \omega(v_1,\dots,v_m)=0 $ wann immer zwei Vektoren gleich sind, d.h.
			\[ v_i = v_j \text{ für } i\neq j \Rightarrow \omega(v_1,\dots, v_m) = 0. \] 
		\end{itemize}
	Die Menge der alternierenden $ m $-Formen wird mit $ \Lambda^mV^* $ bezeichnet. Ist $ \dim V = n $, so heißt ein $ \omega\in \Lambda^nV^* $ auch Determinantenform.
	\end{Definition}

\paragraph{Beispiel}
	Jede Linearform $ \omega\in V^* $ ist eine (alternierende) 1-Form, $ \Lambda^1V^*=V^* $.
\paragraph{Bemerkung}
	$ \Lambda^mV^* $ ist für jedes $ m\in \mathbb{N} $ selbst ein $ K $-VR. 
\subsection{Lemma}
	\begin{Lemma}
		Für eine alternierende $ m $-Form $ \omega \in \Lambda^mV^* $ und $ i\neq j $ gilt:
		\begin{enumerate}[(i)]
			\item $ \omega(\dots,v_i,\dots,v_j,\dots) = -\omega (\dots,v_j,\dots,v_i,\dots)$;
			\item $ \omega(\dots,v_i,\dots,v_is+v_j,\dots) = \omega(\dots,v_i,\dots,v_j,\dots) $ für $ s\in K $;
			\item und $ \omega(v_1,\dots,v_m)=0 $, falls $ (v_i)_{i\in \{1,\dots,m\}} $ linear abhängig ist.
		\end{enumerate}
\paragraph{Beweis}
	Seien $ v_1,\dots,v_m\in V $ und $ i,j\in \{1,\dots,m\} $ mit $ i\neq j $. Dann gilt:
		\begin{align*}
			0 &= \omega(\dots,v_i+v_j,\dots,v_i+v_j,\dots)\\
			&= \omega(\dots,v_i,\dots,v_i,\dots)+\omega(\dots,v_j,\dots,v_j)+\omega(\dots,v_i,\dots,v_j,\dots)+\omega(\dots,v_j,\dots,v_i,\dots)\\
			&= \omega(\dots,v_i,\dots,v_j,\dots)+\omega(\dots,v_j,\dots,v_i,\dots)
		\end{align*}
	und
		\begin{align*}
		0&=\omega(\dots,v_i,\dots,v_is,\dots)\\
		&= \omega(\dots,v_i,\dots,v_is+v_j-v_j,\dots)\\
		&= \omega(\dots,v_i,\dots,v_is+v_j,\dots)-\omega(\dots,v_i,\dots,v_j,\dots)
		\end{align*}
	Dies beweist (i) und (ii). Ist die Familie $ (v_i)_{i\in \{1,\dots,m\}} $ linear abhängig, o.B.d.A
		\[ v_m = \sum_{i=1}^{m-1}v_ix_i \in [(v_i)_{i\in \{1,\dots,m-1\}}] \]
	so gilt
		\begin{gather*}
		\omega(v_1,\dots,v_m)=\omega(v_1,\dots,v_{m-1},\sum_{i=1}^{m-1}v_ix_i)\\
		= \sum_{i=1}^{m-1}\omega(v_1,\dots,v_{m-1},v_i)x_i (=0)
		\end{gather*}
	womit (iii) bewiesen ist.
	\end{Lemma}
\paragraph{Bemerkung}
	(i) liefert eine äquivalente Formulierung von "`alternierend"' für $ m $-Linearformen, wenn $ \Char (K)\neq 2 $.
	Nämlich: sind $ v_1,\dots,v_m\in V $ mit $ v_i=v_j $ für $ i\neq j $, so gilt
		\begin{align*}
		0 &= \omega(\dots,v_i\dots,v_j)+\omega(\dots,v_j,\dots,v_i,\dots)\\
		&= 2\omega(\dots,v_i,\dots,v_j,\dots) \Rightarrow 0 = \omega(\dots,v_i,\dots,v_j,\dots)
		\end{align*}
\paragraph{Buchhaltung}
	Benutzt man (vgl. Gausssches Eliminationsverfahren) die Elementarmatrizen
	\begin{gather*}
		D_i = (d_{kl}) \in Gl(m); d_{kl} = \delta_{kl}+(d-1)\delta_{ik}\delta_{il}\quad (d\in K^\times);\\
		T_{ij} = (t_{kl}) \in Gl(m); t_{kl} = \delta_{kl}-(\delta_{ik}-\delta_{jk})(\delta_{il}-\delta_{jl});\\
		S_{ij} = (s_{kl})\in Gl(m); s_{kl}=\delta_{kl}+s\delta_{ik}\delta_{jl}\quad (s\in K)
	\end{gather*}
	und beschreibt man eine Familie $ (v_i)_{i\in \{1,\dots,m\}} $ von Vektoren $ v_i\in V $ durch ein $ m $-Tupel $ A=(v_1,\dots,v_m) $ von Werten der Familie, so lassen sich die Homogenität und Eigenschaften (i) und (ii) des Lemmas einfach schreiben als
		\[ \omega(AD_i) = \omega(A)d,\ \omega(AT_{ij}) = -\omega(A),\ \omega(AS_{ij}(s)) = \omega(A) \]
\subsection{Wiederholung \& Definition}
	Die bijektiven Abbildungen (Permutationen)
		\[ \sigma: I\to I, i\mapsto \sigma(i),\text{ der Menge }I = \{1,\dots,m\} \]
	bilden eine Gruppe (mit der Komposition als Verknüpfung), die Permutationsgruppe $ S_m $ der Menge $ I $.
	\begin{Definition}[Transposition]
	Eine Transposition $ \tau_{ij} \in S_m, i\neq j $ ist eine Permutation, die zwei Indizes vertauscht,
		\[ \tau_{ij}:I\to I, k\mapsto \tau_{ij}(k):=
		\begin{cases}
			j, &\text{falls } k=i,\\
			i, &\text{falls } k=j,\\
			k & \text{sonst}.
		\end{cases} \]
	Jede Permutation ist eine Komposition von Transpositionen, wie man leicht durch Induktion über m zeigt:
	
	Ist $ \sigma(m) = i<m$, so ist $ \tau_{im}\circ \sigma $ eine Permutation, die $ m $ fixiert, also
		\[ \tau_{im}\circ\sigma\mid_{\{1,\dots,m-1\}}\in S_{m-1} \]
	\end{Definition}
\paragraph{Bemerkung}
	Die Eigenschaft (i) des Lemmas, $ \omega(AT_{ij})=-\omega(A) $, lässt sich mit $ \tau_{ij} $ dann formulieren als
		\[ \omega(v_{\tau_{ij}(1)}, \dots, v_{\tau_{ij}(m)}) = - \omega(v_1,\dots,v_m) \]
	Da jede Permutation $ \sigma\in S_m $ Komposition von Transpositionen ist, folgt
		\[ \forall \sigma\in S_m: \omega(v_{sigma}(1),\dots,v_{sigma(m)})=\pm \omega(v_1,\dots,v_{m}). \]
	Frage: Was ist das Vorzeichen bzw. wie kann man es berechnen?
\subsection{Lemma \& Definition}
	\begin{Definition}[Signum einer Permutation]
	Das Signum einer Permutation $ \sigma\in S_m $ ist die Zahl
		\[ \operatorname{sgn}\sigma := \prod_{i<j} \frac{\sigma(i)-\sigma(j)}{i-j}\in \{\pm 1\}; \]
	ist $ \operatorname{sgn}\sigma = 1 $, so heißt $ \sigma $ gerade, sonst ungerade. Signum liefert einen Gruppenhomomorphismus
		\[ \operatorname{sgn}: S_m\to (\{\pm 1\},\cdot). \]
	\end{Definition}
\paragraph{Beispiel}
	Eine Transposition $ \tau_{ij} $ ist eine ungerade Permutation, da
		\[ \operatorname{sgn}\tau_{ij} = \prod_{k<l}\frac{\tau_{ij}(k)-\tau_{ij}(l)}{k-l} = \frac{j-i}{i-j}\prod_{k\neq i,j}\frac{i-k}{j-k}\frac{j-k}{i-k} = -1 \]
\section{Äquiaffine Geometrie}
\subsection{Definition}
	\begin{Definition}[Parallelotop im affinen Raum]
		Seien $ (A,V,\tau) $ ein $ n $-dimensionaler affiner Raum und $ p_0\in A $; das von einer Familie $ (v_j)_{j\in \{1,\dots,n\}} $ in $ V $ aufgespannte Parallelotop oder Spat ist die (über dem abstrakten Würfel $ \mathbb{Z}_2^n $ indizierte) Familie
			\[ p: \mathbb{Z}_2^n\to A, \epsilon \mapsto p_\epsilon := p_0 +\sum_{j=1}^{n}v_j\epsilon_j. \]
		Ist $ \omega\in\Lambda^nV^*\setminus\{0\} $, so ist das zugehörige (Spat-)Volumen von $ (p_\epsilon)_{\epsilon \in \mathbb{Z}_2^n} $
			\[ \vol(p):= \omega(v_1,\dots,v_n). \]
	\end{Definition}
\subsection{Bemerkung \& Definition}
	\begin{Definition}[Parallelogramm/Flächeninhalt]
		Im Falle $ n=2 $ heißt ein Parallelotop $ p $ auch Parallelogramm, sein Spatvolumen auch sein Flächeninhalt. Der Flächeninhalt ist orientiert:
		Mit
			\[ \epsilon = (\epsilon_j)_{j\in \{1,2\}}\cong (\epsilon_1,\epsilon_2) \]
		und
			\[ v_1 = p_{(1,0)}-p_{(0,0)}\text{ und } v_2 = p_{(0,1)}-p_{(0,0)} \]
		ändert der Flächeninhalt das Vorzeichen, wenn man die Kantenvektoren vertauscht:
			\[ \vol(p) = \omega(v_1,v_2) = -\omega(v_2,v_1) = \vol(p') \]
		mit $ p'_\epsilon = p_\epsilon \circ \tau_{12} $.
	\end{Definition}
\printindex
\end{document}
