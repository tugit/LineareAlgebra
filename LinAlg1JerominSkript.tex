\documentclass[12pt,a4paper,parskip=half-,DIV=15]{scrartcl}
\usepackage[utf8]{inputenc}
\usepackage[T1]{fontenc}
\usepackage{lmodern}%schoeneres Schriftbild
\usepackage[ngerman]{babel}%deutsche Silbentrennung
%\usepackage{tikz}%Zeichnungen
%\usetikzlibrary{arrows}
\usepackage{amsmath,amsfonts,amssymb}%Mathematik-Pakete
\usepackage{graphicx}%?????
\usepackage{paralist}%enumerate mit roemischen Zahlen
\usepackage{float}%fuer H Positionierung

\author{Christoph Fritz, djangonightfall, puenka} 
\title{Skript Lineare Algebra \& Geometrie 1, Hertrich-Jeromin}

\setcounter{section}{-1}%Grundlagen bei 0

\begin{document}
\maketitle
\tableofcontents

\section{Grundlagen}

\subsection{Vorbereitungen, logische Symbolik}
Es existieren zwei Methoden zur präzisen Formulierung:
\begin{itemize}
\item Funktion einer Formulierung wird präzisiert durch: 
	\begin{itemize}
	\item Definition: Begriffsklärung
	\item Satz (Lemma, Proposition, Korollar): Aussage über einen (mathematischen) Sachverhalt
	\item Beweis : eine (logische) Argumentationskette, die erklärt, warum ein Satz/Lemma wahr ist
	\item Bemerkung, Beispiel: zusätzliche Information/Illustration, die oft Eigenarbeit (Beweis) erfordert
	\end{itemize}
\item Formeln und (logische) Symbole werden verwendet:
	\begin{itemize}
	\item $\forall$ -- All-Quantor: \glqq für alle\grqq
	\item $\exists(!)$ -- Existenz-Quantor: \glqq es existiert (genau) ein\grqq
	\item $\lnot$ -- logische Verneinung: $\lnot A$ ist wahr, wenn $A$ falsch ist
	\item $\land ,\lor$ -- logisches \glqq und\grqq{} und \glqq oder\grqq
	\item $\Rightarrow ,\Leftrightarrow$ -- Implikation und Äquivalenz
	\end{itemize}
\end{itemize}

\begin{figure}[H]\centering
\begin{tabular}{c|c|c|c|c|c|c}
$A$ & $B$ & $\lnot A$ & $A\land B$ &$A\lor B$&$A \Rightarrow B$ & $A\Leftrightarrow B$\\\hline
w & w & f & w & w & w & w\\
w & f & f & f & w & f & f\\
f & w & w & f & w & w & f\\
f & f & w & f & f & w & w\\
\end{tabular}
\caption{Wahrheitstafel}
\end{figure}

Beispiele:
\begin{itemize}
\item Implikation: Für $x,y\in\mathbb{R}: xy = 0 \Rightarrow (x = 0\lor y = 0)$
\item Für Aussagen $ A $ und $ B $ gilt: $(A\Rightarrow B)\Leftrightarrow (\lnot A \lor B)$, Beweis durch Wahrheitstafel
\end{itemize}
\begin{figure}[H]\centering
\begin{tabular}{c|c|c|c|c|c}
$A$ & $B$ & $\lnot A$ & $\lnot A\lor B$ & $A \Rightarrow B$ & $(A\Rightarrow B)\Leftrightarrow (\lnot A \lor B)$\\\hline
w & w & f & w & w & w \\
w & f & f & f & f & w \\
f & w & w & w & w & w \\
f & f & w & w & w & w \\
\end{tabular}
\caption{Beweis durch Wahrheitstafel}
\end{figure}

\paragraph{Bemerkung (Kommutativität):} $\land$, $\lor$, und $\Leftrightarrow$ sind kommutativ (symmetisch), $\Rightarrow$ jedoch nicht, d.h.:
\begin{align*}
(A\land B)&\Leftrightarrow (B\land A)\\
(A\lor B)&\Leftrightarrow (B\lor A)\\
(A\Leftrightarrow B)&\Leftrightarrow (B\Leftrightarrow A)\\
(A\Rightarrow B)&\nLeftrightarrow (B\Rightarrow A)\\
\end{align*}
weil beispielsweise formal gilt: $x,y\in\mathbb{R}: x = 0 \Rightarrow xy = 0$, aber nicht $xy = 0 \Rightarrow x = 0$.

\paragraph{Bemerkung (Beweisformen der Implikation):}
Um eine Implikation $A\Rightarrow B$ zu zeigen, bedient man sich häufig auch folgender Äquivalenzen:
\begin{equation*}
(A\Rightarrow B)\Leftrightarrow
\begin{cases}
\lnot B\Rightarrow \lnot A&\text{(Indirekter Schluss)}\\
\lnot (A\land \lnot B)&\text{(Widerspruchsbeweis)}
\end{cases}\end{equation*}

\paragraph{Beispiel} Für reelle Zahlen $x,y\in\mathbb{R}$ gilt:
\begin{equation*}
\left((xy = 0)\Rightarrow (x=0 \lor y=0)\right) \Leftrightarrow \left((xy=0 \land x \neq 0)\Rightarrow (y =0)\right)
\end{equation*}
bzw. allgemein:
\begin{equation*}
(A\Rightarrow (B\lor C))\Leftrightarrow ((A\land\lnot B)\Rightarrow C)
\end{equation*}

\paragraph{Bemerkung (Mengenlehre)} Die Ähnlichkeit mit der Mengensymbolik ist nicht zufällig, z.B. Mengen $X, Y$:
\begin{align*}
(x\in X\cap Y)&\Leftrightarrow (x\in X\land x\in Y)\\
(x\in X\cup Y)&\Leftrightarrow (x\in X\lor x\in Y)\\
(X\subset Y) \Leftrightarrow \{&\forall x : (x\in X \Rightarrow x\in Y)\}
\end{align*}

\subsection{Abbildungen}
\paragraph{Definition:} Eine Zuordnung $f: X\to Y$ zwischen zwei Mengen $X$ und $Y$ heißt eine Abbildung, falls $\forall x\in X: \exists ! y\in Y: y=f(x)$.

X heißt der Definitionsbereich der Abbildung und $f(X):=\{f(x)\mid x\in X \}\subseteq Y$ das Bild.

Eine Abbildung $f: X\to Y$ heißt
\begin{itemize}
\item injektiv, falls $\forall x,x'\in X:f(x) = f(x') \Rightarrow x=x'$
\item surjektiv, falls $\forall y\in Y:\exists x\in X: y = f(x)$
\item bijektiv, falls $\forall y\in Y:\exists !x\in X: y = f(x)$
\end{itemize}

\paragraph{Beispiel} Mit $X=Y=\mathbb{R}$ definiert
\begin{itemize}
\item die Relation $x^2 = y$ eine Abbildung $f:X\to Y, x\mapsto f(x)=x^2$
\item die Relation $x=y^2$ keine Abbildung $f:X\to Y$, denn
	\begin{itemize}
	\item für ein $x$ gibt es zwei $y$-Werte
	\item $x < 0$ ist nicht definiert
	\end{itemize}
\end{itemize}

\paragraph{Beispiel} Die Identität $id_X :X\to X, x\mapsto id_X(x):= x$ ist eine bijektive Abbildung.
\paragraph{Bemerkung} Eine Abbildung ist genau dann bijektiv, wenn sie injektiv und surjektiv ist.
\paragraph{Definition} Sind $ f:X\to Y $ und $ g:Y\to Z$ Abbildungen, so ist ihre Komposition/Verkettung die Abbildung $ g\circ f:X\to Z, x\mapsto (g\circ f)(x):= g(f(x)) $.
\paragraph{Beispiel} Seien $ X = Y = Z = \mathbb{R} $ und $ f:X\to Y, x\mapsto f(x) :=x^2 $, $ g:Y\to Z, y\mapsto g(x):=y^3 + y $, so ist die Verkettung $ g\circ f: X\to Z, x\mapsto (g\circ f)(x) = (x^2)^3+x^2 = x^6 + x^2 $.

\subsection{Inverse}
\paragraph{Lemma} Seien $ f:X\to Y $ und $ g:Y\to X $ Abbildungen. Dann gilt:
\begin{enumerate}[i)]
\item ist $ g $ Linksinverse von $ f $, d.h. $ g\circ f = id_X $, so ist f injektiv
\item ist $ g $ Rechtsinverse von $ f $, d.h. $ f\circ g = id_Y$, so ist f surjektiv
\item ist $ g $ Links- und Rechtsinverse von $ f $, so heißt $ g =f^{-1}$ Inverse von $ f $
\end{enumerate}

\paragraph{Beispiel} $ f:\mathbb{N}\to \mathbb{N}, n\mapsto f(n):= n+1 $ hat Linksinverse 
\begin{equation*}
g:\mathbb{N} \to \mathbb{N}, n\mapsto g(n):=
\begin{cases}
15700, & \text{falls } n=0\\
n-1, & \text{falls } n\neq 0
\end{cases} 
\end{equation*}

Tatsächlich ist $ f $ injektiv, da
\begin{equation*}
\forall n,n'\in \mathbb{N} : n+1 = f(n) = f(n') = n'+1 \Rightarrow n=n'
\end{equation*}
jedoch $ f(\mathbb{N}) = \mathbb{N}\setminus \{0\} $, daher kann keine Rechtsinverse existieren.

\paragraph{Beweis (Lemma)} Zwei Aussagen sind zu beweisen:
\begin{enumerate}[i)]
\item Sei $ g $ Linksinverse von $ f $. Dann gilt für $ x,x'\in X $ mit \\$ f(x) = f(x'): x = g(f(x)) = g(f(x')) = x' $, also ist $ f $ injektiv.
\item Sei $g $ Rechtsinverse von $ f $ und $ y\in Y $. Setze $ x:= g(y)\in X $, dann gilt f(x) = f(g(y)) = y. Damit existiert zu jedem $ y\in Y $ (mindestens) ein $ x = g(y) $, sodass  $ y=f(x) $.
\end{enumerate}

%%%%%%%%%%%%%%%%%%%%%%%%%%%%%%%%%%%%%%%%%%%%%%
%%%%%%%%%%%%%%SECTION-ENDE%%%%%%%%%%%%%%%%%%%%
%%%%%%%%%%%%%%%%%%%%%%%%%%%%%%%%%%%%%%%%%%%%%%

\section{Lineare Räume und Abbildungen}
\subsection{Von Geometrie zu Algebra}
Euklids führte in den \glqq Elementen\grqq{} (ca. 300 v. Chr.) das bis heute gültige Schema ein:
\begin{itemize}
\item Definition
\item Axiom/Postulat
\item Lehrsatz
\item Beweis
\end{itemize}

\paragraph{Parallelenaxiom/-problem (Euklid, Formulierung nach Playfair)} Es existiert genau eine Parallele $ g' $ zum Punkt $ P \notin g $ zur Geraden $ g $.

Kann das Axiom aus den anderen Axiomen hergeleitet/bewiesen werden? Nein, denn es existieren nichteuklidische, hyperbolische Geometrien (18. Jh.) in denen es mehrere derartige Parallelen gibt. Als Beispiel lässt sich eine Geometrie anführen, die nicht auf einer Ebene sondern auf einem Kreis operiert. Dort lassen sich zu einer Sekante mehrere parallele Sekanten betrachten (also Sekanten, die die ursprüngliche nicht schneiden).

%\begin{figure}[H]
%\begin{minipage}{.45\textwidth}
%\begin{tikzpicture}[line cap=round,line join=round,>=triangle 45,x=1.0cm,y=1.0cm]
%\clip(-1.69,-0.64) rectangle (4.14,2.83);
%\draw [domain=-1.69:4.14] plot(\x,{(-1--1*\x)/1});
%\draw [domain=-1.69:4.14] plot(\x,{(-0--1*\x)/1});
%\draw (0.6,1) node[] {P};
%\draw (1.58,0.16) node[] {g};
%\draw (1.52,1.78) node[] {g'};
%\begin{scriptsize}
%\fill [color=blue] (1,1) circle (2pt);
%\end{scriptsize}
%\end{tikzpicture}
%\end{minipage}
%\begin{minipage}{.45\textwidth}
%\begin{tikzpicture}[line cap=round,line join=round,>=triangle 45,x=1.0cm,y=1.0cm]
%\clip(-2.24,-3.38) rectangle (3.15,1.76);
%\draw(0,0) circle (1cm);
%\draw (-0.94,0.35)-- (0.66,0.75);
%\draw (-0.13,0.74) node[] {g};
%\draw (0.32,-0.56) node[] {P};
%\draw (-1,-0.02)-- (0.88,-0.48);
%\draw (-0.35,-0.94)-- (0.91,0.42);
%\begin{scriptsize}
%\fill [color=blue] (0.22,-0.32) circle (1.5pt);
%\end{scriptsize}
%\end{tikzpicture}
%\end{minipage}
%\end{figure}

\paragraph{Was ist eine Geometrie?} Eine Geometrie ist durch eine Menge X und eine auf X operierende Transformationsgruppe gegeben.
%%%%%%%%%%%%%%%%% BEGINN VO3-20151013 %%%%%%%%%%%%%%%%%%%%%

\paragraph{Definition:} Ein Paar $(G,\circ)$ bestehend aus einer Menge $G$ und einer Verknüpfung $(\circ : G\times G \to G) : (g,h) \mapsto g \circ h$ heißt Gruppe, falls:

\begin{enumerate}[(i)]
\item $\forall f,g,h\in G : f\circ (g\circ h) = (f\circ g)\circ h$ (Assoziativität)
\item $\exists e\in G\forall g\in G : e\circ g = g$ (Existenz eines neutralen Elements)
\item $\forall g \in G \exists g^{-1} \in G : g^{-1}\circ g = e$ (Existenz eines inversen Elements)
\end{enumerate}

Die Gruppe heißt kommutativ oder abelsch, falls zusätzlich gilt:
\begin{equation*}
\forall g,h\in G: g\circ h = h\circ g \text{ (Kommutativität)} 
\end{equation*}

\paragraph{Bemerkung:} Das ist eine axiomatische Definition, d.h. der Begriff \glqq Gruppe\grqq{} wird durch (aus vielen (!) Beispielen abstrahierten) \glqq Rechenregeln\grqq{} definiert.
\paragraph{Beispiel:} Die rationalen Zahlen $\mathbb{Q}$ bilden mit der Addition eine Gruppe $(\mathbb{Q} ,+)$.\\
Die rationalen Zahlen ohne $0$, $\mathbb{Q}^{\times} := \mathbb{Q}\setminus \{D\}$, bilden mit der Multiplikation eine Gruppe $(\mathbb{Q}^\times ,\cdot)$.

\paragraph{Definition:} Sind $(G,\circ )$ eine Gruppe und $X$ eine Menge, so heißt eine Abbildung 
\begin{equation*}\cdot : G\times X\to X, (g,x)\mapsto g\cdot x\end{equation*}
eine Gruppenoperation (von $(G,\circ )$ auf $X$), falls
\begin{enumerate}[(i)]
\item $\forall g,h\in G :\forall x\in X: g\cdot (h\cdot x) = (g\circ h)\cdot x$ (entspricht nicht der Assoziativität!)
\item $\forall x\in X: e\cdot x = x$ für das neutrale Element $e$ der Gruppe $(G,\circ )$
\end{enumerate}
$(G,\circ )$ heißt dann Transformationsgruppe von X.

\paragraph{Bemerkung:} Operiert $G$ (kurz für $(G,\circ )$, aus dem Zusammenhang ersichtlich) auf $X$, so ist für jedes $g\in G$ die Abbildung $g:X\to X, x\mapsto g\cdot x$ eine bijektive Abbildung von $X$ auf sich. Wegen der Axiome (i) und (ii) aus der Definition erhält man $g^{-1}: X\to X$ als Inverse der Abbildung.
\paragraph{Beispiel und Definition:} Die bijektiven Abbildungen einer Menge $X$ auf sich, $G:= \{g:X\to X\mid g \text{ bij}\}$, bilden (mit der Komposition $\circ$) eine (Transformations-)Gruppe $(G,\circ )$ (die auf $X$ operiert): die Permutationsgruppe oder symmetrische Gruppe $S_X$ von $X$. Für $X=\{1,2,...,n\}$ schreibt man auch $S_n$ statt $S_{\{1,...,n\}}$.
\paragraph{Bemerkung:} Im Gegensatz zu allgemeinen Abbildungen stimmen in (Permutations-)Gruppen Links- und Rechtsinverse stets überein.
\paragraph{Lemma:} Das neutrale Element einer Gruppe $(G,\circ )$ ist eindeutig und $\forall g\in G: g\circ e = g$. Weiters: \begin{equation*}\forall g\in G \exists ! g^{-1} \in G: g^{-1}\circ g = g \circ g^{-1} = e\end{equation*}

\paragraph{Beweis:} Sei $g\in G$ gegeben und (gemäß Gruppenaxiom (iii)):
\begin{itemize}
\item $h:= g^{-1}$ (Linksinverse von $g$)
\item $k:= h^{-1}$ (Linksinverse von $h$)
\end{itemize}
Damit berechnen wir (multiplikative Schreibweise: $a\circ b = ab$):
\begin{equation*}
hg = e = kh = k((hg)h) = k(h(gh)) = (kh)(gh) = gh
\end{equation*}
und
\begin{equation*}
ge = g(hg) = (gh)g = eg
\end{equation*}
Jedes (links-)neutrale Element $e$ ist also auch rechtsneutral:
\begin{equation*}
\forall g\in G: eg = ge = g
\end{equation*}
und ist $e'\in G$ auch neutrales Element, dann:
\begin{equation*}
e' = ee' = e'e = e
\end{equation*}
Weiters ist jedes (Links-)Inverse auch rechtsinvers:
\begin{equation*}
\forall g \in G: gg^{-1}=g^{-1}g = e
\end{equation*}
und sind $h,h'\in G$ Inverse von $g\in G$, so gilt:
\begin{equation*}
h' = h'(gh) = (h'g)h = h
\end{equation*}
d.h. Eindeutigkeit des Inversen.

\paragraph{Definition (Körper):} Ein Tripel $(K,+,\cdot)$, bestehend aus einer Menge $K$ und zwei Verknüpfungen
\begin{align*}
+:&K\times K\to K,(x,y)\mapsto x+y\\
\cdot : &K\times K\to K, (x,y)\mapsto xy
\end{align*}
heißt Körper, falls:
\begin{enumerate}[(i)]
\item $(K,+)$ ist abelsche Gruppe (mit neutralem Element $0$ und inversem Element $-x$ von $x$)
\item $(K^\times,\cdot)$ ist abelsche Gruppe (mit neutralem Element $1$ und inversem Element $\frac{1}{x} = x^{-1}$ von $x\in K^\times$)
\item die Distributivgesetze gelten:
\end{enumerate}
\begin{equation*}
\forall x,y,z\in K :\begin{cases}x\cdot (y+z) = xy+xz\\ (x+y)\cdot z = xz+yz \end{cases}
\end{equation*}

\paragraph{Bemerkung} In einem Körper gilt stets:
\begin{equation*}
0\cdot x = x\cdot 0 = 0 \Rightarrow
0\cdot x = (0+0)\cdot x = 0\cdot x + 0\cdot x \Rightarrow
0 = 0\cdot x + (-(0\cdot x)) \Rightarrow 0 = 0\cdot x
\end{equation*}
Insbesondere folgt damit: $\forall x,y\in K: xy = yx$ (nicht nur für $K^\times$ (Axiom)).
\paragraph{Beispiel:} Die rationalen Zahlen $\mathbb{Q}$, die reellen Zahlen $\mathbb{R}$ und die komplexen Zahlen $\mathbb{C}$ bilden mit den üblichen Verknüpfungen Körper.




\paragraph{Bemerkung und Beispiel} Aufgrund der Axiome i) und ii) enthält K mindestens 2 Elemente.

\begin{equation*} 
\# K \ge Z,
\end{equation*}

nämlich:

\begin{itemize}
\item 0, das neutrale Elemente bezüglich + und
\item 1 $(\neq 0)$, das neutrale Elemente (in $K\times$ = $K\setminus\{0\}$) bezüglich $\cdot$
\end{itemize}

Es gibt auch einen Körper mit genau 2 Elementen $({0,1},+,*)$, wobei

\begin{equation*}
\begin{tabular}{c|c|c}
	$+$ & 0 & 1\\\hline
	0 & 0 & 1\\
	1 & 1 & 0\\
\end{tabular}
\end{equation*}
\begin{equation*}
\begin{tabular}{c|c|c}
$*$ & 0 & 1\\\hline
0 & 0 & 1\\
1 & 1 & 0\\
\end{tabular}
\end{equation*}

Dieser Körper wird auch $\mathbb{Z}\textsubscript{2}$ bezeichnet.

\paragraph{Bemerkung und Definition} In $\mathbb{Z}\textsubscript{2}:$ 1 + 1 = 0. Allgemeiner definiert man die Charakteristik eines Körpers $(K,+,\cdot)$ (mit neutralen Element 0 und 1 von + und $\cdot$) durch

\begin{equation*}
Char(K,+,\cdot):=
\begin{cases}
0,\text{falls } \forall n \in \mathbb{N}^\times: \sum_{j = 1}^{n} 1 \neq 0\\
min\{n \in \mathbb{N}^\times\mid \sum_{j = 1}^{n} 1 = 1+ ... + 1 = 0\}
\end{cases} 
\end{equation*}

z.B. Char$\mathbb{Z}\textsubscript{2}:$ = $\mathbb{Z}:$, da

\begin{align*}
\{n\in\mathbb{N}^\times\mid 1+...+1=0\}=\\
=\{n\in\mathbb{N}^\times\mid n=0 \text{ mod } 2\}=\\
=\{n\in\mathbb{N}^\times\mid n \text{ gerade }\}
\end{align*}
und damit
\begin{align*}
	min\{n\in\mathbb{N}^\times\mid 1+...+1=0\}=2
\end{align*}
Wir werden mitunter $Char(K,+,\cdot)$, $\neq$ 0 oder öfters $Char(K,+,\cdot)$, =2 ausschließen (müssen).

Translation "der" Ebene bilden eine abelsche Gruppe.

\paragraph{Definition:} Sei K ein Körper; Eine Menge V mit zwei Abbildungen

\begin{itemize}
	\item +: V $\times$ V $\to$ V:(v,u)$\mapsto$ v+w,
	\item $\cdot$: K $\times$ V $\to$ V:(x,v)$\mapsto$ vx,
\end{itemize}

heißt Vektorraum über K (K=VR), falls gilt:
\begin{align*}
	1) (V,+) \text{ ist eine abelsche Gruppe}
	\\2) \forall v\in V: v-1=v \text{ und}
	\\	\forall x,y \in K, \forall v\in V= (vx)y = v(xy)
\\	3) \forall x,y \in K \forall v\in V: v(x+y) = vx + vy
	\\	\forall x\in K \forall v\in V: (v+w)x = vx + wx
\end{align*}

\paragraph{Bemerkung} Wir notieren die Skalarmultiplikation als Rechtsmultiplikation (Skalar steht rechts) 
\begin{align*}
\cdot: K \times V \to V : (x,v) \mapsto vx
\end{align*}
Einen Grund werden wir später sehen
\paragraph{Beispiel} Die Translationen eines affinen Raumes bilden einen Vektorraum (vgl mit der Skizze oben): Diese Beispiel wird im nächsten Kapitel repräsentiert.
\paragraph{Beispiel} Jeder Körper K ist ein K-VR (Vektorraum über sich selbst): das ist ein (trivialer) Spezialfall des folgenden...
\paragraph{Beispiel und Definition} Ist I eine Menge und K ein Körper, so bilden die K-wertigen Abbildungen
\begin{align*}
v: I \to K: i \mapsto v;
\end{align*}
einen Vektorraum mit der punktweise definierten Addition und Skalarmultiplikation:
\begin{align*}
I\ni i \mapsto (v+w)_i := v_i+w_i\in K_i\\
I\ni i \mapsto (vx)_i := v_ix \in K_i
\end{align*}
Dieser Vektorraum wird mit $K^{I}$ bezeichnet und Standardvektorraum (über I und K) genannt. Im Falle I=\{1,...,n\} schreibt man auch $K^{n} := K^{\{1,...,n\}}$
\paragraph{Bemerkung und Definition} Anstelle der normalen Schreibweise
\begin{align*}
I\ni i \mapsto (i) \in K
\end{align*}
für die Auswertung einer Abbildung  $v: I \to K$ um einen Punkt $i\in I$ haben wir die Indexschreibweise verwendet.
\begin{align*}
I\ni i \mapsto v_i \in K_i
\end{align*}
Wir haben damit eine Abbildung $v: I \to K$ als Familie von 
\begin{align*}
(v_i)_{i\in I}
\end{align*}
über der Indexmenge I aufgefasst: Der Begriff Familie $"$ist ein alternativer Begriff für Abbildungen$"$.
\paragraph{Beispiel} Sei i eine $"$Zahl$"$ mit $i^2=-1$\\
 \[[\text{Achtung: i hat hier eine andere Bedeutung als oben!}]\]\\
 Die komplexen Zahlen
 \begin{align*}
 \mathbb{C}:=\{{x+iy\mid x,y\in \mathbb{R}}\}
 \end{align*}
bilden mit der Addition und Multiplikation
\begin{align*}
+:\mathbb{C}\times \mathbb{C} \to \mathbb{C}: ((x+y),(x'+y')) \mapsto ((x+y)+(x'+y')) := (x+x')+i(y+y')\\
\cdot :\mathbb{C}\times \mathbb{C} \to \mathbb{C}: ((x+iy)(x'+iy'))\mapsto (x+iy)\cdot (x'+iy') :=(xx'-yy')+i(xy'+x'y)
\end{align*}
$\mathbb{C}$ bildet einen $\mathbb{R}$-VR mit
\begin{align*}
+:\mathbb{C}\times\mathbb{C}\to\mathbb{C}
\end{align*}
wie oben und der Skalarmultiplikation
\begin{align*}
\cdot:\mathbb{R}\times\mathbb{C}\to\mathbb{C}:(x',(x+iy))\mapsto(x+iy)x':=xx'+iyx'
\end{align*}
Diese Skalarmultiplikation ist also gerade die Einschränkung der komplexen Multiplikation auf $\mathbb{R}\times\mathbb{C}$ wobei die Identifikation
\begin{align*}
\mathbb{R}\cong \{{x+iy\in\mathbb{C}\mid y=0}\}
\end{align*}
verwendet wird.
%%%%%%%%%%%%%%%%% BEGINN VO4-20151015 %%%%%%%%%%%%%%%%%%%%%
\end{document}
